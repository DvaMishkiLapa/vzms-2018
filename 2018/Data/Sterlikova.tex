\vzmstitle{
	ОБ АСИМПТОТИЧЕСКИХ ФОРМУЛАХ ДЛЯ СОБСТВЕННЫХ ЗНАЧЕНИЙ И
	СОБСТВЕННЫХ ФУНКЦИЙ ОДНОГО   ОПЕРАТОРА С ИНВОЛЮЦИЕЙ
}

\vzmsauthor{Стерликова}{Д.\,В.}

\vzmsinfo{Воронеж; {\it sterlikovadi@mail.ru}}

\vzmscaption

 Рассматривается функционально-дифференциальный оператор с инволюцией вида:
$$Ly=y'\left( x \right)+q\left( x \right)y\left( 1-x \right),\text{~~}y\left( 0 \right)=y\left( 1 \right).$$
Будем считать, что область определения оператора  L  состоит из
непрерывно-дифференцируемых функций с заданным краевым условием.

Функционально-дифференциальные операторы с инволютивным отклонением
активно исследуются (см, например, работы [1]-[4]  и библиографию в
них). В  работах [3]-[4] инволюция (т.е. отображение  $J(x)$ такое,
что $J(J(x))=x$)  возникает в аргументе производной.  Но ряд задач,
в том числе начально-краевых, описывающих  процессы с отражением,
могут приводить к исследованию рассматриваемого оператора (как,
например в [5]). Для исследования таких задач необходимо не только
знать главные части асимптотик для собственных значений и
собственных функций, но и иметь уточнённые оценки.

Спектральную задачу для оператора  $L$:
$$y'\left( x \right)+q\left( x \right)y\left( 1-x \right)=\lambda y\left( x \right),\text{~~}x\in \left[ 0;1 \right], \eqno(1) $$
$$y\left( 0 \right)=y\left( 1 \right), \eqno(2) $$
пользуясь методами  работ [3-6], приводим к  исследованию уравнения
Дирака для вектор-функции $z(x){=}(z_{1}(x),z_{1}(x))^{T}$ ($T$~---
знак транспонирования):

$$z'(x)+Q(x)z(x)=\lambda Dz(x), \eqno(3) $$
где $Q(x)=\left( \begin{matrix}
   0 & q(x)  \\
   -q(1-x) & 0  \\
\end{matrix} \right)$, $D=diag\left( 1,-1 \right).$

\textbf{Лемма 1.} {\it Если $\mathrm{Re\,} \mu \geqslant 0$,  $q (x)\in
C^1[0,1]$, то для общего решения уравнения (3) имеем следующую
асимптотическую формулу:
$$ z(x, \lambda )=Z(x,  \lambda) e^{\lambda  {D}x}c, $$
где $ Z(x,  \lambda)= (z_{ij}(x,  \lambda))_{i,j=1,2}$,  $c=(c_1,
c_2)^T$ -- произвольный вектор и $z_{kj}(x,
\lambda)=\widetilde{z}_{kj}(x,\lambda)+ O\left( \frac1{ \lambda^2}
\right)$, где
$$\begin{array}{l} \widetilde{z}_{11}(x, \lambda)=1 -\frac1{2 \lambda}\int\limits_0^x q(1-t) q(t)\, dt, \\
\widetilde{z}_{12}(x, \lambda)=\frac1{2 \lambda}\left(q(x)- q(1)
e^{-2\lambda (1- x)}+ \int\limits_x^1 e^{2\lambda (x- t)} q'(t)\,
dt\right)\!,\\
 %
 \widetilde{z}_{21}(x, \lambda)= {-}\frac1{2 \lambda}\!\left({-}q(1{-}x){+}q(1)
e^{-2\lambda  x}{-}\int\limits_0^x e^{-2\lambda (x{-}t)} q'(1{-}t)  dt\right)\!\!,\\
%
\widetilde{z}_{22}(x, \lambda)=1 +\frac1{2 \lambda}\int\limits_0^x
q(1-t) q(t)\, dt. \end{array}$$} \\
Аналогичные
формулы могут быть получены при $\operatorname{Re}\lambda \leqslant 0$.

\textbf{Теорема 1.} {\it Для собственных значений и собственных
функций оператора $L$ имеют место следующие асимптотические формулы:
${{\lambda }_{n}}=2\pi ni+O(1/n),$ ${{y}_{n}}\left( x
\right)={{e}^{2\pi ix}}+O(1/n).$}

Методами из [6] на базе формул из леммы 1  могут быть получены и
уточнённые асимптотики собственных значений и собственных функций.

\litlist

1. {\it Андреев А.А.} Об аналогах классических краевых задач для одного
дифференциального уравнения второго порядка с отклоняющимся
аргументом // Дифферец. ур-ния. 2004. Т. 40. № 5. С. 1126-1128.

2. {\it Андреев  А.А., Саушкин И.Н.} Об аналоге задачи Трикоми для одного
модельного уравнения с инволютивным отклонением в бесконечной
области// Вестник Самарского государственного технического
университета. Серия Физико-математические науки. --- 2005. № 34.
---С. 10--16.

3. {\it Бурлуцкая М.Ш., Курдюмов В.П.,  Луконина А.С.,  Хромов А.П.}
Функционально-дифференциальный оператор с инволюцией  // Докл. РАН.
- 2007. - Т. 414, № 4. - С. 443-446.

4. {\it Бурлуцкая М.Ш., Хромов А.П.}  Об одной теореме равносходимости  на
всем отрезке для функционально-диффе\-рен\-циаль\-ных операторов //
Изв. Сарат. ун-та. 2009. Т. 9. Сер. Математика. Механика.
Информатика. вып. 4. - С.3-10.

5. {\it Бурлуцкая М.Ш., Хромов А.П.} Метод Фурье в  смешанной задаче для
уравнения первого порядка   с инволюцией // Журнал вычислительной
математики и математической физики. 2011, том 51, № 12, С. 2233-2246

6. {\it Бурлуцкая М.Ш.} Асимптотические формулы для собственных значений и
собственных функций функционально-дифференциального оператора с
инволюцией/ М.Ш. Бурлуцкая // Вестник Воронеж. Гос. Ун-та. Сер.
Физика, математика. 2011, №2, С.64-72
