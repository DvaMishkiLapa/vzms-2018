\begin{center}{ \bf  О ГЛАДКОСТИ РЕШЕНИЙ ВЫРОЖДАЮЩЕГОСЯ ДИФФЕРЕНЦИАЛЬНОГО УРАВНЕНИЯ}\\
{\it П. Н. Зюкин, И. В. Сапронов, В. В. Зенина } \\
(Воронеж; {\it pzukin@mail.ru} )
\end{center}
\addcontentsline{toc}{section}{Зюкин П.Н., Сапронов И. В., Зенина В. В.\dotfill}

Рассматривается дифференциальное уравнение
$$xy\prime \prime+a_{1}(x)y\prime+a_{0}(x)y=h(x)\eqno(1)$$
\noindent где $x\in [0,1]$, $h(x)$ - непрерывная на отрезке $[0,1]$ функция со значениями в комплексном банаховом пространстве $E$, $a_0 (x)$, $a_1 (x)$ - непрерывные на отрезке $[0,1]$ функции со значениями в пространстве $L$ линейных ограниченных операторов, действующих из $E$ в $E$. Уравнение (1) вырождается при $x=0$. В настоящей работе приводятся условия существования гладких вплоть до точки $x=0$ решений дифференциального уравнения (1). Эти условия получены на основе результатов, содержащихся в работах [1], [2].

Пусть $\overline{C}((0,1];E)$ - пространство функций, принадлежащих $C((0,1];E)$, каждую из которых можно продолжить на отрезок $[0,1]$ до функции класса $C([0,1];E)$, $n$ - натуральное число. Введём функциональное пространство с <<весом>>
$$C^n_1([0,1];E)=\{z(x)\in C^{n-1}([0,1];E):xz^{(n)}(x)\in \overline{C}((0,1];E)\}.$$

Пусть $I$ - тождественный оператор в $E$ , $\sigma (a_1(0))$ - спектр оператора $a_1(0)$.

\textbf{Теорема~1.} {\it Пусть $m$ - наименьшее из натуральных чисел $k$ таких, что $\sigma (a_1(0))\subset \{\lambda : Re \lambda > -k\}$. Пусть $m>1$ и в дифференциальном уравнении (1) функции $a_0 (x)$, $a_1(x)$ принадлежат $C^n ([0,1];L)$, $h(x) \in C^n ([0,1];E)$, где $n$ - натуральное число, $n\geqslant m$. Если стистема уравнений
$$ \begin {cases}
a_1(0) \psi_1 +a_0 (0) \psi_0 =h(0),\\
(a_1 (0) +iI)\psi_{i+1}+a_0(0)\psi_i+ \\
+\sum \limits _{j=1}^i C_i^j (a_1 ^{(j)} (+0)\psi_{i-j+1} +a_0^{(j)} (+0) \psi_{i-j} )=h^{(i)}(+0), \\  i=1,2,...,m-1
\end {cases} \eqno (2)$$
\noindent разрешима относительно элементов $\psi_0$, $\psi_1$, ..., $\psi_m$ пространства $E$, то каждому решению $\psi_0$, $\psi_1$, ..., $\psi_m$ этой системы уравнений соответствует единственное решение $y(x)$ дифференциального уравнения (1), принадлежащее \\ $C^{m+1} ([0,1];E)$ и удовлетворяющее условиям
$$y(0)=\psi_0,y\prime (+0)=\psi_1, y\prime \prime (+0)=\psi_2,...,y^{(m)}(+0)=\psi_m;$$
\noindent это решение $y(x)$ принадлежит также $C_1^{n+2} ([0,1];E)$. Если система уравнений (2) неразрешима относительно элементов $\psi_0$, $\psi_1$, ..., $\psi_m$ пространства $E$, то дифференциальное уравнение (1) не имеет решения, принадлежащего    \\ $C^{m+1}([0,1];E)$. }

\textbf{Замечание~1.} {\it Теорема 1 остаётся справедливой и в случае $m=1$, если систему уравнений (2) заменить уравнением $a_1(0)\psi_1+a_0(0)\psi_0=h(0)$.}



%%%%  ОФОРМЛЕНИЕ СПИСКА ЛИТЕРАТУРЫ %%%
\smallskip \centerline{\bf Литература}\nopagebreak

1. {\it Зюкин П.Н.} О гладких решениях линейного вырождающегося дифференциального уравнения $\ell$-го порядка / П.Н. Зюкин // Воронеж. гос. лесотехн. акад. - Воронеж, 2005. - 15с. - Деп. в ВИНИТИ 29.1111.05, №1562 - В2005.

2. {\it Зюкин П.Н.} О гладких решениях линейного вырождающегося дифференциального уравнения / П.Н. Зюкин // Математические модели и операторные уравнения: сб. научн. тр. / ВорГУ. - Воронеж, 2003. - Т. 2. - С. 68-74.

