\begin{center}{ \bf  О ГЕОМЕТРИЧЕСКОМ ОПРЕДЕЛЕНИИ ОТОБРАЖЕНИЙ С S- УСРЕДНЕННОЙ ХАРАКТЕРИСТИКОЙ}\\
{\it А.Н. Малютина, У.К. Асанбеков } \\
(Томск; Национальный исследовательский Томский государственный университет; {\it nmd@math.tsu.ru} )
\end{center}
\addcontentsline{toc}{section}{Малютина А.Н., Асанбеков У.К.\dotfill}


В работе мы продолжаем развивать геометрический метод изучения свойств пространственных отображений с $s$- усредненной характеристикой, основанного на характеристическом законе искажения модулей семейств кривых.

\textbf{Определение~1.} Следуя [1], определим $\alpha$- модуль семейств кривых. Пусть $\Gamma$ семейство кривых в $R^{n}$. Обозначим $\rho\wedge\Gamma$ множество неотрицательных борелевских функций $p:R^{n}\rightarrow[0,\infty]$ таких, что
\begin{center}
	$\int\limits_{\gamma}\rho dl_{x}\geqslant1$
\end{center}

для каждой спрямляемой кривой $\gamma\in\Gamma$, где $dl_{x}=\frac{dl}{1+|x|^{2}}$. Для любого $\alpha\geqslant1$ величину	 $M_{\alpha(\Gamma)}=\inf\limits_{\rho\wedge\Gamma}\int\limits_{R^{n}}\rho^{\alpha}d\sigma_{x}$, где $d\sigma_{x}=\frac{dx}{(1+|x|^{2})^{n}}$, $\frac{1}{n-1}\leqslant\alpha$ назовем сферическим $\alpha$- модулем семейств кривых $\Gamma$.


Дадим, как и в [1], определение отображения с $s$- усредненной характеристикой $1<s<n$, примеры таких отображений и вычисление сферического модуля приведены в [2 - 4].

\textbf{Определение~2.} Пусть отображение $D, D'\subset R^{n}$, $f:D\rightarrow D'$ назовем отображением класса $f\in\tilde{W}^{1}_{n,loc}(D)$, если $f\in W^{1}_{n,loc}(D)$, $f^{-1}\in W^{1}_{n,loc}(D')$ и обладает $N,N^{-1}$- свойствами.


\textbf{Определение~3.} Пусть область $D\subset R^{n}$, $f:G\rightarrow R^{n}$, открыто, непрерывно, изолировано, $f\in\tilde{W}^{1}_{n,loc}(D)$, и сохраняет ориентацию (можно считать, что якобиан отображения, вычисленный в точке $x\in D$, $J(x,f)>0$) и конечны интегралы $\int\limits_{D}K_{I}^{s}(x,f)|J(x,f)|d\sigma_{x}$ и $\int\limits_{D}K_{O}^{s'}(x,f)d\sigma_{x}$.


\textbf{Определение~4.} Скажем, что открытое дискретное непрерывное отображение $f$:

    1) принадлежит классу $Q_{s}(D)$, где $\frac{1}{n-1}\leqslant s<\infty$, если для любого $p$, такого что $\frac{1}{n-1}\leqslant p<s$, существует $\Phi_{p}$- неотрицательная ограниченная $\sigma$- аддитивная функция борелевских множеств в $D$, такая, что для любого семейства кривых $\Gamma$ из $D$ и произвольного борелевского множества $U\subset D$, содержащего все кривые из $\Gamma$, выполняется неравенство
 \begin{equation}\label{eq-1}
 	M_{\frac{np}{(p-1)}}^{p+1}(\Gamma')\leqslant\Phi_{p}(U)M^{p}_{n}(\Gamma),
 \end{equation}
где $\Gamma'=f(\Gamma)$.

Обозначим $q_{s}(D)$- подкласс $q_{s}(D)\subset Q_{s}(D)$ отображений, для которых выше определенные функции $\Phi_{p}$ являются абсолютно непрерывными функциями борелевских множеств. Очевидно, что если $s'<s$, то $Q_{s}(D)\subset Q_{s'}(D)$, $q_{s}(D)\subset q_{s'}(D)$.

2) принадлежит классу $Q_{s'}(D')$, где $n-1\leqslant s'<\infty$, существует $\Psi_{s'}$- неотрицательная ограниченная $\sigma$- аддитивная функция борелевских множеств в $D'$, такая, что для любого семейства кривых $\Gamma'$ из $D'$ и произвольного борелевского множества $U'\subset D'$, содержащего все кривые из $\Gamma'$, выполняется неравенство
\begin{equation}\label{eq-2}
M_{s}^{n}(\Gamma')\leqslant[\Phi_{s'}(U')]^{s}M^{\frac{s}{\beta}}_{n\beta}(\Gamma),
\end{equation}
где $U'=f(U),\Gamma'=f(\Gamma)=\{f(\gamma):\gamma\in\Gamma\},\beta=s(s-1)^{-1}$;

3) принадлежит классу $Q_{s,s'}(D)$, где $n-1\leqslant s,s'<\infty$, если выполняются оба неравенства (1), (2) искажения модулей любых семейств кривых $\Gamma\in D$ и $\Gamma'\in D'$.

Обозначим $q_{s'}(D'),q_{s,s'}(D)$- подкласс $q_{s'}(D')\subset Q_{s'}(D')$, $q_{s,s'}(D')\subset Q_{s,s'}(D')$ отображений, для которых функции $\Phi_{p},\Psi_{s'}$, существование которых доказано в [4] (теоремы 1, 2 и следствия из них) являются абсолютно непрерывными функциями борелевских множеств.

Основным результатом этой работы является следующая теорема.


\textbf{Теорема~1.} Открытое отображение $f:D\rightarrow R^{n}$, $f\in Q_{s}(D)$ при $n-1\leqslant s<\infty$ есть $ACL^{n}$- отображение, дифференцируемое п.в. в области $D$, и такое, что конечен интеграл
\begin{equation*}
\int\limits_{D}K^{s}_{I}(x,f)|J(x,f)|d\sigma_{x}<\infty.
\end{equation*}


%%%%  ОФОРМЛЕНИЕ СПИСКА ЛИТЕРАТУРЫ %%%
\smallskip \centerline{\bf Литература}\nopagebreak

1. {\it Елизарова М.А., Малютина А.Н. } Отображения с s-усредненной характеристикой. Определение и свойства. LAMBERT Academic Publishing, 2013. 121 с. ISBN 978-3-8484-1319-5.


2. {\it Асанбеков У.К., Малютина А.Н. } Вычисление модуля сферического кольца. В книге: Комплексный анализ и приложения материалы VIII Петрозаводской международной конференции, 2016. С. 103-106.

3. {\it	Alipova K., Elizarova M., Malyutina A. } Examples of the mappings with s-averaged characteristic. В сборнике: Комплексный анализ и его приложения материалы VII Петрозаводской международной конференции, 2014. С. 12-17.

4. {\it	Малютина А.Н., Елизарова М.А. } Оценки искажения модулей для отображения с s-усредненной характеристикой //Вестник Томского государственного университета. Математика и механика, 2010. №2 (10). С. 5-15.

