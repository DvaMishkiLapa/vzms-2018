\vzmstitle{ \bf  ОБ ОДНОЙ СМЕШАННОЙ ЗАДАЧЕ ДЛЯ ВОЛНОВОГО УРАВНЕНИЯ НА ГРАФЕ}

\vzmsauthor{{Павлюк}}{Я.\, П.}

\vzmsinfo{Воронеж; {\it ioanna1997@yandex.ru}}

\addcontentsline{toc}{section}{Павлюк Я.П.\dotfill}

На простейшем графе из двух рёбер, первое из которых образует кольцо(цикл), а второе примыкает к нему,  рассматривается следующая смешанная задача, трактуемая как задача в пространстве вектор-функций$u(x,t)=(u_{1} (x,t),u_{2} (x,t))^{T} $(\textit{T~}- знак транспонирования):


$$\frac{\partial ^{2} u_{j} (x,t)}{\partial t^{2} } =\frac{\partial ^{2} u_{j} (x,t)}{\partial x^{2} } ,(j=1,2),t\in (-\infty ,\infty ),x\in [0,1],  \eqno(1) $$
$$u_{1} \left(0,t\right)=u_{1} \left(1,t\right)=u_{2} \left(0,t\right),u_{2} \left(1,t\right)=0, \eqno(2) $$
$$u_{1x}^{{'} } \left(0,t\right)-u_{1x}^{{'} } \left(1,t\right)+u_{2x}^{{'} } \left(0,t\right)=0, \eqno(3) $$
$$u_{1} \left(x,0\right)=\phi _{1} \left(x\right),u_{2} \left(x,0\right)=\phi _{2} \left(x\right),x\in \left[0,1\right],\eqno(4)$$
$$u_{1t}^{{'} } \left(x,0\right)=u_{2t}^{{'} } \left(x,0\right)=0. \eqno(5) $$


Используя методы из [1], получено классическое решение задачи (1)-(5) при минимальных требованиях на $\phi \left(x\right)=\left(\phi _{1} \left(x\right),\phi _{2} \left(x\right)\right)^{T} $:$\phi _{j} \left(x\right)\in C^{2} \left[0,1\right]$ и комплекснозначные,


$$\phi _{1} \left(0\right)=\phi _{1} \left(1\right)=\phi _{2} \left(0\right),\phi _{2} \left(1\right)=0,\phi _{1}^{{'} } \left(0\right)-\phi _{1}^{{'} } \left(1\right)+\phi _{2}^{{'} } \left(0\right)=0,\eqno(6)$$
$$\phi _{1}^{{'} {'} } \left(0\right)=\phi _{1}^{{'} {'} } \left(1\right)=\phi _{2}^{{'} {'} } \left(0\right),\phi _{2}^{{'} {'} } \left(1\right)=0 \eqno(7) $$


(условие (7) следует из дифференциальной системы).

Как известно, решение начальной задачи для уравнения (1) задаётся формулой Даламбера:


$$ u\left(x,t\right)=\frac{1}{2} \left(F(x+t)+F\left(x-t\right)\right). \eqno(8) $$


Покажем, что решение задачи (1)-(5) при условиях (6)-(7) на начальные данные также даётся аналогом формулы Даламбера (см.также [1], [2]).

\textbf{Лемма 1.} Пусть функция $u(x,t)$ из (8), где $F\left(x\right)=\left(F_{1} \left(x\right),F_{2} \left(x\right)\right)^{T} ,$и$F_{k} (x)$ - дважды дифференцируемые функции, удовлетворяет задаче (1)-(5). Тогда $F\left(x\right)=\phi \left(x\right)$ при $x\in \left[0,1\right],$ и $F_{k} (x)$ удовлетворяют следующим соотношениям:

\[F_{1} \left(-t\right)=\frac{1}{2} \left(F_{1} \left(1-t\right. \right)-\left. F_{1} \left(t\right)\right)+F_{2} \left(t\right),F_{2} \left(-t\right)=\frac{1}{2} \left(F_{1} \left(t\right. \right)+\left. F_{1} \left(1-t\right)\right),\]

\[F_{1} \left(1+t\right)=\frac{1}{2} \left(F_{1} \left(t\right. \right)-\left. F_{1} \left(1-t\right)\right)+F_{2} \left(t\right),F_{2} \left(1+t\right)=-F_{2} \left(1-t\right).\]

\textbf{Доказательство.  }Распишем покомпонентно вектор $u\left(x,t\right)$ из (8):

\[u_{1} \left(x,t\right)=\frac{1}{2} \left(F_{1} (x+t)+F_{1} \left(x-t\right)\right),u_{2} \left(x,t\right)=\frac{1}{2} \left(F_{2} (x+t)+F_{2} \left(x-t\right)\right).\]

При $t=0,x\in \left[0,1\right]$ получим: $u_{1} \left(x,0\right)=F_{1} \left(x\right)=\phi _{1} \left(x\right),$$u_{2} \left(x,0\right)=F_{2} \left(x\right)=\phi _{2} \left(x\right).$Выпишем значения вектор-функции $u\left(x,t\right)$при$x=0$и$x=1:$

\[u_{1} \left(0,t\right)=\frac{1}{2} \left(F_{1} (t)+F_{1} \left(-t\right)\right),u_{1} \left(1,t\right)=\frac{1}{2} \left(F_{1} (1+t)+F_{1} \left(1-t\right)\right),\]

\[u_{2} \left(0,t\right)=\frac{1}{2} \left(F_{2} (t)+F_{2} \left(-t\right)\right),u_{2} \left(1,t\right)=\frac{1}{2} \left(F_{2} (1+t)+F_{2} \left(1-t\right)\right).\]

Из краевых условий (2) и (3) получим:


$$F_{1} \left(t\right)+F_{1} \left(-t\right)=F_{1} \left(1+t\right)+F_{1} \left(1-t\right),  \eqno(9) $$
$$F_{1} \left(t\right)+F_{1} \left(-t\right)=F_{2} \left(t\right)+F_{2} \left(-t\right),  \eqno(10) $$
$$F_{2} \left(1+t\right)+F_{2} \left(1-t\right)=0, \eqno(11) $$
$$F_{1}^{{'} } \left(t\right)+F_{1}^{{'} } \left(-t\right)-F_{1}^{{'} } \left(1+t\right)-F_{1}^{{'} } \left(1-t\right)+F_{2}^{{'} } \left(t\right)+F_{2}^{1} \left(-t\right)=0. \eqno(12) $$


Из условий на функцию $\phi \left(x\right)$имеем:

\[F_{1} \left(0\right)=F_{1} \left(1\right)=F_{2} \left(0\right),F_{2} \left(1\right)=0,\]

\[F_{1}^{{'} } \left(0\right)-F_{1}^{{'} } \left(1\right)+F_{2}^{{'} } \left(0\right)=0,\]

\[F_{1}^{{'} {'} } \left(0\right)=F_{1}^{{'} {'} } \left(1\right)=F_{2}^{{'} {'} } \left(0\right),F_{2}^{{'} {'} } \left(1\right)=0.\]

Выразим $F_{k} \left(-t\right)$ и $F_{k} \left(1+t\right)$ через$F_{k} \left(t\right)$ и$F_{k} \left(1-t\right),$$k=1,2.$ Продифференцируем  (9)  по переменной $t,$ и полученное уравнение вычтем из (12):


$$F_{1}^{{'} } \left(-t\right)-F_{1}^{{'} } \left(1-t\right)+F_{2}^{{'} } \left(t\right)+F_{2}^{{'} } \left(-t\right)=0. $$


Проинтегрировав это соотношение с учётом приведённых выше условий на функцию $F(x)$ при $t=0$, получим уравнение, которое будем рассматривать вместо (12):

\[F_{1} \left(-t\right)-F_{1} \left(1-t\right)-F_{2} \left(t\right)+F_{2}^{{'} } \left(t\right)+F_{2} \left(-t\right)=0.\]

Таким образом, система (9)-(12) примет вид:


$$F_{1} \left(t\right)+F_{1} \left(-t\right)=F_{1} \left(1+t\right)+F_{1} \left(1-t\right), $$
$$F_{1} \left(t\right)+F_{1} \left(-t\right)=F_{2} \left(t\right)+F_{2} \left(-t\right),$$
$$F_{2} \left(1+t\right)+F_{2} \left(1-t\right)=0,$$
$${F_{1} \left(-t\right)-F_{1} \left(1-t\right)-F_{2} \left(t\right)+F_{2}^{{'} } \left(t\right)+F_{2} \left(-t\right)=0.} $$


Разрешая эту систему относительно $F_{k} \left(-t\right)$ и $F_{k} \left(1+t\right)$, придём к утверждению леммы${\rm \blacktriangle }$

Непосредственной проверкой можно убедиться, что функция $F(x)$ из леммы 1 является дважды непрерывно-дифференцируемой периодической функциейс периодом 2. Отсюда следует основной результат.

\textbf{Теорема.} Классическое решение задачи (1)-(5) имеет вид $u\left(x,t\right)=\frac{1}{2} \left(F(x+t)+F\left(x-t\right)\right),$ где $F\left(x\right)=\left(F_{1} \left(x\right),F_{2} \left(x\right)\right)^{T} ,$$F_{k} \left(x\right)$ - дважды дифференцируемая периодическая функция (с периодом 2), причём $F\left(x\right)=\phi \left(x\right)$при$x\in \left[0,1\right]$, а на отрезке $\left[-1,0\right]$$F\left(x\right)$ определяется соотношениями:

\[F_{1} \left(-t\right)=\frac{1}{2} \left(F_{1} \left(1-t\right. \right)-\left. F_{1} \left(t\right)\right)+F_{2} \left(t\right),F_{2} \left(-t\right)=\frac{1}{2} \left(F_{1} \left(t\right. \right)+\left. F_{1} \left(1-t\right)\right).\]


\litlist

1. {\it Бурлуцкая М.Ш., Хромов А.П.} Резольвентный подход в методе Фурье / М.Ш.Бурлуцкая, А.П.Хромов //Докл. РАН. -- 2014. - Т. 458, №2. -- С. 138-140.

2. {\it Бурлуцкая М.Ш.} Метод Фурье в смешанной задаче для волнового уравнения на графе / М.Ш.Бурлуцкая// Докл. РАН. - 2015. - Т. 465, № 5. - С.~519--522.
