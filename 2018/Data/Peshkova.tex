\begin{center}{ \bf  НЕТ НАЗВАНИЯ}\\
{\it Е. Пешкова } \\
(Воронеж; {\it evgenya$\_$peshkova@mail.ru} )
\end{center}
\addcontentsline{toc}{section}{Пешкова Е.\dotfill}

Вейвлет - от английского слова "wavelet" означает в переводе "маленькая волна", или "волны, идущие друг за другом". К.И. Осколков предложил термин "всплеск". Именно этот термин мы и будем использовать в дальнейшем. Термин всплеск лучше отражает суть дела, так как, в самом общем виде, ondelette - это затухающее колебание.
Всплески - это семейство функций, которые локальны вовремени и по частоте ("маленькие"), и в которых все функции получаются из одной посредством её сдвигов и растяжений по оси времени (так что они "идут друг за другом").
Все всплеск-преобразования рассматривают функцию (взятую будучи функцией от времени) в терминах колебаний, локализованных по времени и частоте. Всплеск-преобразования бывают следующих типов: дискретноевейвлет-преобразование (ДВП) и непрерывное вейвлет-преобразование (НВП).
Если рассматривать применение, то ДВП обычно используется для кодирования сигналов, в то время как НВП для анализа сигналов. В результате, ДВП широко применяется в инженерном деле и компьютерныхнауках, а НВП в научных исследованиях. Всплеск -преобразования в настоящее время приняты на вооружение для огромного числа разнообразных применений, нередко заменяя обычное преобразование Фурье во многих применениях. Эта смена парадигмы наблюдается во многих областях физики, включая молекулярную динамику, астрофизику, локализацию матрицы плотности, сейсмическую геофизику, оптику, турбулентность, квантовую механику, обработку изображений, анализыкровяного давления, пульса и ЭКГ, анализ ДНК, исследования белков, исследования климата, общую обработку сигналов, распознавание речи, компьютерную графику и мультифрактальный анализ и другие. Всплеск анализ применяется для анализа нестационарных медицинских сигналов, в том числе вэлектрогастроэнтерографии.

Теория всплесков - интенсивно развивающееся межпредметное направление, включающее в себя исследования из области теоретической математики, прикладной математики, информатики. Один из первых примеров всплесковых базисов, построен И. Мейером в 1986 г. и носит его имя. В настоящее время семейство всплеск-функции? Мейера и его модификации находят многочисленные применения в математическом и функциональном анализе, теории функции?, численных методах решения дифференциальных уравнений.
Модифицированные всплески Добеши это компактификация всплесков Мейера.
Мой научный руководитель - И. Я. Новиков - построил семейство модифицированных всплеск-функции? Добеши (всплеск-функции Новикова), имеющих компактный носитель, причем локализованность по времени и частоте автокорреляционной функции, построенной для масштабирующей функции данной всплеск-функции, сохраняется с возрастанием гладкости. Целью моей бакалаврской работы является разработка алгоритма, который раскладывает непрерывную функцию в ряд по модифицированным всплескам Добеши. Данный алгоритм разрабатывается в программе WolframMathematica.
