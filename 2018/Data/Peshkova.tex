\begin{center}{ \bf  Алгоритм разложения функций по модифицированным всплескам Добеши}\\
{\it Е. Пешкова } \\
(Воронеж; {\it evgenya\_peshkova@mail.ru} )
\end{center}
\addcontentsline{toc}{section}{Пешкова Е.\dotfill}

Вейвлет -- от английского слова <<wavelet>> -- означает в переводе <<маленькая волна>>  или <<волны, идущие друг за другом>>.  К.И. Осколков предложил термин <<всплеск>>. Термин всплеск лучше отражает суть дела, так как, в самом общем виде, <<wavelet>> --- это затухающее колебание.
В настоящее время термин <<всплеск>> используется в работах по теоретической математике,  в исследованиях же по прикладным вопросам закрепился термин <<вейвлет>>.

Всплески -- это семейство функций, которые локальны во времени и по частоте и в которых все функции получаются из одной посредством её сдвигов и растяжений по оси времени.
Всплеск-преобразования бывают следующих типов: дискретное всплеск-преобразование (ДВП) и непрерывное всплеск-преобразование (НВП).
ДВП обычно используется для кодирования сигналов, в то время как НВП для анализа сигналов. В результате ДВП широко применяется в инженерном деле и компьютерных науках, а НВП -- в научных исследованиях. Всплеск-преобразования в настоящее время используются в различных областях, заменяя обычное или оконное преобразования Фурье. Это наблюдается во многих областях физики, включая молекулярную динамику, астрофизику, локализацию матрицы плотности, сейсмическую геофизику, оптику, турбулентность, квантовую механику, обработку изображений, анализы кровяного давления, пульса и ЭКГ, анализ ДНК, исследования белков, исследования климата, общую обработку сигналов, распознавание речи, компьютерную графику и мультифрактальный анализ. Всплеск-анализ применяется для анализа нестационарных медицинских сигналов, в том числе в электрогастроэнтерографии.

Теория всплесков -- интенсивно развивающееся межпредметное направление, включающее в себя исследования из теоретической математики, прикладной математики, информатики [1]. Один из первых примеров всплеск-базисов построен И. Мейером в 1986 г. и носит его имя. В настоящее время семейство всплесков Мейера и его модификации находят многочисленные применения в математическом и функциональном анализе, теории функций, численных методах решения дифференциальных уравнений.
 И. Я. Новиков построил семейство модифицированных всплесков Добеши (всплески Новикова), имеющих компактный носитель, причём локализованность по времени и частоте автокорреляционной функции, построенной для масштабирующей функции, сохраняется с возрастанием гладкости [2]. Сохранение локализованности модифицированных всплесков существенно отличает их от классических всплесков Добеши.
Модифицированные всплески Добеши это компактификация всплесков Мейера.
Доклад посвящён алгоритму, который раскладывает функцию в ряд по модифицированным всплескам Добеши. Данный алгоритм разработан в программе WolframMathematica.

\smallskip \centerline{\bf Литература}\nopagebreak

1. {\it Новиков И.\,Я.} Теория всплесков /  И. Я. Новиков, В. Ю. Протасов , М. А. Скопина
//ФИЗМАТЛИТ -- Москва, 2015. -- 612 С.

2. {\it Novikov I.\,Ya.} Modified Daubechies wavelets preserving localization with growth of smoothness / I. Ya. Novikov
// East J. Appr. -- 1995, V. 1, № 3. -- С.~341-348
