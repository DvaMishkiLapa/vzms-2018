\begin{center}{ \bf ТЕОРЕМА СУЩЕСТВОВАНИЯ И ЕДИНСТВЕННОСТИ  РЕШЕНИЯ КРАЕВОЙ ЗАДАЧИ ДЛЯ ДИФФЕРЕНЦИАЛЬНОГО УРАВНЕНИЯ ДРОБНОГО ПОРЯДКА}\\
{\it М.В.Кукушкин } \\
(Железноводск; {\it kukushkinmv@rambler.ru} )
\end{center}
\addcontentsline{toc}{section}{Кукушкин М.В.\dotfill}


  В 1960 году  Киприянов И.А. в своей работе  [1] посвященной свойствам одноименного оператора,
 сформулировал теорему существования и единственности решения краевой задачи для   уравнения  второго порядка в частных производных  с дробной производной в младших членах.
  В дальнейшем   Нахушев А.М., Джрбашян М.М.  одни из первых в своих работах исследова-ли дифференциальный оператор второго порядка с дробными производными в младших членах.

 В данной работе доказана теорема существования и единствен-ности решения краевой задачи для дифференциального уравне-ния второго порядка с дробной производной в смысле Киприянова в младших членах.  Благодаря редукции оператора Киприянова к оператору дробного дифференцирования в смысле Маршо в одномерном случае, полученные результаты можно считать действительными для оператора дробного дифференцирования в смысле Римана-Лиувилля, ввиду известного факта совпадения этих операторов на классах функций представимых дробным интегралом.


  Следуя обозначениям [1] будем полагать $\Omega$ --- выпуклая область $n$ --- мерного евклидова пространства, $P$ --- фиксирован-ная точка границы $\partial\Omega,$
     $Q(r,\vec{\mathbf{e}})$ --- произвольная точка области $\Omega;$  обозначим через $\vec{\mathbf{e}}$ --- единичный вектор имеющий направле-ние от $P$ к $Q,$ через $r$ --- евклидово расстояние между точками $P$ и $Q.$
    Под классом ${\rm Lip}\, \mu,\;0<\mu\leq1 $ будем понимать множество функций удовлетворяющих условию Гельдера-Липшица в области $\bar{\Omega}.$

  Оператор дробного дифференцирования в смысле Киприянова    определенный в  [1]  действует следующим образом
 $$
\mathfrak{D}^{\alpha}:H_{0}^{1}  (\Omega)\rightarrow L_{2}(\Omega),\;0<\alpha<1
$$
 $$
 (\mathfrak{D}^{\alpha}f) (Q)=-\frac{\alpha\, r^{1-n}}{\Gamma(1-\alpha)}\int\limits_{0}^{r} \frac{ \triangle^{-t}_{\bar{\mathbf{e}}}f(Q)}{t^{\alpha }}  \left( r-t \right) ^{n-1} dt +
 $$
 $$
  + C^{(\alpha)}_{n} f(Q)   r ^{  -\alpha},\,\triangle^{-t}_{\bar{\mathbf{e}}}f(Q)=[f(Q-\vec{\mathbf{e}}t)- f(Q)]/t.
 $$
  Рассмотрим оператор
 $$
Lu:=-  D_{j} ( a^{ij} D_{i}u)  +p\, \mathfrak{D}^{ \alpha }u,\;\;  i,j=1,2,...,n\, ,\eqno{(1)}
$$
$$
 \; \mathfrak{D}(L)=H^{2}(\Omega)\cap H^{1}_{0}(\Omega),\eqno{(2)}
$$
$$
 a^{ij}(Q)\in C^{1}(\bar{\Omega})  ,\;a^{ij}\xi _{i}  \xi _{j}  \geq a_{0}  |\xi|^{2},\,a_{0}>0,
 $$
 $$
 \;p(Q)>0,\;p(Q)\in {\rm Lip\,\lambda},\, \lambda>\alpha.
$$
\textbf{Определение~1.}
Будем называть функцию $z\in H^{1}_{0}(\Omega) $ обобщенным решением краевой задачи   (1),(2) если имеет место следующее интегральное тождество
 $$
 B(z,v)= (f,v)_{L_{2}(\Omega)}  ,\;\forall v\in H^{1}_{0}(\Omega),
$$
 $$
 B (u,v)= \int\limits_{\Omega} \left( a^{ij}D_{i}u D_{j}v  +   u \,  \mathfrak{D}^{\alpha}_{d-}p\,  v \right)\,dQ ,\;u,v\in H^{1}_{0}(\Omega).
$$

\textbf{Теорема~1.} {\it Существует единственное    решение обобщенной  краевой задачи (1),(2).}



%%%%  ОФОРМЛЕНИЕ СПИСКА ЛИТЕРАТУРЫ %%%
\smallskip \centerline{\bf Литература}\nopagebreak
1. {\it Киприянов И.А.} Оператор дробного дифференцирования и  степени эллиптических операторов. Доклады Академии наук СССР, 1960. Т.131. №2. С. 238-241.
