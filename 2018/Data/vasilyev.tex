\vzmstitle[\footnote{Работа выполнена
при финансовой поддержке Минобрнауки РФ (проект № 1.73117311.2017/БЧ).}]
{ \bf  ДИСКРЕТНЫЕ ПСЕВДОДИФФЕРЕНЦИАЛЬНЫЕ ОПЕРАТОЫ И УРАВНЕНИЯ}

\vzmsauthor{{Васильев}}{В.\, Б.}

\vzmsinfo{Белгород; {\it vbv57@inbox.ru}}

\addcontentsline{toc}{section}{Васильев В.Б.\dotfill}

Обозначим $u_d(\tilde x)$ функцию дискретного аргумента $\tilde x\in h\mathbb Z^m, h>0,$  и $\tilde u_d(\xi), \xi\in\hbar\mathbb T^m, \hbar=h^{-1}, \mathbb T^m=[-\pi,\pi]^m,$ --  её дискретное преобразование Фурье
\[
(F_du_d)(\xi)=\sum\limits_{\tilde x\in h\mathbb Z^m}u_d(\tilde x)e^{i\tilde x\cdot\xi}h^m.
\]

 Пусть $D=\mathbb R^m_+=\{x\in\mathbb R^m: x=(x',x_m), x_m>0\}$, положим $D_d=h\mathbb Z^m\cap D, h>0$,  $\widetilde A_d(\xi)$ -- периодическая функция на  $\mathbb R^m$ с основным кубом периодов $\hbar\mathbb T^m$ (см. также [1--4]).

\textbf{Определение~1.} {\it Пространство $H^s(h\mathbb Z^m)$ состоит из функций дискретного аргумента $u_d(\tilde x)$, для которых конечна норма
\[
||u_d||_s^2=\int\limits_{\hbar\mathbb T^m}(1+|\zeta^2_h|)^s\tilde u_d(\xi)d\xi,
\]
где $\zeta^2_h=\hbar^{-2}\sum\limits_{k=1}^m(e^{ih\xi_k}-1)^2$.
Пространство  $H^s(D_d)$ состоит из дискретных функций пространства  $H^s(h\mathbb Z^m)$, носители которых содержатся в $\overline{D_d}$. %Норма в пространстве  $H^s(D_d)$ индуцируется нормой пространства  $H^s(h\mathbb Z^m)$.
Пространство
$H^s_0(D_d)$ состоит из дискретных функций  $u_d$ with с носителем в  $D_d$, причём эти функции должны допускать продолжение на все пространство $H^s(h\mathbb Z^m)$. Норма в пространстве  $H^s_0(D_d)$ даётся формулой
\[
||u_d||^+_s=\inf||\ell u_d||_s,
\]
где infimum берётся по всевозможным продолжениям  $\ell$.
}

\textbf{Определение~2.}
{\it Дискретный псе\-в\-до\-диф\-фе\-ре\-н\-ци\-аль-\linebreak ный оператор на функциях дискретного аргумента $u_d(\tilde x)$ определяется формулой
}
\[
(A_du_d)(\tilde x)=\int\limits_{\hbar\mathbb T^m}\sum\limits_{\tilde y\in h\mathbb Z^m}A(\xi)e^{i(x-y)\cdot\xi}\tilde u_d(\xi)h^md\xi,~~~\tilde x\in D_d,\eqno(1)
\]


Рассмотрим уравнение
\[
(A_du_d)(\tilde x)=v_d(\tilde x),~~~\tilde x\in D_d,~~ v_d\in H^s_0(D_d). \eqno(2)
\]

Класс  $E_{\alpha}$ -- символы, удовлетворяющие условию
$$
c_1(1+|\zeta^2|)^{\alpha/2}\leqslant|A_d(\xi)|\leqslant c_2(1+|\zeta^2|)^{\alpha/2}\eqno(3)
$$
с положительными постоянными  $c_1, c_2$, не зависящими от  $h$.

Обозначим  $\Pi_{\pm}=\{(\xi',\xi_m\pm i\tau), \tau>0\}, \xi=(\xi',\xi_m)\in\mathbb T^m$.

\textbf{Определение~3.} {\it Периодической факторизацией эллиптического символа  $A_d(\xi)\in E_{\alpha}$ называется его представление в виде
\[
A_d(\xi)=A_{d,+}(\xi)A_{d,-}(\xi),
\]
где сомножители  $A_{d,\pm}(\xi)$ допускают аналитическое продолжение в полуполосы  $\hbar\Pi_{\pm}$ по последней переменной  $\xi_m$
при почти всех фиксированных  $\xi'\in\hbar{\mathbb T}^{m-1}$ и удовлетворяют оценкам
\[
|A^{\pm 1}_{d,+}(\xi)|\leqslant c_1(1+|\hat\zeta^2|)^{\pm\frac{\ae}{2}},~~~|A^{\pm 1}_{d,-}(\xi)|\leqslant c_2(1+|\hat\zeta^2|)^{\pm\frac{{\alpha-\ae}}{2}},
\]
с постоянными $c_1, c_2$, не зависящими от $h$,
\[
\hat\zeta^2\equiv\hbar^2\!\left(\sum\limits_{k=1}^{m-1}(e^{-ih\xi_k}-1)^2+(e^{-ih(\xi_m+i\tau)}-1)^2\right)\!,~\xi_m+i\tau\in\hbar\Pi_{\pm}.
\]

Число $\ae\in{\mathbb R}$ называется индексом периодической факторизации.

}

\textbf{Теорема.} {\it Пусть $\ae-s=n+\delta, n\in{\mathbb N}, |\delta|<1/2$.
Тогда общее решение уравнения  $(2)$ в образах Фурье имеет следующий вид:
\begin{multline*}
	\tilde u_d(\xi)=
%	\\%=
	\tilde A^{-1}_{d,+}(\xi)X_n(\xi)P^{\it per}_{\xi'}(X_n^{-1}(\xi)\tilde A^{-1}_{d,-}(\xi)\widetilde{\ell v_d}(\xi)) +\tilde A^{-1}_{d,+}(\xi)
	\cdot \\ \cdot
%	\!\!
	\sum\limits_{k=0}^{n-1}
%	\!
	c_k(\xi')\hat\zeta_m^k,
\end{multline*}
\begin{multline*}
	(P_{\xi'}^{\it per}\tilde u_d)(\xi)\equiv
	\\ \equiv
	\frac{1}{2}\left(\tilde u_d(\xi)+\frac{1}{2\pi i}\mathrm{v.p.}\int\limits_{-\hbar\pi}^{\hbar\pi}{\tilde u_d(\xi',\eta_m)}\ctg\frac{h(\xi_m-\eta_m)}{2}d\eta_m\right),
\end{multline*}
где $X_n(\xi)$ -- произвольный многочлен степени  $n$ переменных $\hat\zeta_k=\hbar(e^{-ih\xi_k}-1)$, $k=1,\cdots,m$, удовлетворяющий условию  $(3)$, $c_j(\xi'), j=0,1,\cdots,n-1,$ -- произвольные функции из  $H_{s_j}({h\mathbb T}^{m-1}), s_j=s-\ae+j-1/2$. Имеют место априорные оценки
\[
||u_d||_s\leqslant C(||f||^+_{s-\alpha}+\sum\limits_{k=0}^{n-1}[c_k]_{s_k})
\]
где $[\cdot]_{s_k}$ обозначает норму в пространстве $H^{s_k}({h\mathbb T}^{m-1})$, и постоянная $C$ не зависит от $h$.
}

%Некоторые предварительные результаты описаны в  1--4 .

\litlist

1. \foreignlanguage{english}{
	{\it Vasilyev A.V., Vasilyev V.B.} Discrete singular integrals in a half-space. In:
             Current Trends in Analysis and its Applications. Research Perspectives. Mityushev, V., Ruzhansky, M. (Eds.) Birkh\"auser, Basel, 2015. P. 663-670.
}

2. \foreignlanguage{english}{
	{\it	Vasilyev V.B.} Discreteness, periodicity, holomorphy, and factorization. In: Integral Methods in Science and Engineering. V.1. Theoretical Technique.  C. Constanda, Dalla Riva, M., Lamberti, P.D., Musolino, P. (Eds.)  Birkh\"auser, Cham, 2017. P. 315--324.
}

3. \foreignlanguage{english}{
	{\it Vasilyev V.B.} The periodic Cauchy kernel, the periodic Bochner kernel, and discrete pseudo-differential operators. AIP Conf. Proc.  V. 1863. 2017 (140014).
}

4. \foreignlanguage{english}{
	{\it Vasilyev V.B.} On discrete boundary value problems. AIP Conf. Proc. V. 1880. 2017 (050010).
}

