\begin{center}{ \bf  ОБ ОДНОЗНАЧНОМ ПРОДОЛЖЕНИИ РОСТКОВ РЕШЕНИЙ ЛИНЕЙНЫХ УРАВНЕНИЙ ПЕРВОГО ПОРЯДКА}\\
{\it Н.А. Шананин } \\
(Москва; {\it nashananin@inbox.ru} )
\end{center}
\addcontentsline{toc}{section}{Шананин Н.А.\dotfill}



Пусть $ \Omega$ - открытое множество в ${\cal R}^n$, $X_1$ и $X_2$ - линейно независимые в каждой точке $ \Omega$ векторные поля:
$$
X_j=\sum_{k=1}^na_{j, k}(x)\partial_k,\quad \partial_k=\frac{\partial}{\partial x_k},\quad j=1,2,
$$
с вещественными  коэффициентами $a_{j,k}(x)\in C^{\infty}(\Omega), j=1,2$, $k=1, 2,\dots, n$, комплекснозначная  функция $a(x)\in C(\Omega)$ и
$$
P=X_1+iX_2+a(x), \quad i^2=-1,
$$
- линейный дифференциальный оператор. Пусть $u^j(x)\in L_{2}(U_j)$, где $j=1, 2,$ и $U_j$ - две открытые окрестности точки  $x^0\in \Omega$. Говорят, что ростки  функций $u^1(x)$ и  $u^2(x)$  равны в $x^0$ и пишут  $u_{x^0}^1\cong u_{x^0}^2$, если существует открытая окрестность $V\subset U_1\cap U_1$ этой точки, в которой $u^1(x)=u^2(x)$. Пусть $\Gamma$ - непрерывная кривая в $ \Omega$. Пусть $f(x)\in L_{2, loc}(\Omega)$. Говорят, что функция $u(x)$, определенная в некоторой окрестности $U$ кривой $\Gamma$,  удовлетворяет вдоль $\Gamma$
 уравнению $Pu=f$, если выражение $Pu(x)$ корректно определено в смысле теории обобщенных функций, $Pu(x)\in L_{2}(U)$ и  $Pu_x\cong f_x$ для всех $x\in\Gamma$.
Будем говорить, что ростки решений дифференциального уравнения
$
Pu=f
$
{\it однозначно продолжаются вдоль кривой} $\Gamma$, если для любых двух решений $u^1(x)$ и $u^2(x)$ уравнения вдоль кривой $\Gamma$  из равенства $u^1_{x^{\ast}}\cong u^2_{x^{\ast}}$, выполненного в некоторой точке $x^{\ast}\in\Gamma$,   следует, что $u^1_x\cong u^2_x$ для всех $x\in\Gamma$.

Отображение $x\to {\cal L}_P(x)$, ставящее в соответствие точке $x\in \Omega$
линейное подпространство, натянутое на векторы $X_1(x)$ и $X_2(x)$ в касательном пространстве  $T_x(\Omega)$, определяет двумерное  распределение (дифференциальную систему) на $\Omega$.
Для того чтобы решения линейных дифференциальных уравнений с постоянными коэффициентами
однозначно продолжались вдоль непрерывно дифференцируемой кривой необходимо, чтобы касательный вектор в каждой точке кривой был бы ортогонален характеристическим векторам оператора $P(D)$. Для уравнений первого порядка это условие ортогональности эквивалентно принадлежности касательного вектора распределению ${\cal L}_P$.
Непрерывно дифференцируемую кривую называют {\it интегральной}  для распределения ${\cal L}_P$, если в каждой точке  кривой
касательный  к ней  вектор принадлежит подпространству ${\cal L}_P(x)$.
Обозначим через $[X_1, X_2](x)$ коммутатор векторных полей $X_1$ и $X_2$ в точке $x$.

\textbf{Теорема~1.} {\it Для того чтобы любая интегральная кривая распределения ${\cal L}_P$  обладала свойством однозначного продолжения ростков решений любого дифференциального оператора со старшей частью $X_1+iX_2$ вдоль кривой  необходимо и достаточно, чтобы векторные поля $X_1$ и $X_2$ инволютивны: $$\hspace{2cm} [X_1, X_2](x)\in  {\cal L}_P(x),\quad \forall x\in \Omega.\hspace{2cm}\verb"(1)"$$
}


В силу Теоремы Фробениуса при выполнении условия (1)  через каждую точку $x$ множества $\Omega$ проходит одно и только одно максимальное интегральное многообразие $\Omega_{x, {\cal L}_P}$ распределения ${\cal L}_P$. Вследствие единственности
 включение $y\in \Omega_{x, {\cal L}_P}$ влечет равенство $\Omega_{y, {\cal L}_P}=\Omega_{x, {\cal L}_P}$. Непрерывно дифференцируемая
регулярная кривая является интегральной для распределения ${\cal L}_P$ тогда и только тогда, когда она содержится в одном из максимальных интегральных многообразий распределения ${\cal L}_P$.










\textbf{Теорема~2.} {\it Предположим, что векторные поля $X_1$ и $X_2$ инволютивны,
функции $u^1(x)$, $u^2(x)$, $Pu^1(x)$ и $Pu^2(x)$ $\in L_{2, loc}(\Omega)$ и    $u^1_{x^0}\cong u^2_{x^0}$ в  точке $x^0\in\Omega$.
Тогда $u^1_{x}\cong u^2_{x}$ во всех точках $x$ линейно связной компоненты множества
$$
\Omega_{x^0, {\cal L}_P}\cap\{x \,|\, Pu^1_{x}\cong Pu^2_x\},
$$
содержащей точку $x^0$.}

Доказательство приведенных теорем можно найти в статье  [1].



%%%%  ОФОРМЛЕНИЕ СПИСКА ЛИТЕРАТУРЫ %%%
\smallskip \centerline{\bf Литература}\nopagebreak

1. {\it Шананин Н.А.} Об однозначном продолжении ростков  решений
дифференциальных уравнений уравнений  первого порядка  вдоль кривых. Матем. заметки, 102:6, 2017, 890-903.

