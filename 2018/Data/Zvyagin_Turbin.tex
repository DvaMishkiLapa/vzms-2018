\vzmstitle[
	\footnote{Работа выполнена при финансовой поддержке РНФ (проект №~16-11-10125, выполняемый в Воронежском государственном университете).}
]{
	ЗАДАЧА ОПТИМАЛЬНОГО УПРАВЛЕНИЯ ДВИЖЕНИЕМ ЖИДКОСТИ С ОБРАТНОЙ СВЯЗЬЮ ДЛЯ СИСТЕМЫ НАВЬЕ-СТОКСА С ПЕРЕМЕННОЙ ПЛОТНОСТЬЮ
}
\vzmsauthor{Звягин}{В.\,Г.}
\vzmsauthor{Турбин}{М.\,В.}
\vzmsinfo{Воронеж; {\it mrmike@mail.ru} }

\vzmscaption

Рассматривается задача оптимального управления движением жидкости с обратной связью для системы Навье-Стокса с переменной плотностью:

$$
\left\{
\begin{array}{ccc}
\rho\dfrac{\partial v}{\partial t}+\rho\sum\limits_{i=1}^n v_i\dfrac{\partial v}{\partial x_i}-\nu\Delta v-\nabla{\rm div}\,v+\nabla p=\rho f \in \Psi.\\
\dfrac{\partial\rho}{\partial t}+{\rm div}\,(\rho v)=0, {\rm div}\,v=0,\\
v|_{t=0}=a, \,\, \rho|_{t=0}=\rho_0, \,\, 0<m\leqslant \rho_0 \leqslant M, \,\, v|_{\partial\Omega}=0.
\end{array}
\right.
\eqno{(1)}
$$

Решением поставленной задачи управления движением жидкости является тройка $(v,\rho,\rho f),$ где $v$ --– скорость движения жидкости, $\rho$ --- плотность жидкости, а $\rho f$ --– управление. При этом $\rho f$ принадлежит образу некоторого многозначного отображения, зависящему от скорости движения жидкости $v.$ В связи с тем, что таких пар может быть много, естественным образом возникает понятие оптимального решения – решения, дающего минимум заданному функционалу качества.

Для того чтобы ввести понятие слабого решения рассматриваемой задачи нам потребуется ввести два функциональных пространства:
$$
W = \left\{u: u\in L_2(0,T;V^1), u'\in L_{2}(0,T;V^{-2}) \right\};
$$
$$
E = \left\{\rho: \rho\in L_\infty(0,T;L_2(\Omega)), \rho'\in L_{2}(0,T;H^{-2}(\Omega)) \right\}.
$$

\par Сформулируем условия на многозначное отображение
$$
\Psi: W\multimap L_2(0,T;V^{-2})
$$
в качестве функции управления. Будем предполагать, что $\Psi$ удовлетворяет следующим условиям:
\renewcommand{\theenumi}{\arabic{enumi}}
\renewcommand{\labelenumi}{($\Psi$\theenumi)}
\begin{enumerate}
\item Отображение $\Psi$ определено на $W$ и имеет непустые, компактные, выпуклые значения;
\item Отображение $\Psi$ полунепрерывно сверху и компактно;
\item Отображение $\Psi$ глобально ограничено;
\item  $\Psi$ слабо замкнуто в следующем смысле:
$$
\text{если}\left\{ v_{l}\right\} _{l=1}^{\infty}\subset E_{1},v_{l}
\rightharpoonup v_{0}\,,u_{l}\in \Psi\left( v_{l}\right)\quad \text{и}
$$
$$
u_{l} \to u_{0}\,\,\text{в}\, {L_2(0,T;V^{-2})}, \text{
тогда }u_{0}\in \Psi\left( v_{0}\right) .
$$
\end{enumerate}

Пусть $a\in V^1, \rho_0\in L_\infty(\Omega).$ Дадим определение слабого решения рассматриваемой задачи:

\textbf{Определение~1.}
{\it Слабым решением задачи оптимального управления движением жидкости с обратной связью
для системы Навье-Стокса с переменной плотностью (1) назовём
\linebreak
тройку функций $({v},\rho,\rho f)$,
$$
{v}\in W,\quad \rho \in E, \quad \rho f\in L_2(0,T;V^{-2}),
$$
которая для всех $\varphi\in V^2$ и для почти всех $t\in(0,T)$ удовлетворяет тождеству
$$
\langle(\rho v)',\varphi\rangle-\sum_{i,j=1}^n\int\limits_{\Omega}{v}_i \rho {v}_j\frac{\partial \varphi_i}{\partial x_j}\,dx+\nu \int\limits_{\Omega} \nabla v: \nabla \varphi dx=\langle \rho f,\varphi\rangle,
$$
для всех $\psi\in H^2(\Omega)$ и для почти всех $t\in(0,T)$ удовлетворяет тождеству
$$
\langle\rho',\psi\rangle -\sum_{i=1}^n\int\limits_{\Omega}{v}_i \rho \frac{\partial \psi}{\partial x_i}\,dx=0,
$$
условию обратной связи $\rho f \in \Psi$ и начальным условиям ${v}(0)=a$ и $\rho(0)=\rho_0.$}

Имеет место следующая теорема о существовании решения рассматриваемой задачи:

\textbf{Теорема~1.} {\it Пусть $a\in V^1, \rho_0\in L_\infty(\Omega)$ тогда задача оптимального управления движением жидкости с обратной связью для системы Навье-Стокса с переменной плотностью (1) имеет хотя бы одно слабое решение $({v},\rho, \rho f).$}

\renewcommand{\theenumi}{\arabic{enumi}}
\renewcommand{\labelenumi}{($\Phi$\theenumi)}
\par Обозначим через $\Sigma \subset W\times E\times {L}_2(0,T;V^{-2})$ множество всех слабых решений рассматриваемой задачи (1). Рассмотрим
произвольный функционал качества $\Phi: \Sigma \to \mathbb{R},$
удовлетворяющий следующим условиям:
\begin{enumerate}
\item Существует число $\gamma$ такое, что $\Phi (v,\rho, \rho f) \geqslant \gamma$ для всех $(v,\rho, \rho f)\in \Sigma.$
\item Если $v_m \rightharpoonup v_*$ в $W,$ $\rho_m \rightharpoonup \rho_*$ в $E$ и $(\rho f)_m \to (\rho f)_*$ в
${L}_2(0,T;V^{-2}),$ то $\Phi (v_*,\rho_*,(\rho f)_*) \leqslant
\varliminf\limits_{m\to \infty} \Phi (v_m,\rho_m,(\rho f)_m).$
\end{enumerate}
Основным результатом является следующая теорема.

\textbf{Теорема~2.} {\it Если отображение $\Psi$ удовлетворяет условиям
$(\Psi 1)$--$(\Psi 4),$ а функционал $\Phi$ удовлетворяет условиям $(\Phi 1),(\Phi 2)$,
тогда задача оптимального управления движением жидкости с обратной связью для системы Навье-Стокса с переменной плотностью (1)
имеет хотя бы одно слабое решение $(v_*,\rho_*,$ $(\rho f)_*)$ такое, что
}
$$
\Phi (v_*,\rho_*,(\rho f)_*) = \inf\limits_{(v,\rho, \rho f)\in \Sigma}
\Phi (v,\rho,\rho f).$$
