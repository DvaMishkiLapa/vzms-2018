\begin{center}{ \bf  ВОСПОМИНАНИЯ О Н.А. БОБЫЛЕВЕ}\\
{\it Ю.И. Сапронов} \\
(Воронеж; {\it yusapr@mail.ru} )
\end{center}
\addcontentsline{toc}{section}{Сапронов Ю.И.}




Эти заметки написаны в связи с 70-летием со дня рождения профессора
МГУ Н.А. Бобылева.

 С Николаем Антоновичем Бобылевым я учился в воронежской средней
школе № 58, знаменитой тем, что в ней впервые в Воронеже были
созданы классы с математическим уклоном. В создании и дальнейшем
развитии этих
\linebreak
классов принимали участие такие известные математики,
как М.А. Красносельский, С.Г. Крейн, В.И. Соболев, Б.С. Ми\-тя\-гин,
А.С. Шварц, Б.Н. Садовский. Именно в одном из таких классов мы
обучались вместе с Колей Бобылевым. Коля достаточно быстро стал в
нашем классе заметной и весьма авторитетной личностью. Учителя и
одноклассники сразу обратили внимание на его отчетливо выраженные
научные способности (не только математические) и на открытость его
характера. Если добавить к этим качествам природную склонность к
шуткам и юмору, замешанную на абсолютном дружелюбии и доброте, то
можно легко представить себе, сколь популярен был Коля в классе и
школе. У него было много друзей.

Вместо запланированных трех лет (тогда в СССР было одиннадцатилетнее
образование) Николай проучился лишь два года: с группой некоторых
одноклассников и учащихся из параллельного класса (примерно в 30
человек) он сдал экстерном в 1964 году все необходимые экзамены и,
получив аттестат зрелости, поступил учиться на матмех ВГУ.

Учеба в ВГУ захватила его полностью, точнее, не учеба, а научная
работа. Коля, безусловно, полностью выполнял учебный план и был
отличником в обычном понимании учебного процесса. Но плановые
предметы составляли лишь фон его пребыванию в стенах факультета.
Подлинной страстью для него была научная работа в знаменитом в то
время семинаре М.А. Красносельского по функциональному анализу и
нелинейным на\-чаль\-но-крае\-вым задачам. Николая Антоновича
довольно быстро признали своим все <<взрослые>> участники семинара.
Его авторитет стремительно возрастал не только в семинаре, но и
среди всех воронежских математиков. Уже в то время он проявлял
широту научного кругозора, интересуясь такими разделами математики,
как спектральный анализ операторов, топологические методы анализа
нелинейных дифференциальных уравнений, выпуклый анализ, вариационное
исчисление, теория бифуркаций, приближенные методы анализа и т.д.
Впрочем, это качество было характерным для всей научной школы М.А.
Красносельского, а Н.А. Бобылев был ярчайшим представителем этой
школы. Вспоминаются его семинарские экспромты, дававшие ответы на
только что поставленные <<самим Марком>> серьезные задачи. За этими
потрясающими экспромтами скрывалось обилие <<заготовок>>,
естественно возникающих в процессе длительного и увлекательного
творческого труда.

В стенах ВГУ Н.А. Бобылев общался также с такими известными
математиками, как С.Г. Крейн, Ю.Г. Борисович, П.П. Забрейко, И.С.
Иохвидов, И.А. Киприянов, Б.С. Митягин,  А.Д. Мышкис, А.И. Перов,
Ю.В. Покорный, Я.Б. Рутицкий, Б.Н. Садовский, Е.М. Семенов, П.Е.
Соболевский, В.В. Стрыгин, А.С. Шварц, С.Д. Эйдельман.

Из многочисленных дружеских и научных контактов с Н.А. Бобылевым я
чаще всего вспоминаю один эпизод, оказавший существенное влияние на
всю мою последующую математическую деятельность.

Однажды, где-то в конце сентября 1967 года, я встретил Н.А. Бобылева
возле читального зала на первом этаже главного корпуса ВГУ (тогда он
был студентом 4 курса КФА), и он с ходу, находясь в некотором
возбуждении, спросил меня, слышал ли я о потрясающем результата Б.Н.
Садовского --- новой теореме о неповижной точке, имеющей весьма
хорошую перспективу в нелинейном анализе (что в дальнейшем
подтвердилось). В то время в математическом Воронеже был настоящий
бум вокруг неподвижных точек. Каждый новый результат на эту тему
вызывал всеобщий интерес. Это было вызвано тем, что теоремы о
неподвижных точках составляли главный инструментарий получения
теорем существования решений нелинейных краевых и
на\-чаль\-но-крае\-вых задач. Большинству современных математиков
известны такие классические утверждения, как теорема Банаха о
сжимающих отображениях, теоремы Брауэра, Шаудера, Тихонова. Ряд
теорем о неподвижных точках получили к тому времени М.A.
Красносельский и его ученики --- Ю.Г. Борисович, П.П. Забрейко, А.И.
Перов, В.В. Стрыгин и др. Большое впечатление на всех произвела
новая (в то время) теорема М.А. Красносельского, существенно
усилившая торемы Банаха и Шаудера. Теорема утверждала, что любое
отображение выпуклого ограниченного замкнутого подмножества банахова
пространства в себя, представленное в виде $f+g$, где $f$ ---
сжатие, a $g$ --- вполне непрерывное отображение, имеет неподвижную
точку. Эта теорема казалась пределом совершенства и вызывала
всеобщее восхищение. И вдруг 30-летний Борис Садовский, позже
ставший профессором ВГУ и заведующим кафедрой функционального
анализа (этой кафедрой ранее заведовал М.A. Красносельский), получил
фантастический результат: всякое уплотняющее отображение выпуклого
замкнутого ограниченного подмножества банахова пространства в себя
имеет неподвижную точку. Отображение в теореме М.А.Красносельского
заведомо является уплотняющим, то есть понижающим меру
некомпактности любого ограниченного множества. Н.А. Бобылев узнал о
теореме Б.Н. Садовского непосредственно из уст самого автора,
прослушав его доклад на научном семинаре М.А. Красносельского. И уже
на следующий день он рассказал этот результат мне с пояснением
основных идей доказательства. К моему удивлению, через
непродолжительное время на этот же результат обратил мое внимание и
мой научный руководитель Ю.Г. Борисович, поручив мне проверить
возможность построения топологической степени для уплотняющего
векторного поля. Чуть позже мне удалось доказать теорему о
существовании естественной биекции между гомотопическими классами
уплотняющих и, соответственно, компактных векторных полей, давшую в
итоге решение поставленной задачи о построении степени. Это был мой
первый серьезный научный результат. Предварительный разговор на эту
тему с Колей Бобылевым послужил для меня хорошей стартовой площадкой
для всей траектории моих дальнейших разработок. Все мои более
поздние результаты по анализу нелинейных проблем были получены
благодаря влиянию и помощи моих учителей и наставников --- Б.Н.
Садовского, М.А. Красносельского, Н.А. Бобылева, П.П. Забрейко и
моего непосредственного научного руководителя Ю.Г. Борисовича. Я
бесконечно благодарен им за это.

В 1972 году Н.А. Бобылев переехал на ПМЖ в Москву, вслед за своим
научным руководителем --- М.А. Красносельским. С этого момента вся
дальнейшая на\-уч\-но-про\-из\-вод\-ствен\-ная деятельность Николая
Антоновича протекала в стенах ИПУ и МГУ. Он защитил кандидатскую, а
затем докторскую диссертации. Работал в редколлегии известного
журнала <<Автоматика и телемеханика>>, работал в экспертной комиссии
ВАК РФ по специальности <<Математическое моделирование, численные
методы и комплексы программ>>, написал единолично и в соавторстве
несколько научных монографий.

Свои дружеские и научные связи с воронежскими математиками он
никогда не прерывал. Очень часто приезжал в ВГУ, его приглашали и он
охотно откликался на просьбы принять участие в работе оргкомитетов
многих ВЗМШ и ВВМШ. Приезжал также консультировать молодых
математиков математического факультета, факультета ПММ и НИИМ ВГУ,
принимал участие в защитах диссертаций --- в качестве официального
оппонента. Его вклад в развитие математических исследований ВГУ
трудно переоценить. Минуты общения Николая Антоновича с нами,
воронежскими математиками, останутся в наших сердцах и душах до
самого последнего мгновения каждого из нас.
