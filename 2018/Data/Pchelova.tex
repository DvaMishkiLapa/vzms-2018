\begin{center}{ \bf О ТОЧНЫХ ГРАНИЦАХ ОБЛАСТИ ПРИМЕНЕНИЯ ПРИБЛИЖЕННОГО РЕШЕНИЯ
НЕЛИНЕЙНОГО ДИФФЕРЕНЦИАЛЬНОГО УРАВНЕНИЯ В ОКРЕСТНОСТИ ВОЗМУЩЕННОГО ЗНАЧЕНИЯ ПОДВИЖНОЙ ОСОБОЙ ТОЧКИ}\\
{\it А.З. Пчелова } \\
(Чебоксары; {\it apchelova@mail.ru})
\end{center}
\addcontentsline{toc}{section}{Пчелова А.З.\dotfill}

Рассматривается нелинейное обыкновенное дифференциальное уравнение первого порядка с полиномиальной правой
частью пятой степени, в общем случае не интегрируемое в квадратурах, решение которого обладает подвижными
особыми точками. За счёт нового подхода к оценке приближенного решения в окрестности приближенного значения
подвижной особой точки удаётся значительно расширить область представления приближенного решения. Полученные
результаты сопровождаются расчётами.

Применяется метод построения приближенных решений нелинейных дифференциальных уравнений с подвижными особыми
точками, представленный в работах~[1--3].

Рассмотрим задачу Коши для уравнения в нормальной форме
$$
w'(z)=w^{5}(z)+r(z)\eqno{(1)}
$$
с начальным условием
$$
w(z_{0})=w_{0},\eqno{(2)}
$$
к которому приводится с помощью некоторой замены переменных уравнение $w'(z) = \sum\limits_{i=0}^{5} f_{i}(z)
w^{i}(z)$~[4].

В работе [5] доказана теорема существования и единственности решения задачи (1)--(2), получена оценка
погрешности приближенного решения в случае точного значения подвижной особой точки, а также проведено
исследование влияния возмущения подвижной особой точки на приближенное решение, которое имеет вид
$$
\tilde w_{N}(z) = \sum_{n=0}^{N} \tilde C_{n} (\tilde z^{*} -z)^{\textstyle\frac{n-1}{4}} , \quad \tilde
C_{0}\neq 0. \eqno{(3)}
$$
Для приближенного решения (3) получена оценка погрешности. При этом выяснилось, что область существования
приближенного решения в окрестности возмущённого значения подвижной особой точки $\tilde z^{*}$ значительно
уменьшилась по сравнению с областью, полученной в теореме существования и единственности решения в
окрестности подвижной особой точки. Новый подход в получении оценок приближенного решения, основанный на
замене приращения функции выражением дифференциала~[6], позволяет существенно увеличить область применения
приближенного решения~(3) и получить её точные границы.

Обозначим $\rho_{3} = \min\{ \rho_{1}, \rho_{2} \}$, где
$$
\rho_{2} = 1/\sqrt[\scriptstyle5]{(4M+1)^{4}},\quad M=\sup\limits_{n} \frac{|r^{(n)}(z^{*})| }{n! },\quad
n=0, 1, 2, ...,
$$
а $\rho_{1}$ определяет область для представления функции $r(z)$ в степенной ряд~[5].

\textbf{Теорема.} {\it Пусть выполняются следующие условия:

1)~$r(z)\in C^{1}$ в области $|\tilde z^{*}-z|<\rho_{3}$, где $\rho_{3}=\mathop{\mathrm{const}}>0$ и $\tilde
z^{*}$~--- приближенное значение подвижной особой точки решения задачи Коши (1)--(2);

2)~$\exists M_{1}$: $\displaystyle \frac{|r^{(n)}(\tilde z^{*})| }{n! }\leqslant M_{1}$, где
$M_{1}=\mathop{\mathrm{const}}$, $n=0, 1, 2, ...$;

3)~$|\tilde z^{*}|\leqslant |z^{*}|$;

4)~известна оценка погрешности значения $\tilde z^{*}$: $|z^{*} - \tilde z^{*}|\leqslant \Delta \tilde
z^{*}$;

5)~$\Delta \tilde z^{*} < 1/\sqrt[\scriptstyle5]{16(4M_{2}+4\Delta M+1)^{4}}$, где
$$
M_{2} = \sup_{n} \frac{|r^{(n)}(\tilde z^{*})| }{n! }, \quad \Delta M =\left( \sup_{n, U}
\frac{|r^{(n+1)}(z)| }{n! } \right)\Delta \tilde z^{*},
$$
$$
n=0, 1, 2, ...\,, \quad U= \{ z\colon |\tilde z^{*}-z|\leqslant \Delta \tilde z^{*} \}.
$$

Тогда для приближенного решения (3) задачи (1)--(2) в области
$$
G= G_{1} \cap G_{2} \cap G_{3} \eqno{(4)}
$$
справедлива оценка погрешности
$$
\Delta \tilde w_{N}(z) \leqslant \sum_{i=1}^{4} \Delta_{i},
$$
где
$$
\Delta_{1} = \frac{\Delta \tilde z^{*} }{4\sqrt2 |\tilde z^{*} -z|^{\textstyle\frac{5 }{4 }} },
$$
$$
\Delta_{2} \leqslant \frac{4^{-2}\Delta\tilde z^{*}(4M_{2}+4\Delta M+1) }{1- (4M_{2}+4\Delta M+1)|\tilde
z_{1}^{*}-z|^{\textstyle\frac{5 }{4 }} } \sum_{i=1}^{5} |\tilde z_{1}^{*}-z|^{\textstyle\frac{i-1}{4}},
\eqno{(5)}
$$
$$
\Delta_{3} \leqslant \frac{\Delta M|\tilde z_{1}^{*}-z| }{1-2(4M_{2}+4\Delta M+1)|\tilde
z_{1}^{*}-z|^{\textstyle\frac{5 }{4 }} } \sum_{i=1}^{5} \frac{|\tilde z_{1}^{*}-z|^{\textstyle\frac{i-1}{4}}
}{8+i }, \eqno{(6)}
$$
$$
\Delta_{4} \leqslant \frac{4^{-1}|\tilde z^{*}-z|^{\textstyle\frac{N}{4}} }{1-(4M_{7}+1)|\tilde
z^{*}-z|^{\textstyle\frac{5 }{4 }} } \sum_{i=1}^{5} \frac{(4M_{7}+1)^{\bigl[{\textstyle\frac{N+i}{5}}\bigr]}
}{N+4+i } |\tilde z^{*}-z|^{\textstyle\frac{i-1}{4}},
$$
при этом
$$
|\tilde z_{1}^{*}| = |\tilde z^{*}|+ \Delta\tilde z^{*}, \quad \arg \tilde z_{1}^{*} = \arg\tilde z^{*},
\quad G_{1} = \{ z\colon |z|< |\tilde z^{*}| \},
$$
$$
G_{2} = \left\{ z\colon |\tilde z_{1}^{*}-z|< \frac{1 }{\sqrt[\scriptstyle5]{16(4M_{2}+4\Delta M+1)^{4} } }
\right\},
$$
$$
G_{3} = \left\{ z\colon |\tilde z^{*}-z| < \frac{1 }{\sqrt[\scriptstyle5]{(4M_{2}+1)^{4}} } \right\}.
$$

}

\textbf{Замечание.} {\it Теорема справедлива в области (4), где
$$
G_{1} = \{ z\colon |z|> |\tilde z^{*}| \},
$$
$$
G_{2} = \left\{ z\colon |\tilde z_{2}^{*} -z| < \frac{1 }{\sqrt[\scriptstyle5]{16(4M_{7}+4\Delta
M_{2}+1)^{4}} } \right\},
$$
если вместо условия~3 этой теоремы выполняется условие $|\tilde z^{*}|> |z^{*}|$. В этом случае $|\tilde
z_{2}^{*}| = |\tilde z^{*}|- \Delta \tilde z^{*}$, $\arg \tilde z_{2}^{*} = \arg \tilde z^{*}$ и в (5), (6)
выражение $|\tilde z_{1}^{*}-z|$ заменяется на выражение $|\tilde z_{2}^{*}-z|$.}

\textbf{Пример.} Найдём приближенное решение задачи Коши (1)--(2), где $r(z)\equiv 0$ и $w(i) =
(\sqrt{2+\sqrt2}+i\sqrt{2-\sqrt2})/2$, в окрестности приближенного значения подвижной особой точки.

Имеем точное решение $w(z) = 1/\sqrt[\scriptstyle4]{3i-4z}$. $z^{*} = 0,75i$~--- точное значение подвижной
особой точки; $\tilde z^{*} = 0,0003+ 0,7498i$~--- приближенное значение подвижной особой точки, $\Delta
\tilde z^{*} = 0,00036$, $z_{1} = 0,12+0,57i$ попадает в область действия теоремы. Рассмотрим случай
$C_{0}=1/\sqrt2$. Результаты расчётов представлены в табл.\,1, где $w(z_{1})$~--- значение точного решения,
$\tilde w_{3}(z_{1})$~--- значение приближенного решения, $\Delta$~--- абсолютная погрешность, $\Delta'$~---
априорная погрешность, найденная по теореме, $\Delta''$~--- апостериорная погрешность.

\begin{table}[tb]
\footnotesize

{\raggedleft Таблица 1

}

\smallskip

{\tabcolsep=2.7mm\centering{\bfseries Оценка приближенного решения задачи Коши в~окрестности возмущённого
значения подвижной особой точки}

\bigskip

\begin{tabular}{|c|c|c|c|c|c|}
\hline
 $z_{1}$&$w(z_{1})$&$\tilde w_{3}(z_{1})$&$\Delta$&$\Delta'$&$\Delta''$\\
\hline
 $0,12+$&$0,88954-$&$0,88988-$&$4\cdot 10^{-4}$&$0,02782$&$0,00062$\\
 $+0,57i$&$-0,53266i$&$-0,53286i$&&&\\
\hline
\end{tabular}

}
\end{table}

В следующей табл.\,2 приведено сравнение результатов, полученных по теореме настоящей работы и по теореме~3
работы~[5]. Значение $z_{2} = 0,00029+ 0,71325i$ попадает в область действия указанных выше теорем. Здесь
$w(z_{2})$~--- значение точного решения, $\tilde w_{3}(z_{2})$~--- значение приближенного решения, $|w-\tilde
w_{3}|$~--- абсолютная погрешность, $\Delta'_{\mathrm{I}}$~--- априорная погрешность, найденная по теореме~3
работы~[5], $\Delta'_{\mathrm{II}}$~--- априорная погрешность, найденная по теореме настоящей \mbox{работы}.

\begin{table}[tb]
\footnotesize

{\raggedleft Таблица 2

}

\smallskip

{\tabcolsep=2.1mm\centering{\bfseries Сравнение оценок приближенного решения задачи~Коши в~окрестности
возмущённого значения подвижной особой точки}


\bigskip

\begin{tabular}{|c|c|c|c|c|c|}
\hline
 $z_{2}$&$w(z_{2})$&$\tilde w_{3}(z_{2})$&$|w-\tilde w_{3}|$&$\Delta'_{\mathrm{I}}$&$\Delta'_{\mathrm{II}}$\\
\hline
 $0,00029+$&$1,49083-$&$1,49414-$&$0,00397$&$0,00933$&$0,00835$\\
 $+0,71325i$&$-0,62097i$&$-0,61877i$&&&\\
\hline
\end{tabular}

}
\end{table}

Предложенные исследования позволяют значительно увеличить область применения приближенного решения~(3) задачи
Коши (1)--(2) в окрестности возмущённого значения подвижной особой точки по сравнению с результатами
работы~[5] и найти точные границы этой области. При этом представленные расчёты в табл.\,2 подтверждают
адекватность результата теоремы этой работы с результатом теоремы~3 из работы~[5].

\smallskip
\centerline{\bf Литература}\nopagebreak

1.~{\it Орлов~В.Н.} Точные границы области применения приближенного решения дифференциального уравнения Абеля
в окрестности приближенного значения подвижной особой точки~// Вестник Воронеж. гос. тех. ун-та. 2009. Т.5,
№10. С.\,192--195.

2.~{\it Орлов~В.Н., Редкозубов~С.А.} Математическое моделирование решения дифференциального уравнения Абеля в
окрестности подвижной особой точки~// Известия ин-та инж. физики. 2010. №4(18). С.\,2--6.

3.~{\it Орлов~В.Н.} Метод приближенного решения скалярного и матричного дифференциальных уравнений Риккати:
монография. Чебоксары: Перфектум, 2012. ---~112~с.

4.~{\it Орлов~В.Н., Пчелова~А.З.} Приближенное решение одного нелинейного дифференциального уравнения в
области голоморфности~// Вестник Чув. гос. пед. ун-та им. И.Я.~Яковлева. Серия: Ест. и техн. науки. 2012.
№4(76). С.\,133--139.

5.~{\it Пчелова~А.З.} Улучшенные оценки точности приближенных аналитических решений задачи Коши для
нелинейного дифференциального уравнения в окрестности подвижной особой точки~// Вестник Российской Академии
естеств. наук. Дифф. уравнения. 2017. Т.17, №4. C.\,63--69.

6.~{\it Бахвалов~Н.С.} Численные методы. М.: Наука, 1975. ---~632~с.

