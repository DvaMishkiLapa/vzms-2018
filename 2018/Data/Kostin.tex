\vzmstitle{ \bf  Экспоненциальные косинус
оператор-функции и их связки}

\vzmsauthor{{Костин}}{В.\, А.}

\vzmsinfo{Воронеж}; {\it vlkostin@mail.ru}

\vzmsauthor{{Костин}}{А.\, В.}

\vzmsinfo{{Воронеж}; {\it leshakostin@mail.ru}}

\vzmscaption



Пусть $\mathbb{R}=(-\infty,\infty)$, $E$~--- банахово пространство,
$C(t)$--- операторная косинус-функция класса $C_0$ (КОФ).

Это означает, что выполняются соотношения:\\
$(i)\hspace{10mm}C(t+s)+C(t-s)=2C(t)C(s),\hspace{10mm}t,s\in\mathbb{R};$\\
$(ii)\hspace{10mm}C(0)=I$--- тождественный оператор в $E$;\\
$(iii)\hspace{10mm}\lim\limits_{t\to 0}\|C(t)\varphi-\varphi\|_E=0$
для всех $\varphi\in E$.

Оператор $A=C''(0)$ называется производящим оператором (генератором)
с областью определения $\overline{D}(A)=E$.

{\bf Определение 1 [1]}. Если КОФ представима в виде
$$C(t)=\frac12[\exp(tA)+\exp(-tA)],\hspace{10mm}t\in\mathbb{R},\eqno{(1)}$$
где $\exp(tA)$--- однопараметрическая группа преобразований с
генератором $A$, то она называется экспоненциальной косинус
оператор"=функцией с генератором $A^2$.

Известно (см. [2]), что КОФ является экспоненциальной только тогда,
когда операторы $A$ и $-A$ являются генераторами сильно непрерывных
полугрупп в $E$.

{\bf Определение 2.} Если $C_1(t)$ и $C_2(t)$ экспоненциальные
косинус оператор"=функции в $E$, коммутирующие между собой, то
конструкцию $$C_1(t)\ast
C_2(t)=\frac12[\exp(tA_1)\exp(tA_2)+\exp(-tA_1)\exp(-tA_2)]$$ будем
называть связкой косинус"=функций $C_1(t)$ и $C_2(t)$.

{\bf Утверждение.} Если $C_1(t)$~--- экспоненциальная
косинус"=фу\-н\-к\-ция с генератором $A_1^2$, а $C_2(t)$~--- с генератором
$A_2^2$, то их связка является экспоненциальной косинус"=функцией с
генератором $(A_1+A_2)^2$.

\litlist

1. Васильев В.В. Полугруппы операторов, косинус оператор"=функции и
линейные дифференциальные уравнения// В.В. Васильев, С.Г. Крейн,
С.И. Пискарев/ Итоги науки и техники. Мат. анализ., Т. 28, М.: 1996,
с. 87--202.

2. Костин В.А. Абстрактные сильно непрерывные пары
тригонометрических групп преобразований// В.А. Костин/ № 829--80
Деп. Воронеж.--- 1980, 20 с.
