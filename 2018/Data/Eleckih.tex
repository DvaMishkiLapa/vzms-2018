\addcontentsline{toc}{section}{Елецких К.\,C.}

\begin{center}
	\bf О среднем по окружности радиальной j-функции Бесселя \\
	К.\,C.~Елецких (Россия, Елец)\\
	kostan.yeletsky@gmail.com\\
\end{center}


В этой работе приведены некоторые из формул о весовых средних значениях при интегрировании со степенным весом по окружности.

В евклидовом пространстве $\mathbb{R}_2^+=\{x=(x_1,x_2),\,\,x_1>0\}$ рассматривается смешанное преобразование Фурье-Бесселя [1]. Через $j_\nu$ обозначена j-функция Бесселя
$j_\nu(t){=}
\Gamma(\nu{+}1)\,{J_\nu(t)\over t^\nu}$, $J_\nu$ порядка $\nu{=}{\gamma-1\over2}$, см. [1]. По  переменной $x_1$ действует преобразование Бесселя (по j-функции Бесселя), по $x_2$ --- преобразование Фурье. Применением  формулы  весового интеграла по полуокружности от ядра этого преобразования

$$\int\limits_{S_r^+}{\Lambda(x,\xi)\xi_1^\gamma}dS(\xi)
=\frac{\Gamma\left(\gamma+1\over2\right)\Gamma\left(1\over2\right)}{\Gamma\left(\gamma+2\over2\right)}\,r^{\gamma+1}\,j_{\frac{\gamma}{2}}(r\,\rho),$$
где $\Lambda(x,\xi)=j_\frac{\gamma-1}{2}(x_1,\xi_1)e^{-ix_2\xi_2}$ --- ядро смешанного преобразования, $\rho=|x|=\sqrt{x_1^2+x_2^2}$,
получено обратное преобразование Фурье-Бесселя финитной радиальной функции\\ $\psi(\xi)=\left\{ \begin{array}{cc}
          (a^2-|\xi|^2)^{{\nu-{2+\gamma\over2}}} , & |\xi|\leqslant a\,, \\
					   0  ,& |\xi|>a\,.
\end{array}\right.\,\,$ для случая $\nu>{\gamma\over2}$:

$F^{-1}_B[\psi]=\left\{ \begin{array}{cc}
          C(\gamma)\,|S^+_1(2)|_\gamma \,\,\frac{\Gamma\left(\nu-\frac{\gamma}{2}\right)\,\Gamma\left(\frac{2+\gamma}{2}\right)}
{2\Gamma\left(\nu+1\right)}\,\,\,a^{2\nu}\,
j_{\nu}(a|x|) , & |\xi|\leqslant a\,, \\
					   0  ,& |\xi|>a\,,
\end{array}\right.\,\,$\\
где $C(\gamma)=\frac{1}{2^{\gamma}\,\pi\,\,\Gamma^2\left(\frac{\gamma+1}{2}\right)}\,,$ $|S^+_1(2)|_\gamma=\sqrt{\pi}\frac{\Gamma(\frac{\gamma+1}{2})}{\Gamma(\frac{\gamma+2}{2})}.$

Применяя эту формулу найдено преобразование Фурье-Бесселя функции $j_\nu(a|x|)$ для случая $\nu>{\gamma+1\over2}$. Приведем полученную формулу

$F_B[j_\nu(a\,|x|)](\xi)=\left\{\begin{array}{cll}
\frac{2^{\gamma+1}\sqrt{\pi}\,\Gamma(\frac{\gamma+1}{2})\Gamma\left(\nu+1\right)}{\Gamma\left(\nu-\frac{\gamma}{2}\right)\,a^{2\nu}}(a^2-|\xi|^2)^{{\nu-\frac{2+\gamma}{2}}} &, &  |\xi|\leqslant a \\
{} & {} & \\									0 \, &, & |\xi|> a\,.\end{array} \right.$
% Для оформления теорем, пожалуйста, используйте следующий образец
%
% \textbf{Теорема.} \textit{ Текст формулировки теоремы.}
%
%нумерацию формул можно делать как автоматическую, так и через команду \eqno.

% Утверждения, определения, замечания и следствия оформляются так же, как и теоремы

% Для оформления благодарностей, пожалуйста, используйте следующий образец
%
% {\small Работа выполнена при поддержке Российского фонда фундаментальных
% исследований, проект №~00-00-00888).}
%
%
% Для оформления списка литературы, пожалуйста, используйте следующий образец
%
\smallskip
{\small
\centerline{\bf Литература}
\vspace{-0.2cm}
\begin{enumerate}
\item {\it Киприянов~И.А.}{ Преобразование Фурье-Бесселя и теоремы вложения для весовых классов. /\!/ Тр. МИАН. 1967. Т. LXXXIX. С. 130-213.}
%\item {\it Фамилия~И.О.} {Название книги. Город: Издательство, год. ??~c.}
%\item {\it Фамилия~И.О.}{ Название статьи в журнале~/\!/ Название журнала. Год. Т.~?, №~?. С.~??-??.}
%\item {\it Фамилия1~И1.О1., Фамилия2~И2.О2.}
%{Название статьи в сборнике~/\!/ Название сборника. Город: Издательство, год. С.~??-??.}
\end{enumerate}
