\begin{center}{ \bf ТОПОЛОГИЧЕСКИЙ АНАЛИЗ ЗАДАЧИ О ГЕОДЕЗИЧЕСКОМ ПОТОКЕ ЭЛЛИПСОИДА В ПОЛЕ УПРУГОЙ СИЛЫ}\\
{\it И.Ф. Кобцев } \\
(Москва; {\it int396.kobtsev@mail.ru} )
\end{center}
\addcontentsline{toc}{section}{Кобцев И.Ф.\dotfill}


Как известно [1], геодезический поток двумерного эллипсоида в $\mathbb{R}^3$
$$
\frac{x^2}{a}+\frac{y^2}{b}+\frac{z^2}{c}=1,\quad a>b>c>0
$$
является вполне интегрируемой по Лиувиллю задачей. Класс интегрируемых задач о движении материальной точки по эллипсоиду существенно расширяется, если добавить потенциальную силу, действующую на точку. В [2] показано, что введение в исходную систему силы с потенциалом более общего вида
$$
\frac{k}{2}\left( x^2+y^2+z^2\right)+\frac{\alpha}{x^2}+\frac{\beta}{y^2}+\frac{\gamma}{z^2}
$$
сохраняет интегрируемость задачи.

Исследование проводилось как для притягивающей силы ($k>0$), так и для отталкивающей ($k<0$).


Задачу о движении точки по эллипсоиду под действием упругой силы можно исследовать с точки зрения гамильтонова формализма. Поскольку в рассматриваемой системе нет диссипации энергии, полная механическая энергия точки
$$
H=\frac{1}{2}\left(\dot x^2+\dot y^2+\dot z^2\right)+\frac{k}{2}\left(x^2+y^2+z^2\right)
$$
является интегралом движения. Следовательно, топологию слоения Лиувилля можно описывать изоэнергетическими инвариантами, определяемыми для многообразия постоянной энергии $Q^3_h=\left\{H=h \right\}$. Кроме того, исследуемая система имеет две степени свободы; для таких гамильтоновых систем в [3] построена соответствующая теория и  классификация особенностей.


Результаты представлены в виде изоэнергетических инвариантов --- молекул Фоменко--Цишанга; это удобный и наглядный способ. Исследование проводится алгебраическим методом, предложенным в М.П. Харламовым в [4]. Использование такого подхода значительно упрощает вычисления; это было наглядно показано при анализе других задач классической механики [5].

В результате исследования задачи о геодезическом потоке эллипсоида в поле упругой силы получены новые изоэнергетические молекулы Фоменко--Цишанга, проведена топологическая классификация слоений Лиувилля. Также обнаружено, что лиувиллево эквивалентны некоторым другим задачам динамики твердого тела.


%%%%  ОФОРМЛЕНИЕ СПИСКА ЛИТЕРАТУРЫ %%%
\smallskip \centerline{\bf Литература}\nopagebreak

1.  {\it Якоби К.} Лекции по динамике. Пер. с нем. М.;~Л.: Гл. ред. общетехн. лит-ры, 1936.

2.  {\it Козлов В.В.} Некоторые интегрируемые обобщения задачи Якоби о геодезических на эллипсоиде // Прикладная математика и механика, том 59, вып. 1, 1995.

3.  {\it Болсинов А.В., Фоменко А.Т.} Интегрируемые гамильтоновы системы. Геометрия, топология, классификация. Ижевск: Издательский дом <<Удмуртский университет>>, 1999.
% Статьи, тезисы и прочее оформляются следующим образом:

4. {\it Харламов М.П.} Топологический анализ и булевы функции. Методы и приложения к классическим системам // Нелинейная динамика. - 2010. - Том 6. - N.4 - C. 769--805.

5. {\it Николаенко С. С.} Топологическая классификация интегрируемого случая Горячева в динамике твердого тела // Математический сборник. - Том 207. - N.1.

