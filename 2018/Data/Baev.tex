\vzmstitle[
	\footnote{
		Работа выполнена при финансовой поддержке Министерства образования и науки
		РФ (проект 14.Z50.31.0037).
	}
]{О свойствах сопряжённых операторов для одного класса весовых
псевдодифференциальных операторов }


\vzmsauthor{Баев}{А.\,Д.}
\vzmsauthor{Бахтина}{Ж.\,И.}
\vzmsauthor{Бунеев}{С.\,С.}

\vzmsinfo{Воронежский государственный университет}


\vzmscaption

{\sloppy

Рассмотрим достаточно гладкую функцию $\alpha (t),\,\,\,\,t \in R_ + ^1 $,
для которой $\alpha ( + 0) = {\alpha }'( + 0) = 0$, $\alpha \mbox{(}t\mbox{)
> 0}$ при $t > 0$, $\alpha \mbox{(}t\mbox{) = const}$ для $t \geqslant d$ для
некоторого $d\mbox{ > 0}$.

Рассмотрим функциях $u(t) \in C_0^\infty (R_ + ^1 )$ интегральное
преобразование, определённое формулой
\begin{equation}
\label{eq4400}
F_\alpha [u(t)](\eta ) = \int\limits_0^{ + \infty } {u(t)\exp (i\eta }
\int\limits_t^d {\frac{d\rho }{\alpha (\rho )}} )\frac{dt}{\sqrt {\alpha
(t)} },
\end{equation}

Это преобразование было введено в [1]. В [1] показано, что преобразование
$F_\alpha $ связано с преобразованием Фурье
\[
F_{\tau \to \eta } [u] = \int\limits_{ - \infty }^{ + \infty } {u(\tau )\exp
(i\eta } \tau )d\tau ,\,\,\,\,\eta \in R^1
\]
следующим равенством $F_\alpha [u(t)](\eta ) = F_{\tau \to \eta } [u_\alpha
(\tau )],$ здесь $u_\alpha (\tau ) = \left. {\sqrt {\alpha (t)} u(t)}
\right|_{t = \varphi ^{ - 1}(\tau )} ,\,\,\,t = \varphi ^{ - 1}(\tau )$ -
функция, обратная к функции $\tau = \varphi (t) = \int\limits_t^d
{\frac{d\rho }{\alpha (\rho )}} .$ В [1] и [2] показано, что преобразование
$F_\alpha $ может быть продолжено до преобразования, осуществляющего взаимно
однозначное и взаимно непрерывное преобразование пространств $L_2 (R_ + ^1
)$ и $L_2 (R^1 )$, а также может быть рассмотрено на некоторых классах
обобщённых функций.

Обозначим через $F_\alpha ^{ - 1} $ обратное к $F_\alpha $ преобразование,
которое можно записать в виде $F_\alpha ^{ - 1} [w(\eta )](t) = \left.
{\frac{1}{\sqrt {\alpha (t)} }F_{\eta \to \tau }^{ - 1} [w(\eta )]}
\right|_{\tau = \varphi (t)} $, где $F_{\eta \to \tau }^{ - 1} $ - обратное
преобразование Фурье. Можно показать, что на функциях $u(t) \in C_0^\infty
(\bar {R}_ + ^1 )$ выполняются соотношения $F_\alpha [D_{\alpha ,t}^j
u](\eta ) = \eta ^jF_\alpha [u](\eta ),\,\,j = 1,2,...$, где $D_{\alpha ,t}
= \frac{1}{i}\sqrt {\alpha (t)} \partial _t \sqrt {\alpha (t)}
,\,\,\,\,\partial _t = \frac{\partial }{\partial t}.$

С помощью преобразования (\ref{eq4400}) и преобразования Фурье $F_{x \to \xi } = F_{x_1
\to \xi _1 } F_{x_2 \to \xi _2 } ...F_{x_{n - 1} \to \xi _{n - 1} } $
определим весовой псевдодифференциальный $P(D_x ,D_{\alpha ,t} )$ оператор
по формуле $P(t,D_x ,D_{\alpha ,t} )v(x,t) = F_{\xi \to x}^{ - 1} F_\alpha
^{ - 1} [p(t,\xi ,\eta )F_\alpha F_{x \to \xi } [v(x,t)]]$,
где символ $p(t,\xi ,\eta )$ есть бесконечно дифференцируемая функция по
совокупности переменных, растущая по переменным $\xi ,\,\,\eta $ не быстрее
некоторого многочлена.

\textbf{Определение 1.} Будем говорить, что символ $p(t,\xi ,\eta )$
весового псевдодифференциального оператора $P(t,D_x ,D_{\alpha ,t} )$
принадлежит классу символов $S_{\alpha ,\rho }^\sigma (\Omega )$, где
$\Omega \subset \bar {R}_ + ^1 $ (открытое множество), если функция $\lambda
(t,\xi ,\eta )$ является бесконечно дифференцируемой функцией по переменной
$t \in \Omega $ и по переменной $\eta \in R^1\,$. Причём, при всех $j =
0,\,\,1,\,\,2,...,\,\,\,\,l = 0,\,\,1,\,\,2,...$ справедливы оценки $\left|
{(\alpha (t)\partial _t )^j\partial _\eta ^l \lambda (t,\xi ,\eta )} \right|
\leqslant c_{jl} (1 + \left| \xi \right|^ + \left| \eta \right|)^{\sigma - \rho
l}$ с константами $c_{jl} > 0$, не зависящими от $\xi \in R^{n -
1},\,\,\,\eta \in R^1$, $t \in K$, где $K \subset \Omega $ - произвольный
отрезок. Здесь $\sigma $ - действительное число, $\rho \in (0;1]$.

Пусть выполнено следующее условие.

\textbf{Условие 1.} Существует число $\nu \in \mbox{(0,1]}$ такое, что
$\left| {\alpha '(t)\alpha ^{ - \nu }(t)} \right| \leqslant c < \infty $ при всех
$t \in [0, + \infty )$. Кроме того, $\alpha (t) \in C^{s_1 }[0, + \infty )$
для некоторого $s_1 \geqslant 2N - \left| \sigma \right|$, где
\[
N \geqslant \mathop {\max }\limits_{0 \leqslant p_1 \leqslant l} \{2p_1 + \frac{l - p_1 +
\frac{3}{2}}{\nu } + 1,\,\,\sigma + 1,\,\,\sigma + \frac{l}{2}\},\,\,l =
1,\,\,2...,
\]
$\sigma$ --- некоторое действительное число.



Заметим, что указанное выше число $\nu $ существует, если $\alpha ( + 0) =
\alpha '( + 0) = 0$.

\textbf{Определение 2}. Сопряжённым оператором к весовому
псевдодифференциальному оператору $P(t,D_x ,D_{\alpha ,t} )$ назовём
оператор $\,P^\ast (t,D_x ,D_{\alpha ,t} )$, удовлетворяющий равенству
\begin{multline*}
(P(t,D_x ,D_{\alpha ,t} )u(x,t),v(x,t))_{L_2 (R_ + ^n )} =
\\=
(u(x,t),P^\ast
(t,D_x ,D_{\alpha ,t} )v(x,t))_{L_2 (R_ + ^n )}
\end{multline*}
для всех $v(x,t) \in L_2 (R_ + ^n ),\,\,\,\,u(x,t) \in L_2 (R_ + ^n )$
таких, что $P(t,D_x ,D_{\alpha ,t} )u(x,t) \in L_2 (R_ + ^n )$.

Здесь $\,( \cdot , \cdot )$ - скалярное произведение в $L_2 (R_ + ^n )\,$.

Справедлива следующая теорема.

\textbf{Теорема 1.} Пусть $p(t,\xi ,\eta ) \in S_{\alpha ,\rho }^m (\Omega
),\,\,\,\,\Omega \subset \bar {R}_ + ^1 ,\,\,\,\,\,m \in R^1$, $\rho \in
(0;1]$. Тогда оператор $P^\ast (t,D_x ,D_{\alpha ,t} )$, сопряжённый к
весовому псевдодифференциальному оператору $\,P(t,D_x ,D_{\alpha ,t} )$ с
символом $p(t,\xi ,\eta )$, является весовым псевдодифференциальным
оператором с символом $\,p^\ast (t,\xi ,\eta ) \in S_{\alpha ,\rho }^m
(\Omega )$. Причём справедливо соотношение
\[
\,\,p^\ast (t,\xi ,\eta ) - \sum\limits_{j = 0}^{N - 1} {\frac{(i)^j}{j!}}
(\alpha (t)\partial _t )^j\partial _\eta ^j \left. {\bar {p}(t,\xi ,\eta
)\,} \right|_{y = t} \in S_\alpha ^{m - N} (\Omega )
\]
для любых $N = 1,2,...$.



При $\rho = 1$ теорема, аналогичная теореме 1, доказана в [3]. Некоторые
другие свойства весовых псевдодифференциальных операторов с символом из
класса $S_{\alpha ,\rho }^m (\Omega )$ доказаны в [4] - [8].

\litlist

1. Баев А. Д. Вырождающиеся эллиптические уравнения высокого порядка и
связанные с ними псевдодифференциальные операторы / А. Д. Баев // Доклады
Академии наук. -- 1982. - Т. 265, № 5. - С. 1044 -- 1046.

2. Баев А.Д. Об общих краевых задачах в полупространстве для вырождающихся
эллиптических уравнений высокого порядка /А.Д. Баев// Доклады Академии наук,
2008, т. 422, №6, с. 727 -- 728.

3. Баев А. Д., Садчиков П. В. Априорные оценки и существование решений
краевых задач в полупространстве для одного класса вырождающихся
псевдодифференциальных уравнений / А. Д. Баев, П. В. Садчиков // Вестник
ВГУ. Серия: Физика. Математика.-- 2010, №1. -- С. 162-168.

4. Баев А.Д. О некоторых свойствах одного класса псевдодифференциальных
операторов с вырождением /А.Д. Баев, П.А. Кобылинский// Вестник ВГУ. Серия:
Физика. Математика.-- 2014, №2. -- С. 66-73.

5. Баев А.Д. О свойствах коммутации одного класса вырождающихся
псевдодифференциальных операторов /А.Д. Баев, П.А. Кобылинский// Вестник
ВГУ. Серия: Физика. Математика.-- 2014, №4. -- С. 102 -- 108.

6. Баев А.Д. О некоторых свойствах одного класса вырождающихся
псевдодифференциальных операторов /А.Д. Баев, П.А. Кобылинский// Доклады
Академии наук. -- 2015. - Т. 460, № 2. - С. 133 -- 135.

7. Баев А.Д. Теоремы об ограниченности и композиции для одного класса
весовых псевдодифференциальных операторов /А.Д. Баев, Р.А. Ковалевский//
Вестник Воронежского государственного университета Серия: Физика.
Математика.-- 2014, №1. -- С. 39- 49.

8. Баев А.Д. О некоторых краевых задачах для псевдодифференциальных
уравнений с вырождением /А.Д. Баев, П.А. Кобылинский// Доклады Академии
наук. -- 2016. - Т. 466, № 4. - С. 385 -- 388.

}
