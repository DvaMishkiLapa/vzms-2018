

\begin{center}{ \bf  К ВОПРОСУ ОБ АСИМПТОТИКЕ ДВИЖЕНИЯ ВЯЗКОЙ НЕСЖИМАЕМОЙ ЖИДКОСТИ ПРИ МАЛОЙ ВЯЗКОСТИ}\\
{\it В.Л. Хацкевич } \\
(Воронеж; {\it vlkhats@mail.ru} )
\end{center}
\addcontentsline{toc}{section}{Хацкевич В.Л.\dotfill}
Пусть  $\Omega$ – ограниченная область в $R^n$, ($n=2, 3$) с липшицевой границей  $S$, $T>0$  -- заданное число. Обозначим $Q_T:=\Omega\times (0, T)$  и $S_T:= S\times [0, T]$ . Движение вязкой несжимаемой жидкости в  $Q_T$ описывается уравнениями Навье-Стокса
\begin{equation}
\frac{\partial u}{\partial t} + u_k\frac{\partial u}{\partial x_k} - \mu\Delta u + \nabla p_{*} = f(x, t),\,\, div u = 0\,\,(x, t\in Q_T);
\end{equation}
\begin{equation}
u(x,t) = 0 \,\,(x,t\in S_T);\,u(x, 0) = a(x)\,\,(x\in\bar{\Omega}).
\end{equation}
Здесь векторная функция  $u$ и скалярная функция $p_{*}$ -- искомые скорость жидкости и давление. Векторные функции  $a$ и  $f$  -- соответственно начальное условие на скорость жидкости и вектор плотности внешних сил заданы,   $\mu$ -- кинематический коэффициент вязкости.

В настоящем сообщении обсуждается вопрос обоснования поведения решений начально-краевых задач для нестационарных уравнений Навье-Стокса (1), (2) при стремлении коэффициента вязкости к нулю. Существует большое количество работ по асимптотике решений уравнений Навье-Стокса в случае малой вязкости. Однако, вопросы обоснования изучены до настоящего времени недостаточно.

В случае задачи Коши результат о предельном поведении решений нестационарной системы Навье-Стокса при исчезающей вязкости известен. Различные результаты в этом направлении для начально-краевых задач получили О.А. Ладыженская, Ж.Лионс, В.П. Маслов, Т.Като, Р. Темам, Ф.Л. Черноусько, С.Н. Алексеенко, Г.В. Сандраков и др.. Большинство результатов получено для линеаризованных задач, либо в случае двумерной области, либо для частных ситуаций.

Важную роль при исследовании предельного поведения решений задачи (1), (2) при  $\mu\rightarrow 0$ играют решения вырожденной задачи
\begin{equation}
\frac{\partial v}{\partial t} + v_k\frac{\partial v}{\partial x_k} = f(x,t) - grad p_0,\,\,div v = 0,\,\, x, t\in Q_T;
\end{equation}
\begin{equation}
v(x,0) = a(x) \,\,(x\in\bar{\Omega}),\,\,v\cdot\bar{n}|_{S_T}=0.
\end{equation}
где $\bar{n}$ -- единичный вектор внешней нормали к границе  $S$.

Задача (3), (4) описывает движение идеальной жидкости.
В силу наличия пространственного пограничного слоя трудно ожидать в общей ситуации хорошей сходимости при $\mu\rightarrow 0$  решений задачи (1), (2) к решению вырожденной задачи (3), (4).

Нами в работе [1] предложено рассматривать задачу (1) при условиях
\begin{equation}
u|_{t=0} = a(x),\,\,u|_{S_T}= v^0|_{S_T},
\end{equation}
где $v_0$ -- решение вырожденной задачи (3), (4).

Показано, что имеет место сходимость решений задачи (1), (5) к решению вырожденной задачи (3), (4) при $\mu\rightarrow 0$ в пространстве $L_2(Q_T)$.

Полученный в работе [1] результат подтверждает тот факт, что при малой вязкости решение задачи (1), (2) можно приближенно представить в виде суммы решения вырожденной задачи (3), (4) и некоторой погранслойной функции.

%%%%  ОФОРМЛЕНИЕ СПИСКА ЛИТЕРАТУРЫ %%%
\smallskip \centerline{\bf Литература}\nopagebreak

1. {\it Хацкевич В.Л.} Об асимптотике движения вязкой несжимаемой жидкости 	при малой вязкости. Дифференциальные уравнения, 2017, Т. 53, №6, С. 830 - 840

