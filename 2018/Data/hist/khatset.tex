\begin{center}{ \bf  ЧУДНЫЙ ПАРЕНЬ ИЗ НАШЕЙ ЮНОСТИ}\\
{\it И.\,П.~Грагеров, Е.\,Е.~Крисе, Б.\,И.~Хацет } \\
%(Липецк; {\it travkin@lipetsk.ru} )
\end{center}
\addcontentsline{toc}{section}{Грагеров И.П., Крисе Е.Е., Хацет Б.И.}

Селим Григорьевич Крейн~--- выдающийся учёный, замечательный педагог,
неутомимый организатор науки и математического просвещения, человек,
наделённый столь не\-о\-бы\-ч\-ной притягательностью и столь неотразимым обаянием,
что далеко не каждому в течение жизни выпадает счастье встречи с такой личностью.

Нам выпало.

Мы знаем, конечно, что Селим Григорьевич, энергичный, доброжелательный, общительный, душевно щедрый, всегда был окружён многочисленными коллегами, учениками, друзьями, почитателями и почитательницами. Его имя, его труды, его облик знакомы и дороги многим людям в разных странах. Так что воспоминания о нём, будучи собраны, могли бы образовать многотомное издание.

Нам хотелось, чтобы несколько этих страничек заняли бы своё, почти незаметное место в этом собрании. Мы --- это сверстники, которые стали друзьями Селима в теперь столь далекие тридцатые годы и сохранили эту дружбу до конца. Мы знаем многих, слишком многих из нашего поколения и из следующего, кто ушёл из жизни раньше Селима. Нам выпал другой жребий, и эта несправедливость судьбы добавляет горечи к боли утрат. Добавить несколько тёплых слов о дорогом нам светлом человеке --- наша потребность и наш долг.

Мы не будем писать ничего о научных трудах и о многогранной деятельности Селима Григорьевича. Есть много людей, которые сделают (и отчасти уже сделали) это лучше, чем смогли бы мы. Мы хотели бы просто рассказать тем, кто моложе и кто встретил Селима позднее, несколько впечатлений и эпизодов, связанных с этим красивым, лёгким, открытым человеком в наши очень молодые годы, из тех, что ещё сохранились в нашей уже ненадёжной памяти.

1.

Первым из нас с Селимом познакомился и подружился я, Изя Грагеров,
поступив в 10-й класс средней школы города Киева (сейчас в этом помещении расположен Киевский автодорожный институт),
в котором учился и Селим.
Впрочем, и тогда, и позднее все друзья и родные именовали его Мирой.
Насколько помню из Мириных рассказов, его имя Селим было производным от Шолом, что в переводе с иврита означает Мир.

Мира отличался от других учеников не только блестящими способностями и успехами по многим предметам, тем, что, будучи с детства поражённым анкелозом коленного сустава (из-за чего нога была согнута под прямым углом, и для перемещения требовались палка и подскоки на здоровой ноге), он, не обращая внимания на недуг, активно участвовал во всех играх, включая спортивные. В последней четверти 10-го класса Миру прооперировали --- ногу выпрямили, но в колене она так и не сгибалась. Зато можно было отбросить палку и ещё активнее участвовать во всех играх.

После окончания школы в 1935 году друзья поступили в Киевский университет:
Крейн на математический факультет, Грагеров~--- на химический.
Учёба их отличалась высокой успеваемостью и общественной активностью,
что по тогдашним нормам награждалось походами в театры, поездками в студенческий дом отдыха (под Житомиром)
в каникулярное время или в университеты других городов страны (например, в Тбилисский в 1940 году).
Всё это способствовало возникновению новых дружеских связей.
Так образовался сплочённый кружок, состоящий в основном из сту\-де\-н\-ток"=хи\-ми\-чек,
химика Грагерова и студентов-математиков во главе с Селимом.

В учебное время почти регулярно в конце недели в течение нескольких предвоенных лет (1939-1941) в просторной и гостеприимной квартире Хацетов, расположенной неподалеку от университета, собиралось 6-10 друзей за чаем с чудесными домашними коржиками. Мама математика Бориса и химички Фриды --- светлой памяти Нина Борисовна --- была удивительно хлебосольной и неравнодушной к друзьям своих детей хозяйкой. Время проходило быстро за обсуждением стихов, книг, событий, за различными литературными играми (буриме, викторины и др.). И, как всегда и везде, Мира --- искромётно остроумный, обаятельный --- был душой этого сообщества, победителем в состязаниях и верным другом каждого из нас. Мы охотно признавали лидерство этого чарующе красивого, никогда не унывающего человека; правда, иногда ему не удавалось <<перетанцевать>> другого партнера выбранной им девушки, но зато в волейболе неизменно переигрывал всех, даже классных игроков.

Нередко наша компания, в том или ином составе, отправлялась на прогулки, с которыми были связаны многие забавные эпизоды, инициатором которых обычно бывал Мира. И сейчас, по прошествии более шестидесяти лет, в памяти сохранилась такая история. Один из нас (Люсик Раппопорт) заключил с нами пари на 50 рублей, утверждая, что просидит 5 минут на тротуаре в самом людном месте Крещатика~--- у входа в главный довоенный киевский кинотеатр (после войны в этом месте разрушенного Крещатика было построено здание горисполкома) в разгар вечернего гуляния публики. Были назначены секунданты, но большинство из нас отправились вместе с участниками, чтобы стать свидетелями незаурядного события. На месте имеющего состояться происшествия Люсик как бы споткнулся и упал на тротуар, и тут оказалось, что он припас коробку с зубным порошком, которая якобы случайно раскрылась, а Люсик сел и стал не спеша собирать щепотками рассыпавшийся порошок в эту коробку. Собралась толпа, тут же милиционер, и Люсику стали помогать подняться. Однако он не давался, ссылаясь на исключительную ценность порошка. Наконец, пять минут, казавшиеся нам вечностью, истекли, секундант подал знак, Люсик вскочил и бросился наутёк, а изумлённая толпа во главе с остолбеневшим ментом ещё долго обсуждала увиденное в уверенности, что виновник события --- помешанный. Естественно, Мире не давала покоя слава Люсика, и он тут же вызвался на тех же условиях повторить его подвиг, но уже не на улице, а в людном вестибюле второго этажа Публичной библиотеки (Центральная библиотека АН УССР на Владимирской). Конечно, милиционера могло тут и не оказаться, но зато какая избранная публика! Многие из нас отправились полюбоваться зрелищем. И было на что посмотреть! Усевшись на полу, Мира артистично пользовался своей - действительно больной - ногой, чтобы отклонять все попытки его поднять. В течение обусловленных пяти минут, конечно \ldots

После окончания с отличием университета Селим был принят в аспирантуру Института математики Академии наук УССР; его научным руководителем был глава известной математической школы академик М.\,А.~Лаврентьев. Из-за неполноценной ноги Селим был освобождён от службы в армии и в начале войны эвакуирован вместе с Академией в город Уфу, где закончил аспирантуру и защитил
кандидатскую диссертацию.

Связи Миры с друзьями, военные адреса которых удалось установить, не прерывались. Так и я, Женя Крисе, медсестра военного госпиталя, бывшая студентка-химичка, получала даже в самые мрачные военные годы добрые, жизнеутверждающие и всегда оптимистичные Мирины письма. А Изе Грагерову удалось <<вживую>> встретиться и пообщаться с ним в Уфе, куда Изя специально заехал в 1942 году, возвращаясь после тяжёлого ранения из госпиталя к родным.

После защиты диссертации Селим получил назначение на должность преподавателя в Московский энергетический институт.
И в Москве продолжались дружеские встречи с Грагеровым, заканчивавшим аспирантуру в Московском
\linebreak
университете.

Примерно к середине 1946 года все мы возвратились в Киев на пепелище наших прежних мест общения.

Мира был востребован Институтом математики АН \linebreak УССР и даже получил комнату в помещении института, в которой прожил больше года. Женившись на химичке Евгении Кострюковой, он переехал в её семью. Переженились и другие друзья, но встречи и взаимная симпатия не иссякали. Академические успехи Миры радовали нас всех. Он был одним из самых молодых, популярных и уважаемых докторов физико-математических наук, получившим признание в кругу истинно творческих математических авторитетов.

Но, увы, грянула антисемитская кампания конца со\-ро\-ко\-вых--начала пятидесятых годов, которая коснулась и Селима.
Он переехал в Воронеж, где создал математическую школу --- одну из лучших в стране.

Наши семьи не прекращали дружить. Мы, Грагеровы, гостили у Крейнов в Воронеже и часто встречались с Мирой в Киеве, куда он приглашался в качестве руководителя или оппонента многочисленных диссертационных работ, участника математических семинаров, школ, конференций, и всегда --- воспоминания о юности, обсуждение жизненных обстоятельств и свершений. Возможность видеть друг друга согревала наши сердца. Вспоминается одно из шутливых поздравлений, посланное нами в Воронеж к юбилею Селима:

Славят нашего Селима

от Воронежа до Рима,

да и Киев к юбилею шлёт привет!

Город наш на древних кручах математиков могучих

наплодил, но вот такого - больше нет.

Приезжай к нам чаще, Мира,

чтоб на киевских квартирах

от пиров и танцев лопнул весь паркет.

И не вздумай задаваться возрастом: <<чуть-чуть за двадцать>>. Мы-то знаем --- ты в душе юнец-корнет!

Светлая память о Селиме Григорьевиче Крейне, нашем любимом Мире, замечательном человеке, учёном, близком и верном друге, навсегда запечатлена в сердцах всех нас, кому выпало счастье знать его лично, дружить, делить радость и горе.

2.

Я пытаюсь, но не могу вспомнить, когда именно познакомился с Мирой. Я поступил на физико-математический факультет Киевского университета (он тогда имел три отделения --- математики, физики и механики) тремя годами позже Миры (в 1938 г.) и не мог не обратить внимания на старшекурсника, который выделялся из общей студенческой массы красивым вдохновенным лицом, быстрой и уверенной, несмотря на прихрамывание, походкой, общей осанкой. Мы, первокурсники, особенно девочки, бегали из наших аудиторий в нижнем этаже исторического красного корпуса на Владимирской полюбоваться (через забор) его виртуозной игрой на волейбольной площадке напротив здания президиума АН УССР. Не знаю, было ли это звание официальным, но мы все называли его чемпионом и поражались силе воли юноши с покалеченной ногой.

Видимо, я попросил познакомить меня с Мирой другого старшекурсника, вскоре аспиранта, общение с которым тогда очень поднимало меня в моих глазах, а затем превратилось в близкую и прочную дружбу на всю жизнь. Я говорю об Иосифе Ильиче Гихмане (Жозике, как мы его называли) --- впоследствии крупном математике, возглавившем Киевскую школу теории вероятностей, позднее работавшем в Донецке. С Жозиком судьба свела меня ещё летом 1936 года в Коростышеве (вблизи Житомира) на берегу реки Тетерев. Я брёл на дачу, снимаемую родителями, с удочками, но без улова. Жозик сидел на большом камне с книгой в руках. Как-то возник разговор, и я спросил, что за книга; оказалось --- <<О философии математики>> Г.Вейля. Поскольку я по-мальчишески мнил себя и математиком, и философом, я попросил книгу почитать. Вскоре я пришёл к Жозику на дачу вернуть книгу. Глядя на мое сконфуженное лицо, Жозик рассмеялся и протянул мне книгу Теплица и Радемахера. А я укрепился в намерении поступать на математический.

Но вернёмся в 1938 год, так как именно в конце этого года или в начале следующего начала складываться вокруг Миры, как аттрактора, та тесная дружеская компания, которая затем превратилась в как бы клуб <<Мамины коржики>> и о которой выше писали Грагеровы. Лишь тогда я узнал, что настоящее имя Миры --- Селим, а Мирой (по одной из версий) его называли с детства потому, что все остальные мальчики в этой многодетной семье (Селим был младшим ребёнком) имели имена, начинающиеся с буквы <<М>>. Вначале (впрочем, и потом) мы устраивали совместные прогулки в центре Киева (приднепровские сады, Крещатик и прилегающие улицы и т.п.), обсуждая на ходу все те темы, вокруг которых в дальнейшем строились застольные беседы. При этом мы любили жевать ещё горячие вкуснейшие бублики, которые покупали, нанизанными на шпагат, в магазинчике-пекарне внизу Михайловской улицы.

Допускались и забавы, по современным понятиям совершенно невинные,
и тут Мира был неистощимым выдумщиком.
Помню, например, как однажды довольно поздним вечером, когда мы гурьбой шли по Крещатику от Владимирской горки,
Мира широким жестом пригласил нас перейти на запретную для пешеходов проезжую часть
(тогда почти свободную) со словами: <<Пошли, я угощаю!>>
Милиционер не замедлил появиться, отвёл нас на тротуар и потребовал уплатить штраф (что-то около 3-5 рублей).
Мира попросил минутку, чтобы разменять деньги, и надолго исчез в находящемся рядом (кажется, по сей день)
магазине минеральных вод. Милиционер, окружённый нами, не скупящимися на язвительные шуточки,
вынужден был ждать, испытывая возрастающую неловкость.
Но главное было впереди. Мира разменял деньги на копейки, и милиционеру довелось стоять с протянутой ладонью, в которую Мира отсчитывал мелкие медяки, иногда требуя пересчёта. Собрались зеваки, добавляющие свои реплики, страж порядка стоял пунцовый, пока, наконец, не сбежал, объявив, что на первый раз он нас прощает и отменяет штраф.

Мне тоже как-то пришлось стать жертвой Мириной шутки.
Не помню, где и по какому поводу мы собрались <<в расширенном составе>> --- человек пятнадцать.
Мира объявил, что проведёт сеанс гипноза, а мне была отведена роль подопытного.
Меня удалили из комнаты с тем, чтобы согласовать несколько слов, которые затем под гипнозом я должен был отгадывать.
Всё шло гладко, Мира совершал надо мной свои пассы, я по его приказу называл первое пришедшее в голову слово,
встречаемое всеобщим восторгом, так как оно <<совпадало>> с задуманным.
Возможно, теперь все знают эту шутку, но тогда я довольно долго чувствовал себя потерянным и озадаченным,
пока Мира не сжалился и не объяснил мне, что в некоторых фокусах не один дурит всех, а все дурят одного.

Осенью 1939 года я ввел в наш коржичный клуб девушку, в которую был влюблён, и она немножко разбавила наше химико-математическое сообщество, будучи студенткой гуманитарного факультета. Признаюсь, что с замиранием сердца ожидал суждения <<самого Миры>>. Оно оказалось благоприятным, и я ещё много лет (поженились мы с той девушкой аж в 1946 году) изливал ему свои неудачи на сердечном фронте. О своих переживаниях Мира не очень-то распространялся, но для меня, как и для других друзей, был превосходным <<транквилизатором>> благодаря неизменному оптимизму и жизнелюбию.

Во время войны я не поддерживал связи с Мирой, хотя краткую информацию о нём получал от его брата, знаменитого математика Марка Григорьевича Крейна. Марк Григорьевич жил тогда в Куйбышеве (Самаре) и заведовал кафедрой механики в авиационном институте. В это время я работал на военном заводе в 12 км. от города и иногда встречался с Марком Григорьевичем. Он сыграл большую роль в моей жизни, настойчиво побуждая меня окончить университет экстерном (до войны я окончил три курса), а затем немедленно начать работать ассистентом в авиаинституте на полставки, несмотря на 12-тичасовые смены на заводе и большие длины сторон треугольника завод-институт-жильё. Марк Григорьевич любил повторять, что педагогический опыт необходим каждому молодому математику, и чем в более трудных условиях он приобретается, тем он полезнее. Позднее я понял, насколько он был прав.

С Мирой я встретился вновь, когда он вернулся в Киев из Москвы в 1946 году. Вокруг него снова образовалась группа молодых людей, куда входили ставшие впоследствии известными математиками Георгий Исаакович Кац, Юрий Макарович Березанский, другие талантливые ребята (помню, например, Сергея Авраменко, Соломона Шахновского). Мы встречались, вместе прогуливались, но теперь не так часто, да и темы наших бесед стали глубже: позади была война, вокруг нас - разрушенный город, некоторые из нас уже были семейные люди. Помню рассказ Миры об одном `- в общем-то незначительном - эпизоде его преподавательской деятельности в Московском энергетическом институте. Во время его лекции погас свет (в послевоенной Москве перебои в подаче электроэнергии случались часто). Мира в темноте сказал: <<Известно, что коммунизм --- это советская власть плюс электрификация всей страны. Если в этом уравнении перенести один член в другую сторону, получим: советская власть --- это коммунизм минус электрификация всей страны>>. Человек, родившийся во времена <<перестройки>>, только усмехнётся: что-то такое я уже слышал. А человек из нашего поколения содрогнётся. В те сталинские времена практически в каждом студенческом потоке был хоть один <<информатор органов>>. Доложи он - и мир потерял бы Селима Крейна. Видимо, студенческая любовь к нему была столь сильна, что даже эти люди не могли ей не поддаться.

Более регулярно мы встречались с Мирой в Институте математики на семинаре по функциональному анализу, которым он руководил (одно время там был другой номинальный руководитель, но это продлилось недолго). Будучи студентом, я не имел случая слышать и видеть Миру в роли преподавателя. Теперь я мог и слышать, и видеть, и восхищаться, и учиться. Он обладал даром вскрывать живую душу математического рассуждения и помогать слушателям разделять с ним наслаждение, получаемое при этом. В связи с работой семинара Мира как-то попросил меня во время командировки в Москву зайти домой к Израилю Моисеевичу Гельфанду, который обещал Мире дать нам во временное пользование тогда труднодоступную книгу Хопфа по эргодической теории. Мне впервые предстояло повидаться с крупным учёным, об особенностях характера которого у нас ходили легенды. Мои страхи оказались напрасными: он принял меня на кухне, где чинил развороченный электроутюг, и мы мило побеседовали о нашем семинаре; я был рад, что Мира дал мне возможность познакомиться с этим человеком. Ещё в связи с семинаром вспоминается одна Мирина шутка, которая нас очень позабавила. Однажды к началу семинара, где-то к полудню Миры не было на месте. Подождав немного, направились к комнате, предоставленной ему институтом в качестве жилья. Мы постучали и услышали задорный Мирин голос: <<Погодите, мы ещё не проснулись!>>

Но суровые времена становились всё более суровыми, а государственный антисемитизм быстро крепчал. Многие ответственные работники Института математики за глаза называли наш семинар по функциональному анализу <<семинаром по национальному анализу>>. Институт срочно чистил свои и без того весьма чистые ряды (в последующие годы он твёрдо держал абсолютный рекорд даже среди академических институтов Украины). Директор Ишлинский сказал Мире: здесь всё-таки не Израиль. Словом, ещё одно хорошее дело было загублено. Дальше --- больше. Вскоре Мире пришлось работу в институте прекратить.

В 1948 году после окончания аспирантуры и защиты диссертации я был направлен
в Житомирский пединститут заведовать кафедрой на вновь открываемом фи\-зи\-ко"=ма\-те\-ма\-ти\-че\-с\-ком факультете.
Естественно, что контакты с киевскими друзьями стали менее частыми, но они не прекращались.
Кроме личных встреч с Мирой в дни, когда я приезжал в Киев,
я обращался к нему за советом и помощью по делам кафедры и факультета. Упомяну только о двух из них.
Я~считал важным для студентов и преподавателей провинциального вуза время от времени видеть и слышать выдающихся математиков и деятелей науки и приглашал многих ученых выступить с лекцией или докладом на нашем факультете. Естественно, что Мира был одним из первых, кто откликнулся на мою просьбу, и в институте долго ещё находились под впечатлением от его блестящей лекции. Очень доброжелательно отнёсся Мира (к этому времени он уже был не в Киеве) к моей просьбе принять на себя научное руководство диссертационной работой молодого члена кафедры, впоследствии моего друга и соавтора В.Н. Костарчука, много лет бывшего затем ректором Черниговского пединститута.

Мракобесам удалось выдавить Селима из Украины и тем нанести посильный вред своей <<неньке>>. Но не такой был Мира человек, чтобы можно было подавить его творческий потенциал, его энергию и его волю. Все эти замечательные качества проявились в воронежский период его плодотворной деятельности, но это принадлежит уже не моим воспоминаниям.

Впрочем, и после отъезда Миры в Воронеж наши связи не прерывались. Где-то в середине 60-х, когда я уже вернулся из Житомира в Киев, я обратился к нему с просьбой посодействовать переводу в Воронежский университет из Житомирского пединститута начинающего математика с незаурядными способностями, более склонного к научной работе, чем к учительской деятельности. Мира приложил значительные усилия, и в результате Воронежская школа приобрела теперь достаточно известного математика и хорошего человека --- Томаса Яковлевича Азизова. В 1965 году мы встретились с Мирой в Новосибирском Академгородке, где я должен был выступить с докладом о работе по новой для меня проблематике в области методов решения оптимизационных задач. Узнав, что Мира будет председательствовать на соответствующем заседании конференции, я изрядно струхнул, ибо хорошо знал, что личная дружба не защитит меня от острых реплик Миры, если он сочтёт их уместными. Впрочем, всё благополучно обошлось. Несколько позднее мне удалось побывать в Воронеже и пообщаться с Мирой. Я также старался не упускать случая увидеться с Мирой в Киеве во время его кратких наездов.

Годы шли, но время было бессильно лишить его поседевшую голову обаяния красоты и неувядающей молодости. Очень горько было сознавать, что тяжёлый недуг мешает ему жить и работать, но вызывала восхищение Мирина воля сопротивляться ему и продолжать своё дело. Как-то уже после операции шунтирования он приехал в Киев в качестве оппонента по диссертации. Мы встретились, как договорились, в Институте математики, защита уже началась. Перед тем, как попрощаться, Мира наклонился ко мне и тихо сказал: <<Мне сейчас выступать, но что-то прихватило \ldots>> Я тогда постоянно носил с собой нитропрепараты, и он, глотнув таблетку, направился к залу. Я не слышал его выступления, но уверен, что оно, как всегда, было впечатляющим.
<<Печаль моя светла.>> --- сказал поэт. Моя тоже. Воспоминания о Селиме Григорьевиче принадлежат к числу самых светлых и дорогих в моей долгой жизни.
