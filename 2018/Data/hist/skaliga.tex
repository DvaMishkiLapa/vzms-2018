\begin{center}{ \bf  ВОСПОМИНАНИЯ О Н.А. БОБЫЛЕВЕ}\\
{\it В.И. Скалыга} \\
%(Воронеж; {\it yusapr@mail.ru} )
\end{center}
\addcontentsline{toc}{section}{Скалыга В.И.}
Мы с Колей дружили 40 лет. Особенно часто мы общались в его московский период жизни Николая.
В Москве я нашёл Колю в аспирантском общежитии на улице Дм. Ульянова в феврале 72-го  года.
Там же   я встретил   своего друга по колмогоровскому интернату Сергея  Михайловича Воронина.
Затем наши контакты не прекращались до их ухода. Мы часто встречались в разных комбинациях.
Помимо математики мы трое были любителями хорошей литературы, поэзии, музыки.
Сергей и Николай со временем стали  известными математиками после публикаций их ярких результатов.
Сергей~--- в 75 он  доказал свою знаменитую теорему об универсальности дзета-функции Римана,
Коля~--- в  81ом опубликовал свой известный гомотопический инвариант минимума.
Одно время я занимался обобщением этого результата Коли на различные   пространства  и классы экстремальных задач.
Коля потом использовал мои наработки и конструкции в двух своих книгах.
С Сергеем многие годы я сотрудничал в области квадратурных формул, алгоритмов их построения.
Многолетнее сотрудничество с Сергеем и Колей, а также мои исследования в теории приближений, где я получил ряд точных оценок,
дало много работ, совместных и моих.
Я познакомился со многими известными математиками, стал выступать на известных семинарах,
публиковаться в центральных журналах и т.д.
Коля всегда бережно относился к друзьям, особенно школьным.
Мне Коля часто в жизни помогал: в  устройстве на интересную работу, где я мог заниматься математикой,
в организации  моих защит, кандидатской и докторской. На улице Коля не мог пройти мимо протянутой руки.
В тот страшный вечер гибели  Коли я был рядом с ИПУ.  Если бы я зашёл к Коле, мы бы пошли домой другой дорогой.
Но не было явной причины зайти, а Коля не любил праздных встреч. Николай был человеком дела.
Просто поразительно~--- сколько Коля успел создать за свою недолгую жизнь. В  этом есть что-то лермонтовское.
Коля и был поэтом~--- поэтом в математике с  тонкой, виртуозной техникой построения математических конструкций.
Спектр интересов его был широк, от топологии до линейной алгебры, и везде были результаты  мирового уровня.
С уходом Коли математика потеряла одного из ярких своих создателей, а мы бесценного друга.
Боль утраты не проходит. Но мы  всегда рядом~--- на сайте  math-net.ru.

Вечная Вам память мои великие друзья.   До встречи.


