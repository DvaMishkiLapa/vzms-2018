\begin{center}{ \bf  ВОСПОМИНАНИЯ}\\
{\it С.Г. Крейн} \\
%(Воронеж; {\it yusapr@mail.ru} )
\end{center}
\addcontentsline{toc}{section}{Крейн С.Г.}
\begin{flushright}
Данные материалы,
написанные рукой Селима Григорьевича Крейна,
были переданы в оргкомитет конференции,
посвящённой 100-летию С.Г.Крейна,
 Адамовой (Показеевой) Риммой  Сергеевной
\end{flushright}

Марк Григорьевич Крейн родился в апреле 1907 года. В нашей семье было семеро детей. М.Г. был четвёртым (я седьмым). Он успел закончить 7 классов официальной школы и один класс какой-то частной школы, созданной на средства родителей. Полного среднего образования он не имел. (Кстати ряд крупнейших наших математиков не имел среднего образования: Боголюбов, Гельфанд, Лаврентьев, Шнирельман и др.). М.Г. подрабатывал распиловкой дров. Дрова были тогда единственным источником тепла. В то же время М.Г. заинтересовался математикой и стал посещать некоторые лекции и семинары в Киевском университете. Там он обратил на себя внимание. Я помню, как к нам пришёл профессор Б.Н. Делоне (известный алгебраист) и просил родителей отпустить М.Г. с ним в Ленинград, куда он переезжал из Киева. Однако согласия он не получил.

Однажды утром я проснулся и узнал, что Марк ушёл из дому. В записке он написал, что ему уже 16 лет и он больше не может сидеть «на шее у родителей». Одновременно сбежал его приятель, проживавший в нашем доме. Его родители как-то узнали, что беглецы поехали в Одессу, и старший брат поехал на розыски. Поиски не увенчались успехом и он, собираясь уехать, решил сходить на пляж. Там он увидел толпу, которая наблюдала за гимнастическими трюками Марка и его товарища. Как выяснилось, они собирались наняться юнгами на пароход и поехать заграницу, стремясь попасть к моему дяде, который с 1908 г. жил в Берлине. Это не удалось. Тогда они нанимались в цирк. Не знаю почему, но их не приняли. Дальше в моих сведениях разрыв. Знаю только, что брата приютила семья его будущей жены. На жизнь он зарабатывал уроками. Вскоре он связался с университетом и затем поступил в аспирантуру (без среднего и высшего образования) к Николаю Григорьевичу Чеботареву (известному алгебраисту). Закончил аспирантуру в 22 года, представив семь работ по алгебре, геометрии и теории функций. Одновременно с обучением в аспирантуре читал лекции в Донецком горном институте (по-моему, в ранге профессора).

Я проживал в Киеве и виделся с братом редко. В 1939 г. его избрали членом-корреспондентом Академии наук УССР и после этого он систематически приезжал в Киев. Первые его лекции были по теории конусов в банаховом пространстве. Они увлекли многих. Мой руководитель Н.Н. Боголюбов предложил мне получить методом конусов обобщения одной теоремы Фреше об интегральных уравнениях. С его помощью я это сделал. В процессе работы у меня возникла мысль о структуре универсально пространства для пространств с миниэдральным конусом. Первые шаги были сделаны, а окончательное доказательство было получено с М.Г. во время моей поездки в Одессу. Этот результат в литературе носит название теоремы братьев Крейн-Какутани (японский математик).
Война нарушила наше сотрудничество. После войны М.Г., будучи в Одессе, заведовал отделом функционального анализа в Институте математики АН\linebreak УССР. У меня с ним образовался своеобразный тандем: я руководил постоянно действующим семинаром, а М.Г. периодически приезжал, делал доклады, ставил задачи. Я начинал работать с молодым киевским математиком, а затем его руководителем становился М.Г. Такую эволюцию прошёл один из лучших учеников М.Г.  – Марк Александрович Красносельский. Аналогичная ситуация была с известными киевскими математиками Ю.М. Березанским и Ю.Л. Далецким.

В 1951 году нас с М.Г. вынудили уйти из Института математики и наши творческие пути разошлись.

В статье в журнале «Успехи математических наук», посвящённой 70-летию М.Г., описывается ряд направлений, в которых М.Г. получил существенные результаты. Приведу их:  1) Проблема моментов, теория приближений. 2) Теория устойчивости решений дифференциальных уравнений. 3) Геометрия банаховых пространств. 4) Теория расширений эрмитовых операторов. 5) Спектральная теория. 6) Операторы в пространствах с индефинитной метрикой. 7) Теория несамосопряженных операторов. 8) Теория пространств с конусом. 9) Осцилляционные матрицы и операторы. 10) Теория гамильтоновых уравнений. 11) Теория представлений компактных групп.

