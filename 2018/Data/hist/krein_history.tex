\begin{center}{ \bf ИЗ ИСТОРИИ ВОРОНЕЖСКОЙ МАТЕМАТИКИ}\\
{\it С.Г. Крейн} \\
%(Воронеж; {\it yusapr@mail.ru} )
\end{center}
\addcontentsline{toc}{section}{Крейн С.Г.}

{ \bf Теория уравнений в частных производных в Воронеже}

Настоящая статья не претендует на обзор результатов, полученных в Воронеже в 50-60-е гг. по теории уравнений в частных производных. Её также нельзя рассматривать как работу по истории математики.

Статья скорее носит характер мемуаров~--- автор пытается рассказать, как возникала та или иная тематика,
кто и как включался в её выполнение, какой характер носила творческая обстановка.
В журнале «Успехи математических наук» в 1964 г. была опубликована статья «О математической жизни в Воронеже»
четырёх авторов: М.\,А.~Красносельского, С.\,Г.~Крейна, Я.\,Б.~Рутицкого, В.\,И.~Соболева [1].
Автор предполагает рассказать обо всех работах по теории уравнений в частных производных, выполненных за период 1954-1964 гг. После этого коснёмся, в основном, тех результатов, в получении которых автор участвовал как исполнитель или как руководитель. В связи с этим статья носит несколько субъективный характер. Другой автор, вероятно, делал бы акценты в других местах и излагал бы многое с иной точки зрения.

К статье прилагается список литературы, в который
\linebreak
включены лишь обзорные доклады на съездах и конференциях,
обзорные статьи в журналах и монографии. Ознакомление с этой литературой может позволить читателю избавиться от чувства субъективизма, о котором говорилось выше.

{\bf Предыстория}

Когда меня спрашивают, что это за наука~--- функциональный анализ, я отвечаю, что это не наука, а мировоззрение.
К моменту моего приезда, в 1954 году, в Воронеже это мировоззрение усилиями М.А.~Красносельского и В.И.~Соболева уже пустило глубокие корни. Поэтому мне было легко включиться в работу математического коллектива. Обсуждая с Марком Александровичем Красносельским перспективы, мы пришли к решению начать совместную работу по теории дифференциальных уравнений в банаховых пространствах. Два раза в неделю я приезжал к нему домой. Мы разбирали книгу Сансоне и пытались
перенести изложенные в ней факты на случай дифференциальных уравнений в банаховых пространствах с ограниченными операторами. Нам удалось получить теоремы существования, единственности, неограниченной продолжимости решений, обоснование принципа усреднения.

Некоторые теоремы были новыми даже для системы
\linebreak
обы\-к\-но\-вен\-ных дифференциальных уравнений.
К разработке этой теории вскоре была привлечена большая группа учеников Марка Александровича. Взаимодействие функционального анализа с теорией уравнений в частных производных, как, впрочем, и с другой, более конкретной систематической дисциплиной, может идти по таким путям:
\begin{itemize}
	\item
		результаты,   полученные   в процессе внутреннего развития функционального анализа, получают приложения в конкретной дисциплине;
	\item
		задачу, возникшую, например, в математической физике, удаётся
	 сформулировать в терминах функционального анализа и применить к её решению известные, или близкие к известным результаты;
	\item
		переформулированная задача упирается в ещё не разработанные разделы функционального анализа и становится стимулом для их разработки. При этом разработка новых вопросов иногда уходит так далеко, что, подобно Ивану, не помнящему родства, не возвращается к искомой задаче.
\end{itemize}
Конечно, эти различные пути часто переплетаются.

{\bf Гидродинамический стимул}

В 1953 г. мною была опубликована в ДАН статья, в которой, по-видимому, впервые, был проанализирован с точки зрения функционального анализа ряд основных операторов векторного анализа и гидродинамики вязкой жидкости, предложен метод проектирования исследования уравнений Навье-Стокса, применена тогда только недавно возникшая теорема М.~Келдыша о полноте нормальных колебаний. При выполнении этой работы я пошёл по второму пути. Когда эти же соображения мы попытались применить к нестационарной задаче о движении вязкой жидкости, которая существенно нелинейна, то известных нам средств функционального анализа явно не хватало. А именно, требовалось развить теорию дифференциальных уравнений в банаховом пространстве с неограниченными операторами. Для меня было неожиданным открытием то, что в теории полугрупп неограниченных линейных операторов фактически изучается теория простейших дифференциальных уравнений в банаховом пространстве (линейные с постоянными коэффициентами). Таким образом, задача гидродинамики как стимул плюс наш с Марком Александровичем опыт по дифференциальным уравнениям с ограниченными операторами, плюс освоение теории полугрупп — все это дало возможность нам с М.\,А.~Красносельским и П.\,Е.~Соболевским, начиная с 1956 года, выполнить ряд работ по теории дифференциальных уравнений с неограниченными операторами. Эта тематика на многие годы стала одной из ведущих для воронежских математиков.

{\bf Семинар в ВЛТИ}

В те годы я работал в Воронежском лесотехническом институте (ВЛТИ).
Всех наиболее сильных математиков - выпускников ВГУ брал к себе в аспирантуру Марк Александрович.
Поэтому мне пришлось привлекать молодёжь, являющуюся необходимой «живительной средой» для учёного, со стороны.
Так, в организованном мной при ВЛТИ семинаре участвовали:
два бывших студента Киевского госуниверситета, не имевших возможности поступить там в аспирантуру
(О.\,М.~Козлов и П.\,Е.~Соболевский); приехавший в 1956 г. из Челябинска кандидат наук И.\,А.~Киприянов;
выпускник ВГУ, механик, отработавший по назначению В.\,П.~Глушко, поступивший в 1956г. на полставки ассистента ко мне на кафедру,
хотя имел более выгодные предложения из других вузов; выпускник ВГУ, механик-экспериментатор С.\,С.~Литвинов. Этот семинар по отношению к ВГУ носил характер «Запорожской Сечи». Правда, на многих заседаниях присутствовал и активно в них участвовал Марк Александрович.

Сохранился интересный документ того времени - «Тезисы докладов на научной конференции ВЛТИ по итогам исследовательских работ за 1957г». Сборник печатался в типографии, не имевшей математических символов, поэтому их приходилось описывать словами. А вот список докладов: «Дробная производная и теорема вложения» (И.\,А.~ Киприянов), «О дробных степенях дифференциальных операторов» (С.\,Г.~Крейн, В.\,П.~Глушко), «О некоторых граничных задачах для линеаризованных уравнений гидродинамики» (С.\,С.~Литвинов), «О связи между нормами решений задач Неймана и Дирихле» (О.\,М.~Козлов), «О некоторых свойствах операторов типа потенциала» (В.\,П.~Глушко). Замечу, что первый доклад П.\,Е.~Соболевского о суммируемых полугруппах операторов, сделанный им на семинаре, был опубликован в аналогичном сборнике в СХИ. Все эти доклады уже относились к теории уравнений в частных производных, позднее они были опубликованы в центральной печати и получили развитие, о котором будет сказано ниже.

{\bf Расширенное заседание семинара по фу\-н\-к\-ци\-о\-на\-ль\-но\-му анализу}

Без преувеличения можно сказать, что оно стало выдающимся событием в математической жизни Воронежа в марте 1957~г. [2].
На заседание были приглашены десять иногородних математиков:
Ю.\,М.~Березанский (Киев),
М.\,И.~Ви\-шик (Москва), И.\,И.~Ворович (Ростов-на-Дону), А.\,Г.~Костюченко (Москва), А.\,И.~Кошелев,
О.\,А.~Ладыженская (Ленинград), А.\,Д.~Мышкис (Харьков), О.\,А.~Олеиник и
\linebreak
А.\,Я.~По\-в\-з\-нер (Москва),
Л.\,Д.~Фаддеев (Ленинград).
Самым опытным участником был А.\,Я.~Повзнер, самым молодым - Л.\,Д.~Фаддеев, ученик О.\,А.~Ладыженской.

Заседания проходили в течение шести дней по программе, выработанной после того, как все съехались.
Регламентом докладчики не были стеснены.
Практически вечерние доклады длились по полтора часа,
продолжительность
\linebreak
утренних колебалась от 5-10 минут до двух часов.
Все заседания проходили оживлённо и в весьма непринуждённой обстановке. Особенно это относится к утренним заседаниям, которые, в основном, носили дискуссионный характер и на которых многие участники выступали по нескольку раз.

В вечерних заседаниях участвовали почти все математики и многие механики Воронежа.
Утренние заседания проходили при небольшом числе слушателей.
От Воронежа с вечерними докладами выступали:
М.\,А.~Красносельский,
\linebreak
С.\,Г.~Крейн, В.\,И.~Соболев~--- с обзорами работ молодых воронежских математиков по функциональному анализу и его приложениям (18 марта): они же рассказали о своих работах по применению теории полугрупп к решению линейных эволюционных уравнений в банаховом пространстве.

В широком плане обсуждались проблемы, связанные с уравнением Навье-Стокса, по которым основной доклад был сделан О.\,А.~Ладыженской. Вопросы о существовании и \linebreak свойствах\,разрывных\,решений нелинейных модельных уравнений газовой динамики, решённые О.\,А.~Олеиник с учениками, освещались в её содержательном докладе. Сильное впечатление произвёл на нас доклад М.\,И.~Вишика о его совместных с Л.\,А.~Люстерником работах по теории пограничного слоя для уравнений в частных производных. Об этих проблемах в Воронеже раньше ничего не знали.

Ряд докладов и выступлений был посвящён спектральной теории дифференциальных операторов в частных производных и,
в частности, получению аналогов различных свойств преобразования Фурье для разложений по собственным функциям,
возможно обобщённым, таких операторов
(Ю.\,М.~Березанский, А.\,Г.~Костюченко, Л.\,Д.~Фаддеев).
Слушатели с большим интересом восприняли доклад И.\,И.~Воровича по нелинейной теории оболочек,
в котором было продемонстрировано глубокое взаимное переплетение методов функционального анализа и интуитивных механических соображений. Казалось, нет такого факта функционального анализа, которому И.\,И.~Ворович не мог бы придать механический смысл.

А.\,И.~Кошелев рассказал свою известную теорему о коэрцитивной разрешимости первой краевой задачи для сильно эллиптических систем произвольного порядка в L.-нормах.

П.\,Е.~Соболевский ввёл понятие операторов, образующих острый угол,  которое весьма полезно для сравнения сложных операторов с более простыми: А.\,Я.~Повзнер сделал доклад о методе получения решений различных нестационарных линейных уравнений предельным переходом с помощью <<ступенчатой>>  аппроксимации входящих в уравнение операторов.

Новаторское впечатление произвёл доклад А.\,Д.~Мышкиса по теории особых точек так называемых «бушующих» динамических систем.

 	Следует ещё рассказать, что иногородние участники семинара жили в общежитии Воронежского лесотехнического института, которое только что было введено в строй, и студентам его ещё нельзя было заселять. Последние дни марта выдались очень холодными, поэтому только неограниченные возможности использования матрацев и научный энтузиазм спасали жильцов от холода и сырости. Но несмотря на трудности у всех участников этого семинара остались самые светлые воспоминания о той атмосфере дружелюбия, взаимной доброжелательности и раскованности, которая установилась на заседаниях. Достаточно сказать, что с большинством участников семинара воронежцы до сих пор поддерживают дружеские отношения.

А для математиков Воронежа семинар был чрезвычайно полезен тем, что показал большую широту и разнообразие проблем теории уравнений в       частных производных и её приложений, а также недостаточное ещё развитие методов функционального анализа для решения этих проблем.
Отметим, что участники семинара решили встретиться снова в Харькове. И встреча состоялась, однако в ней приняли участие уже 35 человек, доклады были жёстко регламентированы, все участники размещались в гостиницах. То есть это была уже обычная конференция. Романтика исчезла.

{\bf Дифференциальные уравнения в банаховых пространствах}

Итак, в первой работе М.\,А.~Красносельского, С.\,Г.~Крейна и П.\,Е.~Соболевского (ДАН, 1956г.) рассматривалась задача Коши для уравнения:
\begin{equation}
\frac{dx}{dt} = A(t)x + f(t, x)
\end{equation}
где $x(t)$ - искомая функция со значением в банаховом пространстве $E$, $A(t)$ и $f(t,x)$ --- операторы, действующие в $E$, причём $A(t)$ при каждом $f$ --- неограниченный замкнутый линейный оператор.

Используя теорию полугрупп, результаты Филлипса, Като относительно линейной части уравнений, уточняя и обобщая их, мы получили некоторые теоремы существования решений задачи Коши, локальные во времени. В этих теоремах предполагалось, что нелинейный член $f(t,x)$ обладает определённой гладкостью и заведомо непрерывен. Однако, уже простейший нелинейный оператор возведения функции в квадрат не является непрерывным в пространстве $L_p$, хотя он непрерывен в С. Для ослабления требований на нелинейность мы перешли к рассмотрению уравнения (1) в гильбертовом пространстве с положительно определённым самосопряжённым оператором $A(t)$. Для таких операторов определены функции от них и, в частности, дробные степени.

Тогда в уравнении (1) делается формально замена \linebreak $x = A^{-\alpha} y$, которая приводит к уравнению:
\begin{equation}
\frac{dy}{dt} = A(t)\frac{d A^{-\alpha} (t)}{dt} y + f(t, A^{\alpha} y)
\end{equation}

Может, например, случиться, что оператор, который, как правило, улучшающий, переводит $L_p$ в $C$. Оператор $f(t,x)$ непрерывен в $C$, а оператор $A^{-\alpha}$  возвращает $f(t, y)$ снова в $L_2$ (эта идея, в основном, принадлежала Марку Александровичу).

Это соображение позволило рассмотреть уже большой класс нелинейностей,
не являющихся непрерывными в исходном пространстве.
При исследовании второго члена
\linebreak
справа использовались наши с Ю.\,Л.~Далецким (Киев) результаты о дифференцировании по параметру t функций от самосопряжённых операторов. Рассмотренный метод мы назвали методом дробных степеней операторов и рассказывали о нем в совместном докладе на совещании в Харькове.

Примерно в этом же плане была построена часть нашего с М.А.Красносельским доклада на III Всесоюзном математическом съезде, который назывался «О дифференциальных уравнениях в банаховом пространстве» [3]. Основным примером применения общих теорем являлось уравнение теплопроводности с нелинейностью, где оператором $A(t)$ был оператор Лапласа при нулевых граничных условиях. Как раз в это время В.А. Ильин (Москва) нашёл в явном виде ядра интегральных операторов, являющихся отрицательными дробными степенями оператора Лапласа.
Мы с П.\,Е.~Соболевским обобщили то, что делалось для уравнения с положительно определённым оператором $A$, на случай оператора $A+B(t)$, где $B(t)$ подчинён дробной степени оператора $A$. Такой оператор мы назвали абстрактным эллиптическим. В дифференциальных уравнениях это означало, что к главной линейной части можно добавлять члены с младшими производными, не изменяя существенных свойств уравнения. Здесь же было показано, что определение того, что оператор $В$ подчинён дробной степени $А$ эквивалентно некоторому неравенству, в котором участвуют только нормы $A$ и $B$.

Мы с В.\,П.~Глушко заметили, что в методе дробных степеней не нужно знать вид отрицательных дробных степеней оператора,
а достаточно знать, в каких пространствах действуют эти операторы.
Для широкого класса самосопряжённые эллиптических краевых задач произвольного чётного порядка
(для которых в то время были получены неравенства коэрцитивности) мы указали, в какое пространство из пространства $L_2$
действуют дробные отрицательные степени соответствующих операторов,
их композиции с операторами дифференцирования и операторами умножения на сингулярные коэффициенты.
Доказательства основывались на уточнении роли неравенств, о которых говорилось выше, и на получении этих неравенств.
Для этого пришлось обобщить неравенства, которые назывались неравенствами Эрлинга-Ниренберга,
с помощью новых теорем об операторах типа потенциала, полученных В.П.Глушко.
При этом нам пришлось исследовать геометрические свойства областей, звёздных относительно шара.
Упоминавшиеся неравенства в $L_2$
с весом и свойства звёздных областей с полными доказательствами были позднее опубликованы
в сибирском математическом журнале (в 1960 г., а через несколько лет статья была переведена в США).
Попутно замечу, что В.\,П.~Глушко дал положительный ответ на вопрос С.\,Л.~Соболева о том, является ли область со свойством конуса объединением конечного числа областей, звёздных относительно шара.

К сожалению, все наши ухищрения не удовлетворяли нас,
так как «...и локальные (not) теоремы существования решений в гильбертовом пространстве
в применении к нелинейным параболическим уравнениям давали теоремы существования решений граничных задач
с жёсткими условиями на рост нелинейности по $x$,
что с точки зрения классической теории казалось неестественным.
Сейчас стала ясна связь этих ограничений с тем, что начальные данные и, соответственно,
все решения рассматривались в $L_2$.
Возможность рассмотрения нелинейных параболических уравнений путём сведения их к операторным уравнениям в пространствах $L_p$
со сколь угодно большим $p$ позволила получить ряд локальных теорем без всяких ограничений на рост нелинейности».
Здесь приведена цитата из нашего с М.\,А.~Красносельским доклада на Всесоюзном математическом съезде [4].

Получение последних результатов потребовало преодоления целого ряда трудностей.
Во-первых, требовалось построение и изучение свойств дробных степеней операторов в банаховом пространстве.
Подходящим для этого оказался класс производящих операторов аналитических полугрупп,
на который обратил наше внимание М.\,З.~Соломяк, и более широкий класс позитивных
(по терминопогии М.\,А.~Красносельского и П.\,Е.~Соболевского) операторов. Почти одновременно, хотя и независимо друг от друга, дробные степени таких операторов были рассмотрены индийским математиком Балакришнаном и М.\,А.~Красносельским с П.\,Е.~Соболевским. В дальнейшем глубокие результаты были получены Като.

Во-вторых, требовалось изучить линейные уравнения с переменным оператором указанного типа.
Это было сделано П.\,Е.~Соболевским и, независимо от него, японским математиком Танабе.
Для доказательств был применён аналог метода Лови (метод «замороженных коэффициентов»).
Полученная теорема в литературе называется теоремой Соболевского"=Танабе и под таким названием
включена в учебник Иосиды по функциональному анализу [5].
Таким образом, все основные результаты,
полученные ранее для дифференциального уравнения с положительно определённым оператором (или с эллиптическим оператором)
в гильбертовом пространстве,
удалось перенести на уравнения в банаховом пространстве с производящим оператором аналитической полугруппы. Описанные результаты легли в основу докторской диссертации П.\,Е.~Соболевского, которую он защитил в 1962 г. (Это был первый доктор наук из «нашего леса»).

Ещё более общий класс уравнений был рассмотрен бакинским математиком Я.\,Д.~Мамедовым, одно время проживавшим в Воронеже,
и П.\,Е.~Соболевским. Дальнейшее исследование дробных степеней позитивных операторов
(в частности, позитивных эллиптических операторов) проводилось в работах М.\,А.~Красносельского с П.\,П.~Забрейко, Е.\,И.~Пустыльником, а также П.\,Е.~Соболевским. Полученные результаты изложены в книге [6].

{\bf Ну, а что же с гидродинамикой?}

Это было знойным летом, когда мы с П.\,Е.~Соболевским, покрытые минимумом одежд, «долбили» локальную теорему существования решения задачи Коши для нестационарного уравнения гидродинамики вязкой жидкости (уравнение Навье-Стокса). После длительных усилий мы её доказали и на последнем листе рукописи написали: «Ура! Принцип Шаудера». Этот результат был анонсирован в моем докладе на конференции по функциональному анализу «Дифференциальные уравнения в банаховом пространстве и их приложения в гидродинамике» [7], а также на съезде механиков. Однако эта работа не была опубликована. Дело в том, что в ней мы использовали классические результаты Одквиста об оценках функции Грина для оператора Навье-Стокса. Когда я дал О.\,А.~Ладыженской прочитать весь текст работы (включая «Ура!»), она сказала, что у них на семинаре изучались работы Одквиста, и была обнаружена их ошибочность. Узнав об этом, мы не решились на публикацию. Хотя через некоторое время работы Одквиста были реабилитированы, но уже появились другие локальные теоремы существования. Отмечу, что нелокальная теорема существования сильных решений задачи Коши для трёхмерных уравнений Навье-Стокса при любых числах Рейнопьдса до сих пор ещё не доказана.

{\bf Ещё раз о семинаре в ВЛТИ}

Проследим результаты деятельности участников семинара до 1964г., когда была опубликована статья М.\,А.~Красносельского, С.\,Г.~Крейна, Я.\,Б.~Рутицкого и В.\,И.~ Соболева [1], в которой был параграф №3 --- «Уравнения математической физики». О П.\,Е.~Соболевском уже было сказано, что он вырос в крупного
специалиста по теории дифференциальных уравнений в банаховом пространстве и их приложений, в частности, к гидродинамике.

С.\,С.~Литвинов изучил сходимость рядов Фурье по обобщённым сферическим функциям
(введённым И.\,М.~Гель\-фа\-н\-дом и З.\,Я.~Шапиро) и применил их к решению некоторых задач гидродинамики. Защитив диссертацию в 1962г. он перешёл к другой тематике. О.\,М.~Козлов изучил простейший случай уравнений, в которых оператор $A(t)$ при каждом $t$ является самосопряжённым расширением одного и того же симметрического оператора. В дифференциальных уравнениях это соответствует тому, что коэффициенты уравнения постоянные, а коэффициенты граничных условий зависят от $t$. Им было обнаружено, что при некоторых условиях удаётся установить гладкость оператора $A^{\alpha}(t)$ при $A^{\alpha} < \frac{1}{2}$  и постоянство его области определения.

Далее О.\,М.~Козлов методом теории расширения симметрических операторов в гильбертовом пространстве исследовал краевые задачи для эллиптических уравнений второго порядка с разрывными коэффициентами граничных условий, а также с некоторыми нелинейными граничными условиями. Диссертацию он защитил в 1961 г. после чего возвратился в Киев, где и работает над проблемами прикладной математики.

Весьма целеустремлённо работал И.\,А.~Киприянов. В течение ряда лет он развивает теорию пространств функций, имеющих производные дробного порядка в том или ином смысле. Для этих пространств им были получены теоремы вложения, мультипликативные неравенства и др. Затем, по моему совету, он применил теорию преобразования
Фурье-Бесселя к построению новых классов пространств, для которых получил теоремы вложения, прямые и обратные теоремы о следах, построил операторы дробного дифференцирования связанные с оператором Бесселя. Полученные теоремы вложения он применил для исследования вариационным методом краевых задач для некоторых вырождающихся эллиптических уравнений, установил априорные оценки решений уравнений, содержащих операторы Бесселя. Этот материал вошёл в докторскую диссертацию, которую он защитил в 1964 г.

Большую самостоятельность, я бы даже сказал,
сме-\linebreak лость,
проявил В.\,П.~Глушко.
Он вторгся в область, которой в Воронеже до него никто не занимался. Тематика эта началась с работы M.\,B.~Келдыша, развивалась С.\,Г.~Михлиным и М.\,И.~Вишиком. В.\,П.~Глушко получил новые теоремы об интегральных операторах, действующих в пространствах  $L_p$ с весом, и, в частности, об операторах типа потенциала. На базе этих теорем устанавливаются новые теоремы вложения и мультипликативные неравенства. Далее рассматриваются вырождающиеся эллиптические уравнения и схема, разработанная М.\,И.~Вишиком для уравнений второго порядка, обобщается на специальный вид эллиптических уравнений произвольного чётного порядка, вырождающихся или имеющих особенности на многообразии любой размерности, меньшей, чем размерность области (до этого вырождение допускалось лишь по границе области или в конечном числе точек двумерной области). Детально был исследован вопрос о снятии или сохранении граничных условий. Перечисленные здесь и выше результаты В.\,П.~Глушко легли в основу его кандидатской диссертации, которую он защитил в 1961 г.

В дальнейшем В.\,П.~Глушко построил теорию общих краевых задач для вырождающихся эллиптических уравнений второго порядка, изучив предварительно поведение решений вырождающегося обыкновенного уравнения второго порядка в банаховом пространстве, и в 1970 г. защитил докторскую диссертацию.

В 1960 г. на кафедру ВЛТИ пришёл И.\,И.~Шмулев. Он исследовал методом Лере Шаудера вопрос о разрешимости нелинейной системы эллиптических уравнений второго порядка, возникающей в связи с задачей о периодических решениях нелинейных параболических уравнений. По этим вопросам он защитил диссертацию в 1962 г. Аналогичными задачами, но с применением более совершенной техники, он продолжает заниматься и сейчас.

В настоящее время в университете работают В.\,П.~Глушко (с 1963г.), И.\,А.~Киприянов (с 1967г.) и возглавляют кафедры, на которых ведутся интенсивные исследования по теории уравнений в частных производных.

{\bf Некорректные задачи}

Задача Коши для дифференциальных уравнений называется корректной, если решение её существует и непрерывно зависит от начальных данных. Некорректные задачи встречаются в математической физике. Однако такие задачи могут стать корректными в некотором, априори заданном, классе решений (ограниченных, положительных и др.).

И.\,М.~Гельфанд в докладе на Всесоюзном совещании по функциональному анализу в 1966г. обратил внимание на этот класс условий корректных задач для уравнений в частных производных. Меня этот вопрос, как говорится, «зацепил», и мне захотелось что-нибудь сделать для абстрактных дифференциальных уравнений. Я рассмотрел уравнение $u + A(t)u = 0$. Если $А(t)$ --- положительно определённый, то задача Коши $u(0)=u_0$ при определённых условиях на гладкость оператора корректна в гильбертовом пространстве. Я рассмотрел случай просто самосопряжённого оператора $A(t)$, у которого производная по $t$ ограниченный сверху оператор. Оказалось, что для всякого решения на $[0,T]$ справедливо неравенство вида:
$$||u(t)||\leqslant||u_{\alpha}||^{1-\alpha(t)} ||u(t)||^{\alpha(t)},$$
где $0<\alpha(t) < 1$.
Отсюда сразу видно, что в классе решений, априори ограниченных фиксированной константой,
задача Коши корректна (при малой норме $u_0$ решение также мало по норме).
В качестве приложений получались теоремы о корректности в норме $L_p$
ограниченных решений для различных типов уравнений в частных производных.
К дальнейшей разработке этих вопросов я привлёк О.\,И.~Прозоровскую. Мы рассмотрели уравнение $U = Au$ с производящим оператором аналитической полугруппы в банаховом пространстве. Задача Коши для него корректна. Пользуясь принципом Неванлинна для аналитической функции, мы смогли получить неравенство типа (3), где $\alpha(t)$
является гармонической мерой отрезка $[u, \infty)$  относительно угла аналитичности полугруппы. Из этого неравенства вытекает, что в классе решений, априори ограниченных константой $(||u||<M)$, корректно решение обратной задачи Коши $u(t) = u$ (см. 10)

{\bf Интерполяция линейных операторов}

Просматривая как-то книги М.\,А.~Красносельского, я наткнулся на том журнала «Am.J.Math»,
в котором прочёл статью Кальдерона и Зигмунда о доказательстве теоремы М.~Рисса.
В простейшей формулировке она говорит o том, что если линейный оператор ограниченно действует в пространствах $L_{p_0}$ и  $L_{p_1} (1\leqslant p_0 \leqslant p_1 \leqslant x)$, то он ограниченно действует и во всех проме-жуточных пространствах $L_p$ c $p \in [p_0;p_1]$

На меня эта теорема и её доказательство с помощью теоремы о трёх прямых из теории аналитических функций произвели большое впечатление. Возникло желание найти другие семейства пространств, обладающих аналогичными свойствами. На этом пути было введено понятие аналитической шкалы пространств, для которой был справедлив аналог теоремы М.~Рисса. Были получены интерполяционные теоремы для новых семейств пространств: уточнения теорем вложения, полученных ранее М.\,А.~Красносельским и Е.\,И.~Пустыльником, В.\,П.~Глушко и мною; теоремы о дробных степенях положительно определённого самосопряжённого оператора и, в частности, известное неравенство Гайнца с уточнением Като и др.

Интересно, что близкая конструкция независимо и почти одновременно была предложена Кальдероном и Лионсом, что является ещё одним подтверждением существования объективных закономерностей развития науки. С этого момента интерполяция линейных операторов и теория шкал стали полем деятельности большой группы молодых (в то время) и сильных воронежских математиков. Но об этой деятельности следует писать отдельную статью. Укажу только, что соответствующие результаты излагались мною на IV Всесоюзном математическом съезде [8], а также в совместном с Ю.\,И.~Петуниным (аспирантом из Тамбова) докладе на международном конгрессе математиков в Москве в 1966 г.

Рассмотрение интерполяционных теорем привело нас к пересмотру наших позиций в функциональном анализе.
\linebreak
Процитирую отрывок из введения к нашей с Ю.\,И.~Петуниным обзорной статьи по теории шкал в журнале
«Успехи математических наук»:
«Непрерывно расширяющийся круг приложений функционального анализа приводит к систематическому пересмотру его методологических положений. Одно из них утверждает, что первичным и основным понятием функционального анализа является понятие пространства (нормированного, метрического, топологического и др.). Для исследования задачи нужно выбрать пространство и в нем изучать соответствующие функционалы, операторы и т.п.

Однако при рассмотрении сложных задач обычно приходится вводить множество различных пространств.

Многочисленные примеры таких задач даёт теория уравнений в частных производных. Для изучения гладкости решений вводят одни серии пространств, для изучения поведения вблизи границы области или вблизи каких-либо особых точек --- другие; изучение значений решений на многообразиях меньшей размерности проводится в новых пространствах.

Мы уж не говорим о том, что имеются ещё пространства коэффициентов,
пространства граничных значений, начальных значений и~т.\,п.
Таким образом, введение пространства, в котором исследуется задача, зачастую связано с теми су\-бъ\-е\-к\-ти\-в\-ны\-ми целями, которые ставит перед собой исследователь. Объективными данными являются, по-видимому, лишь те операторы, которые входят в уравнение задачи.

В силу изложенного, нам кажется, что понятие оператора является первичным и основным в функциональном анализе. Операторы обычно имеют конкретное аналитическое задание с помощью дифференциального выражения, интегрального оператора, суммы ряда и т.п. Аналогично тому, как функция имеет свою полную область определения, так и оператор имеет полный набор пространств, в которых его можно рассматривать. Весь этот набор является мало обозримым. Чаще всего выделяют семейства пар пространств, зависящих от одного или нескольких параметров, в которых действует данный оператор. Одним из важных методов выделения однопараметрических пар пространств являются интерполяционные теоремы» [9]

{\bf Теоремы о гомеоморфизмах}

Во время одной из конференций меня поселили в номере вместе с Ю.\,М.~Березанским, учёным из Киева. Он поделился со мной тем, что они с Я.\,А.~Ройтбергом из Чернигова исследуют, как ведут себя решения эллиптических уравнений при негладких правых частях, например, в правой части $\delta$ - функция. Первые результаты здесь были получены Гордингом. В беседе мы выяснили, что этот вопрос может быть решён с помощью интерполяции линейных операторов. В простейшем случае в условиях единственности самосопряжённый эллиптический оператор порядка $2m$ устанавливает гомеоморфизм между пространством Соболева $W_2^{2m}$ и $L_2$. Тогда сопряжённый к нему оператор, который совпадает с ним, осуществляет гомеоморфизм между сопряжёнными пространствами $L_2$ и $W_2^{-2m}$. Так как гильбертовы шкалы пространств обладают интерполяционным свойством, то из предыдущего следует гомеоморфизм между $W_2^{2m}$ и $W_2^{-(1-\alpha)2m}$. Замечу, что пространства с отрицательными индексами уже состоят из обобщённых функций. Это отправное соображение позволило Ю.\,М.~Березанскому и Я.\,А.~Ройтбергу получить большую серию теорем о гомеоморфизмах, осуществляемых эллиптическими операторами в различных условиях (большая часть которых получилась уже без интерполяции).

{\bf Советско-американский симпозиум}

В 1963 г. в Новосибирске проходил симпозиум,
на который приехали наиболее крупные специалисты по теории уравнений в частных производных из США и СССР.
Воронежская математика была представлена двумя докладами «О некоторых нелинейных задачах для уравнений в частных производных>>
(М.\,А.~Красносельский и П.\,Е.~Соболевский) и «Теорема о гомеоморфизмах и локальное повышение гладкости вплоть до границ решений эллиптических уравнений»
 (Ю.\,М.~Березанский, С.\,Г.~Крейн, Я.\,А.~Ройтберг).

{\bf Ещё раз о гидродинамике}

В течение ряда лет я все не решался подступиться к задаче гидродинамики вязкой жидкости при наличии свободной поверхности (жидкость в сосуде). В 1961 г. ко мне в аспирантуру поступил Н.\,К.~Аскеров, приехавший из Баку. Он хотел заниматься прикладными задачами, и я предложил исследовать указанную выше задачу. Трудность её решения состояла в том, что на свободное поверхности жидкости должно выполняться достаточно сложное граничное условие. Н.\,К.~Аскеров рассмотрел модельную задачу, где оператор Навье-Стокса был заменён на оператор Лапласа, а граничное условие значительно упрощено. На этой модельной задаче проявились некоторые особенности: её несамосопряженность, сведение её к граничной задаче со спектральным параметром как в уравнении, так и в граничном условии. Однако гидродинамическую задачу до отъезда Н.\,К.~Аскерова в Баку нам исследовать не удалось.

Приближался Всесоюзный съезд механиков. В связи с этим я снова занялся описанной выше задачей. Появились некоторые соображения, и за два дня до окончания срока подачи докладов на съезд я послал телеграмму с просьбой включить мой доклад «О колебаниях вязкой жидкости в сосуде» в программу съезда. Благодаря близкому знакомству с Н.\,Н.~Моисеевым, который был членом оргкомитета, доклад был включён в программу без представления тезисов. После этого я уже был вынужден вплотную заняться задачей и на съезде докладывал её достаточно полное решение. Задача о колебаниях сводилась к исследованию квадратичного операторного пучка. В то время многие авторы усиленно занимались операторными пучками, но тот который появился в задаче гидродинамики, не был исследован. Важным качественным результатом было то, что возможно лишь конечное число колебательных движений, а остальные затухающие (подсчитать бы конечное число движений в озере Байкал!). Развёрнутое изложение полученных результатов было опубликовано в наших с Н.\,К.~Аскеровым и Г.\,И.~Лаптевым статьях.

{\bf О других работах М.А.Красносельского}

Новый метод исследования сходимости рядов Фурье фу\-н\-к\-ци\-ям дифференциальных эллиптических операторов,
основанный на  использовании дробных степеней операторов и применимый к операторам любого порядка,
М.А.~Красносельский предложил и разработал совместно с Е.И.~Пустыльником.  Тот же метод применим для обоснования метода Фурье
решения эллиптических, гиперболических и параболических уравнений.

Ряд теорем существования решений нелинейных уравнений производных установлен М.\,А.~Красносельским и
\linebreak
М.\,П.~Семеновым.
Совместно с П.\,Е. Соболевским  была доказана
положительность функции Грина первой краевой задачи для эллиптических уравнений второго порядка.
\linebreak
М.\,А.~Красносельский, А.\,И.~Перов и П.\,Е.~Соболевский показали,
что известная для систем обыкновенных уравнений теорема Кне\-зе\-ра-Ху\-ку\-ха\-ра о связности множества решений при естественных предположениях переносится на решения задачи Коши для дифференциального уравнения в банаховом пространстве.

М.\,А.~Красносельский и И.\,Я.~Бакепьман (Ленинград) исследовали уравнения с дифференциальными операторами Монжа-Ампера и дополнительными сильными нелинейностями. Пользуясь методами нелинейного функционального анализа, они нашли новые условия разрешимости краевых задач для таких уравнений и указали число решений таких задач.

{\bf Приезд С.\,Д.~Эйдельмана}

В 1963 г. в Воронеж из Черновиц переехал С.\,Д.~Эйдельман, известный
специалист по теории параболических уравнений, и стал заведовать
кафедрой в политехническом институте. В 1964г. вышла в свет его монография «Параболические системы» [11], подводившая итог многолетним исследованиям. Вместе с Эйдельманом на кафедру ВПИ приехал ряд его учеников из Черновицкого университета. С другими учениками он продолжал работать на расстоянии. Вместе с С.\,Д.~Ивасишен (Черновцы) он исследовал матрицы Грина параболических граничных задач, как с гладкими, так и с разрывными коэффициентами. Полученные результаты докладывались на международном конгрессе в Москве в 1966г. С тем же соавтором было проведено исследование класса параболических систем, названных Эйдельманом «2В»-параболическими. Совместно с М.\,И.~Матийчуком (Черновцы) изучались фундаментальные
решения задачи Коши для параболических по Петровскому систем c коэффициентами, обладающими минимальной падкостью. С.\,Д.~Эйдельман и
В.\,Д.~Реппиков (ВПИ) установили необходимые и достаточные условия поточечной стабилизации решений задачи Коши в классах ограниченных функции для
модельный параболических уравнений. В 1965г. Репников защитил диссертацию. Эта тематика успешно рассматривается им и до сегодняшнего дня.

Вместе с Н.\,Д.~Житарашу (ВПИ) Эйдельман получил ряд глубоких результатов о нормальной разрешимости ряда задач сопряжений.
Житарашу защитил диссертацию в 1967 г.
Ф.\,У.~Порпер (ВПИ) изучал вопросы качественной теории параболических уравнений второго порядка с переменными коэффициентами.
Защитил диссертацию в 1965 г. Исследования по этой тематике продолжаются.
Ф.\,Г.~Селезнева (ВПИ) изучала начальную задачу нелинейных систем с частными производными и постоянными коэффициентами,
а также граничные задачи для корректных по Петровскому систем. Защитила диссертацию в 1968 г. С.\,Д.~Шмулевич (ВПИ) получил ряд результатов по спектральной теории эллиптических уравнений с растущими коэффициентами.

Следует отметить, что С.\,Д.~Эйдельман активно участвовал в работе семинара по уравнениям в частных производных. Он прочитал интересный спецкурс по современной теории эллиптических задач. Главная его заслуга перед коллективом воронежских математиков была в том, что он неоднократно нам показывал силу классических методов в теории уравнений в частных производных. Его отъезд в 1968 г. из Воронежа по семейным обстоятельствам прервал эту весьма полезную деятельность.

\smallskip \centerline{\bf Литература}\nopagebreak

1. {\it Красносельский М.\,А., Крейн С.\,Г., Рутицкий Я.\,Б, Соболев В.\,И.} О математической Жизни в Воронеже //Успехи математических наук. -1964. - Т. ХIХ - Вып.З. - С 225-245

2.	{\it Красносельский М.\,А., Крейн С.\,Г., Мышкис А.\,Д.} Расширенное Заседание Воронежского семинара по функциональному анализу в марте 1957г.// Успехи математических наук. - 1957.-Т ХII -Вып.4. -С.241-250.

3. {\it Красносельский М.\,А., Крейн С.\,Г.}
О дифференциальных уравнениях в банаховом пространстве // Труды III Всесоюзного математического съезда. - Т III – М.  Издание АН СССР 1958.-С.73-80.

4. {\it Красносельский М.\,А., Крейн С.\,Г.}
Об операторных уравнениях в функциональных пространствах // Труды IV Всесоюзного математического съезда. - М. 1964. - Т II - С 292-299.

5.	{\it Иосида К.} Функциональный анализ. М. Мир, 1967.

6.	{\it Красносельский М.\,А., Забрейко П.\,П., Пустыльник Е.\,И., Соболевский П.\,Г.}
Интегральные операторы в пространствах суммируемых функций - М. Наука, 1966 -С.499.

7. {\it Крейн С.\,Г.}
Дифференциальные уравнения в банаховом пространстве и их приложения в гидродинамике // Ус\-пе\-хи математических наук. -1957 - Т XII - Вып.1 С.208-211

8.	{\it Крейн С.\,Г.} Интерполяционные теоремы в теории операторов и теоремы вложения // Труды IV Всесоюзного математического съезда. - М.1964. - Т.П. - С.504-510

9.	{\it Крейн С.\,Г., Петунин Ю.\,И.} Шкалы банаховых пространств // Успехи математических наук,- 1966. -T.XXI – Вып. 1.-С.89-168.

10.	{\it Крейн С.\,Г.} Линейные дифференциальные уравнения в банаховом пространстве М. Наука, 1967.

11.	{\it Эйдельман С.\,Д.} Параболические системы - М: Наука, 1964,-443с.

12. Воспоминания о Крейне — Воронеж: ВорГУ, 2002. — 104 с.

13.	 Материалы к истории математического факультета ВГУ. Сборник  Воронеж:
Воронежский университет, 1998. 118 с.: ил.

{\bf Воронежская зимняя математическая школа. Доклад на открытии 20-й школы.}

{\bf Предыстория.}

В 1966 г. в Москве проходил Международный конгресс математиков. Наряду с яркими впечатлениями от многих докладов, бесед с известными зарубежными математиками (Филлипс, Иосида, Комацу, Маженес, Кальдерон и др.) у меня осталось чувство неудовлетворения. Мне показалось, что у нас имеется по ряду направлений отставание от современного на тот момент уровня. В недавно вышедшей в «Итогах науки» статье В.\,И.~Арнольд цитирует Пуанкаре: «Часто достаточно изобрести слово, и это слово становится творцом». Конечно, под словом Пуанкаре понимал определение. Вот у меня и создалось впечатление, что мы не знали многих новых слов.

Осенью 1966 г. мне рассказал Б.\,С.~Митягин, что в Горьком летом проходила школа по топологии, в которой участвовало 15 сильных молодых математиков. В школе читались лекции и проводились практические занятия. Меня очень захватила идея проведения в Воронеже математической школы.
Нам показалось естественным проводить её во время зимних каникул в вузах. Мы надеялись ещё на то, что зимой легче доставать путёвки в дома отдыха и турбазы.

В ноябре проходила университетская партийная конференция. Выступая на ней, как секретарь партийной организации факультета, я просил помочь нам в организации зимней школы. Присутствовавший на конференции секретарь горкома партии поддержал наше предложение и обещал содействие горкома.

На следующее утро мы с моим заместителем С.\,А.~Скля\-дневым были в кабинете секретаря горкома.
Он позвонил в Облпрофсовет и просил выделить нам путёвки в дом отдыха им. Горького.
Однако, ему ответили, что все путёвки на январь уже розданы в организации. Когда мы пошли в Облпрофсовет, то два молодых человека, имевших отношение к распределению путёвок, подтвердили невозможность нашего мероприятия. Несмотря на это, мы с Сергеем Анатольевичем решили поговорить с тогдашним председателем Облпрофсовета Василием Аносовичем Фетисовым. Нас отговаривали, говоря, что он человек жёсткий и своевольный. В разговоре Фетисов сначала не понял цель нашего мероприятия, считая, что оно связано со средней школой, но когда он
узнал, что там будут учить друг друга учёные-математики, то пришёл в неописуемый восторг. Немедленно были вызваны два молодых человека, о которых речь шла выше, им было предложено отобрать путёвки в д.о. им. Горького от других организаций и передать в университет. Таким образом, с благословения Фетисова, в 1967 г. школа родилась.

{\bf Первая школа}

Дом отдыха стоял па берегу реки Воронеж, а лекции проходили в лесотехническом институте, который расположен на горе. Возвращаясь с лекции, участники спускались с горы на различных точках. Приехало в школу 109 чел. из 22 городов Союза, в том числе 53 воронежца. Неожиданным был мороз в 35-36 градусов в первые дни работы школы. От него в первую очередь полегли с простудами южане. Через три дня после открытия школы, в субботу, заболело 14 человек. Я думал, что школу придётся закрыть, но с воскресенье все начали поправляться и к концу работы вся школа была в полном здравии.

Следовало решить вопрос о методике проведения школы. Мы считали и считаем сейчас, что в школе, в отличие от конференций, должны читаться циклы лекций, освещающие состояние различных областей математики. Как правило, выбирались области, с которыми были мало знакомы воронежские математики.
В первой школе работал лишь один учебный семинар по топологии. В связи с ним в стенгазете школы была помещена детская песенка:

«Топ-топ ножками, топ-топ ручками -

Учат топологию дедушки и внучки.

Ton-топ ножками, топ-топ ручками —

Знают в топологии только сферу с ручками».

В дальнейших школах уже работало от пяти до восьми семинаров. При чтении лекций в специфической аудитории, где сидят слушатели с различной подготовкой, некоторые из которых являются специалистами по освещаемым вопросам, имелась возможность возникновения дискуссий между лектором и этими специалистами. При этом большая часть аудитории ничего не понимает и отключается от слушания. В связи с этим было принято решение, что вопросы разрешается задавать только одному участнику школы --- Е.\,А.~Горину.
В первой школе были прочитаны следующие циклы лекций: топология функциональных пространств и вариационное исчисление в целом --- С.\,И.~Апьбер; кольца операторов --- Г.\,И.~Кац: многомерные дифференциальные уравнения --- А.\,И.~Перов.

{\bf Вторая школа}

Эту школу я открывал в предынфарктном состоянии и в тот же день у меня случился микроинфаркт.
В связи с этим, я могу лишь сказать, что она проходила в том же месте. Среди лекторов были Б.\,Я.~Левин, Ю.\,И.~Любич, В.\,И.~Мацаев, С.\,В.~Фомин.

{\bf Третья школа}

В школе приняло участие 170 чел. из 26 городов Союза, в том числе воронежцев 47. Мне кажется, что третья школа была в определённом смыслу переломной. Если в первых двух школах лекторами были математики, близко связанные с воронежцами личной дружбой или тематикой работ, то большинство лекторов третьей школы были нам совсем незнакомы.

Здесь хотелось бы в первую очередь отметить две лекции С.\,П.~Новикова на тему «Гладкие многообразия, К-теория, эллиптические операторы», которые для нас были ценными не столько по фактическому материалу, сколько по новому для нас мировоззрению, которое Новиков со страстью пропагандировал. (Эффект второй лекции был несколько снижен предшествующим ей банкетом).

Большой интерес вызвали циклы лекций,
прочитанные Б.\,В.~Шабатом, С.\,Г.~Гиндикиным,
Г.\,М.~Хенкиным и
\linebreak
Е.\,М.~Чиркой по теории функций многих комплексных переменных.
В этой области у нас никто ничего не знал.
Г.\,И.~Эскин прочёл лекции по псевдодифференциальным операторам. Был заслушан часовой доклад Г.\,А.~Маргулиса (лауреат Филдсовской премии).

Четвёртая школа проходила так же, как и третья, в д.о. им. Дзержинского.
В ней участвовало 195 чел. из 22 городов, в том числе 60 воронежцев.
Были прочитаны циклы лекций:
«Метод орбит в теории групп Ли»~--- А.\,А.~Кириллов;
«Аксиоматическая теория поля»~--- Ю.\,М.~Березанский;
«Адиабатическая и физическая матрица рассеяния квантовой теории поля»~--- А.\,С.~Шварц;
«Бесконечномерные многообразия» --- Ю.\,Л.~Далецкий;
«Топология банаховых пространств» --- Б.\,С.~Митягин, А.\,С.~Дынин.

Была проведена дискуссия по проблеме образования на математических факультетах университетов.
В ней участвовали Ю.\,М.~Березанский, Ю.\,Л.~Далецкий,
Е.\,А.~Горин, В.\,А.~Ефремович, Б.\,В.~Шабат, В.\,П.~Хавин, Ю.\,И.~Любич, С.\,Г.~Крейн.

{\bf Совещание в Минвузе РСФСР}

Как-то Минвуз РСФСР собрал 27 профессоров для обсуждения проблемы развития науки в вузах. Я посвятил часть своего выступления работе Воронежской зимней математической школы. В заключение министр Столетов горячо одобрил нашу деятельность. На следующий день я передал ему проект организации 1-й Всероссийской математической школы на базе Воронежского университета. Одним из пунктов было выделение 0,5 ставки старшего лаборанта для работы в школе. Референт мне несколько раз говорил, что проект лежит на столе министра. Но, по-видимому, он потом попал в корзину. Вскоре я понял, что моё импульсивное действие, направленное на поднятие авторитета Воронежского университета, могло принести вред школе. Действительно, если бы школа была официальным учреждением при Минвузе, то в неё посылали бы, как на ФПК, в обязательном порядке, математиков, не интересующихся тематикой школы. Сила нашей школы в том, что она нигде не числится (правда, сейчас отчёт о работе школы передаётся в Региональный совет). Всякий покупающий путёвку уже на правах туриста может участвовать в работе школы. Ниже это будет проиллюстрировано на примере.

{\bf Шестая школа}

Снова угроза нависла над школой. Дом отдыха им. Горького стал санаторием, дом отдыха им. Дзержинского --- пансионатом. Кто-то из начальства мне сказал сакраментальную фразу: «Наши путёвки предназначены для трудящихся». Мы уже были в отчаянии. Неожиданно две наши девушки Таня Гареева и Надя Лаптева прочитали объявление о том, что турбаза «Коммунальник» приглашает туристов для краткого зимнего отдыха. Ю.\,Г.~Борисович сразу отправился по указанному адресу и довольно быстро договорился о том, что нас принимают на турбазу. Турбаза находилась в живописном месте, но состояла она из двухкомнатных деревянных домиков без фундаментов и без удобств. Как назло, в каждом домике висел термометр. Занятия можно было проводить лишь на веранде и в столовой. Приглашая в школу Б.\,Я.~Левина, я описал ему состояние базы. Он приехал и сказал, что приехал только потому, что я всем стал бы говорить, что Левин испугался.

Одним из лекторов был Ю.\,И.~Манин. Он подошёл ко мне и сказал: «Я все понимаю, но - 8 градусов!». Немедленно он был переселён в другой домик, где было 13, и остался доволен.

Оргкомитет закупил электроплитки, которые спасали от холода. Кроме того, Ю.\,Г.~Борисович каждую ночь будил истопника и заставлял его топить (естественно, за дополнительное поощрение).

В школе было 115 участников из 19 городов, включая 43 воронежца.
Были прочитаны циклы лекций: «Внутренние гомологии и формальные группы»~--- В.\,М.~Бухштабер, А.\,С.~Мищенко;
«Теория аналитических $J$-растягивающих матриц» --- И.\,В.~Ковалишина и В.\,П.~Потапов;
«Алгебраическая геометрия» Ю.\,И.~Манин; «Функции от некоммутирующих операторов и их применение» --- В.\,П.~Маслов; «Дифференциальные уравнения и группы Ли» --- А.\,Л.~Онищик. Лекцию на тему «Топология алгебраических многообразий» прочитал В.\,И.~Арнольд.

Состоялась дискуссия в связи с сообщением В.\,П.~Маслова и В.\,В.~Грушина
о создании в институте электронного машиностроения группы с углублённой математической подготовкой.
В частности, В.\,В.~Грушин читает теорию дифференцируемых функций как предельную для теории многочленов. С критикой этого метода выступили В.\,И.~Арнольд, Ю.\,И.~Любич и другие. Я занял примирительную позицию, считая, что никакой метод преподавания (квалифицированным лектором) не может испортить группу сильных студентов.
Несмотря на бытовые трудности, а, может быть, и благодаря им, у всех участников остались самые тёплые воспоминания о шестой школе как о самой «романтической».

{\bf Седьмая школа}

Здесь я хочу рассказать лишь об одном эпизоде.
Председателем оргкомитета был С.\,А.~Скляднев.
Встречаясь со мной до начала работы школы, он говорил,
что ректор
\linebreak
Н.\,А.~Плаксенко разрешил пригласить воронежских профессоров, доцентов, ассистентов.
Я делал вид, что не понимаю, почему в этом перечне нет старших преподавателей. А дело было в том, что ректор плохо относился к одному выпускнику ВГУ, который работал старшим преподавателем. Когда раздавали путёвки в дом отдыха «Углянец», этому математику отказали. Воронежская молодёжь очень этим возмущалась и готовилась к бунту во время открытия школы, что могло привести к её ликвидации. Выход был найден простой. Я пошёл к директору дома отдыха и попросил продать ещё одну путёвку, что он охотно сделал (план нужно выполнять). Опальный старший преподаватель стал просто отдыхающим и таким образом смог слушать лекции, активно участвовать в работе семинаров.

Надо сказать и об оргкомитете школы. Пока я работал в ВГУ, в оргкомитет входили: Галина Григорьевна Трофимова и я. Ректорат ВГУ писал лишь одну бумажку с просьбой выделить определённое количество путёвок. После моего ухода между зав. кафедрами начались трения из-за желания руководить школой. Тогда ввели очерёдность. Состав оргкомитета назначался ректоратом. Каждый из руководителей вносил свой вклад в систему проведения школы. Так В.\,П.~Глушко и П.\,Е.~Соболевский «открыли» турбазу «Берёзка», которая стала постоянным местом работы школы. Это позволило увеличить число участников до 260-280 человек. Б.\,Н.~Садовский ввёл очень чёткую систему подготовки работы школы. По программе мне ближе школы, руководителем которых был Ю.\,Г.~Борисович.

{\bf О значении школы}

Молодые математики очень легко воспринимают новые идеи и смело начинают их использовать.
Мы тоже, по мере сил пытались «задрав штаны бежать за комсомолом». Расскажу только о влиянии школы на моих учеников.
Лекции Г.\,И.~Каца повлияли на творчество В.\,И.~Овчинникова и его диссертацию.
Прослушав лекции А.\,И.~Перова, мы поняли, что он занимается теорией дифференциальных уравнений на коммутативной группе Ли,
и развили эту теорию для некоммутативной группы.
По этой тематике защитили диссертации А.\,М.~Шихватов. Ю.\,С.~Сысоев, А.\,И.~Фурменко. И.\,З.~Песенсон (с ним я познакомился в школе).
Лекции по теории функций многих комплексных переменных оказали влияние на М.\,Г.~Зайденберга, П.\,А.~Кучмента, А.\,А.~Панкова
(об этом говорит наш совместный обзор в журнале «Успехи математических наук»).
В диссертации П.\,А.~Кучмента широко использовались знания, полученные на лекциях Б.\,С.~Ми\-тя\-ги\-на и А.\,С.~Дынина. Диссертация Н.\,И.~Яцкина была связана с лекциями А.\,Л.~Онищика. Позднее мы опубликовали небольшую монографию «Линейные дифференциальные уравнения на многообразиях».

В школе возникали творческие содружества.
М.\,Г.~Зай\-ден\-берг и В.\,Я.~Лин вместе работали над проблемами алгебраической геометрии, В.\,А.~Кондратьев и С.\,Д.~Эйдельман - над положительными решениями эллиптических задач. Под влиянием лекций Р.\,Л.~Волевича были выполнены последние работы Я.\,А.~Ройтберга по теории гиперболических уравнений.

Трудно переоценить значение математических разговоров и дискуссий между участниками школы.
Иногда до поз\-д\-ней ночи работали небольшие внеплановые «самодеятельные» семинары.

Характерной чертой воронежской школы является доброжелательность, взаимопомощь, отсутствие споров за приоритеты. В школе нет выборов, нет премий, нет демократии, поэтому работа проходит в спокойной, творческой обстановке.

Не забыли и об отдыхе участников. Одной из его главных форм были, конечно, лыжные прогулки. Хотя находились и другие занятия. Так, В.\,И.~Арнольд «заразил» школьников игрой в нарды, в которую они играли с большим воодушевлением. Традиционно 28 января проводился праздник, посвящённый дню рождения неизменного участника всех школ Евгения Алексеевича Горина.

Замечательным явлением в жизни школы были вечерние лекции А.\,Я.~Хелемского по истории различных стран и народов мира.
Нас поражали его обширные знания дат, имён исторических деятелей, их произведений и взглядов,
взаимоотношений, войн и т.\,п.
Запомнился также доклад А.\,Т.~Фоменко о методике датирования исторических событий, вызвавший бурную дискуссию.
Во время одной из школ была организована выставка картин А.\,Т.~Фоменко,
а Лаптев рассказывал о своей поездке в Англию. Впрочем, культурных мероприятий было так много, что трудно все перечислить.

Открывая 20-ю школу, хочу выразить уверенность, что Воронежская зимняя математическая школа будет существовать ещё многие годы.

\begin{flushright}
{\it 1987 г.}
\end{flushright}

{\bf Организация института математики}

В 1966-67 годах на факультете создалась неустойчивая обстановка. Дело в том, что М.А. Красносельский с группой своих активно работающих учеников перешёл на работу в Институт автоматики и телемеханики АН СССР, оставаясь жить в Воронеже. Возникла идея о создании при ВГУ научно-исследовательского института с тем, чтобы вернуть эту группу математиков в ВГУ.

Когда меня уговаривали стать деканом, то в качестве одного из условий я
поставил оказание ректоратом помощи в организации института. Такая помощь была обещана. Однако, как нужно было начинать это мероприятие, было совершенно неясно. Дело в том, что организация института не значилась ни в каких планах нашего планового хозяйства. Кого-то надо было убедить в необходимости создания института. Здесь нам повезло. В университет приехал заместитель министра Высшего и Среднего специального образования РСФСР Алексей Иванович Попов. Ректор ВГУ проф. В.\,П.~Мелешко вместе с ним зашли ко мне на кафедру. Я показал ему статью академика Колмогорова в газете «Известия», в которой тот очень лестно отозвался о воронежских математиках. На Попова эта статья произвела впечатление. Когда я ему изложил идею о создании института, он её поддержал. «Приезжай ко мне в понедельник в 9 ч. 30 мин. и мы начнём действовать». В понедельник я был у него, и мы узнали, с какого конца начинать действовать.

Первым делом было решение Учёного совета ВГУ. Я его написал, и ректорат, идя навстречу, провёл это решение без заседания, путём опроса членов совета. Затем требовалось ходатайство Обкома партии. Здесь большую помощь оказал секретарь по пропаганде Вячеслав Павлович Усачев (впоследствии --- зам. министра), который всегда хорошо относился к воронежским математикам. Ходатайство было быстро подписано первым секретарём.

Следующим этапом было получение поддержки различных отделов Минвуза. С активной помощью А.\,И.~Попова это было быстро сделано.

Неожиданно потребовалась поддержка Академии Наук СССР. Председателем бюро Отделения физ. мат. наук АН СССР был академик Н.\,Н.~Боголюбов, у которого я обучался в аспирантуре. Соответствующее письмо я подготовил и нашёл Боголюбова на защите диссертации на физфаке МГУ. Он тут же подписал письмо и просил меня, только для соблюдения коллегиальности, согласовать его с заместителем С.\,Н.~Мергеляном.

В министерстве были поражены тем, что Академия Наук за один день решила вопрос о поддержке коллектива воронежских математиков. Через Совмин РСФСР и Минвуз СССР наш вопрос прошёл без моего участия. Здесь действовал А.\,И.~Попов самостоятельно.

Наиболее трудной инстанцией был Государственный Комитет по науке и технике при Совмине СССР.
Нашим куратором стал Иван Сергеевич Герасимов, кандидат физ.-мат. наук, механик. Мы с ним нашли ряд общих знакомых механиков и разговор шёл в доброжелательной форме. Оказалось, что для дальнейшего необходимо подробное обоснование необходимости института и перспективный план его развития. Стало ясно, что составление такого документа является для Ивана Сергеевича, впрочем как и для любого постороннего человека, трудной задачей, для выполнения которой нужны месяцы.

В связи с этим я предложил, что сам составлю этот документ. Предложение было принято, мне выделили стол, за которым я работал, и в течение двух
дней перспективный план развития института по 16-ти научным направлениям был составлен. Этот документ был подписан, и я с ним направился к зав. отделом Е.\,И.~Склярову. Этого властного человека с аристократическими манерами сотрудники, как мне показалось, сильно побаивались. Несмотря на это, удалось уговорить его послать в Воронеж комиссию для ознакомления с ситуацией на месте.

Комиссию возглавила зам. начальника Управления университетов Инесса Сергеевна Дубинина, членами комиссии были И.\,С.~Герасимов и ещё один сотрудник Комитета по науке и технике. Главным вопросом, который они выясняли, было наличие площадей для размещения института. Им был вручён документ за подписью ректора, в котором указывался ряд комнат и аудиторий, где будет размещён институт. Естественно, что в дальнейшем этот план не был выполнен.

Во время работы комиссии произошло событие, которое могло поставить под удар открытие института.
Руководство механиков, поддержанное ректором В.\,П.~Мелешко,
у которого к тому времени резко ухудшились отношения с математиками,
выдвинуло предложение о включении слова «механика» в название института.
Я выступил резко против такого изменения.
Решила вопрос И.\,С.~Дубинина, которая сказала, что если ВГУ будет настаивать на создании института с другим названием,
то все ходатайство и документацию нужно начинать сначала.

Атака механиков (двоих из которых затем изгнали из ВГУ) была отбита.
Следует заметить, что отдел механики сплошной среды планировался в проекте перспективного плана.
Комиссия встретилась с секретарём обкома В.\,П.~Усачевым и после его энергичной поддержи приняла положительное решение.

Замечу, что А.\,И.~Попов встретился в президиуме какого-то торжественного собрания
с председателем Комитета по науке академиком В.\,А.~Кириллиным и попросил его помочь в организации института.

\ldots Ноябрь 1967 года. Заседание Комитета по науке.
В течение двух с лишним часов обсуждается важнейший вопрос о применении ингибиторов для борьбы с коррозией металла.
После перерыва --- наш вопрос. Ректор В.\,П.~Мелешко просит меня представить его.
После моей краткой информации В.\,А.~Кириллин сказал, что ему известны достижения коллектива воронежских математиков.
Институт, по его мнению, нужен для того, чтобы разгрузить их от учебной работы.
Академик Гвишиани высказал пожелание~--- чтобы институт доводил свои разработки до постановки решений задач на ЭВМ.
Я сказал, что у нас есть вычислительный центр, с помощью которого это будет делаться.
Без каких-либо замечаний было принято положительное решение.
И.\,С.~Герасимов поздравил меня с открытием института. Приведу документ, давший основание для рождения института:


\begin{center}
{\it ВЫПИСКА\\
из протокола заседания коллегии\\
Государственного комитета СССР по науке и технике\\
от 30 ноября 1967 г. N. 74}
\end{center}

IV. 06 организации научно-исследовательского института математики в Воронежском государственном университете.

Принять предложение Министерства высшего и среднего специально образования РСФСР и Воронежского обкома КПСС, согласованное с Министерством высшего и среднего специального образования СССР и Академией наук СССР, об организации научно-исследовательского института математики при Воронежском государственном университете и об установлении следующих основных направлений деятельности этого института:
\begin{itemize}
\item создание качественной теории дифференциальных \linebreak уравнений в функциональных пространствах:

\item разработка асимптотических методов решения нелинейных задач с приложением к теории колебаний и теории автоматического регулирования;

\item развитие теории функциональных пространств и операторов, действующих в них;

\item исследование нетрадиционных задач для уравнений в \linebreak частных производных:

\item разработка конечно-разностных методов решения \linebreak дифференциальных уравнений, наиболее пригод\-ных
\linebreak для применения электронных\,вычислительных\,машин;

\item развитие вероятностно-статистических методов решения задач кибернетики с применением их к задачам экономики и проблемам физиологии.
Организацию \linebreak указанного института провести в пределах фонда заработной платы и ассигнований, установленных для научно-исследовательских учреждений Министерства высшего и среднего специального образования РСФСР на
1968 г.
\end{itemize}
ПРЕДСЕДАТЕЛЬ КОЛЛЕГИИ\;\;\; 	{\it /В.КИРИЛЛИН/}\\

Как видно из последнего пункта, вопрос о финансировании института оставался пока открытым. Некоторые чиновники из Минвуза РСФСР хотели начать финансирование с 1969 г. однако снова пришёл на помощь А.\,И.~Попов, и с 1968. г институт получил бюджет, составивший 60 тыс. рублей в год.

Задача  на  будущее.

Последним этапом явилось присвоение институту категории. Этот вопрос решала зам. министра финансов РСФСР М.\,Л.~Рябова. Я к этому времени заболел и хлопоты принял на себя Е.\,М.~Семенов. Несмотря на неоднократные визиты к М.\,Л.~Рябовой, ему не удалось сломить её сопротивление. Институт получил лишь третью категорию, а это противоречило тому, что НИС университета имел первую категорию. (Замечу, что категорию до сих пор так и не удалось повысить).

Надо сказать, что институт был создан благодаря инициативе и настойчивости нескольких лиц. В отличие от нынешних времён, тогда инициатива поощрялась. На меня \linebreak произвело большое впечатление, как Алексей Иванович Попов, бывший тогда партийным работником с большим стажем, настойчиво «пробивал» наш вопрос, преодолевая все трудности на пути к цели. Безусловно, его надо считать одним из основных создателей института.

Первым\,директором\,института\,математики\,при\,ВГУ\,стал \linebreak профессор В.\,И.~Соболев. Впоследствии институтом математики руководили С.\,А.~Скяднев, В.\,П.~Трофимов, Ю.\,В.~Покорный, В.\,Г.~Звягин.
