\vzmstitle{НЕСКОЛЬКО СЛОВ О НИКОЛАЕ АНТОНОВИЧЕ БОБЫЛЕВЕ}
\vzmsauthor{Обуховский}{В.\,В.}
\vzmsinfo{Воронеж; {\it valerio-ob2000@mail.ru} }
\vzmscaption

В ушедшем 2017 году мы, те кому посчастливилось знать Н.А. Бобылева и дружить с ним, отметили две памятные даты. 28 октября исполнилось 70 лет с его дня рождения, а 17 декабря – 15 лет со дня его трагической гибели.

	… Несколько лет назад известная в городе Гимназия им. Басова (а для нас, одних из первых её выпускников, все та же, ставшая такой близкой пятьдесят восьмая) праздновала какой-то юбилей, на который пригласили и нас. Со сцены звучало много хороших слов о выпускниках математических классов. Один из них был (вполне справедливо) назван гордостью науки в Австралии, о другом (тоже совершенно заслуженно) было сказано как о выдающемся математике Германии, но, увы, никто не вспомнил о человеке, который за всю свою блестящую научную карьеру за границей ни разу, по-моему, толком и не был.

	Расцвет научной деятельности Н.А. Бобылева пришёлся на девяностые годы прошлого века, очень непростое время в истории нашей страны. С одной стороны это было время больших ожиданий, а с другой - очень часто казалось, что наука в России больше никому не нужна, так же, как и люди, которые, не найдя себе лучшего применения, ею все ещё занимались. Я помню, что в те времена Институт проблем управления РАН, где работал заведующим лабораторией Николай Антонович, представлял собой довольно странное зрелище – весь первый этаж являл почти сплошное торжище – на сданных в аренду площадях теснились какие-то лотки, киоски и т.д. Сам Коля часто ворчал, что некоторые его сотрудники даже за зарплатой не приходят, найдя себе, очевидно, более выгодный род занятий.

	Но сам он не собирался изменять своему призванию.  Его переполняли идеи настолько, что достаточно было поговорить с ним «о науке» минут двадцать, и ты снова обретал пошатнувшуюся было уверенность в том, что математика все ещё необходима человечеству.

	Подтверждением моих слов может служить хотя бы список монографий, опубликованных Николаем Антоновичем в эти годы. Он включает в себя следующие книги:

«Методы нелинейного анализа в задачах негладкой оптимизации» (М., 1992);

«Approximation procedures in nonlinear oscillation theory» (Berlin, 1993);

«Топологические методы в вариационных задачах» (М., 1997); «Геометрические методы в вариационных задачах» (М., 1998);

«Гомотопии экстремальных задач» (М., 2001);

«Методы нелинейного анализа в задачах управления и оптимизации» (М., 2002).

	Думаю, что за прошедшие пятнадцать лет этот список очень существенно бы расширился. У него было много ярких, блестящих идей, планов, которыми он щедро делился с друзьями. Но, увы, всем  этим мечтам не суждено было сбыться из-за нелепого случая… Поздним декабрьским вечером 2002 года он, засидевшись на работе, возвращался домой (по всегдашнему обыкновению пешком, с рюкзаком за плечами) и при переходе улицы в районе метро «Беляево» его сбила мчавшаяся куда-то по важным, наверное, делам машина.

Огромное количество раз за эти годы меня охватывало одно и то же чувство – эх, как хорошо было бы сейчас посоветоваться с Колей или рассказать ему пришедшую в голову идею - за которым всегда следовало горькое ощущение утраты. Утешением служило и служит лишь одно – осталась память о замечательном человеке и прекрасном друге, которая и сейчас помогает жить и надеяться.
