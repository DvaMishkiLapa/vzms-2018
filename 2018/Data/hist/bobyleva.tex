\begin{center}{ \bf  ВОСПОМИНАНИЯ ОБ ОТЦЕ}\\
{\it О.\,Н. Бобылева} \\
(МГУ, Москва; {\it o\_bobyleva@mail.ru } )
\end{center}
\addcontentsline{toc}{section}{Бобылева О.\,Н.}

Когда мы прощались с папой, сотрудник, проработавший с ним всю жизнь в одной комнате, сказал мне “Вы, наверное, его и не видели совсем. Николай Антонович проводил на работе не только понедельник-пятница до ночи, а ещё и выходные иногда. Как же вы по нему не скучали?”

А мы с мамой скучали. Очень. Но его хватало всем, тем более семье. А так же друзьям, которые могли считать себя абсолютно близкими папе людьми. Ученикам, в которых он вкладывал душу, выходя далеко за рамки необходимого для их работ. Питомцы отца не подвели – его дело продолжается в их научных результатах.

Приходя с работы, папа ужинал на скорую руку, а затем начинались телефонные звонки. Звонили те, кому я с извинениями сказала чуть раньше: “Его ещё нет дома, перезвоните позже, пожалуйста”. Краткий отдых наступал только когда все рабочие вопросы были решены. Отец обладал потрясающим кругом интересов, но с его графиком времени вечерами хватало только на чтение. Огромная папина библиотека до сих пор жива: классика в томах, современная проза, букинистические издания. В нашем стареньком, его руками сделанном книжном шкафу, умещалась тогда вся мировая история, география, мир приключений, естествознание, мемуары, и все, чем ещё может интересоваться человек. Но для множества математических книг был выделен отдельный шкаф. Потрясающе эрудированный во многих областях, отец, тем не менее, всегда был человеком, в жизни которого главное место занимала наука. Математика. В детстве я любила сидеть рядом и смотреть, как папа работает. Как на чистом листе появляются чётким папиным почерком выведенные неведомые мне греческие буквы и символы. В то время это казалось мне недосягаемым искусством. Письменный стол и тишина, это все, что нужно было ему для работы.

Дело его жизни не отпускало отца даже на отдыхе. А отдых он любил самый простой: подмосковье, рыбалку, лес. Папа обычно говорил, что идёт больше “погулять по лесу” - он очень любил природу средней полосы и пешие прогулки, но корзиночку грибов все равно всегда приносил и сам их отменно жарил. Ко всем увлечениям он относился с полной отдачей. На рыбалке несколько часов мог ждать “ту самую” поклёвку, и в пять утра профессионально “подсекал” огромного леща. Отец обладал свойством заражать всех вокруг тем, что ему интересно, и вся молодёжь на нашей турбазе болела рыбалкой, а те, кто ещё не дорос, выстраивались в очередь за выточенными из досок деревянными ножиками.

Папы нет уже 15 лет. О нем помнят его друзья, сотрудники, ученики. Для моих друзей он остался в доброй памяти «Дядей Колей»…
