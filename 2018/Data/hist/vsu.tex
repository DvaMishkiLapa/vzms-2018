\vzmstitle{О ЮРЬЕВСКО-ВОРОНЕЖСКОМ ОЧАГЕ ВЫСШЕГО ПРОСВЕЩЕНИЯ}
\vzmsauthor{Костин}{В.\,А.}
\vzmsauthor{Сапронов}{Ю.\,И.}
\vzmsauthor{Удоденко}{Н.\,Н.}
\vzmsauthor{Бочаров}{В.\,Л.}
\vzmsauthor{Кадменский}{С.\,Г.}
\vzmsauthor{Негробов}{О.\,П.}
\vzmsauthor{Овчинников}{В.\,И.}
\vzmsauthor{Селеменев}{В.\,Ф.}

\vzmscaption

\begin{flushright}
«Принимая во внимание государственную

необходимость сохранить для России Юрьевский

Университет, как очаг высшего просвещения, \ldots

\ldots Остановиться на г. Воронеже, как месте, где в

случае необходимости, открыть деятельность университета».

{\it Выписка из журнала заседания Совета

Юрьевского Университета от 20(7) февраля 1918 г.}
\end{flushright}
В этом году наш Воронежский государственный университет отмечает свой вековой юбилей.
Однако, мы не знаем точной даты, с которой ведётся отсчёт.
До недавнего времени это было 18 мая 1918~г.~---
дата подписания В.\,И.~Лениным Постановления Большой государственной комиссии по просвещению
о переводе Юрьевского Университета в Воронеж.
Однако, теперь эта дата почему-то не связывается с учреждением Воронежского государственного университета.
Также без объяснений меняются и гербы университета с обозначенными на них датами.

Все это говорит о необходимости изучения истории нашего университета и, в частности, истории математического факультета.

В	то же время, не только нам, но и властям нашего города нелишне задуматься о том, как и почему Воронеж получил столь высокий статус университетского в кошмарных условиях Мировой и Гражданской войн.

В	своей монографии «Воронежский университет. Вехи истории 1918—2003» профессор М.\,Д.~Карпачев, обращаясь к истокам университета в Воронеже, справедливо утверждает, что долгое время вопрос об утверждении университета трактовался весьма упрощённо, являясь плодом только волевых усилий Советского правительства, что, по его мнению, иногда сознательно, а чаще по установившейся традиции соответствовало лишь части правды. Устраняя этот недостаток, он, со своей стороны, указывает на определяющую роль общественных организаций города Воронежа и губернии в подготовке условий для учреждения университета в нашем городе. В связи с	этим М.\,Д.~Карпачев достаточно подробно анализирует действия этих организаций, начиная с шестидесятых годов XIX века до 21 фев-раля 1918 г.~--- даты роспуска земских учреждений большевиками.

Безусловно, желания и старания прогрессивных слоёв нашего города способствовали успеху в учреждении университета в Воронеже, но они не могли быть решающими. К тому же их деятельность протекала в условиях относительно спокойного хода событий в стране, когда проблемы учреждения высших учебных заведений касались только центральных и региональных властей, и правительство царской России даже планировало открыть десять новых университетов в связи с нехваткой специалистов со среднетехническим и общеуниверситетским образованием.

Но к началу 1918 г. ситуация резко и качественно изменилась, приобретая турбулентный характер, когда каждый день вносил свои жёсткие коррективы. Власть в стране перешла к Советам большевиков, а затем появились и внешние угрозы, связанные с оккупацией Лифляндии и Эстляндии немецкими войсками. Это сразу приводит к появлению новой стороны, связанной с учреждением университета в Воронеже --- Юрьевского Императорского Университета во главе с ректором, доктором чистой математики, профессором В. Г. Алексеевым.

В	результате образовался четырёхугольник со сторонами, заинтересованными в судьбе Юрьевского,
а следовате\-льно, и Воронежского университетов:

\begin{itemize}
  \item немецкая оккупационная власть в Лифляндии и Эстляндии;
  \item советская власть в России;
  \item власти и общественные организации Воронежа;
  \item Юрьевский Университет.
\end{itemize}

Уникальность такого расклада состоит в том, что отсутствие любой из этих сторон приводило бы к отрицательному результату для Воронежа:
\begin{itemize}
  \item без немецкой угрозы снималась потребность юрьевцев в эвакуации, как это происходило в 1915 и 1917 гг.;
  \item отказ Советского правительства от решения университетских проблем, также приводил к отрицательному результату;
  \item без готовности воронежских властей принять университет, вопрос просто не имел бы смысла;
  \item наконец, в случае нежелания Юрьевского Университета переезжать в Воронеж ответ также был бы очевиден.
\end{itemize}

Ситуация осложнилась ещё и тем, что некоторая часть русского общества приветствовала наступление немцев,
поскольку отрицательно относилась к власти большевиков.
Вот как это описывает Иван Бунин: «В газетах --- о начавшемся наступлении немцев. Все говорят: «Ах, если бы!» … Вчера были у Б. ... Собралось порядочно народу --- и все в один голос: немцы, слава Богу, продвигаются, взяли Смоленск и Бологое…»

А	вот что Михаил Пришвин записал в дневнике 19 февраля 1918 г. о разговорах на Невском проспекте: «Сегодня о немцах говорят, что в Петроград немцы придут скоро, недели через две. Попик, не скрывая, радостно говорит: «Ещё до весны кончится». Ему отвечают: «Конечно, до весны нужно: а то и землю не обсеменят, последнее зерно выбирают». Слабо возражают: «Думаете, немцы зерно себе не возьмут?» Отвечают убеждённо: «Возьмут барыши, нас устроят, нам хорошо будет и себе заработают, это ничего».

Среди преподавателей Юрьевского Университета также были разные мнения. К.\,К.~Сент-Илер пишет, что некоторые профессора и доценты русского Юрьевского Университета, находившиеся на службе русского правительства, получавшие от него ордена и пенсии, признали себя германскими подданными (например, профессор К. Дегио).

И вот при такой неопределённости, основная роль при выборе решения выпала на долю Юрьевского Университета с его администрацией и Учёным Советом во главе с В.\,Г.~Алексеевым.

И	руководствуясь государственной необходимостью сохранить для России Юрьевский Университет как очаг высшего просвещения, за три дня до прихода немцев в Юрьев 20 февраля 1918 г. Учёный Совет Университета единогласно постановил, что ввиду возможного занятия Юрьева немцами, «остановиться на г. Воронеже как месте, где в случае необходимости открыть деятельность университета».

Вместе с тем, опираясь на условия Брестского мира, подписанного 3 марта 1918 г., и положения 4-й Гаагской конвенции 1907 г., правоведы (В.\,Э.~Грабарь) доказали неправомерность объявления Германскими оккупационными властями Юрьевского Университета немецким.

Эти решительные шаги коллектива Юрьевский Университет и	его ректора В.\,Г.~Алексеева в сочетании с действиями Советского правительства обеспечили в условиях военного времени достойную эвакуацию персонала Юрьевского Университета в Воронеж летом 1918 года вместе с администрацией во главе с ректором, при этом была сохранена факультетская структура.
Вместе с физико"=математическим, медицинским, историко-филологическим и юридическим факультетами в г. Воронеж прибыли:

профессоров — 39,

преподавателей — 45,

канцелярских служащих — 25,

служащих библиотеки — 6,

служителей — 12,

около 800 студентов и 9/10 наиболее ценного имущества Юрьевского университета,
включая архив, библиотеку, коллекцию университетских музеев, инструменты и при\-на\-д\-ле\-ж\-но\-с\-ти клиник и других учебно"=вспомогательных учреждений университета, без которых не было возможности начать занятия в университете (всего 50 вагонов).

Так было спасено для России бесценное дерево высшего образования и пересажено вместе с корнями в наш город. И теперь, готовясь к его юбилею, не следует восхищаться только кроной этого дерева, разросшегося на нашей чернозёмной почве, а нужно помнить, что если «корни подрубить, то оно засохнуть может».
