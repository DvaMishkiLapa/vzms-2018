\begin{center}{ \bf  БОБЫЛЕВ НИКОЛАЙ АНТОНОВИЧ}\\
{\it А.М. Красносельский} \\
%(Воронеж; {\it yusapr@mail.ru} )
\end{center}
\addcontentsline{toc}{section}{Красносельский А.М.}

\begin{flushright}
%\begin{quote}
{\it Один бедный малый, не найдя ничего лучшего, объявил не без гордости:

 – А меня Том Сойер здорово отколотил как-то раз!

 Но его попытка покрыть себя  славой  не  увенчалась успехом. Ведь то же
 самое могли сказать о себе чуть ли не все }

Марк Твен. Приключения Тома Сойера.

{\it Помню – папа ещё молодой.}

Давид Самойлов. Выезд.
%\end{quote}
\end{flushright}

Я сразу прошу прощения  за то, что пишу этот текст не только лично о Коле, но о многом другом, о его отношениях с Красносельскими... и о себе... Но мне кажется, что его отношения с папой настолько определили всю его жизнь, что отделить Колю от Красносельских просто невозможно.

У папы было несколько учеников, которых он называл самыми любимыми\footnote{Читатель, если ты тоже папин ученик, но не Забрейко, не Садовский, не Покровский, не Лифшиц и не Соболевский или Рутицкий~--- не обижайся! Сердцу не прикажешь.}. Он любил иногда перебирать чётки всех имён (большинство можно найти в списке его соавторов) своих учеников, тасовать их по местам, вслух (!) обсуждать, кто главнее (Соболевский, Забрейко, Садовский или кто-нибудь ещё). Было дело давно, я был «маленьким» и плохо знал папиных «старых» учеников (до Забрейко) и слушал перестановки фамилий Забрейко, Бобылева, Лифшица, Покровского и ещё временно вторгавшихся (видимо, благодаря каким-то последним теоремам) других учеников (чаще других Козякина). Он расставлял их в списки, приговаривал что-то (чего я тогда совсем не понимал) математическое (типа: «Конечно, теоремы о ... войдут в историю», впрочем – скорее мне это кажется), тасовал, руководствуясь настроением и их последними успехами, даже что-то записывал. Потом Женя Лифшиц выпал из этих обсуждений... долго, лет 10-15 папа его ещё называл, ругал меня, когда я возражал, говорил, что я ничего не понимаю. Потом появилась новая фамилия – Рачинский... Много времени прошло,  прежде  чем  я  понял  (уже после папиной смерти или незадолго до неё), что и Петя Забрейко, и Коля Бобылев, и Лёша Покровский были всегда не просто любимыми учениками, а членами семьи моего папы. Не просто особями из огромной стаи, которую он возглавлял, не просто младшими друзьями, а такими же (ну, может быть, кроме мамы и меня) или даже более близкими членами ЕГО семьи. И дело не в том, что для Петра Петровича когда-то каждый день мама сама жарила ЛИЧНЫЙ ПЕТИН биток (Забрейко не ел молотого мяса, то есть нормальных котлет), за поедание которого без спроса однажды меня сильно ругали. И не в том, что Коля просто жил со мной и с папой в тогда ещё пустой московской квартире, обживал её (одна из комнат в квартире долгие годы потом называлась бобылевской). И не в том, что Коля был один из немногих папиных учеников, которые любили рыбалку почти так же как сам папа. Это была некая душевная близость, которая бывает между родными, и не просто родственниками, а между близкими родственниками.

 А неплохое было время! Все были живы, зима, 1974 год, мне 19 лет, мы втроём (папа, я и Николай) живём в квартире на улице Островитянова, её в конце 1973 года, наконец, достроили. Каждый вечер он ходит звонить своей будущей жене Тане (он называл её из конспирации Клавушкой). Я увязываюсь с ним вместе, покурить, но я курю быстро, а он говорит долго. Я тоже начинаю звонить... якобы от нечего делать. В результате женился я тогда ещё раньше, чем он! Он должен был быть свидетелем на моей первой женитьбе, однако у него в ЗАГСе не оказалось московской прописки (из ДАСа его уже выгнали, а его кооперативную квартиру ещё не достроили) и моим свидетелем стал Покровский (я тогда и не понимал, как все это круто!). По субботам утром они с папой начинали жарить рыбу (иногда они надо мной издевались и жарили мозги, которые я не ел). Каждый жарил по большой чугунной сковородке трески! А я потом был рефери и решал, у кого рыба в этот день оказалась вкуснее. По крайней мере, один обиженный был всегда! После еды мы садились разыгрывать в преферанс генеральную уборку квартиры, дороже всего ценилась чистка газовой плиты. Бобылев проигрывал часто (впрочем, плиту все равно чистил папа), но уж когда выигрывал – сиял от счастья. Этот стиль соперничества во всяких важных делах типа преферанса или кто наловит больше карасиков величиной с ладошку ребёнка (более крупных они с папой почему-то не ловили, видимо считая ловлю крупной рыбы неэстетичной), кто наберёт больше вёдер опёнка, кто быстрее разгадает бобылевскую шараду\footnote{Про шарады следовало бы сказать особо. Их ввёл в жизнь лаборатории 61 ИПУ, тогда ещё папиной, Бобылев. Почти никто из <<неместных>> решать их так и не научился (Шурик Соболев их здорово разгадывал!), придумывали их в 61й почти все. Типичный пример совершенно честной бобылевской шарады (рассказанной мне лично автором~– Колей): <<первое>> – угроза ввергнуть мир во тьму с помощью канцелярской принадлежности, <<второе>> – домашнее животное, не бросающее слов на ветер, «все вместе» - человек, который не может работать в ИПУ. Истории про бобылевские шарады всегда и везде пользовались большим успехом. Насколько мне известно, список шарад так и не составлен. Я тоже придумал одну, высоко оценённую Бобылевым: «первое» – гласная буква, <<второе>> – два еврея, <<все вместе>>~– дерево. Конечно, это все простые шарады. Дело в том, что одним из основных достоинств бобылевской шарады полагалась её разгадываемость, кому интересны шарады, которые невозможно разгадать… Это же не теоремы\ldots} и (самое главное) кто над кем лучше подшутит, очень украшал жизнь всех вокруг Коли. Когда они с папой ходили ловить рыбу, то всегда один из них был главным рыбаком. Не надо думать, что «главный рыбак» – это тот, кто поймал больше рыбы, нет, наоборот! Ну, сами подумайте – кто ловит больше рыбы (доказывает теорем, печёт сладких булочек и пр.): сам рыбак – или его начальник?!

 	     Слева Коля позирует в Тишково с ведром белых. Как специалист, замечу: староваты, конечно, белые грибочки, однако все равно... ведро белых грибов в Подмосковье набрать совсем не легко – это не Воронеж. Бобылев любил собирать белые, с палочкой, по опушкам. А справа он чистит опёнка, это Дубечино. Я даже не знаю, какой гриб красивее белый – или опёнок, когда он молодой и растёт кучкой.

За грибами я с Бобылевым ходил великое множество раз. Он давно перестал ходить ловить рыбу, а за грибами ходил каждый год. Даже дачу себе он построил в самом грибном месте Московской области – деревне Дубечино. Доказать, что это самое грибное место, невозможно, но косвенные улики лично для меня весьма существенны. Я ехал в 2001 году за белым грибом первым автобусом от Тёплого Стана (Коля со мной не поехал, сказал, что времени нет даже половины дня, так сильно занят!). Ехал я довольно далеко, за бетонку на Вороновском автобусе и разговорился с соседом по сидению в автобусе. Как всегда в таких ситуациях мы говорили о грибах, о том, что опёнок был не сильный в этом году, что чернушки не было совсем, зато белый есть, о том кто, где и когда взял особенно много гриба. И тут этот совершенно мне незнакомый мужик начинает рассказывать о Дубечино! О том, что именно там самые лучшие и обильные грибы. Ну не может же быть таких совпадений без основания! Бобылев в тот год собирал (брал) у себя в Дубечино волнушку. Он рассказывал мне, как за 3 часа нарезал 5 корзин, долго напоминал мне о каких именно корзинах идёт речь (...нет, не та, средняя, а та, помнишь, я с ней в Тишково всегда за опёнком хожу...). А я завидовал ему, люблю солёную волнушку, но ни разу её много не собирал. И даже не видел, чтобы собирали другие.

Помню, как году в 1984-85 вечером в воскресенье мы ехали из Тишково в ПАЗике, набитом опятами. Тот, кто знает, как остро пахнет свежесрезанный опёнок, меня поймёт. Так вот, опёнком в ПАЗике пахло сильнее, чем бензином. Был он, я с сыном Мишкой, тогда ещё довольно мелким, полный автобус людей и опята. Год на опёнка выдался удачный, лично я вёз полную выварку уже сваренного совсем молодого опёнка-гвоздика, собранного в пятницу вечером и в субботу утром, уже нагревшийся рюкзачок сухого опёнка покрупнее, собранного в субботу после обеда и большой рюкзак ядрёного опёнка, собранного в воскресенье утром. Ещё была полная корзина, набранная в самом лагере и в 20 метрах рядом с ним уже перед самым отъездом. Примерно столько же гриба везли все люди в ПАЗике (ну, может, я все-таки был чуть-чуть пожаднее!). Народ устал после резанья опят, и все начали дремать. Мы с Мишкой сидели на сиденье, а Бобылев сидел на ступеньках автобуса рядом. В некоторый момент мой сын начал нудить и Коля начал его развлекать. Ещё через 10 минут все проснулись и смотрели бесплатное представление. Начальную стадию его я не помню – сам дремал под сыновьи вопли. Помню только, что Мишка решал какие-то смешные задачи (книжек Остера тогда ещё не было, но, думаю, Бобылев мог бы и фору дать уважаемому автору!), неизменно требуя за каждую решённую задачу сколько-то там опёнков. Потом  проснулся и я, и тогда мы с Мишкой решали Специальные Бобылевские Задачи Для Двух Решающих. Я запомнил одну: вот мы с Мишей в городе зашли в магазин ЦУМ, разминулись и потеряли друг друга. Дальше каждый из нас называл своё место встречи. Встретимся ли мы? Мы встретились дома... Финалом (за 25 опят) была шарада, уже решённая лично мной: «первое» – выдающаяся часть человеческого недостатка, «второе» – Бобылев один в большой пустой тюремной камере, «все вместе» – апробация диссертации. Мы и не заметили, как доехали до дома...

Мы все любили Николая Антоновича, но больше всего его любили дети и собаки (недавно мне Таня рассказала о том, как к нему клеились знакомые вороны в Тропаревском парке, но сам я того не видел!). Собаки – и бездомные беспородные, и домашние в ошейниках – знали его самого и его слова: «Ух, ты моя девочка!», произносимые под разворачивание принесённого пакета с едой. Заинтересованные лица помнят любовь Коли к моей племяннице Ленке... примерно году так в 1969... ей было года 4 (на самом деле мне кажется, что 5, но она всегда отказывается это признавать). Он приходил к нам на Театральную заниматься с папой и немедленно оказывался втянутым в игру в прятки с Ленкой. Вначале это были обычные прятки (благо было, где играть!), но потом он, будучи в особенном расположении духа, подойдя к двери, за которой стояла Ленка, произнёс: «Заглянуть что ли за дверь... Или не стоит... Если за ней кто-то есть, то пусть ОНО молчит. А если там никого нет, то пусть ОНО скажет «ОГУРЕЦ!»... Это был экспромт! Папа, я и Алка (моя сестра и Ленкина мать, если кто вдруг не знает!) стояли рядом и молчали, поражённые самой идеей. И были даже как-то сначала не сильно удивлены, когда из-за двери раздалось тихое: «Огурец!»... Это потом в результате длительных экспериментов выяснилось, что так говорят все дети (мои старшие точно говорили), что они все отлично понимают и поддерживают игру из инстинктивного чувства прекрасного. Это прекрасное придумывал Коля и щедро разбрасывал вокруг себя.

Любил Бобылев всякие интересные «штучки», у него
\linebreak
был к этому вкус.
Последние несколько лет он отовсюду возил камни. Коллекционировал пейзажную яшму, даже специально для этого ездил в Башкирию, в Сибай к Марату Юмагулову. Лекции читал и камней привёз оттуда. Он любил показывать свои камни всем, кто был готов их смотреть. А как Коля любил книги! Книги он начал собирать в самом начале учёбы в Воронежском университете. Никогда не забуду его рассказа о купленном им ещё в конце 60х двухтомнике Тацита в Литпамятниках... Жаль, не дожил тот Тацит до московских времён. Канул в бездны воронежского «Букиниста». Впрочем, заваленный книгами бывший папин кабинет в 578... заставленная книгами квартира... Когда мы жили втроём с папой и мебель у нас была только кухонная, расставленная по всей квартире (в одной комнате одна полка стояла на полу, в другой – другая полка и стол), то для одежды места не было, а книжки лежали в полках, чтобы не запылиться. При нынешнем книжном изобилии это довольно странно выглядит, но тогда все было именно так.
Жалко, что у меня осталось так мало его фотографий. Цветная фотография – из Ирландии, нас с ним Покровские возили смотреть на Атлантический океан. А вот ещё одна фотография, чёрно-белая. На ней – неузнаваемо юные Лёша Покровский, Валерий Иванович Опойцев, Шурик Соболев. Ну и Коля с папой. Господи, насколько я сам уже старше их с этой фотографии... Теперь такие фотографии уже никто не делает. Жалко, что не осталось фотографий воронежского периода, когда у нас не было ни качественных аппаратов, ни хорошей плёнки.

Все математики, кто кончал в Воронеже школу в 60х-70х годах, учились с Бобылевым в одной и той же школе №58 города Воронежа. И всех нас в школе учил Давид Борисович Сморгонский («Идите, идите к доска, троечник Бaбилев» - это Коля так очень смешно передразнивал идишный акцент Давида). Воронежские зимние математические школы. Коля любил туда ездить, его все знали и любили, не только воронежцы и москвичи. Остались ли бытовые фотографии с этих школ? Сколько там было приключений! Как-то мы с ним пошли рыбу ловить. Взяли с собой все, что нужно (вот, кстати, ещё бобылевская шарада: «первое» – принадлежность зимнего рыбака, «второе» – женщина еврейской национальности, въезжающая в новую квартиру, «все вместе» – философское животное), оделись, как положено, пошли. А на бобылевские валенки не было калош. И он выпросил калоши у папы (у папы с собой тоже был комплект одежды). Половили мы с ним рыбу, начало смеркаться, пошли мы с ним назад. Устали уже, времени часа четыре. Ну~– два умника решили срезать путь. Через такие заросли камыша... а под коркой то ли льда, то ли наста было болото. Ладно, идти недалеко, метров 100 всего. И тут потерял Коля в болоте калошу, да ещё и заметил это не сразу. Идти в «Берёзку» без калоши он отказался, папу боялся, знал что за калошу ему мало не покажется. Стемнело... мы шаг в шаг искали калошу на ощупь до 8 часов вечера... Нашли, однако! Или другая история из тех же мест. Спортсмены-лыжники (в тот год это были, кажется, Покровский с Грачевым) каждый день бегали 30 километров туда, 30 километров сюда, а потом долго обсуждали свои пробежки. Бобылев с Шуриком Соболевым уговорили меня тоже пойти покататься на лыжах. Прошлись мы по лыжне где-то с километр (один!) и стало мне не интересно. И пошёл я на реку, посмотреть, как рыбаки рыбу ловят. Тут и Шурик с Колей меня догнали, как они выразились, на всякий случай. И я их подначил прогуляться вверх по речке Усманке, приговаривая, что по лыжне ходить не интересно, вот вдоль речки по льду будет здорово и не ходит там никто. Ну, сначала все и было хорошо. А потом мы поняли, почему там никто не ходит: Усманка речка быстрая и в середине начались промоины. Сначала мы все шли вдоль одного берега, но наклонные ивы и ольхи, растущие в изобилии вдоль берега, заставили нас переходить на другой берег. И это было не страшно, но потом настал момент совпадения промоины с необходимостью перехода на другой берег. На лыжах такое оказалось вполне возможно: надо было стоять над промоиной, опираясь носками лыж на лёд с одной стороны реки, а задниками – на лёд другой стороны. После первого же такого вынужденного перехода стало ясно, что он далеко не последний и нам стало грустно. Поход вдоль речки надо было отменять. Ладно, решили мы, пойдём назад. Подумали (а надо было бы раньше это сделать!) и поняли, что вдоль реки мы до дому можем и не дойти. До «Берёзки» по прямой было не больше километра. Мы прорубались через обледеневший сухой камыш, которым зарос этот километр, примерно часа три. Когда мы благополучно пробились к дороге, и я начал лицемерно говорить «как хорошо мы прогулялись», они меня чуть не убили.

Лаборатория 61 ИПУ. Папина лаборатория, потом – лаборатория Бобылева. Изначально всю эту лабораторию пронизывал дух Бобылева. Он ещё в 70-80х годах определял отношения в лаборатории, их стилистику, их форму и ту огромную любовь, которая там царила. Эту любовь замечательно чувствовали и видели все приходящие люди (вроде меня), у которых по месту их постоянной работы ТАКОГО не было. Не было Марка Александровича завлаба и Бобылева – стилиста. Он был настоящий артист\footnote{Профессионал (театральный педагог и телезвезда) натуралист Павел Любимцев (он знал Колю ещё с тех времён, когда папа жил у них под роялем, а Коля жил в ДАСе и ходил к ним в гости) полностью подтверждает высокий Колин артистизм в совершенно профессиональном смысле.}. Я никогда с ним это не обсуждал, но мне кажется, на сцене он был бы великолепен. Мы его видели там только во время его замечательных докладов (кстати, папа всегда выделял Колю в числе своих учеников, по-настоящему умевших делать доклады; я понимаю, теперь эта фраза смешна, все знают его мастерство; но папа отмечал Колино умение делать доклады и в 70х!). Один из его докладов (о критических точках и локальных и глобальных экстремумах гладких скалярных функций двух переменных) я с удовольствием повторял много раз.  Я стараюсь немножко подражать Коле, тема благодатная, знать слушателям ничего специального вроде и не надо... Однако, на внешне тривиальные вопросы (их автор – Бобылев) мало кто из специально не подготовленных слушателей знает правильные ответы.

Замечательно смотрелось желание каждого нового пришедшего в лабораторию поколения переплюнуть
«стари-\linebreak ков»~---
в основном самого Бобылева – в области шуток, подначек и розыгрышей.
В конце 80х там уже настали совершеннейшие джунгли, где воровали чужие пирожные и ромовые бабы...
фиксировали сотрудников на их собственных столах с полным прекращением доступа кислорода к телу организма...
Впрочем, рассказы об этом хороши в изустной форме. Иных уж нет, а те далече...
Я все-таки там не так уж и часто бывал, но всегда там происходили какие-то удивительные истории,
нигде более не возможные, только там.

Сидит Бобылев и вставляет чёрной ручкой формулы в статью\footnote{Кстати, Бобылев был великий умелец красиво формулы вставлять! Саше Владимирову за 2 компота вставил первый экземпляр диссертации... Умение, ныне совсем утратившее свою ценность... Скоро и рассказы о нем станут не понятны новым поколениям математиков.}. Лето... Жарко...
По комнате 635 слоняется очумевший от жары и безделья Игорь Фоменко (привет из США).
В комнате ещё я, Покровский (Ирландия) и Саша Веретенников (Англия). Не выдерживает Фома, подходит к Бобылеву сзади, делает тому «короткое замыкание» и быстро начинает бежать зигзагами мимо столов и стульев к спасительной двери. Я открываю рот... Опытные сотрудники лаборатории 61 быстро прячутся под столами... Бобылев, не вставая, спокойно, но быстро берет со стола стакан с чаинками и остатками заварки и через всю большую комнату накрывает Фому чаем. Коричневая струя остатков заварки пролетает над столами и оказывается вся на майке Игоря. Он по инерции вылетает за дверь... и исчезает. Я захлопываю рот со словами, что-де молодёжь пошла в нынешнее время особенно наглая... Умудрённые сотрудники лаборатории из-под столов не вылезают... Бобылев, так и не встав, продолжает вставлять закорючки в текст. Входит разъярённый Фома со стаканом прозрачной воды. Бобылев его игнорирует спиной, если так можно выразиться. Фома подходит к Бобылеву, мечется, со стороны от меня и из-под столов доносятся подначки (была ли горячая вода, легко ли отстиралась майка, была ли очередь к умывальнику). Тогда в ярости Фома хватает пиджак, висевший на стуле, на котором сидел Бобылев, и ВЫЛИВАЕТ ВОДУ во внутренний карман!!! И начинает все так же бежать зигзагами к двери, один стол, другой... И в эту вот секунду все понимают, что что-то не так: Бобылев сидит и продолжает безмолвно вставлять закорючки... Не добежав до двери, даже Игорь останавливается... я столбенею... сотрудники опасливо вытягивают шейки из-под столов... и тут Веретенников говорит: «Ой! Это же мой пиджак...!». Как нам троим (Коле, мне и Лёше) было хорошо в тот момент! Фома полз на коленях к Веретенникову, обещая «больше никогда так не делать»... Как хохотал Бобылев...

Это январь 1997 года, папа ещё жив... Мы с Колей находимся в Австралии, в Джилонге в ботаническом саду, мы стоим рядом с фикусом–переростком. Фотографирует нас Борис Николаевич Садовский. Бобылев решил, что если он раздвинет руки, то станет если и не такого размера, как фикус, то такого размера, как я.

Вообще, несмотря на то, что мы с ним в Москве последнее время не очень часто виделись, (раз в 2-3 месяца, больше по телефону общались) благодаря Лёше Покровскому нам удались две совместные поездки. В Австралию и в Ирландию. В Ирландии мы с ним сидели в одной комнате, он писал попеременно тo статью про устойчивость интервальных самосопряжённых матриц, тo воспоминания о папе. При этом хихикал, взвизгивал, иногда даже в голос смеялся каким-то своим историям. Написав, он рассказал мне, что получил огромное удовольствие от того, о чем писал. Что многие истории, о которых он писал, он пережил заново. Не думал я, что мне придётся писать воспоминания так скоро... и уж точно не о нем!

Вот, напоследок, ещё пара фотографий. Какое у него красивое лицо!
