\vzmstitle{О совпадениях отображений упорядоченных множеств}

\vzmsauthor{Фоменко}{Т.\,Н.}

\vzmsinfo{Москва; {\it tn-fomenko@yandex.ru}}

\vzmscaption

Доклад посвящён проблемам существования общих неподвижных точек и совпадений семейств многозначных отображений упорядоченных множеств.

Классическими результатами о существовании неподвижной точки отображения упорядоченного множества в себя являются теоремы Кна\-стера-Тарского (Knaster-Tarski), Смитсона (Smithson) и Цермело (Zer\-melo) (см. [1,2]), имеющие многочисленные применения. Результаты о существовании совпадений двух многозначных отображений получены в [3,4], затем развиты и обобщены  в [5--7]. В  [3--7] получены результаты, обобщающие теоремы Кнастера-Тарского и Смитсона. Однако из них не следует теорема Цермело.

В докладе будут представлены теоремы об общих неподвижных точках и совпадениях семейств многозначных отображений, уточняющие и обобщающие некоторые результаты [5-8] и  соответствующие теоремы из [3,4]. В отличие от  работ [3--7],  на отображения не накладываются требования типа изотонности или накрываемости. Требуется лишь наличие специальных цепей в упорядоченном множестве и их нижних границ с определёнными свойствами.

Известно, что проблемы существования  общих неподвижных точек и точек совпадения {\it семейств отображений}, рассматриваемые в [6,7,8],  не сводятся к случаю  одного или двух отображений соответственно, представляют самостоятельный интерес и важны в приложениях. В частности, многими авторами изучаются задачи об общих неподвижных точках  полугрупп отображений (например, особые точки динамических систем), об общих неподвижных точках семейств  монотонных операторов в упорядоченных банаховых пространствах, рассматриваются  варианты таких задач для семейств отображений с различными свойствами типа сжимаемости в метрических и обобщённых метрических пространствах, задачи о совпадениях семейств отображений в духе обобщений теоремы Хопфа о склеивании диаметрально противоположных точек сферы (см., например,  [9--12] и многие другие работы российских и зарубежных авторов.)

{\bf Необходимые обозначения.} Сим\-вол  $\rightrightarrows$  обозначает многозначное отображение. $(X,\preceq_{X}), (Y,\preceq_{Y})$ -- упорядоченные множества , $x_{0}\in X$, ${\cal F}=\{F_{1},F_{2}\}$, где
$F_{1},F_{2}: X\rightrightarrows Y$.

Обозначим через  ${\bf {\cal C}(x_{0}; {\cal F})}$ совокупность пар вида $(S,f)$, где цепь
$S\subseteq T_{X}(x_{0}):=\{x'\in X |x'\preceq_{X} x_{0}\}$,
$f=\{f_{1}, f_{2}\}, f_{i}: S\to Y,  i=1,2, f_{2}(x)\preceq_{Y} f_{1}(x), \forall x\in S$, причём $f_{1}(x) \in  F_{2}(T_X(x_0))\cap F_{1}(x), f_2(x) \in F_2(x)$ для любого $x\in S$. Кроме того, для любых $x,z\in S$, $x\prec z\Longrightarrow f_{1}(x)\preceq f_{2}(z).$

Приведём  одно из следствий  результатов [8] для случая совпадений двух многозначных  отображений.

\textbf{Теорема~1.} {\it  Пусть $(X,\preceq), (Y,\preceq)$  - частично упорядоченные множества, $x_{0}\in X$, ${\cal F}=\{F_{1},F_{2}\}$,  $F_{1},F_{2}: X\rightrightarrows Y$ - многозначные отображения. Пусть множество ${\cal C}(x_{0}; {\cal F})$ непусто, и для любой пары $(S,f)\in {\cal C}(x_{0}; {\cal F})$, где $f=\{f_{1},f_{2}\}$, цепь $S$ имеет нижнюю границу $w\in X$, и существуют значения  $\{z_{1},z_{2}\}$, $z_{1}\succeq z_{2}$, $z_{j}\in F_{j}(w),$  где $z_{j}$ есть нижняя граница множества $\{f_{j}(x)| x\in S\}, j=1,2.$ При этом, если $z_{2}\prec z_{1}$, то  $\exists w_{1}\in X, w_{1}\preceq w$, что   $z_{2}\in F_{1}(w_{1})\cap F_{2}(w)$ и  $\exists v\in F_{2}(w_{1})$, $v\preceq z_{2}$. Тогда   $Coin(F_{1},F_{2})=\{x\in X| F_{1}(x)\cap F_{2}(x)\ne \emptyset\}\ne \emptyset$.}

Можно показать, что из Теоремы 1 следуют Теорема 1 из [3,4], а также Теорема 3 из [6,7], при $n=2$.

Если в условиях Теоремы 1 положить  $X=Y$, $F_{1}=Id_{X}, F_{2}=F: X\rightrightarrows X$, то  получается, например, следующее утверждение.

\textbf{Теорема~2.} {\it Пусть $(X,\preceq)$  - частично упорядоченное множество, $x_{0}\in X$,   $F: X\rightrightarrows X$, и ${\cal C}(x_{0}; \{Id_{X}, F\})\ne \emptyset$. Пусть для каждой пары $(S,f)\in {\cal C}(x_{0}; \{Id_{X}, F\})$ существует нижняя граница $w$  цепи $f(S)$, и существует $z\in F(w)$, $z\preceq w$. Тогда $Fix(F)\ne \emptyset$.}

Можно показать, что из Теоремы 2 следует результат Ячимского (Ya. Yachymski) [1, Глава 18, Теорема 3.13], являющийся обобщением теоремы Цермело. Поэтому Теоремы 1 и 2 следует рассматривать и как обобщения  теорем Ячимского и  Цермело, не вытекающих из [3--7].



\litlist


1. {\it Kirk W.A., Sims B. (eds.)} Handbook of metric fixed point theory. Springer Science \& Business Media,  2001.

2. {\it Smithson R.E.} Fixed points of order preserving  multifunctions. Proc.Amer.Math.Soc., 28(1971), 304--310.

3. {\it Arutyunov A.V., Zhukovskiy E.S., Zhukovskiy S.E.} Coincidence points of set-valued mapping in partially ordered spaces, Dokl. Math., 88 (3), 2013, 727--729.

4. {\it Arutyunov A.V., Zhukovskiy E.S., Zhukovskiy S.E.} Coincidence points principle for set-valued mappings in partially ordered spaces. Topology and its Applications, 201:330--343, 2016.

5. {\it Fomenko T.N., Podoprikhin D.A.} Fixed points and coincidences of mappings of partially ordered sets, Journal of Fixed Point Theory and its Applications, 18 (2016), c.823--842.

6. {\it Подоприхин Д.А., Фоменко Т.Н.} О совпадениях семейств отображений  упорядоченных множеств. // ДАН,  {\bf 471}:1, 2016, с.16--18.

7. {\it Fomenko T.N., Podoprikhin D.A} Common fixed points and coincidences of mapping families on partially ordered sets. Topology and its Applications, 2017, vol.221, p.275-285.

8. {\it Фоменко Т.Н.} Неподвижные точки и совпадения в упорядоченных множествах // ДАН, 2017, {\bf 474}:5,  с.550--552.

9. {\it А. Ю. Воловиков}, Точки совпадения отображений $\mathbb Z ^{n}_{p}$ -пространств // Изв. РАН. Сер. матем. 2005. том 69, выпуск 5, с. 53–106.

10. {\it Р. Демарр}, Общие неподвижные точки коммутирующих нерастягивающих отображений, Математика, 1965, том 9, выпуск 4, с. 139–141.

11. {\it R.N. Karasev, A.Yu. Volovikov} Knaster's problem for almost $(\mathbb Z_{p})^{k}$-orbits // Topology and its Applications 157 (2010), p. 941 -- 945.

12. {\it Бахтин И.А., Попова В.В.} Существование общих неподвижных точек монотонных не коммутирующих операторов. // Депонированная рукопись, 2299-В 92 Деп., РЖМв., 1992, ИБИ79. 27 c.

