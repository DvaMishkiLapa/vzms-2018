\begin{center}{ \bf  АНАЛИЗ РАЗЛИЧНЫХ МЕТОДОВ ИНТЕРПОЛЯЦИИ ДЛЯ ВОССТАНОВЛЕНИЯ ФИЗИЧЕСКИХ СПЕКТРОВ}\\
{\it Н. Акиньшина } \\
({\it akinshina$\_$nadya@mail.ru} )
\end{center}
\addcontentsline{toc}{section}{Акиньшина Н.\dotfill}

Сегодня вычислительная математика включает в круг своих проблем изучение особенностей вычисления с применением компьютеров и обладает широким кругом прикладных применений для проведения научных и инженерных расчётов. Любые математические приложения начинаются с построения модели явления.

Важнейшими средствами изучения математических моделей являются аналитические методы: получение точных решений в частных случаях, разложения в ряды. Большую роль издавна играли приближенные вычисления. Множество численных методов решения задач так или иначе связано с аппроксимацией функций. Это и задачи приближения функций (интерполяция, сглаживание, наилучшее приближения) и задачи, в которых аппроксимация присутствует как промежуточный этап исследования (численное интегрирование и дифференцирование, численное решение интегральных и дифференциальных уравнений).

Типичная задача приближения - задача интерполяции, которая заключается в восстановлении функции $f(x)$ с той или иной точностью на отрезке $[a:b]$ действительной оси по заданной таблице чисел $(x_i, f(x_i)$, где $i=0,...,N$. Существуют различные способы решения данной задачи. Классический метод ее решения состоит в построении интерполяционного многочлена Лагранжа.
В данной исследовательской работе я рассмотрела метод интерполяции эрмитовыми кубическими сплайнами и метод интерполяции кубическими сплайнами класса $C^2$ .

Кубическим интерполяционным сплайном дефекта 2
\linebreak
(или эрмитовым кубическим сплайном) будем называть фу\-н\-к\-цию
$S_{3,2}(f;x)=S_{3,2}(x)$, удовлетворяющую двум условиям:

1. На каждом из промежутков $[x_i,x_i+1]$
$$S_{3,2}(x)=a_{i0}+a_{i1}(x-x_i)+a_{i2}(x-x_i)^2+a_{i3}(x-x_i)^3$$

2.$S_{3,2}(x_i)=f_i$, $S_{3,2}^{'}(x_i)=f_i^{'}$, $i\in 0,...,N$

Интерполяционным кубическим сплайном $S(f;x)$  называется сплайн, удовлетворяющий условиям
 $S(f;x_i)=f_i$, $i\in 0,...,N$
 Геометрически он представляет собой ломанную, проходящую через точки $(x_i;y_i)$, где $y_i=f_i$ .
