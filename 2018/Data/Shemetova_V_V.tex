\begin{center}{ \bf ОБ ОДНОЙ НАЧАЛЬНОЙ ЗАДАЧЕ ДЛЯ ДИФФЕРЕНЦИАЛЬНЫХ УРАВНЕНИЙ С ОТКЛОНЯЮЩИМСЯ АРГУМЕНТОМ В БАНАХОВОМ ПРОСТРАНСТВЕ}\\
{\it В.В. Шеметова } \\
(Иркутск; {\it valentina501@mail.ru} )
\end{center}
\addcontentsline{toc}{section}{Шеметова В.В.}


Пусть $E$~--- банахово пространство, $u$ и $f$~--- неизвестная и заданная функции со значениями в $E$. Рассматривается функционально-дифференциальное уравнение вида
$$
u'(t)=Au(t)+Bu(t-h)+f(t),\,t>0.\,\eqno(1)
$$
Здесь $A$ и $B$~--- линейные непрерывные операторы из $E$ в $E$, $h>0$. Для уравнения (1) зададим начальные условия
$$
u(t)=\varphi(t),\,-h\leq t<0,\,\,\,u(0)=u_{0}, \eqno(2)
$$
где функция $\varphi(t)\in C(\left[-h,\,0\right];\,E)$ и вектор $u_{0}\in E$ известны. Под {\it классическим} решением задачи (1), (2) будем понимать функцию $u(t)\in C(t\geq-h;\,E)\cap C^{1}(t>0;\,E)$, обращающую в тождество уравнение (1) и удовлетворяющую начальным условиям (2). Для существования классического решения необходимо $u_{0}=\varphi(0)$, т.~е. $u(t)=\varphi(t),\,-h\leq t\leq0$, и задача (1), (2) становится обычной задачей с начальной функцией.

Проблема однозначной разрешимости рассматриваемой начальной задачи (1), (2) изучается с помощью аппарата обобщенных функций Соболева--Шварца со значениями в бесконечномерном банаховом пространстве [1]. Концепция фундаментальной оператор-функции [1] линейного интегро-дифференциального оператора в банаховых пространствах распространяется на класс дифференциальных операторов семейства уравнений (1) с отклоняющимся аргументом.

Пусть $u$~--- классическое решение, продолжим его нулем при $t<-h$ следующим образом:
$$
\tilde{u}(t)=\varphi(t)\bigl(\theta(t+h)-\theta(t)\bigr)+u(t)\theta(t),
$$
где $\theta$~--- {\it функция Хевисайда}. В классе $K'_{+}(E)$ распределений с ограниченным слева носителем начальная задача (1), (2) принимает вид сверточного уравнения
$$
\bigl({\mathbb I}\delta'(t)-A\delta(t)-B\delta(t-h)\bigr)\ast\tilde{u}(t)=\tilde{g}(t)\,\eqno(3)
$$
с правой частью
$$
\tilde{g}(t)=f(t)\theta(t)-A\varphi(t)\bigl(\theta(t+h)-\theta(t)\bigr)+
$$
$$
\delta'(t)\ast\varphi(t)\bigl(\theta(t+h)-\theta(t)\bigr)+u_{0}\delta(t),
$$
Здесь и далее $\delta$~--- {\it дельта-функция Дирака}. Единственным решением уравнения (3) в $K'_{+}(E)$ ({\it обобщенным} решением задачи (1), (2)) является распределение $\tilde{u}(t)={\cal E}(t)\ast\tilde{g}(t)$,
где ${\cal E}$~--- обобщенная опе\-ра\-тор-функ\-ция такая, что
$$
\forall\,v(t)\in K'_{+}(E)\,\,\,\,\,\,\bigl({\mathbb I}\delta'(t)-A\delta(t)-B\delta(t-h)\bigr)\ast{\cal E}(t)\ast v(t)=v(t),
$$
$$
\forall\,v(t)\in K'_{+}(E)\,\,\,\,\,\,{\cal E}(t)\ast\bigl({\mathbb I}\delta'(t)-A\delta(t)-B\delta(t-h)\bigr)\ast v(t)=v(t),
$$
называемая {\it фундаментальным решением} функционально-дифференциального оператора $\left({\mathbb I}\delta'(t)-A\delta(t)-B\delta(t-h)\right)$.

\textbf{Теорема 1.} {\it Пусть композиция операторов $A$, $B\in{\cal L}(E)$ коммутативна, тогда оператор $\left({\mathbb I}\delta'(t)-A\delta(t)-B\delta(t-h)\right)$ имеет фундаментальное решение вида
$$
{\cal E}(t)=\sum\limits_{k=1}^{+\infty}U_{k-1}\bigl(t-(k-1)h\bigr)\theta\bigl(t-(k-1)h\bigr),
$$
где оператор-функции $U_{k}(t)$ задаются следующим образом:
$$
U_{k}(t)=\frac{t^{k}}{k!}e^{At}B^{k}.
$$}
При выполнении условий теоремы 1 начальная задача (1), (2) имеет единственное обобщенное решение вида
$$
\tilde{u}(t)=\varphi(t)\bigl(\theta(t+h)-\theta(t)\bigr)+\bigl(e^{At}\varphi(0)+\int\limits_{0}^{t}e^{A(t-s)}f(s)ds\bigr)\theta(t)+
$$
$$
\sum\limits_{k=1}^{+\infty}\biggl\lbrace\biggl[ U_{k}(t-kh)u_{0}+\int\limits_{(k-1)h}^{kh}U_{k-1}(t-s)B\varphi(s-kh)ds+\biggr.\biggr.
$$
$$
\biggl.\int\limits_{kh}^{t}U_{k}(t-s)f(s-kh)ds\biggr]\theta(t-kh)+
$$
$$
\biggl.\int\limits_{(k-1)h}^{t}U_{k-1}(t-s)B\varphi(s-kh)ds\bigl(\theta(t-(k-1)h)-\theta(t-kh)\bigr)\biggr\rbrace.
$$
которое является регулярным распределением и порождено обычной функцией $u:\,[-h;\,+\infty)\to E$, заданной кусочно на полуинтервалах $[(k-1)h,\,kh)$, $k\in\lbrace0\rbrace\cup{\mathbb N}$. Эта функция является классическим решением рассматриваемой задачи при условиях $f(t)\in C(t\geq0;\,E)$ и $u_{0}=\varphi(0)$.

Пусть $f(t)\in C^{n-1}(t>0;\,E)$, тогда решение задачи (1), (2) в точках $t=(k-1)h$, где $k=1,\ldots,n$, имеет $k-1$ порядком сильной гладкости, а в других точках интервала $(0;\,+\infty)$ он равен $n$. Это согласуется с известными фактами  [2, с.~20] о скалярных ($E={\mathbb R}$) уравнениях с отклоняющимся аргументом. При $u_{0}\neq\varphi(0)$ функция $u=u(t)$ разрывна в точке $t=0$, имеет сильную гладкость порядков $k-1$ в точках $t=kh$ для $k=1,\ldots,n$ и $n$ во всех остальных точках. Если $B\in{\cal L}(E)$ необратим и $\bigl(u_{0}-\varphi(0)\bigr)\in N(B)$, то ситуация улучшается: разрывное при $t=0$ решение задачи (1), (2) обладает при $t>0$  свойствами ее классического решения.

\smallskip \centerline{\bf Литература}\nopagebreak

1. {\it Sidorov~N. et al.} Lyapunov--Schmidt Methods in Non-linear Analysis and Applications. Dordrecht--Boston--London: Kluwer Academic Publishers, 2002. 568~p.

2. {\it Эльсгольц~Л.Э., Норкин~С.Б.} Введение в теорию дифференциальных уравнений с отклоняющимся аргументом. М.: Наука, 1971. 296~с.
