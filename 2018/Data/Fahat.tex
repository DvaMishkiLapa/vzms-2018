\begin{center}
{\bf Фахад Дульфикар Али\\ (Воронежский государственный
университет)\\ Задача со смешанным краевым условием для уравнения
В.С. Голубева}
\end{center}
\addcontentsline{toc}{section}{Фахад Дульфикар Али}

Исследуется общая краевая задача без начальных условий для
дифференциального уравнения
$$L_tu(t,x)=\frac\nu a\frac{\partial u(t,x)}{\partial
t}+\frac{(1-\nu)\gamma}au(t,x)+$$
$$+\frac{(1-\nu)\gamma^2}a\int_{-\infty}^te^{\gamma(s-t)}u(t,s)ds\eqno{(1)}$$
с граничным условием $$\lambda u(t,x)-\left.\frac
d{dx}u(t,x)\right|_{x=0}=\varphi(t)\eqno{(2)}$$
$$u(t,\infty)=0\eqno{(3)}$$

Справедлива следующая

{\bf Теорема 1.} Если в условии (2) функция $\varphi(t)$ имеет вид
$$\varphi(t)=\cos \omega
t,\hspace{5mm}t\in(-\infty,\infty)\eqno{(4)}$$ и выполнено условие

$$\lambda+\sqrt{\frac{\rho+\alpha}2}>0.\eqno{(5)}$$ где
$$\alpha = \frac{\omega^2(1- \nu)\gamma}{(\gamma^2 + \omega^2)}$$
$$\rho = \sqrt{(\frac{\omega^2(1-\nu)\gamma}{(\gamma^2 + \omega^2)})^2 +
(\frac{\omega(\gamma^2 + \nu\omega^2)}{(\gamma^2 + \omega^2)})^2},
$$ то задача (1)---(3) имеет единственное ограниченное решение и
оно представимо в виде

$$u(t,x)=\frac{e^{-(\lambda+B)x}}{\sqrt{(\lambda+B)^2+B^2}}\cos(Bx-\omega
t-\Theta)\eqno{(6)}$$

где $B=\sqrt{\frac{\rho+\alpha}2}$,
$\Theta=\arccos\frac{B}{\sqrt{B^2+\sqrt{(B+\lambda)^2}}}$.

{\bf Литература.}

1. Голубев В.С. Уравнение движения жидкости в пористой среде с
застойными зонами. ДАН СССР, Т.238, №6, 1978, С. 1318---1320.

2. Бабенко Ю.И. Методы дробного интегродифференцирования в
прикладных задачах теории тепломассообмена. // Ю.И. Бабенко.---
СПБ.: НПО "Профессионал", 2009, 584 с.

3. Kostin D.V. On well-posed solvability of boundary value problems
for equations with fractional derivatives in hyper-weight spaces of
continuous functions on $R^+$// D.V. Kostin, Applicable Analysis
Volume 96, 2017 -- Issue 3, p.396-408.
