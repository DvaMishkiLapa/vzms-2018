\begin{center}{ \bf СХОДИМОСТЬ КОМБИНАТОРНОГО ПОТОКА РИЧЧИ С ОБОБЩЕННЫМИ ВЕСАМИ}\\
{\it Р.Ю. Пепа } \\
(Москва; {\it pepa@physics.msu.ru} )
\end{center}
\addcontentsline{toc}{section}{Пепа Р.Ю.\dotfill}


Б. Чоу и Ф. Луо в 2003 году доказали аналог теоремы Р. Гамильтона о сходимости потока Риччи на дискретных многообразиях с заданной метрикой для  упаковки окружностей У. Тёрстона. Комбинаторная структура включает в себя набор весов, определенных для каждого ребра симплекса. Необходимым условием теоремы Б. Чоу и Ф. Луо является неотрицательность весов. В данной работе показано, что тех же результатов о сходимости потока Риччи можно достигнуть и при более слабых условиях для весов на ребрах симплекса: некоторые веса могут быть отрицательными и должны удовлетворять определенным неравенствам.

В работе обсуждается наиболее
естественная, с точки
зрения геометрии, дискретизация потока
Риччи в двумерном случае.
Пусть задана триангуляция некоторой
двумерной замкнутой поверхности $T$.
Перенумеруем каким-либо образом вершины
многогранника. Для ребра,
соединяющего вершины с номерами $i$ и $j$,
выберем число $ w_{ij} =
w_{ji} \in [-1,1]$. Наконец, в каждой вершине
зафиксируем
положительное число $r_i$. Набор данных,
состоящий из триангуляции
$T$, набора весов $W = \{ w_{ij} \}$ и положительных чисел
$R = \{ r_i \}$, определяет на многограннике
метрику следующим
образом. Длину ребра $l_{ij}$, соединяющего
вершины $i$ и $j$,
определим по формуле $l_{ij}^2 = {r_i}^2 + {r_j}^2 +
2r_i
r_jw_{ij}$. Заданные таким образом длины
ребер многогранника
определяют плоские углы всех граней, которые в свою очередь вычисляются через множество длин ребер.
С комбинаторной точки зрения, такое определение длин ребер многогранника имеет простую геометрическую интерпретацию. Пусть  $C_i$, $C_j$ --- окружности радиуса $r_i$, $r_j$ соответственно  в евклидовой плоскости, $ \theta_{ij} $ определяют их угол пересечения, причем $ w_{ij} = \cos \theta_{ij} $. Тогда, $l_{ij}$ --- расстояние между центрами окружностей $C_i$, $C_j$.
Далее, гауссова кривизна многогранника с
евклидовой метрикой на
гранях сконцентрирована в его вершинах и
определятся формулой
$$
K_i = 2 \pi - \sum_{j} \alpha_{ij},
$$
где $i$ --- номер вершины, а $\alpha_{ij}$ --- все
плоские углы в
вершине $i$.\textbf{Определение~1.} {\it
Дискретным потоком Риччи называется
система дифференциальных уравнений}

$$
\dfrac{dr_i}{dt} = -K_i r_i,
$$
где триангуляция $T$ и веса $W$ считаются
фиксированными.
\newline
В статье [2] доказано, что для любого
набора неотрицательных
$w_{ij}$ и при некоторых ограничениях на
триангуляцию существует и
притом единственная метрика постоянной
кривизны. Далее, в статье [1]
показано, что при неотрицательных $w_{ij}$
для любых начальных
условий нормализованный поток Риччи
сходится к метрике постоянной
кривизны тогда и только тогда, когда эта
метрика существует для
данных $(T, W)$.
\newline
В этой работе рассматриваются примеры
триангуляций с весами, которые
могут принимать и отрицательные
значения.

\textbf{Теорема~1.} {\it Пусть задан многогранник, зафиксируем на нём триангуляцию и набор весов $(T, W)$, и пусть для каждой грани многогранника, с набором весов $ \alpha, \beta, \gamma $ выполнено условие: $\alpha, \beta$ --- неотрицательные, $\gamma < 0$ и $ \alpha \beta + \gamma > 0 $ , тогда дискретный поток Риччи (1) сходится к метрике постоянной кривизны.}

%\begin{center}
%Файл должен компилироватьс без ошибок \\
%и переполнений ("overfull").
%\end{center}
%
%\textbf{Электронную версию тезисов необходимо выслать по
%электронному адресу vzms@mail.ru.}

%%%% ОФОРМЛЕНИЕ СПИСКА ЛИТЕРАТУРЫ %%%
\smallskip \centerline{\bf Литература}\nopagebreak

1. {\it B.Chow, F. Luo} Combinatorial Ricci flows on surfaces . J. of differential geometry 63 (2003) 97–129

2. {\it A. Marden, B. Rodin} On Thurston’s formulation and proof of Andreev’s theorem, Computational methods and function theory (Valparaiso, 1989), 103–115, Lect. Notes in Math., 1435, Springer, Berlin, 1990

3. {\it A.Akopyan  } Matematicheskoe prosveshchenie, Ser. 3, N 13, 2009, 155–170 (in English: arXiv:1105.2153v1 [math.MG])

4. {\it W. Thurston } Geometry and topology of 3-manifolds, Princeton lecture notes, 1976, http://www.msri.org/publications/books/gt3m/. 