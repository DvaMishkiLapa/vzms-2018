\begin{center}{ \bf   О СЛАБОЙ РАЗРЕШИМОСТИ  ОБОБЩЕННОЙ МОДЕЛИ ВЯЗКОУПРУГОСТИ  ФОЙГТА}\\
{\it Орлов В.П.  } \\
(Воронеж; {\it orlov$\_$vp@mail.ru} )
\end{center}
\addcontentsline{toc}{section}{Орлов В.П. }


В  ограниченной области $\Omega$ в ${R}^N$, $N=2,3$, $\partial\Omega\in C^2$, рассматривается   начально-краевая задача
%%%%5
%\begin{multline}\label{1a}
$${\partial v}/{\partial t}+\sum_{i=1}^N v_i {\partial v}/{\partial x_i}-\mu_0\Delta v-
$$
$$
\mu_1\frac{1}{\Gamma(1-\alpha)}\mathrm{Div}\,\int\limits_{0}^t(t-s)^{-\alpha}\,\mathcal{E}(v)(s, x)ds\,+ \nabla p=
$$
%\end{multline}\begin{equation}\label{2a}
$$f(t,x), \, (t, x) \in Q_T;\ \ \mathrm{div}\,v(t, x)=0, \quad (t, x)\in Q_T;\eqno{(1)}$$
%\end{multline}%{equation}
%\begin{equation}\label{3a}
$$v(0, x)=v^0(x),\  x\in \Omega, \quad
v(t, x)=0, \quad (t, x)\in [0, T]\times \partial\Omega.\eqno{(2)}$$
%\end{equation}
Здесь  $v(t,x)=(v_1(t,x),\ldots,v_N(t,x))$~-- вектор скорости частицы в точке $x$ области $\Omega$ в момент времени $t,$ $p=p(t,x)$ - давление жидкости в точке $x$ в момент времени $t$,  $f$~-- плотность внешних сил, действующих на жидкость.  ${\rm Div}\,\sigma$ суть вектор, координатами которого являются дивергенции векторов-столбцов матрицы $\sigma$, $\mu_0>0$, $\mu_1\ge 0$, $0<\alpha<1$.


Введем функциональное пространство (обозначения $H$  и $V$ см. [1])
$$
W(a,b)\equiv  L_2(a,b;V)\cap L_{\infty}(a,b;H)\cap W_1^1(a,b;V^{-1}). % \text{ при}\ n=2,
$$
\textbf{ Определение.}{\it
 Слабым решением задачи (1)-(2) называется функция $v\in W(0,T)$
удовлетворяющая тождеству
%\begin{multline}\label{1c}
$$
{d}(v, \varphi)/{dt}-\sum_{i=1}^N(v_iv, {\partial \varphi}/{\partial x_i})+\mu_0(\mathcal{E}(v), \mathcal{E}(\varphi))\, +
$$
$$
\mu_1\frac{1}{\Gamma(1-\alpha)}\left(I_{0t}^{1-\alpha}\mathcal{E}(v)(s,x))\,ds,
\mathcal{E}(\varphi)\right)=\langle f,\varphi\rangle\eqno{(2.1)}
$$
при любой $\varphi\in V$  и п.в. $t\in[0,T]$ и условию  (2).
}

Сформулируем основные результаты.

\textbf{ Теорема 1.}
{\it Пусть $f\in L_2(0,T;V^{-1})$, %$v^0\in V^{-1}$
$v^0\in H$. Тогда задача (1)-(2) имеет по крайней мере одно слабое решение.}

\textbf{ Теорема 2.}
{\it При  $N=2$ в условиях теоремы 1 %. Пусть $f\in L_2(0,T;V^{-1})$, $v^0\in H$. Тогда
 слабое решение задачи (1)-(2) единственно.}

Для доказательства Теоремы 1 строятся последовательные приближения $v^n$, $n=1,2,\dots$, определяемые как слабые решения вспомогательных задач
%\begin{equation}\label{1d}
$$
{\partial v^n}/{\partial t}+\sum_{i=1}^Nv_i^n(1+n^{-1}|v^n|^2)^{-1}{\partial v^n}/{\partial x_i}-\mu_0\Delta v^n+\nabla p^n=w^n ;
\eqno{(3)}
$$
$$
\mathrm{div}\,v^n=0;\ \ v^n(0, x)=v^0(x), \ x\in\Omega; \quad v^n|_{[0,T]\times\partial \Omega}=0.
$$
Здесь $|z|=(\sum_{i=1}^N z_i^2)^{1/2}$ для $z=(z_1,\dots,z_N)$,  приближение $v^0(t, x)$ определяется как $v^0(t, x)=v^0(x)$, а
%\begin{equation}\label{4d}
$$w^n=f+\mu_1\frac{1}{\Gamma(1-\alpha)}\mathrm{Div}\,\int\limits_{0}^t(t-s)^{-\alpha}\,\mathcal{E}(v^{n-1})(s, x)\,ds.$$
%\end{equation}
Обозначим через
 $W^*(0, T)$ пространство
 $$
   L_2(0, T; V)\cap C_w([0, T]; H)\cap L_\infty(0, T; H)\cap W_2^1(0, T; V^{-1}) .
 $$
 %банахово пространство с естественной нормой %пересечения.

Под слабым решением задачи (3) будем понимать  функцию $v^n\in  W^*(0,T)$,
%$v(t, x)\in L_2(0, T; V)\cap C_w([0, T]; H)\cap L^\infty(0, T; H)\cap W_1^1(0, T; V^{-1})\equiv W$,
удовлетворяющую тождеству
$$
{d}(v^n, \varphi)/{dt}-\sum_{i=1}^N(v_i^n(1+n^{-1}|v|^2)^{-1}v^n,{\partial \varphi}/{\partial x_i})+
$$
$$
\mu_0(\mathcal{E}(v^n), \mathcal{E}(\varphi))=\langle w^n, \varphi\rangle,%\eqno{(3.4)}
$$
%\end{equation}
при любой $\varphi\in V$ и почти всех $t$ и начальному условию (2).

Оценки решений $v^n$ задачи (3) при  достаточно малом $T$
%\begin{equation}\label{d9}
$$\sup_t|v^n(t, \cdot)|_H+\|v^n\|_{L_2(0, T; V^{1})}\le M(\|f\|_{L_2(0, T; V^{-1})}+|v^0|_H),\eqno{(4)}$$
%\end{equation}
%\begin{equation}\label{d9+}
$$\|\partial v^n/\partial t\|_{L_1(0,T; V^{-1})}\le M(\|f\|_{L_2(0, T; V^{-1})}+|v^0|_H+1)^2\eqno{(5)}$$
%\end{equation}
%Здесь  $C(s)$ некоторая не зависящая от $n$ функция переменной $s$.
позволяет перейти к пределу в определяющем слабое решение тождестве и получить разрешимость задачи  (1)-(2) при малом $T$.

 Разрешимость задачи  (1)-(2) при произвольном  $T$ устанавливается с помощью продолжения решения на последовательные равные промежутки с использованием
 не зависящих от промежутков оценок решений приближенных задач.

Для доказательства Теоремы 2 сначала устанавливается однозначная разрешимость задачи  (1)-(2) при малом  $T$  на основе априорных оценок решений задачи  (1)-(2) в плоском случае. Затем, используя методику доказательства  Теоремы 1 единственность  устанавливается на произвольном промежутке $[0,T]$.



Работа выполнена совместно с М.А. Плиевым и Д.А.Роде.

Работа поддержана  РФФИ (грант 16-01-00370).

%%%%  ОФОРМЛЕНИЕ СПИСКА ЛИТЕРАТУРЫ %%%
\smallskip \centerline{\bf Литература}\nopagebreak
1. {\it Темам Р.}  Уравнения Навье-Стокса. М.: Мир,  1981.

2. {\it Орлов В. П.,  Роде Д. А., Плиев М. А.} О слабой разрешимости обобщенной
модели вязкоупругости  Фойгта// Сибирский математический журнал, 2017, Том 58, № 5, с.1110-1127.
