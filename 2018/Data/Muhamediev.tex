\begin{center}{ \bf О ПОЛОЖИТЕЛЬНЫХ ПЕРИОДИЧЕСКИХ РЕШЕНИЯХ ОДНОЙ МОДЕЛЬНОЙ СИСТЕМЫ НЕЛИНЕЙНЫХ ДИФФЕРЕНЦИАЛЬНЫХ УРАВНЕНИЙ}\\
{\it Э. Мухамадиев, А.Н. Наимов, М.К. Собиров} \\
(Вологда; {\it emuhamadiev@rambler.ru, nan67@rambler.ru}, Душанбе;
{\it msobirov@yahoo.com})
\end{center}
\addcontentsline{toc}{section}{Мухамадиев Э.М., Наимов А.Н., Собиров М.К.\dotfill}

Статья посвящена исследованию положительных $\omega$-периодических
решений следующей системы нелинейных обыкновенных дифференциальных
уравнений:
$$
\left\{
\begin{array}{l}
  x'(t)=c(t)(Y(t)-y(t))x(t)-k_1(t)x(t), \\
  y'(t)=a(t)(Y(t)-y(t))x(t)-k_2(t)y(t). \\
\end{array}
\right.
   \eqno (1)
$$
Предполагается, что функции $c(t)$, $a(t)$, $Y(t)$, $k_1(t)$,
$k_2(t)$ известны, $\omega$-периодичны и непрерывны.


Система уравнений (1) в случае, когда функции $c(t)$, $a(t)$,
$Y(t)$, $k_1(t)$, $k_2(t)$ постоянны и положительны исследована в
работах [1, 2]. В этих работах анонсировано и доказано, что если
$cY>k_1$, то любое положительное решение системы уравнений (1) при
неограниченном возрастании $t$ приближается к точке $(x_*, y_*)$,
где
$$
y_*=Y-\frac{k_1}{c}, \qquad x_*=\frac{k_2y_*}{a(Y-y_*)}.
$$

В настоящей работе найдены условия, при которых множество
положительных $\omega$-периодических решений системы уравнений (1)
ограничено по норме пространства $C[0, \omega]$ и непусто.

Справедливы следующие две теоремы.

\textbf{Теорема~1.} {\it Пусть функции  $c(t)$, $a(t)$, $Y(t)$,
$k_1(t)$, $k_2(t)$ положительны, $\omega$-периодичны и непрерывны,
и пусть $c(t)Y(t)>k_1(t)$. Тогда существуют положительные числа
$B_1$, $B_2$, $D_1$, $D_2$, зависящие лишь от максимальных и
минимальных значений функций $c(t)$, $a(t)$, $Y(t)$, $k_1(t)$,
$k_2(t)$ и модуля непрерывности функции $Y(t)$, такие, что для
любого положительного $\omega$-периодического решения системы
уравнений (1) имеют место оценки}
$$
B_1<x(t)<B_2, \qquad D_1<y(t)<D_2. \eqno (2)
$$


\textbf{Теорема~2.} {\it Если выполнены условия теоремы 1, то
существует хотя бы одно положительное $\omega$-периодическое
решение системы уравнений (1).}

Существование положительного $\omega$-периодического решения
системы уравнений (1) доказано с применением аппарата вычисления
вращения конечномерных векторных полей ([3, 4, 5]) следующим
образом. Рассмотрим двумерное векторное поле
$$
\Phi(x_0,y_0)\equiv(x_0,y_0)-(\varphi(\omega,x_0,y_0),\psi(\omega,x_0,y_0)),
$$
где $(\varphi(t,x_0,y_0),\psi(t,x_0,y_0))$ - решение системы
уравнений (1) с начальными значениями $x_0$ и $y_0$. Отметим, что
в условиях теоремы 1 решение
$(\varphi(t,x_0,y_0),\psi(t,x_0,y_0))$ определено и положительно
при всех положительных $t$, $x_0$, $y_0$. Из теоремы 1 следует,
что двумерное векторное поле $\Phi$ не обращается в ноль на
границе прямоугольника $\Pi=[B_1, B_2]\times[D_1, D_2]$. В ходе
доказательства теоремы 2 установлено, что вращение векторного поля
$\Phi$ на границе прямоугольника $\Pi$ равно $1$. Отсюда, в силу
принципа ненулевого вращения векторных полей ([3]), вытекает
существование положительного $\omega$-периодического решения
системы уравнений (1).


\smallskip \centerline{\bf Литература}\nopagebreak

1. {\it  Горский~А.~А., Локшин~Б.~Я., Розов~Н.~Х. } Режим
обострения в одной системе нелинейных уравнений // Дифференц.
уравнения. - 1999. - Т.~35, №~11. - С.~1571.


2. {\it Мухамадиев~Э., Наимов~А.~Н., Собиров~М.~К. } Исследование
положительных решений динамической модели производства и продажи
товара // Современные методы прикладной математики, теории
управления и компьютерных технологий: сборник трудов X междунар.
конф. «ПМТУКТ-2017».  - Воронеж: Изд-во  «Научная книга», 2017. -
С.~268–271.


3. {\it Красносельский М.А., Забрейко П.П. } Геометрические методы
нелинейного анализа. - М.: Наука, 1975. - 512 с.


4. {\it Мухамадиев~Э., Наимов~А.~Н.} К теории двухточечных краевых
задач для обыкновенных дифференциальных уравнений второго порядка
// Дифференц. уравнения. - 1999. - Т.~35, №~10. - С.~1372-1381.


5. {\it Мухамадиев~Э., Наимов~А.~Н.} Критерий разрешимости одного
класса нелинейных двухточечных краевых задач на плоскости
// Дифференц. уравнения. - 2016. - Т.~52, №~3. - С.~334-341.


