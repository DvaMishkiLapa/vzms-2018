\begin{center}{ \bf  ТОПОЛОГИЧЕСКИЕ ИНВАРИАНТЫ ГАМИЛЬТОНОВЫХ СИСТЕМ}\\
{\it К.И. Солодских } \\
(Москва; {\it solodskihkirill@gmail.com} )
\end{center}
\addcontentsline{toc}{section}{Солодских К.И.\dotfill}


Если гамильтонова система с двумя степенями свободы является интегрируемой по лиувиллю при помощи боттовского интеграла $F$, то она допускает качественное описание в терминах инвариантов Фоменко-Цишанга (подробнее см. [1]). Оказвается класс изоэнергитических многообразий гамильтоновых систем интегрируемых при помощи боттовского интеграла совпадает с классом граф-многообразий Вальдхаузена(см. [2]). Изучение топологии этих многообразий очень важно, так как это позволяет сильно ограничить лиувиллевы слоения, которые могут возникнуть в конкретной задаче. Многие топологические свойства гамильтоновых систем были изучены в [3]. В данной работе демонстрируется  применение кручения Рейдемейстера для установления топлогического типа изоэнергетического многообразия.

\textbf{Определение 1.} Молекула интегрируемой системы называется простой, если она имеет всего один седловой атом без звездочек, который является плоским.

Фундаментальная группа таких многообразий имеет следующее копредставление
$$
\pi_1(\mathbf{Q}^3) = \bigm\langle \lambda_V, \mu_1, \dots\, \mu_{m} \bigm|  [\lambda_V, \mu_i], \lambda_V^{\alpha_i}\mu_i^{\beta_i} ,~ \mu_1\dots\mu_m,\\ i = 1,\dots, m \bigm\rangle,
$$
где $\lambda_V, \mu_1, \dots, \mu_m$ -- образующие в фундаментальной группе седлового атома $V$, которые соответствуют гомотопным допустимым циклам на граничных торах, $\alpha_i, \beta_i$ -- элементы матриц склейки
$$
C_i = \begin{pmatrix} \alpha_i && \beta_i \\ \gamma_i && \delta_i \end{pmatrix},\quad i = 1,\dots, m.
$$

\textbf{Теорема 1 [см. 4].} {\it Пусть гомоморфизм колец в поле
$$
h \colon \mathbb{Z}[\pi_1(\mathbf{Q}^3] \rightarrow \mathbb{F},
$$
такой, что $h(\lambda_V)^{\gamma_k}h(\mu_k)^{\delta_k} \ne 1, k = 1,\dots, m$. Тогда кручение многообразия $\mathbf{Q}^3$ не равно $0$ в том и только том случае, когда~$h(\lambda_V)~\ne~1$. Если~$h(\lambda_V)~\ne~1$, то
$$
\tau_h(\mathbf{Q}^3) = (h(\lambda_V) - 1)^{m-2}{\underset{k=1}{\overset{m}{\prod}}(h(\lambda^{\gamma_k}_V\mu^{\delta_k}_k) - 1)^{-1}} \in \mathbb{F}^*/\pm h(\pi_1(\mathbf{Q}^3)).
$$}

Пусть все $r$-метки простой молекулы равны нулю, а $n$-метка не равна $0, 1, -1$. Без ограничения общности матрицы склейки в этом случае имеют следующий вид
$$
C_m = \begin{pmatrix} n && \varepsilon_m \\ \varepsilon_m && 0 \end{pmatrix}, \quad C_i = \begin{pmatrix} 0 && \varepsilon_i \\ \varepsilon_i && 0 \end{pmatrix},\quad i = 1,\dots, m-1.
$$
Фундаментальная группа многообразия $\mathbf{Q}^3$ является циклической группой порядка $n$
$$
\pi_1(\mathbf{Q}^3) = \bigm\langle \lambda_V \bigm| \lambda_V^n \bigm\rangle,
$$
поэтому многообразие $\mathbf{Q}^3$ гомеоморфно некоторой линзе $L(n, q)$.
\\

\textbf{Следствие.}{ \it
Многообразие $\mathbf{Q}^3$ гомеоморфно линзе $L(n, 1)$.}
\\

Далее рассмотрим такие простые молекулы, что только две $r$-метки не равны 0. Без ограничения общности можно считать, что матрицы склейки следующие
$$
C_1 = \begin{pmatrix} \alpha_1 && \beta_1 \\ \gamma_1 && \delta_1 \end{pmatrix}, \quad
C_2 = \begin{pmatrix} \alpha_2 && \beta_2 \\ \gamma_2 && \delta_2 \end{pmatrix},
\quad C_i = \begin{pmatrix} 0 && \varepsilon_i \\ \varepsilon_i && 0 \end{pmatrix},
$$
$$
i = 3,\dots, m.
$$
Фундаментальная группа таких многообразий имеет следующее копредставление
$$
\pi_1(\mathbf{Q}^3) = \bigm\langle \lambda_V, \mu_2 \bigm|  [\lambda_V, \mu_1],
\lambda^{\alpha_1}_V\mu^{-\beta_1}_2, \lambda^{\alpha_2}_V\mu^{\beta_2}_2 \bigm\rangle.
$$
\textbf{Следствие.} {\it
Многообразие $\mathbf{Q}^3$ гомеоморфно линзе $L(p, q)$, где
$$
p = \alpha_1\beta_2 + \alpha_2\beta_1, \quad q = \alpha_1\gamma_2~+~\beta_1\delta_2.
$$
}
\smallskip \centerline{\bf Литература}\nopagebreak

1. {\it Болсинов А.В., Фомнеко А.Т.}  Интегрируемые Гамильтоновы системы. Геометрия, топология, классификация. Том 1. -- Ижевск.: Издательский дом <<Удмуртский университет>>, 1999. - 444 С.

2. {\it Матвеев С.В., Фомнеко А.Т.} Алгоритмические и компьютерные методы в трехмерной топологии.  - М.: Наука, 1998. - 304 С.

3. {\it Фоменко А. Т., Цишанг Х.} “О типичных топологических свойствах интегрируемых гамильтоновых систем”, Изв. АН СССР. Сер. матем., 52:2 (1988), 378–407

4. {\it Солодских К. И.} “Граф-многообразия и интегрируемые гамильтоновы системы”, Математическиий сборник(в печати) 2017 г.

