\begin{center}{ \bf  О КОЛИЧЕСТВЕ ГОЛОМОРФНО-ОДНОРОДНЫХ ГИПЕРПОВЕРХНОСТЕЙ В $\mathbb{C}^3$\footnote{Работа выполнена при частичной финансовой поддержке РФФИ (проект \No\ 17-01-00592-a).}}\\
{\it А.В. Атанов, А.В. Лобода } \\
(Воронеж; {\it lobvgasu@yandex.ru} )
\end{center}
\addcontentsline{toc}{section}{Атанов А.В., Лобода А.В.\dotfill}

В сообщении обсуждается один фрагмент сложной задачи описания голоморфно-однородных вещественных гиперповерхностей 3-мерных комплексных пространств. Рассматриваются строго псевдо-выпуклые (СПВ) гиперповерхности с дискретными группами изотропии. Такая поверхность определяется не более чем тремя вещественными коэффициентами (см. [1]) своего нормального мозеровского уравнения (см. [2]), если
многочлен 4-й степени
$
   N_{220} $ из этого уравнения имеет специальный вид
$$
   N_{220} = \varepsilon (|z_1|^4 -  4 |z_1|^2 |z_2|^2 +  |z_2|^4), \ \varepsilon = \pm 1.
\eqno (1)
$$

При этом известны (см. [3]) семейства аффинно-раз\-лич\-ных аф\-финно-однородных СПВ-ги\-пер\-по\-верх\-но\-стей 3-мер\-ных комплексных пространств, оп\-ределяемые тремя вещественными параметрами. Можно показать, что после голоморфной нормализации многочлен $ N_{220} $ в уравнениях этих поверхностей имеет именно вид (1). Представляется естественной ситуация, в которой три вещественных параметра, позволяющие устанавливать аффинное различие поверхностей, <<удерживали>> бы и голоморфные различия этих же многообразий.
Например, известным аффинно-однородным трубчатым поверхностям (трубкам) из двухпа\-ра\-мет\-ри\-чес\-ко\-го семейства
$$
   Re\,z_3 =  (Re\,z_1)^{\alpha} (Re\,z_2)^{\beta}, \quad \alpha, \beta \in \mathbb{R}
\eqno (2)
$$
соответствуют, как несложно проверить, различные алгебры также из двухпараметрического семейства $ g_{5,33} $ работы [4].
В силу этого и голоморфная эквивалентность (как более слабая по сравнению с абстрактной алгебраической эквивалентностью)
невозможна для различных поверхностей семейства (2), по крайней мере, при малых возмущениях пары $ (\alpha, \beta) $.

Однако, компьютерные вычисления показывают, что в нормальных уравнениях поверхностей из [3], обобщающих известные примеры [5--6], фактически совпадают не только многочлены $ N_{220} $, но и два других многочлена $N_{320}$, $N_{330}$.

Напомним, что именно три вещественных коэффициента пары ($ N_{320} $, $N_{330}$) позволяют определить однородную СПВ-поверхность при выполнении условия (1). В итоге оказываются справедливыми следующие утверждения.

\textbf{Теорема~1.} {\it Все алгебраические СПВ-гиперповерхности 4-го порядка из [3]
$$\begin{array}{c}
    v = x_2y_1 + \dfrac{m}{n}y_1y_2 - \dfrac{1}{2n}x_1y_1^2 - \dfrac{m}{6n^2}y_1^3 - \\[10pt]
    - \dfrac{s(y_1^2 - 2ny_2)^2}{4n(nx_1 - my_1)} + C(nx_1 - my_1)^3,
  \end{array}\eqno{(3)}
$$
где $m^2 + n^2 - ns = 0$, $n < 0$,
голоморфно эквивалентны друг другу и поверхности $v = {x_2^2}/{x_1} + x_1^3$.}


\textbf{Теорема~2.} {\it
Любая из алгебраических СПВ-ги\-пер\-по\-верх\-нос\-тей 6-го порядка из [3]
$$v = x_2y_1 + \frac{m}{n}y_1y_2 - \frac{1}{2n}x_1y_1^2 - \frac{m}{6n^2}y_1^3 + $$
$$+ C\left|(nx_1 - my_1)^2 - N(y_1^2 - 2ny_2)\right|^{3/2} + \eqno{(4)}$$
$$+ \frac{(N + 2ns)(nx_1 - my_1)}{6n^2N^2}\left(2(nx_1 - my_1)^2 - 3N(y_1^2 - 2ny_2)\right),$$
где $n \neq 0$, $N = m^2 + n^2 - ns \neq 0$,
голоморфно эквивалентна аффинно-однородной трубке
$$
  (v - x_1 x_2 - x_1^3)^2 = C^2(x_2- x_1^2)^3
\eqno (5)
$$
с тем же, что и в формуле (4), значением параметра $C$.}

\textbf{Замечание.} {\it Всем трубкам (5) отвечает одна и та же алгебра Ли $ g_{5,33} $ из работы [4].}

\


\smallskip \centerline{\bf Литература}\nopagebreak

1. {\it Лобода А.В.} Коэффициентный анализ в задаче описания голоморфно-однородных гиперповерхностей в $\mathbb{C}^3$ / А. В. Лобода
// Междунар. конф., посв. 100-летию С.Г.Крейна. -- Воронеж, 2017. -- С.~130--131.

2. {\it Лобода А.В.} Однородные строго псевдо-выпуклые гиперповерхности в $\mathbb{C}^3$ с двумерными группами изотропии/
 А.В. Лобода // Матем. сборник. -- 2001. -- Т. 192, №12. -- С.~3--24.

3. {\it Лобода А.В.} О различных способах представления матричных алгебр Ли, связанных с однородными поверхностями / А.~В.~Лобода, В.~К.~Евченко // Изв. вузов. Матем. -- 2013. -- № 4. -- C.~42--60.

4. {\it Мубаракзянов Г.М.} Классификация вещественных структур алгебр Ли пятого порядка / Г.М. Мубаракзянов // Изв. вузов. Матем. -- 1963. -- № 3. -- С.~99--106.

5. {\it Doubrov B.M.} Homogeneous surfaces in the three-di\-men\-si\-onal affine geometry / B.~M.~Doubrov, B.~P.~Komrakov, M.~Ra\-bi\-no\-vich // Geometric Topology of Submanifolds VIII, World Scientific. -- 1996. -- P.~168--178.

6. {\it Beloshapka V.K.} Homogeneous hypersurfaces in $ \Bbb C^3 $, asso\-ciated with a model CR-cubic / V.~K.~Beloshapka, I.~G.~Kossov\-skiy // J. Geom. Anal. -- 2010. -- V. 20, № 3. -- P.~538--564.


