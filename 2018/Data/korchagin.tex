\begin{center}{ \bf  О ТОПОЛОГИЧЕСКОЙ СЛОЖНОСТИ $C^*$-АЛГЕБР}\\
{\it Корчагин А.И. } \\
(Санкт-Петербург; {\it mogilevmedved@yandex.ru} )
\end{center}
\addcontentsline{toc}{section}{Корчагин А.И.\dotfill}


В работе [1] была введена топологическая сложность $TC(X)$ для компакта $X$. Она является любопытным гомотопическим инвариантом, а так же полезна при изучении механических систем, если в качестве $X$ брать конфигурационное пространство. В работе [2] был построен некоммутативный вариант этого понятия для $C^*$-алгебр. По определению топологическая сложность $TC(A)$ унитальной $C^*$-алгебры $A$ это такое наименьшее число $n$ (или бесконечность, если такого числа не существует), что существуют $n$ эпиморфизмов $\beta_j:A\otimes A\to B_j$ в некоторые $C^*$-алгебры $B_j$ такие, что $\oplus_{j=1}^n\beta_j$ - инъективно, и $\beta_j\circ\alpha_0$ гомотопично $\beta_j\circ\alpha_1$ для любого $j$, где $\alpha_i:A\to A\otimes A$, $\alpha_0(a)=a\otimes1$, $\alpha_1(a)=1\otimes a$. Отметим, что гомотопическая теория $C^*$-алгебр находится ещё на своём раннем этапе развития, а потому полезна любая топологическая информация о $C^*$-алгебрах.

На докладе предполагается рассказать о топологической сло\-жности и её значениях на важнейших примерах $C^*$-алгебр, таких как коммутативные $C^*$-алгебры, AF-алгебры и многие другие. Особое внимание планируется уделить следующим теоремам:

\textbf{Теорема~1.} {\it Пусть $A$~--- унитальная AF-алгебра.
Тогда
\linebreak
${TC(A)=1}$ тогда и только тогда, когда $A$~--- UHF-алгебра. В противном случае $TC(A)=\infty$.}

\textbf{Теорема~2.} {\it Верно равенство $TC(\mathcal{O}_{2n})=1$, где $\mathcal{O}_{2n}$ - алгебра Кунтца.}

До сих пор остаётся открытым вопрос значения $TC(\mathcal{O}_{2n+1})$ в случае <<нечётных>> алгебр Кунтца.

%%%%  ОФОРМЛЕНИЕ СПИСКА ЛИТЕРАТУРЫ %%%
\smallskip \centerline{\bf Литература}\nopagebreak

1. {\it Farber M.} Topological Complexity of Motion Planning, Dis\-c\-re\-te Comput. Geom. 29, 211–221, 2004

2. {\it Manuilov V.} A Noncommutative Version of Farber`s Topo\-lo\-gi\-cal Complexity, Topol. Methods Nonlinear
Anal. 49 (4), 287-298, 2017
