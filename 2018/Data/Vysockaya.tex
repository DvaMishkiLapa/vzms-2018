\begin{center}{ \bf ПОЧТИ ПЕРИОДИЧЕСКИЕ НА БЕСКОНЕЧНОСТИ ФУНКЦИИ КАК РЕШЕНИЯ ДИФФЕРЕНЦИАЛЬНЫХ УРАВНЕНИЙ}\\
{\it И.А. Высоцкая } \\
(Воронеж; {\it i.a.trishina@gmail.ru} )
\end{center}
\addcontentsline{toc}{section}{Высоцкая И.А.}

Пусть $C_b(\mathbb{J},X)$~--- банахово пространство непрерывных ограниченных функций,
определённых на $\mathbb{J}$ со значениями в комплексном банаховом пространстве $X$.

Пусть $C_{b,u}(\mathbb{J},X)$~--- замкнутое подпространство равномерно непрерывных ограниченных функций. Через $C_0(\mathbb{J},X)$ обозначим (замкнутое) подпространство функций $x\in C_b$,
 исчезающих на бесконечности, т.~е. $\lim\limits_{|t|\rightarrow\infty}\|x(t)\|=0$, $x\in C_b(\mathbb{J},X)$.

В пространстве $C_{b}(\mathbb{J},X)$ рассмотрим операторы сдвига $S(t):C_{b}(\mathbb{J},X)\rightarrow C_{b}(\mathbb{J},X),$ $(S(t)x)(\tau) = x(\tau + t),$ $\tau\in\mathbb{J},$ $t\in\mathbb{J},$ $x\in C_{b}(\mathbb{J},X)$.


\textbf{Определение 1.} Функцию  $x$ из $C_{b,u}(\mathbb{J},X)$ назовём \emph{интегрально убывающей на бесконечности},
если $$\lim\limits_{\alpha\rightarrow\infty}\frac{1}{\alpha}\sup\limits_{t\in\mathbb{J}}\int\limits_0^{\alpha}\|x(t+s)\|\,ds=0.$$
Множество интегрально убывающих на бесконечности фун\-к\-ций будем обозначать символом $C_{0,int}=C_{0,int}(\mathbb{J},X)$.

Введённый класс является более широким по сравнению с
классом почти периодических на бесконечности функций, введённым в работах А. Г.
Баскакова [1-3].

\textbf{Определение 2.} Далее символом $\mathcal{C}_0=\mathcal{C}_0(\mathbb{J},X)$ обозначим замкнутое (с нормой из $C_{b,u}$) подпространство функций из $C_{b,u}(\mathbb{J},X)$, обладающих свойствами:

1) $S(t)x\in \mathcal{C}_0$ для любого $t\in\mathbb{J}$ и любой функции $x\in\mathcal{C}_0$;

2) $C_0 \subset \mathcal{C}_0 \subset C_{0,int}$;

3) $e_{\lambda}x \in \mathcal{C}_0$ для любого $\lambda\in\mathbb{R}$, где $e_{\lambda}(t)=e^{i\lambda t},$ $t\in\mathbb{R}$.


\textbf{Определение 3.}
Функция $x\in C_{b,u}(\mathbb{J},X)$ называется медленно
меняющейся на бесконечности функцией, относительно подпространства $\mathcal{C}_0$, если для каждого $\alpha \in \mathbb{J}$ выполнено $S(\alpha)x-x\in \mathcal{C}_{0}$.



Отметим что в работах [2] и [4] давалось определение медленно меняющейся функции с использованием подпространства $C_0=C_{0,int}(\mathbb{R},X)$. Свойства медленно меняющихся функций относительно подпространства $C_0$ также были отмечены в работах [1],[2].

Множество всех медленно меняющихся на бесконечности функций из $C_{b,u}(\mathbb{J},X)$ относительно подпространства $C_{0,int}$ будем обозначать через $C_{sl,int}(\mathbb{J},X)$ и через $C_{sl}(\mathbb{J},X)$ - относительно подпространства $C_0$. Символом $\mathcal{C}_{sl,\mathcal{C}_0}$ будем обозначать подпространство обладающее свойством $C_{sl}(\mathbb{J},X) \subset \mathcal{C}_{sl,\mathcal{C}_0} \subset C_{sl,int}(\mathbb{J},X)$ .

\textbf{Определение 4.}
Пусть $\varepsilon>0$.Число $\omega\in\mathbb{J}$ называется \emph{ $\varepsilon$-периодом функции} $x\in C_{b,u}(\mathbb{J},X)$ на бесконечности относительно подпространства $\mathcal{C}_0(\mathbb{J},X)$ исчезающих на бесконечности функций, если существует функция $x_0\in \mathcal{C}_0$ такая, что $\|S(\omega)x-x-x_0\|<\varepsilon.$
Множество $\varepsilon$-периодов функции $x\in C_{b,u}(\mathbb{J},X)$ обозначим через $\Omega_{\infty}(x;\mathcal{C}_0;\varepsilon)$.


\textbf{Определение 5.}
Подмножество $\Omega$ из $\mathbb{R}$ называется относительно плотным на $\mathbb{J}$, если существует такое $l>0$, что $[t,t+l]\cap\Omega\neq\varnothing$, для любого $t\in\mathbb{J}$.


\textbf{Определение 6} (классическое определение Бора). %7
Фун\-к\-ция $x$ из $C_{b,u}(\mathbb{J},X)$ называется \emph{почти периодической на бесконечности}
относительно подпространства $\mathcal{C}_0$ исчезающих на бесконечности функций,
если для любого $\varepsilon>0$ множество $\Omega_{\infty}(x;\mathcal{C}_0;\varepsilon)$ относительно плотно на $\mathbb{J}$.



Множество почти периодических на бесконечности фун\-к\-ций относительно подпространства $\mathcal{C}_0=\mathcal{C}_0(\mathbb{J},X)$ обозначим символом $AP_{\infty}(\mathbb{J},X;\mathcal{C}_0)$.

Рассмотрим дифференциальное уравнение

$$
\dot{x}(t)-Ax(t)=\psi(t), t\in\mathbb{R},\psi\in C_{0}(\mathbb{R},X) \eqno (1)
$$


Справедлива следующая

\textbf{Теорема 1.} {\it Пусть для оператора $A\in End X$ выполнено условие $\sigma(A)\cap(i\mathbb{R}) = \{i\lambda_1,i\lambda_2...,i\lambda_N\}$, где $i\lambda_1 ,...,i\lambda_N$ - полупростые собственные значения оператора $A$. Тогда Каждое ограниченное решение $x:\mathbb{R}\rightarrow X$ уравнения (1) является почти периодической на бесконечности функцией
$x \in AP_{\infty}(\mathbb{J},X,\mathcal{C}_0)$, которая допускает представление вида

$$x(t)=\sum\limits_{k=1}^N y_k(t)e^{i\lambda_kt}+z_0(t),t\in\mathbb{R},$$

где $y_k\in \mathcal{C}_{sl,\mathcal{C}_0}(\mathbb{R},X), z_0\in \mathcal{C}_{0}(\mathbb{R},X) $.}




%%%%  ОФОРМЛЕНИЕ СПИСКА ЛИТЕРАТУРЫ %%%
\smallskip \centerline{\bf Литература}\nopagebreak

1. {\it Баскаков~А.~Г.} Теория представлений банаховых алгебр, абелевых групп и полугрупп в спектральном анализе линейных операторов/ А.Г.~Баскаков // Функциональный анализ, СМФН. МАИ М. -- 2004. Т.~9. -- C.~3--151.


2. {\it Баскаков~А.~Г., Калужина Н. С.} Теорема Берлинга для функций с~существенным спектром из однородных пространств и стабилизация решений параболических уравнений // Матем. заметки. --2012. -- Т.~92. -- № 5. -- С.~643--661.


3. {\it Баскаков~А.~Г., Калужина Н. С., Поляков Д. М.} Медленно меняющиеся на бесконечности полугруппы операторов // Изв. вузов. Матем. --2014. --№ 7. -- С.~3--14.

4. {\it Тришина И. А.} Почти периодические на бесконечности функции относительно
подпространства интегрально убывающих на бесконечности функций // Изв.
Сарат. ун-та. Нов. сер. Сер. Математика. Механика. Информатика. -- 2017. Т. 17, -- № 4. -- С. 402--418.
