\vzmstitle{Неподвижные точки отображений упорядоченных множеств}

\vzmsauthor{Подоприхин}{Д.\,А.}

\vzmsinfo{Москва; {\it podoprikhindmitry@gmail.com}}

\vzmscaption


Доклад посвящён вопросам существования неподвижных точек и точек совпадения  отображений упорядоченных множеств и состоит из двух частей.
В первой части будут рассмотрены вопросы, касающиеся существования общих неподвижных точек семейства многозначных отображений. В частности, будут изложены
достаточные условия, гарантирующие существование наименьшего элемента во множестве
общих неподвижных точек, а также представлен конструктивный метод итерационного поиска общей неподвижной точки конечного семейства отображений.
Будет показана связь полученных результатов с недавними результатами, представленными в работах [1, 2].


Вторая часть доклада посвящена вопросам сохранения свойства отображения иметь неподвижные точки при упорядоченной гомотопии. Для отображений банаховых пространств известны результаты об инвариантности свойства отображения иметь неподвижную точку при гомотопии (см. [3]). В докладе
данная проблема будет рассмотрена для случая отображений упорядоченных множеств,
где отображения связаны упорядоченной гомотопией. Понятие упорядоченной гомотопии между изотонными отображениями упорядоченных множеств было введено Уолкером
(Walker) в 1983 году в [4]. Также в докладе будут рассмотрены теоремы, полученные в соавторстве с Т.\,Н. Фоменко, о сохранении парой отображений упорядоченных множеств свойства иметь точку совпадения при упорядоченных гомотопиях.

Все изложенные в докладе результаты представлены в работах [5, 6].



\litlist

1. {\it Подоприхин Д.\,А., Фоменко Т.\,Н.} О совпадениях семейств отображений упорядоченных множеств // Доклады Академии наук, 2016. Т. 471. No. 1. С. 16-18.

2. {\it Fomenko T.\,N., Podoprikhin D.\,A.} Common fixed points and coincidences of mapping families on partially ordered sets // Topology and its Applications, 2017.

3. {\it Frigon M.} On continuation methods for contractive and nonexpansive mappings // Recent Advances on Metric Fixed Point Theory, 1996. Т. 48. С. 19-30.

4. {\it  Walker J.\,W.} Isotone relations and the fixed point property for posets // Discrete Mathematics, 1984. Т. 48. No. 2-3. С. 275-288.

5. {\it Podoprikhin D.\,A.} Fixed Points of Mappings on Ordered Sets // Lobachevskii Journal of Mathematics, 2017, Т. 38, No. 6, С. 1069–1074.

6. {\it Подоприхин Д.\,А., Фоменко Т.\,Н.} Сохранение свойства неподвижной точки и свойства совпадения при гомотопии отображений упорядоченных множеств // Доклады академии наук, 2017, Т. 477, No. 4, С. 1–4
