\vzmstitle{ \bf  Некоторые подходы к непараметрической идентификации диффузионных моделей}

\vzmsauthor{{Макарова}}{И.\, А.}

\vzmsinfo{Томск; {\it irina\_ makarova\_mmf@mail.ru}}

\addcontentsline{toc}{section}{Макарова И.А.\dotfill}


Пусть на вероятностном пространстве с фильтрацией $(\Omega, F, (F)_{t\geqslant 0}, P)$ задано стохастическое дифференциальное уравнение следующего вида
 \begin{equation}\label{sec:In.1}
d y_{t}=S(y_{t})\,d t +d w_{t}\,,
 \end{equation}
где $(w_{t})_{t\geqslant 0}$ стандартный винеровский процесс,
начальное значение $y_{0}$~--- некоторая заданная константа,
а $S(\cdot)$~--- неизвестная функция, называемая коэффициентом сноса.
Напомним, что уравнение \eqref{sec:In.1} определяет диффузионный случайный процесс Ито.
Диффузионные процессы широко используются в прикладных задачах биомедицины, экономики, финансовой математики и особенно в физике в задачах обработки сигналов при их передаче по зашумлённым каналам связи.
Задача состоит в том, чтобы оценить функцию $S(x)$, $x\in[a,b]$, по наблюдениям процесса
$(y_{t})_{0\leqslant t\leqslant T}$.

В данной работе рассматриваются некоторые подходы к оцениванию неизвестной функции $S(x)$, $a<x<b$ в смысле среднеквадратической точности:
\begin{equation}\label{sec:In.2}
R({S}_{T},S)=E_{S}\|{S}_{T}-S\|^2\,,
\quad
\|S\|^2=\int^b_{a}\,S^2(x)d x\,,
\end{equation}
где ${S}_{T}$ - оценка
$S$ по наблюдениям $(y_{t})_{0\leqslant t\leqslant T}$ (т.е. измеримая функция относительно $\sigma$ - алгебры, порождённой процессом $(y_{t})_{0\leqslant t\leqslant T}$) . Здесь $E_{S}$ обозначим как математическое ожидание относительно распределения случайного процесса
 $(y_{t})_{0\leqslant t\leqslant T}$ при истинной функции $S$.

 Основной целью является построение адаптивной процедуры выбора модели на основе улучшенных взвешенных оценок МНК для оценивания функции сноса $S$ в \eqref{sec:In.1}. Применение улучшенных оценок в регрессионных моделях стало возможно благодаря работам $[3]$. Такой подход позволяет повысить среднеквадратическую точность статистической идентификации модели \eqref{sec:In.1} по сравнению с оценками по методу наименьших квадратов, предложенных в работе $[1]$. Такой подход позволяет повысить среднеквадратическую точность статистической идентификации модели \eqref{sec:In.1} по сравнению с оценками по методу наименьших квадратов, предложенных в работе $[2]$. В этой статье предложен новый метод построения адаптивных оценок, основанных на использовании усечённых последовательных процедур. Такой метод позволяет свести исходящую задачу к задачи оценивания в дискретной регрессионной модели. А также  даёт возможность получения точных неасиптотических оракульных неравенств для среднеквадратических рисков оценок МНК и доказательства их асимптотических свойств.
 Как и в работе $[2]$ предположим, что на множестве $\Gamma\subseteq\Omega$
задана непараметрическая регрессионная модель
\begin{equation}\label{sec:In.3}
Y_{l}=S(x_{l})+\zeta_{l}\,,\quad  1\leqslant l\leqslant n\,,
\end{equation}
где $\zeta_{l}\,=\,\sigma_{l}\,\xi_{l}\,+\,\delta_{l}$. Расчётные точки
$(x_{l})_{1\leqslant l\leqslant n}$
определим далее. Функция $S(\cdot)$ неизвестна и оценивается по наблюдениям $Y_1,\ldots,Y_n$.

Качество любой оценки  ${S}$ будет измеряться эмпирической квадратической ошибкой
$$
\|{S}-S\|^2_n=({S}-S,{S}-S)_{n}
=\frac{b-a}{n}\sum^n_{l=1}({S}(x_{l})-S(x_{l}))^2\,.
$$
Зафиксируем ортогональный базис $(\phi_{j})_{1\leqslant j\leqslant n}$
\begin{equation}\label{sec:In.6}
(\phi_i\,,\,\phi_{j})_{n}=
\frac{b-a}{n} \sum^n_{l=1}\,\phi_i(x_{l})\phi_{j}(x_{l})= {\bf Kr}_{ij}\,,
\end{equation}
где ${\bf Kr}_{ij}$ - постоянная Кронекера.
Используя этот базис, применим дискретное преобразование к
\eqref{sec:In.3} и получим коэффициенты Фурье:
$$
{\hat \theta}_{j,n}=\frac{b-a}{n}\sum^n_{l=1}\,Y_{l}\phi_{j}(x_{l})\,,
\quad
\theta_{j,n}=\frac{b-a}{n}\sum^n_{l=1}S(x_{l})\,\phi_{j}(x_{l})\,.
$$

Для такой модели предложена взвешенная оценка МНК следующего вида:
\begin{equation}\label{sec:In.7}
\hat {{S}}_{\lambda}(x_{l})\,=\,\sum^n_{j=1}\,\lambda(j)\,{\hat \theta}_{j,n}\,\phi_{j}(x_{l})\,
_{\Gamma}\,,\quad 1\leqslant l\leqslant n\,,
\end{equation}
где $\lambda=(\lambda(1),\ldots,\lambda(n))'$
принадлежит некоторому конечному множеству $\Lambda\subset [0,1]^n$.

Далее предположим, что первые $d \leqslant  n$ координат вектора $\lambda$ равны 1, т.е. $\lambda(j)=1$ для всех $1 \leqslant j \leqslant d$.

Определим новый класс оценок для $S$ в \eqref{sec:In.3}:
\begin{equation}\label{sec:In.8}
 {{S}}^*_{\lambda}(x_{l})\,=\,\sum^n_{j=1}\,\lambda(j)\,{\theta}^*_{j,n}\,\phi_{j}(x_{l})\,
_{\Gamma}\,,\quad 1\leqslant l\leqslant n\,,
\end{equation}
где
\begin{equation}\label{sec:In.9}
{\theta}^*_{j,n}\,=\,\left( 1-\frac{c(d)}{||\widetilde{\theta}_{n}||}
1_{\Gamma} \right)\hat {\theta}_{j,n}
\end{equation}
и
\begin{equation}\label{sec:In.10}
c(d)\,=\,\frac{(d-1){\sigma}^2_*L(b-a)^{\frac{1}{2}}}{n(s^*+({\frac {d \sigma_*}{n} })^{\frac{1}{2}})}.
\end{equation}

Тогда для каждого $a \leqslant x \leqslant b$  положим
\begin{equation}\label{sec:In.11}
{{S}}^*_{\lambda}(x)\,=\,{{S}}^*_{\lambda}(x_{l}) 1_{a \leqslant x \leqslant  x_{l}}
+{{S}}^*_{\lambda}(x_{l}) 1_{x_{l-1} \leqslant x \leqslant  x_{l}}.
\end{equation}

Обозначим разность рисков оценок \eqref{sec:In.11} и \eqref{sec:In.8}, как
 \begin{equation}\label{sec:In.12}
\Delta_n (S)\,:=\,E_S {{||{S}}^*_{\lambda}-S||}^2 _n -E_S {||\hat {{S}}_{\lambda}-S||}^2 _n.
\end{equation}

\textbf{Теорема~1.}
Оценка \eqref{sec:In.11} превосходит по среднеквадратической точности оценку \eqref{sec:In.8}, т.е.
\begin{equation}
\sup \Delta_n (S) \leqslant -c^2 (d).
\end{equation}


\litlist

1. {\it Galtchouk L., Pergamenshchikov S.} Nonparametric sequential estimation of the drift in diffusion processes // Mathematical Methods of Statistics. 2004. V. 13. No. 1. P. 25-49.

2. {\it Galtchouk L., Pergamenshchikov S.} Asymptotic efficient sequential kernel estimates of the drift coefficient in ergodic diffusion processes // Statistical Inference for Stochastic Process. 2006. V.9. No. 1. P. 1-16.

3. {\it Pchelintsev E.} Improved estimation in a non-Gaussian parametric regression //
Statistical Inference for Stochastic Processes. 2013. V. 16. No. 1. P. 15-28.

4. {\it Kutoyants Yu.}Statistical inference for ergodic diffusion processes. Sprinter-Verlag, London.2004.

