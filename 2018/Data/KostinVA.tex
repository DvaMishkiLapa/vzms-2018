\vzmstitle{К 100-летию Воронежского государственного университета. О Юрьевско-Воронежском очаге просвещения}

\vzmsauthor{{Костин}}{В.\, А.}

\vzmsinfo{Воронеж; {\it vlkostin@mail.ru}}

\vzmscaption


Доклад посвящён истории Воронежского университета,
начиная с февраля 1918 года с момента оккупации немецкими войсками города Юрьева (Дерпта).
После чего Русский императорский университет, открытый в Прибалтике императором Александром I в 1802 году,
был эвакуирован в Воронеж.
На основе сохранившихся документов проводится анализ действий администрации и Учёного совета Юрьевского университета во главе с его ректором доктором чистой математики, профессором Виссарионом Григорьевичем Алексеевым и доказывается незаконность оккупационных германских властей сделать университет немецким. В связи с этим приводятся данные как в военной обстановке за три дня до занятия немцами Юрьева Совет университета на экстренном заседании избрал город Воронеж местом своей дальнейшей деятельности. Отмечается, что этот выбор был сделан из числа различных городов, при этом указывается авторитетное мнение В.Г.  Алексеева, как председателя Совета и как ректора трижды избранного на этот пост. Указываются меры, проводимые Советской властью, позволившей успешно эвакуировать 4/5 учебно-преподавательского персонала и имущества университета из оккупированной зоны, что послужило основой для создания Воронежского университета в условиях гражданской войны.

\litlist

1. История отечественной математики (отв.редактор И.З.~Штокало) Т.3. Наукова думка, 1968. 728с.\\
2. {\bf Карпачев М.Д.} Воронежский университет: Начало пути / М.Д. Карпачев. – Воронеж: Изд. ВГУ. 1998. – 112 с.\\
3. {\bf Карпачев М.Д.} Воронежский университет: Вехи истории 1918 - 2003/ М.Д.Карпачев. – Воронеж: Изд. ВГУ. 2003. – 472 с.\\
4. {\bf Карпачев М.Д.} Воронежский университет: Вехи истории 1918 - 2013/ М.Д.Карпачев. – Воронеж: Изд. ВГУ. 2013. – 560 с.\\
5. {\bf Сент-Илер К.К.} К истории воронежского университета/ Труды воронежского университета 1925 Т.1\\
6. {\bf Сент-Илер К.К.} К истории воронежского университета/ Воронеж. Издательский Дом ВГУ. 2016. – 48 с.\\
7. Воронежский университет. Страницы истории. Хронология / Ред. И.И. Борисов, С.А. Запрягаев. – Воронеж: Изд. ВГУ. 2003. – 168 с.\\
8. Рождённый революцией: Документы, воспоминания / Сост. доц. Кройчик, отв.ред. проф. В.В. Гусев. – Воронеж: Изд. ВГУ. 1988.\\
9. {\bf Костин В.А., Сапронов Ю.И., Удоденко Н.Н.} Профессор Виссарион Григорьевич Алексеев и Воронежский университет /Воронеж: Типография Концерна ОАО «Созвездие». 2012. –  83 с.\\
10. {\bf Костин В.А., Сапронов Ю.И., Удоденко Н.Н.} Виссарион Григорьевич Алексеев – забытое имя в математике (1866-1943) Вестник ВГУ, серия «Физика, математика» - 2003. №1.  с. 132-152\\



