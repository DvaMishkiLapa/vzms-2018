\begin{center}{ \bf  ИНТЕГРИРУЕМЫЕ ДИНАМИЧЕСКИЕ СИСТЕМЫ С ДИССИПАЦИЕЙ}\\
{\it М.В. Шамолин } \\
(Москва; {\it shamolin@rambler.ru,~shamolin@imec.msu.ru})
\end{center}
\addcontentsline{toc}{section}{Шамолин М.В.\dotfill}

Работа посвящена новым случаям интегрируемости систем на касательном
расслоении к конечномерной сфере. К такого рода задачам приводятся
системы из динамики многомерного твердого тела, находящегося в
неконсервативном поле сил. Исследуемые задачи описываются
динамическими системами с переменной диссипацией с нулевым средним
[1]. Обнаружены случаи интегрируемости уравнений движения в
трансцендентных (в смысле классификации их особенностей) функциях и
выражающихся через конечную комбинацию элементарных функций [2, 3].

Построение неконсервативного силового поля, действующего на
закрепленное многомерное твердое тело, опирается на результаты из
динамики реальных закрепленных твердых тел, находящихся, в поле силы
воздействия среды. Становится возможным изучение уравнений движения
для многомерного тела в аналогично построенном поле сил и получение
полного набора, вообще говоря, трансцендентных первых интегралов,
выражающихся через конечную комбинацию элементарных функций.
Полученные результаты особенно важны в смысле присутствия в системе
именно неконсервативного поля сил, поскольку ранее другими авторами
использовалось поле сил лишь потенциальное.



В общем случае построить какую-либо теорию интегрирования
неконсервативных систем (хотя бы и невысокой размерности) довольно
затруднительно. Но в ряде случаев, когда исследуемые системы
обладают дополнительными симметриями, удается найти первые интегралы
через конечные комбинации элементарных функций.

Получены новые случаи интегрируемости неконсервативных динамических
систем, обладающих нетривиальными симметриями. При этом почти во
всех случаях интегрируемости каждый из первых интегралов выражается
через конечную комбинацию элементарных функций, являясь одновременно
трансцендентной функцией своих переменных. Трансцендентность в
данном случае понимается в смысле комплексного анализа, когда после
продолжения данных функций в комплексную область у них имеются
существенно особые точки. Последний факт обуславливается наличием в
системе притягивающих и отталкивающих предельных множеств (как,
например, притягивающих и отталкивающих фокусов или узлов,
предельных циклов).



Рассматриваемые ранее автором задачи из динамики $n$-мерного
твердого тела в неконсервативном силовом поле породили системы на
касательном расслоении к $(n-1)$-мерной сфере. В работе будет
тщательно разобран индуктивный переход от систем на касательных
расслоениях к маломерным сферам до систем на касательных расслоениях
к сферам произвольной размерности. При этом исследование начинается
для систем при отсутствии силового поля и продолжается системами при
наличии некоторых неконсервативных силовых полей (см. также [2, 3]).

В качестве примера рассмотрим системы вида
$$
%\begin{equation}\label{s.1p}
%\left\{
\dot{\alpha}=-z+bg(\alpha),~\dot{z}=F(\alpha),\eqno (1)
%\right.
%\eqno(1)
%\end{equation}
$$
на двумерном цилиндре --- касательном расслоении
$T_*\mathbf{S}^1\{z;\alpha\}$ к одномерной сфере
$\mathbf{S}^1\{\alpha:~\alpha~\textrm{mod}~2\pi\}$.

Функции $F(\alpha)$ и $g(\alpha)$ --- периодические, достаточно
гладкие и определяют силовое поле. Первое уравнение системы (1)
задает координату $z$ в касательном пространстве к сфере (является
кинематическим соотношением).

Система (1) также может быть представлена в виде маятникового
уравнения
%\begin{equation}\nonumber%\label{s.5-9-0g}
$ \ddot{\alpha}-bg'(\alpha)\dot{\alpha}+F(\alpha)=0.
%\end{equation}
$

При $b=0$ система (1) является консервативной и обладает одним
(полным набором) гладким первым интегралом:
%$$
\begin{equation}\nonumber%\label{s.1}
F_1(z;\alpha)=z^2+z_2^2+2\int_{\alpha_0}^\alpha
F(\xi)d\xi=C_1=\textrm{const}.
\end{equation}



При $b>0$ система (1) перестает быть консервативной и является
динамической системой с переменной диссипацией с нулевым средним
[1--3].

%\begin{rem}\label{rem1}
\textbf{Замечание.} {\it В случае если
$g(\alpha)=F(\alpha)/\cos\alpha$, то система (1) описывает
плоскопараллельное движение твердого тела во внешнем силовом поле
$F(\alpha)$, а также под действием следящей силы [2]. В частности,
если $
%\begin{equation}\nonumber%\label{n.5p}
%\begin{aligned}%{c}
F(\alpha)=\sin\alpha\cos\alpha,$ $g(\alpha)=\sin\alpha,
%\end{aligned}
%\eqno(5)
%\end{equation}
$
то система (1) описывает также плоский (цилиндрический) маятник,
помещенный в поток набегающей среды [2], и обладает одним (полным
набором) трансцендентным первым интегралом, выражающимся через
конечную комбинацию элементарных функций. Трансцендентность в данном
случае понимается в смысле комплексного анализа, когда функция имеет
существенно особые точки, соответствующие имеющимся притягивающим
или отталкивающим предельным множествам системы. }
%\end{rem}

Введем ограничение на силовое поле для полной интегрируемости
системы.

%\begin{thm}\label{thm0}
\textbf{Теорема.} {\it
Если существует такая постоянная
$\lambda\in\mathbf{R}$, что выполнено равенство
$
F(\alpha)=\lambda g(\alpha)g'(\alpha),
$
то при $b\ne 0$ система (1) обладает одним (полным набором) первым
интегралом (вообще говоря, трансцендентным) следующего вида:
$$
\Phi_1\left(g(\alpha),\frac{z}{g(\alpha)}\right)=g(\alpha)\exp\left\{\int\frac{(u-b)du}{u^2-bu+\lambda}\right\}=
$$
$$=C_1=\textrm{const},~u=\frac{z}{g(\alpha)}.$$
}
%\end{thm}

Если функция $g(\alpha)$ не является периодической, то
рассматриваемая диссипативная система является системой с переменной
диссипацией с ненулевым средним (т.е. она является собственно
диссипативной). Тем не менее, и в этом случае (благодаря теореме)
можно получить явный вид трансцендентных первых интегралов,
выражающихся через конечную комбинацию элементарных функций.
Последнее также является новым нетривиальным случаем интегрируемости
диссипативных систем в явном виде (см. также [1--3]).



%Для нумерации формул используйте, пожалуйста, команду \verb"\eqno".

Работа выполнена при финансовой поддержке РФФИ (проект
\No~15--01--00848--а).

%\textbf{Электронную версию тезисов необходимо выслать  по
%электронному адресу vzms@mail.ru.}

%%%%  ОФОРМЛЕНИЕ СПИСКА ЛИТЕРАТУРЫ %%%
\smallskip \centerline{\bf Литература}\nopagebreak

%1. {\it Шамолин М.В.} Методы анализа динамических систем с
%переменной диссипацией в динамике твердого тела. М.:~Экзамен, 2007.
%-- 352 с.

1. {\it Шамолин М.В.} Интегрируемые системы с переменной диссипацией
на касательном расслоении к многомерной сфере и приложения //
Фундам. и прикл. матем. -- 2015. -- Т.~20. -- Вып.~4. -- С.~3--231.

2. {\it Шамолин М.В.} Маломерные и многомерные маятники в
неконсервативном поле. Часть 1 // Итоги науки и техн. Сер. Соврем.
мат. и ее прил. Темат. обз. -- 2017. -- Т.~134. -- С.~6--128.

3. {\it Шамолин М.В.} Маломерные и многомерные маятники в
неконсервативном поле. Часть 2 // Итоги науки и техн. Сер. Соврем.
мат. и ее прил. Темат. обз. -- 2017. -- Т.~135. -- С.~3--93.




%2. {\it Львовский С. М.} Набор и верстка в системе LATEX. М.: МЦНМО, 2003. — 448 с.




