\vzmstitle{
	СУЩЕСТВОВАНИЕ ПЕРИОДИЧЕСКОГО РЕШЕНИЯ У НЕЛИНЕЙНЫХ
	ДИФФЕРЕНЦИАЛЬНЫХ УРАВНЕНИЙ ВЫСШИХ ПОРЯДКОВ
}

\vzmsauthor{Каверина}{В.\,К.}

\vzmsinfo{Воронеж; {\it  lera\_evk@mail.ru}}

\vzmsauthor{Перов}{А.\,И.}

\vzmsinfo{Воронеж; {\it anperov@mail.ru}}

\vzmscaption


В работе изучается вопрос существования периодических решений у возмущённого   дифференциального уравнения.
Полученные теоремы навеяны проблемой В.И. Зубова  [1], [3, с. 220]
и связаны с понятиями асимптотической устойчивости в целом и устойчивости по Дирихле.


Рассмотрим обыкновенное нелинейное дифференциальное
\linebreak
скалярное  уравнение  $n$-го порядка следующего вида

$$x^{(n)}=f(x,\dot{x},\ldots, x^{(n-1)}), \eqno(1)$$
где $f(x_1,x_2,\ldots, x_n):\mathbb{ R}\times\ldots\times \mathbb{ R}$ ($n$-раз) $\rightarrow \mathbb{ R}$ обладает следующими свойствами

$1^0$. $f(0,\ldots,0)=0$,

$2^0$. $f(x_1,0,\ldots,0)\neq 0$ при $x_1\neq 0$,

$3^0$. $f(x_1,x_2,\ldots, x_n) $ непрерывная функция,

$4^0$. $f(x_1,x_2,\ldots, x_n)$ локально липшицева
$$\|f(x_1,x_2,\ldots, x_n) - f(y_1,y_2,\ldots, y_n) \|\leqslant \sum_{i=1}^{n}l_i|x_i-y_i|, \eqno(2)$$
где $x,y\in K$, где $K$ -- любое ограниченное множество в фазовом пространстве $\mathbb{ R}^n$, причём
$l_i=l_i(K)$, $i=1,\ldots, n$.


Одно уравнение $n$-го порядка (1) равносильно системе $n$ уравнений первого порядка
$$\dot{x}_1={x}_2,
\dot{x}_2={x}_3,
\ldots
\dot{x}_{n-1}={x}_n,
\dot{x}_n=f(x_1,x_2,\ldots, x_n),\eqno(3)$$
или
$$\dot{x}=F(x),\eqno(4)$$
где $x\in \mathbb{ R}^n$, $F(x):\mathbb{ R}^n\rightarrow \mathbb{ R}^n$ определяемое естественным образом.

Наряду с уравнением (1) рассмотрим возмущённое уравнение
$$x^{(n)}=f(x,\dot{x},\ldots, x^{(n-1)})+h(t), \eqno(5)$$
где $h(t): \mathbb{ R}\rightarrow \mathbb{ R}$ непрерывная $\omega$-периодическая функция:
$h(t+\omega)=h(t),$
где  $\omega>0$ -- некоторое положительное число.

Это уравнение равносильно периодически возмущённой системе
$$\dot{x}=F(x)+\mathbf{h}(t),$$
где $\mathbf{h}(t)$ -- непрерывная $\omega$-периодическая векторная функция:
$\mathbf{h}(t+\omega)=\mathbf{h}(t)$,
$\mathbf{h}(t)=col(0,\ldots, 0,h(t)):\mathbb{ R}\rightarrow \mathbb{ R}^n$.

Пусть выполнены ещё  два условия:

$5^0$. Любое решение $x(t)$ уравнения (1)  определено при всех $t:0\leqslant t< +\infty$.
(Это относится и к системе (4).)

$6^0$. Дифференциальное уравнение (1) асимптотически устойчиво в целом, т.е. её нулевое решение
$x(t)\equiv 0$ устойчиво и для любого другого решения $x(t)$:
$x(t), \dot x(t),\ldots , x^{(n-1)}(t)\rightarrow 0$ при  $t\rightarrow\infty$ (это относится и к системе (4)).

\textbf{Теорема~1.} {\it Рассмотрим дифференциальное уравнение} (1) {\it при условиях $1^0$-$6^0$. Тогда
	при любой непрерывной $\omega$-пе\-ри\-о\-ди\-чес\-кой функции  $h(t)$ периодически возмущённое дифференциальное
	уравнение} (5) {\it имеет по крайней мере одно $\omega$-периодическое решение  $x(t)$
		$$x(t+\omega)=x(t)$$
	(вынужденное колебание), если выполнено условие}

$$f(\mathbb{ R},0,\ldots,0)=\mathbb{ R}.$$



Пусть выполнены следующие три условия

$7^0$. Любое решение $x(t)$ уравнения (1) определено при всех $t$, $-\infty<t<+\infty$.

$8^0$. Любое решение $x(t)$ является ограниченным вместе с производными до $(n-1)$-го порядка включительно
$$\|x(t)\|\leqslant c_1, \|\dot{x}(t)\|\leqslant c_2, \ldots, \|x^{(n-1)}(t)\|\leqslant c_n,\;  -\infty<t<+\infty.$$

$9^0$. Предположим, что периоды всех собственных колебаний системы (3) (см. также (4)) ограничены снизу

$$ 0<\sigma_0 \leqslant \sigma. $$

\textbf{Теорема~2.} {\it Рассмотрим дифференциальное уравнение} (1) {\it при выполнение условий $1^0$-$4^0$ и $7^0$-$9^0$. Тогда
	при любой непрерывной  $\omega$-периодической функции  $h(t)$  периодически возмущённое дифференциальное
	уравнение} (5) {\it имеет по крайней мере одно $\omega$-периодическое решение  $x(t)$
		$$x(t+\omega)=x(t) $$
	(вынужденное колебание), если период $\omega$ достаточно мал
		$$0<\omega< \sigma_0 $$
	и}
$$f(\mathbb{ R},0,\ldots,0)=\mathbb{ R}.$$

\litlist

1.  {\it Евченко~В.\,К.} Об одной задаче из теории колебаний.
- Вестник Тамбовского университета. - Тамбов: 2015. - Том 20, вып.5. - С. 1136-1137.

2. {\it Боровских~А.\,В., Перов~А.\,И.} Лекции по обыкновенным дифференциальным
уравнениям. Москва-Ижевск: Регулярная и хаотическая динамика, 2004. - 540~с.

3. {\it Зубов~В.\,И.} Теория колебаний. М.: Высшая школа, 1979. - 400~с.

4. {\it Перов~А.\,И., Евченко~В.\,К.} Метод направляющих функций.
Воронеж: Издательско-полиграфический центр ВГУ, 2012. - 182~с.










