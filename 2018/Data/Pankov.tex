\begin{center}
\textbf{Об оценке решений одной краевой задачи в полосе для вырождающегося
эллиптического уравнения}
\end{center}

\begin{center}
\textbf{Баев А.Д., Панков В.В. }
\end{center}
\addcontentsline{toc}{section}{Баев А.Д., Панков В.В.}

\begin{center}
\textbf{Воронежский государственный университет}
\end{center}





В настоящее время интенсивно исследуются процессы с вырождением, то есть
процессы, в которых граница области оказывает существенное влияние на
процессы, происходящие вблизи границы. В этом случае на границе области
может меняться как тип уравнения, так и его порядок. В данной работе
рассматриваются краевые задачи для уравнений, являющихся эллиптическими
внутри области, которые на границе области меняют порядок по одной из
переменных. К таким уравнениям приводит математическое моделирование
процессов фильтрации идеального баротропного газа в неоднородной
анизотропной пористой среде, различных процессов гидродинамики с сингулярной
особенностью у параметров. Подобные уравнения возникают при моделировании
процесса распространения примеси в жидкокристаллическом растворе,
находящемся во внешнем электрическом поле, при исследовании стационарной
задачи о контакте мягкой оболочки с препятствием, при расчете линейных
стационарных магнитных осесимметричных полей в неоднородных анизотропных
средах. Такие уравнения являются, например, обобщением сингулярно
возмущенных уравнений конвекции -- диффузии. В работе В.П. Глушко [1] были
получены оценки решений общей краевой задачи в полосе для вырождающегося
эллиптического уравнения высокого порядка, вырождающегося а границе области
в уравнение первого порядка по переменной $t$. В работе А.Д Баева, В.П.
Глушко [2] были получены априорные оценки общей краевой задачи для одного
вырождающегося уравнения высокого порядка, которое вырождается на границе
области в уравнение второго порядка по переменной $t$. Уравнения,
вырождающиеся в уравнения третьего порядка по переменной $t$, были изучены в
[3], [4]. Некоторые другие вырождающиеся уравнения были рассмотрены в
[5]-[7].

Рассмотрим в полосе $R_d^n = \{x \in R^{n - 1},\,\,0 < t < d\}$, где $d >
0$ --- некоторое число, уравнение
\begin{equation}
\label{eq4500}
A(D_x ,D_{\alpha ,t} ,\partial _t )v(x,t) = F(x,t),
\end{equation}
где
$A(D_x ,D_{\alpha ,t} ,\partial _t )v = L_{2m} (D_x ,D_{\alpha ,t} )v -
b( - 1)^{k - 1}\partial _t^{2k - 1} v$,
\linebreak
$L_{2m} (D_x ,D_{\alpha ,t} ) =
\sum\limits_{\left| \tau \right| + j \le 2m} {a_{\tau j} D_x^\tau D_{\alpha
,t}^j } $, $b,\,\,a_{\tau j} $ - комплексные числа, $Im\,\overline b a_{02m}
= 0$, $D_{\alpha ,t} = \frac{1}{i}\sqrt {\alpha (t)} \partial _t \sqrt
{\alpha (t)} ,\,\,\,\,\partial _t = \frac{\partial }{\partial t}$, $D_x^\tau
= i^{\left| \tau \right|}\partial _{x_1 }^{\tau _1 } \partial _{x_2 }^{\tau
_2 } ...\partial _{x_{n - 1} }^{\tau _{n - 1} } $.

На границе $t = 0$ полосы $R_d^n $ задаются условия вида
\begin{equation}
\label{eq4501}
B_j (D_x )\left. v \right|_{t = 0} = \!\!\! \sum\limits_{\left| \tau \right| \le
m_j } {b_{\tau j} D_x^\tau \partial _t^{j - 1} \left. v \right|_{t = 0} =
G_j (x)} ,\,\,j = 1,2,...,k - 1
\end{equation}
с комплексными коэффициентами $b_{\tau j} $.

На границе $t = d$ полосы $R_d^n $ задаются условия вида
\begin{equation}
\label{eq4502}
\left. v \right|_{t = d} = \partial _t \left. v \right|_{t = d} = ... =
\partial _t^{m - 1} \left. v \right|_{t = d} = 0.
\end{equation}

Пусть выполнены следующие условия.

\textbf{Условие 1.} При всех $(\xi ,\eta ) \in R^n$ справедливо неравенство
$\,Re\overline b L_{2m} (\xi ,\eta ) \ge c(1 + \left| \xi \right|^2 + \left|
\eta \right|^2)^m$, где постоянная $c > 0\,$ не зависит от $(\xi ,\eta )$.

\textbf{Условие 2.} Для некоторого $s \ge 2m + \mathop {\max }\limits_{1 \le
j \le k} (m_j )$ функция $\alpha (t)$ принадлежит $C^{s - 1}[0,d]\,$, причем
$\,\alpha (0) = \,\alpha '(0) = 0,\,\,\,\alpha (t) > 0$ при $\,t > 0$.

\textbf{Условие 3.}
$\sum\limits_{\left| \tau \right| \le m_j } {b_{\tau
j} \xi _\tau \ne 0,\,\,} \,j = 1,2,...,k - 1$ при всех $\xi \in R^{n - 1}$.

Рассмотрим интегральное преобразование $F_\alpha $, которое на функциях
$u(t) \in C_0^\infty (R_ + ^1 )$ может быть записано в виде
\[
F_\alpha [u(t)](\eta ) = \int\limits_0^{ + \infty } {u(t)\exp (i\eta }
\int\limits_t^d {\frac{d\rho }{\alpha (\rho )}} )\frac{dt}{\sqrt {\alpha
(t)} }.
\]

Это преобразование было введено в [7]. Для этого преобразования можно
построить обратное преобразование $F_\alpha ^{ - 1} $, которое можно
записать в виде
$$F_\alpha ^{ - 1} [w(\eta )](t) = \left. {\frac{1}{\sqrt
{\alpha (t)} }F_{\eta \to \tau }^{ - 1} [w(\eta )]} \right|_{\tau = \phi
(t)} ,$$ где $\tau = \phi (t) = \int\limits_t^d {\frac{d\rho }{\alpha (\rho
)}} ,\,\,\,\,\,F_{\eta \to \tau }^{ - 1} $ - обратное преобразование Фурье.
Кроме того, для преобразования $F_\alpha $ доказан аналог равенства
Парсеваля, что дает возможность рассмотреть это преобразование не только на
функциях из $L_2 (R_ + ^1 )$, но и на некоторых классах обобщенных функций.
Из определения преобразования $F_\alpha $ следует, что если $u\left( t
\right) \in C^s\left[ {0,d} \right]$ и удовлетворяет условиям $u\left( 0
\right) = \partial _t u\left( 0 \right) = ... = \partial _t^{s - 1} \left( 0
\right) = 0$,
то справедливо равенство $F_\alpha \left[ {D_{\alpha ,t}^j u} \right]\left(
\eta \right) = \eta ^jF_\alpha \left[ u \right]\left( \eta \right)$ при всех
$j = 0,1,2,...,s$.

С помощью преобразования $F_\alpha $ были построены псевдодифференциальные
операторы с вырождением. Исследование таких псевдодифференциальных уравнений
позволило получить априорные оценки и теоремы о существовании граничных
задач в полупространстве для новых классов вырождающихся уравнений.

Введем пространства, в которых будет изучаться задача (\ref{eq4500})-(\ref{eq4502}).

\textbf{Определение 1.} Пространство $H_{s,\alpha ,\frac{2m}{2k - 1}} (R_d^n
)$ ($s \ge 0$ - целое число) состоит из тех функций $v(x,t) \in L_2 (R_d^n
)$, для которых конечна норма
\begin{multline*}
	\left\| v \right\|_{s,\alpha ,\frac{2m}{2k - 1}} =
	%\\=
	\{
		\sum\limits_{l = 0}^{[\frac{(2k - 1)s}{2m}]}
		\left\|
			F_{\xi \to x}^{ - 1} F_\alpha ^{ - 1}
			[(1 + \left| \xi \right|^2 +
		\right. \\ + \left.
			\left| \eta \right|^2 )^{\frac{1}{2}(s -
			\frac{2m}{2k - 1}l)}F_\alpha F_{x \to \xi } [\partial _t^l v(x,t)]]
		\right\|_{L_2 (R_d^n )}^2
	\}
	^\frac{1}{2} ,
\end{multline*}
где $[\frac{(2k - 1)s}{2m}]$ - целая часть числа $\frac{(2k - 1)s}{2m}$.

Здесь $F_{x \to \xi } $ ($F_{\xi \to x}^{ - 1} )$ -- прямое (обратное)
преобразование Фурье

Если $s$ - натуральное число такое, что число $\frac{(2k - 1)s}{2m}$
является целым числом, то эта норма эквивалентна следующей норме
\[
\left\| v \right\|_{s,\alpha ,\frac{2m}{2k - 1}} = \{\sum\limits_{\left|
\tau \right| + j + \frac{2m}{2k - 1}l \le s} {\left\| {D_x^\tau D_{\alpha
,t}^j \partial _t^l v} \right\|} _{L_2 (R_d^n )} \}^{\frac{1}{2}}.
\]



Обозначим через $\,\,H_s (R^{n - 1})$ пространство Со\-бо\-ле\-ва--Сло\-бо\-де\-ц\-ко\-го,
норму в котором обозначим через $\left\| \cdot \right\|\,_s \,$

Справедливы следующие теоремы.

\textbf{Теорема 1.} $Пусть s \ge \max \{2m,\,\,\mathop {\max }\limits_{1 \le j \le
k} (m_j + \frac{2m(j - 1)}{2k - 1}) + \frac{m}{2k - 1}\}$\textit{ - целое число, }$m \ge 2k - 1$\textit{ и выполнены условия 1 - 3. Тогда для любого решения }$v(x,t)$\textit{ задачи (\ref{eq4500}) - (\ref{eq4502}), принадлежащего пространству }$H_{s,\alpha
,\frac{2m}{2k - 1}} (R_d^n )\,$\textit{ справедлива коэрцитивная априорная оценка }
\begin{multline*}
\left\| v \right\|_{s,\alpha ,\frac{2m}{2k - 1}} \le
\\ \le
c(\left\| {Av}
\right\|_{s - 2m,\alpha ,\frac{2m}{2k - 1}} + \sum\limits_{j = 1}^{k - 1}
{\left\| {B_j \left. v \right|_{t = 0} } \right\|_{s - m_j - \frac{2m(j -
1)}{2k - 1} - \frac{m}{2k - 1}} } ),
\end{multline*}
\textit{где постоянная }$c > 0\,$\textit{ не зависит от }$v.$



\begin{center}
Литература
\end{center}

1. Вишик М.И., Грушин В.В. Математический сборник, 1969, вып. 79 (121), с.
3-36.

2. Баев А.Д. Доклады Академии наук, 2008, т. 422, №6, с. 727 -- 728.

3. Баев А. Д. Вестник Самарского гос. ун-та. Сер. Ес\-теств. науки, 2008,
№\textbf{3} (62), с. 27-39.

4. Левендорский С.\,З. Математический сборник, 1980, №111 (153), с. 483-501.

5. Баев А.Д., Бунеев С.С. Доклады Академии наук, 2013, т. 448, №1, с. 7-8.

6. Баев А.Д., Бунеев С.С. Известия вузов. Серия Математика. 2012. №7.С. 1-4.

7. Баев А.Д. Доклады академии наук СССР. 1982. т.265, № 5, с. 1044-1046.

8. Баев А.Д.,Кобылинский П.А. Доклады академии наук. Математика. 2016. т.
466. № 4, с. 385 -- 388.

9. Баев А.Д., Ковалевский Р.А., Кобылинский П.А. Доклады академии наук 2016,
т. 471, № 4, с. 387--390.





Работа выполнена при финансовой поддержке Ми\-ни\-с\-те\-р\-с\-т\-ва образования и науки
РФ (проект 14.Z50.31.0037).

