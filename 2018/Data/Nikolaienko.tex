\begin{center}{ \bf  ОСОБЕННОСТИ СЛОЕНИЙ ДВУМЕРНЫХ НЕКОМПАКТНЫХ МНОГООБРАЗИЙ НА ЛИНИИ УРОВНЯ ГЛАДКИХ ФУНКЦИЙ}\\
{\it С.С. Николаенко} \\
(Москва, МГУ им. М.В. Ломоносова; {\it nikostas@mail.ru} )
\end{center}
\addcontentsline{toc}{section}{Николаенко С.С.\dotfill}

Рассматривается классическая задача описания топологии слоений, порождаемых гладкими функциями на двумерных поверхностях. Помимо того, что эта задача представляет самостоятельный интерес, она может быть рассмотрена в контексте теории интегрируемых гамильтоновых систем. Действительно, изучаемые слоения можно интерпретировать как слоения Лиувилля гамильтоновых систем с одной степенью свободы. Кроме того, такие слоения могут возникать как сечения фазовых пространств интегрируемых систем с большим числом степеней свободы (см.~[1]). Для копмактных многообразий рассматриваемый вопрос хорошо изучен. Некомпактный же случай оказывается гораздо сложнее (см., например, [2]).

Пусть $M^2$ -- гладкое некомпактное многообразие, $f\in C^\infty(M^2)$ -- вещественнозначная функция, все критические точки которой изолированы. Линией уровня функции $f$ будем называть связную компоненту множества $f^{-1}(c)\subset M^2$, где $c\in\mathbb R$. Назовём линии уровня $l$ и $\tilde l$ эквивалентными, если найдётся такая конечная последовательность линий уровня $l_1=l,l_2,\ldots,l_n=\tilde l$, что для всех $i=1,2,\ldots,n-1$ линии $l_i$ и $l_{i+1}$ не имеют инвариантных (т.е. целиком состоящих из линий уровня) непересекающихся окрестностей. %Очевидно, на эквивалентных линиях уровня функция $f$ принимает одинаковые значения.

\textbf{Определение 1.}
{\it \textbf{Слоем} слоения, порождаемого функцией $f$ на многообразии $M^2$, будем называть объединение всех линий уровня из некоторого класса эквивалентности.}

\textbf{Определение 2.}
{\it Слой $L$ называется \textbf{бифуркационным}, если он не имеет инвариантной окрестности, послойно гомеоморфной прямому произведению $L\times(-1,1)$.}

\textbf{Определение 3.}
{\it \textbf{Атомом} называется связная инвариантная окрестность бифуркационного слоя, не содержащая иных бифуркационных слоёв и рассматриваемая с точностью до послойного гомеоморфизма.}

Для простоты будем рассматривать лишь атомы $V$, бифуркационный слой $L$ которых состоит из конечного числа компонент связности и содержит конечное число критических точек функции $f$. Пусть $f|_L\equiv c$. Тогда компонеты связности множества $V\setminus L$ можно разделить на положительные (на которых $f>c$) и отрицательные (на которых $f<c$). Наибольший интерес представляют неминимаксные атомы $V$, для которых множество $V\setminus L$ содержит как положительные, так и отрицательные компоненты.

%\begin{figure}[h]
%	\centering
%		\includegraphics[scale=0.3]{mmm.pdf}
%		\caption{}
%	\label{fig:mmm}
%\end{figure}

\textbf{Определение~4.}
{\it \textbf{$\mathbf f$-атомом} называется атом $V$ с бифуркационным слоем $L$, для которого фиксировано разделение компонент связности множества $V\setminus L$ на положительные и отрицательные.}

В компактном случае для функций Морса А.А.~Ошемковым в [3] был предложен изящный и эффективный метод перечисления всех $f$-атомов с помощью так называемых $f$-графов. С некоторыми модификациями эту конструкцию можно обобщить на некомпактный случай и свести тем самым задачу классификации атомов к гораздо более простой задаче классификации графов специального вида.

\textbf{Определение~5.}
{\it Конечный связный граф с ориентированными рёбрами назовём \textbf{$\mathbf{\bar f}$-графом}, если он удовлетворяет следующим условиям:\\
 1) все вершины графа имеют степень 1, 2, 3 или 4;\\
 2) все рёбра графа окрашены в 2 цвета (синий и красный) так, что:\\
		-- каждой вершине степени 1 инцидентно синее ребро;\\
		-- каждой вершине степени 2 инцидентны либо два синих ребра, либо одно синее и одно красное;\\
		-- каждой вершине степени 3 инцидентны два синих ребра и одно красное;\\
		-- каждой вершине степени 4 инцидентны два синих и два красных ребра;\\
	3) если какой-то вершине инцидентны два ребра одного цвета, то одно из них входит в эту вершину, а другое выходит из неё; \\
	4) каждой вершине степени 3 или 4 приписана метка $\varepsilon=\pm1$;\\
	5) каждому циклу из красных рёбер приписана метка $\delta\in\{0,1\}$.
}

Введём на множестве $\bar f$-графов отношение эквивалентности $\sim$, полагая два $\bar f$-графа эквивалентными, если один из них можно получить из другого последовательностью замен $\varepsilon$-меток всех вершин, а также ориентации всех рёбер некоторого максимального связного одноцветного подграфа.

Пусть $\mathfrak\Gamma$ -- множество всех $\bar f$-графов, за исключением графа, состоящего из двух вершин, соединённых синим ребром.

\textbf{Теорема~2.} {\it Имеется естественное взаимно-однознач\-ное соответствие между множеством всех $f$-атомов и множеством классов эквивалентности $\mathfrak\Gamma/\sim$.}

Исследование выполнено за счет гранта Российского научного фонда (проект №17-11-01303).

%%%%  ОФОРМЛЕНИЕ СПИСКА ЛИТЕРАТУРЫ %%%
\smallskip \centerline{\bf Литература}\nopagebreak

1. {\it Болсинов А.В., Фоменко А.Т.} Интегрируемые гамильтоновы системы. Геометрия, топология, классификация.\\ Ижевск: изд. дом <<Удмуртский университет>>, 1999.

2. {\it Шарко В.В.} Гладкие функции на некомпактных поверхностях. // Збiрник праць Iн-ту математики НАН України, 2006, т. 3, № 3, с. 443--473.

3. {\it Ошемков А.А.} Функции Морса на двумерных поверхностях. Кодирование особенностей. // Труды МИРАН, 1994, т. 205, с. 131--140.

