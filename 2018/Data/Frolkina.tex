\begin{center}{ \bf  О СЕМЕЙСТВАХ ЛИСТОВ МЕБИУСА В $R^3$}\\
{\it О.Д. Фролкина } \\
(Москва; {\it olga-frolkina@yandex.ru} )
\end{center}
\addcontentsline{toc}{section}{ФРОЛКИНА О.Д.}


Теория диких вложений
восходит к работам Л.Антуана, П.С.Урысона,
Дж.Александера 1920-х годов;
книги [2], [10] и обзор [3]
содержат сотни ссылок по данной области
топологии.

\textbf{Определение~1.} {\it
Полиэдром в $R^N$ называется
конечное объединение симплексов.
Подмножество $X\subset R^N$, гомеоморфное полиэдру,
называется ручным, если существует такой гомеоморфизм
$h$ пространства $R^N$ на себя, что
$h(X)$  --- полиэдр (в $R^N$);
в противном случае $X$ называется диким.
}

Из классических теорем Антуана и
Шенфлиса [2, II.4] вытекает, что
всякое вложение
окружности или отрезка в плоскость является
ручным.
Знаменитая рогатая сфера Александера [2, IV.3] --- пример
дикой 2-сферы в $R^3$.
Вообще, при $N\geqslant 3$
всякий полиэдр $P\subset R^N$, $\dim P\geqslant 1$,
обладает дикими вложениями в $R^N$.

Как показал Бинг,
в $R^3$ невозможно разместить
несчетное семейство попарно непересекающихся
замкнутых поверхностей,
каждая из которых вложена дико [4], [7, Th.~3.6.1].
Предположение дикости, очевидно, существенно;
к примеру, концентрические сферы всевозможных
положительных радиусов дают несчетное семейство попарно дизъюнктных ручных поверхностей.
В 1960 г. Столлингс построил континуальное
семейство попарно дизъюнктных диких 2-дисков
в $R^3$ [13].
В 1968 г. Шер улучшил конструкцию Столлингса,
получив, дополнительно, что все 2-диски
семейства вложены в $R^3$ неэквивалентными способами [12].
Автором получено обобщение построения Столлингса и Шера на случай
произвольного $R^N$, $N\geqslant 3$,
и произвольного совершенного компакта,
вложимого в гиперплоскость $R^{N-1}$ [11].
(На данный момент теорема Бинга
имеет лишь частичные обобщения на случай более высоких размерностей,
см. [6, Thm. 1, 2],
[8, Th.10.5],
[9, p.383, Th.3C.2].)

В случае неориентируемой гиперповерхности ситуация иная.
В работе [1]
доказан следующий результат.
{\it Пусть $P$ --- 2-мерный полиэдр.
В пространстве $R^3$ можно разместить несчетное
семейство попарно непересекающихся
полиэдров, гомеоморфных $P$,
в том и только том случае, когда
$P$ ориентируем и у каждой точки
$p\in P$
имеется планарная окрестность.}
(Двумерный полиэдр называется
ориентируемым, если он не содержит подмножества,
гомеоморфного листу Мебиуса.)
Подчеркнем, что
здесь каждый экземпляр пространства $P$
является полиэдром в $R^3$;
доказательство,
приведенное в [1],
существенно использует это предположение;
это сильное ограничение, поскольку,
рассуждая аналогично [5], можно построить лист Мебиуса в
$R^3$, дикий в каждой своей точке.

Автором получены
следующие результаты:

\textbf{Теорема~1.}
{\it Пусть $P$ --- 2-мерный полиэдр.
В пространстве $R^3$ можно разместить несчетное
семейство подпространств, гомеоморфных $P$,
в том и только том случае, когда
$P$ ориентируем и у каждой точки
$p\in P$
имеется планарная окрестность.}

Как следствие, в пространстве $R^3$ нельзя разместить более чем счетное число
попарно дизъюнктных листов Мебиуса.

\textbf{Теорема~2.} {\it
В $R^N$, $N\geqslant 3$,
нельзя разместить несчетное семейство попарно непересекающихся
ручных гомеоморфных копий
компактного триангулируемого
неориентируемого $(N-1)$-мерного многообразия с границей.}

Предполагается обсудить эти результаты,
а также проблемы,
к которым
на сегодняшний день
сводится возможность дальнейших
обобщений.


\smallskip \centerline{\bf Литература}\nopagebreak

1.
{\it Грушин В.В., Паладомодов В.П.}
О максимальном количестве попарно непересекающихся гомеоморфных между собою фигур, которые можно разместить в трехмерном пространстве
 // УМН 17 (1962), 3(105), 163--168.

2.
{\it Келдыш Л.В.}
Топологические вложения в евклидово пространство
// Тр. МИАН СССР, 1966,	 т. 81, 3-184.

3.
{\it Чернавский А.В.}
О работах Л.В.Келдыш и ее семинара //
УМН, 60:4(364) (2005), 11–36.

\selectlanguage{english}

4.
{\it Bing R.H.}
$E^3$ does not contain
uncountably many mutually exclusive wild
surfaces //
Bull. Amer. Math. Soc.
63 (1957) 404.
Abstract \# 801t.

5.
{\it Bing R.H.}
A wild surface each of whose arcs is tame //
Duke Math. J. 28 (1961), 1--15.

6.
{\it Bryant J.L.}
Concerning uncountable families of $n$-cells in $E^n$ //
Michigan Math. J. 15 (1968) 477--479.


7.
{\it Burgess C.E., Cannon J.W.}
Embeddings of surfaces in $E^3$ //
Rocky Mountain J. of Math.
1 (2) (1971) 259--344.

8.
{\it Burgess C.E.}
Embeddings of surfaces in Euclidean three-space //
Bull. Amer. Math. Soc.
81(1975), 5, 795--818.

9.
{\it Daverman R.J.}
Embeddings of $(n-1)$-spheres in
Euclidean $n$-space //
Bull. Amer. Math. Soc.
84 (1978), 3, 377--405.

10.
{\it Daverman R.J., Venema D.A.}
Embeddings in Manifolds.
Graduate Studies in Mathematics 106. Providence, RI: American Mathematical Society.
2009.

11.
{\it Frolkina O.D.}
Wild high-dimensional Cantor fences in $\mathbb R^n$, Part I // Topol. Appl.
\foreignlanguage{russian}{
	(представлено к публикации)
}

12.
{\it Sher R.B.}
A note on the example of Stallings //
Proc. Amer. Math. Soc., 19 (3) (1968) 619--620.

13.
{\it Stallings J.R.}
Uncountably many wild disks //
Annals of Math. 71 (1) (1960) 185--186.

