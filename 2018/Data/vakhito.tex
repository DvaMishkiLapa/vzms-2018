\begin{center}{ \bf О ДОКАЗАТЕЛЬСТВЕ ЧАСТНОГО СЛУЧАЯ ТЕОРЕМЫ ДИРИХЛЕ ОБ АРИФМЕТИЧЕСКОЙ ПРОГРЕССИИ}\\
{\it Е. В. Вахитова, С. Р.  Вахитова, Т. М. Сивоплясов} \\
 (Воронеж)
\end{center}
\addcontentsline{toc}{section}{Вахитова Е. В.,  Вахитова С. Р., Сивоплясов Т. М.\dotfill}



В настоящей работе рассмотрен частный случай теоремы Дирихле об арифметической прогрессии. Сформулируем теорему Дирихле.

\textbf{Теорема 1.} \emph{Арифметическая прогрессия вида $nk+a$:
$$a,\,\, a+ k,\,\, a+2k,\,\, …,\,\, a+ (n -1)k,\,\, …$$
при $a,n,k \in N$ и $(a,k)= 1$ содержит бесконечное множество простых чисел.}

Отметим, что требование $(a,k)= 1$  важно, так как если $(a,k)> 1$, то арифметическая прогрессия будет содержать самое большее одно простое число.

Доказательство теоремы 1 получил Л. Дирихле в 1837 г. Оно основано на применении характеров (по модулю $k$) и $L$ -- рядов Дирихле с использованием теории функций комплексной переменной.

Доказательство теоремы 1 или более сильного результата можно найти в книгах [1] -- [3]. О характерах можно узнать из книги [4].

Некоторые частные случаи теоремы 1 могут быть просто доказаны, например, для арифметических прогрессий видов $4k +1,\, 4k+ 3,\, 6k+ 1,\, 6k+ 5\, (k \in N).$ Эти доказательства приведены в книге [5].

В 1949 г. А. Сельберг получил элементарное доказательство теоремы 1, то есть без использования теории функций комплексной переменной. Это доказательство является технически сложным, использует аппроксимацию числовой функции. Элементарное доказательство теоремы 1 приведено в книге [6].

В 1961 г. А. Роткевич получил элементарное доказательство для частного случая теоремы 1, а именно, для арифметической прогрессии вида $nk+ 1\, (n,k \in N).$

\textbf{Теорема 2.} \emph{Для любого натурального числа $n$ существует бесконечное множество простых чисел вида $nk+1,$ где $k\in N.$}

Элементарное доказательство теоремы 2, то есть частного случая теоремы 1 при $a= 1$, приведено в статье [7], в книгах его нет. Идея доказательства состоит в том, что достаточно было доказать существование простого числа вида $nk +1$ для любого натурального числа $n$, то есть простого числа вида $n m t+ 1,$ которое больше $m$  и имеет вид $nk +1$ при  $k= m t \,(n,m,k,t \in N).$ При этом использованы понятие показателя числа по простому модулю, функция Мёбиуса и теорема Ферма.






\centerline{\bf Литература}

1. {\it Гельфонд А. О.} Элементарные методы в аналитической теории чисел / А. О. Гельфонд, Ю. В. Линник. -- М.: ГИФМЛ, 1962. -- 272 с.

2. {\it Карацуба А. А.} Основы аналитической теории чисел / А. А. Карацуба. -- 2-е изд., перераб. и доп. -- М.: Наука, 1983. -- 240 с.

3. {\it Чандрасекхаран К.} Введение в аналитическую теорию чисел / К. Чандрасекхаран. -- перевод с англ. С. А. Степанова. -- М.: Мир, 1974. -- 187 с.

4. {\it Виноградов И. М.} Основы теории чисел / И. М. Виноградов. -- 9-е изд., перераб. -- М.: Наука, 1981. -- 176 с.

5. {\it Бухштаб А. А.} Теория чисел / А. А. Бухштаб. -- 2-е изд., исправл. -- М.: Просвещение, 1966. -- 384 с.

6. {\it Трост Э.} Простые числа / Э. Трост. -- перевод с нем. Н. И. Фельдмана. -- М.: ГИФМЛ, 1959. -- 135 с.

7. {\it Rotkiewicz A.} Demonstraton arithmetique de l`existence d`une infinite de nombres premiers de la forme $nk+ 1$ / A. Rotkiewicz // L`enseignement Mathematique, 1961. -- T. VII. -- P. 277--280.




