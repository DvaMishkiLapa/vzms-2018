\begin{center}{ \bf  ИНВАРИАНТЫ ЖОРДАНА--КРОНЕКЕРА ПОЛУПРЯМЫХ СУММ НЕКОТОРЫХ АЛГЕБР ЛИ}\\
{\it К.С. Ворушилов } \\
(Москва; {\it ksvorushilov@gmail.com} )
\end{center}
\addcontentsline{toc}{section}{Ворушилов К.С.\dotfill}

В докладе будет рассказано о связи инвариантов Жорда\-на--Кронекера алгебр Ли с методом сдвига аргумента, изложенным А.\,С.\,Мищенко и А.\,Т.\,Фоменко в работе [4], а также о вычислении инвариантов Жордана--Кронекера для некоторых классов алгебр Ли. В частности, эти инварианты вычислены для полупростых алгебр Ли и алгебр малой размерности; данные результаты можно найти в работах [1] и [2] . Интересным примером для вычисления инвариантов Жордана--Кронекера являются полупрямые суммы полупростых алгебр Ли с несколькими экземплярами пространств их стандартных представлений.

Согласно теореме Болсинова из [3], полнота коммутативного набора, построенного методом сдвига аргумента на некоторой алгебре Ли, означает, что эта алгебра Ли имеет кронекеров тип. Используя этот факт и некоторые дополнительные соображения, автору удалось вычислить инварианты Жордана-Кронекера для полупрямых сумм  $so(n)\oplus (\mathbb{R}^n)^k$ и  $sp(n)\oplus (\mathbb{R}^n)^k.$ При этом в случае $sp(n)$ при различных значениях $n$ и $k$ полупрямая сумма может иметь как кронекеров, так и смешанный тип.

Для аналогичных полупрямых сумм в случаях $sl(n)$ и $gl(n)$ полный ответ пока не получен, но удалось вычислить инварианты Жордана--Кронекера алгебры $sl(n)\oplus (\mathbb{R}^n)^k$ для некоторых значений $k$ и $n$, в частности, для случаев $k\geq n.$



%%%%  ОФОРМЛЕНИЕ СПИСКА ЛИТЕРАТУРЫ %%%
\smallskip \centerline{\bf Литература}\nopagebreak

1. {\it Bolsinov A.V., Zhang P.} Jordan–Kronecker invariants of finite-dimensional Lie algebras. Transform. Groups Vol. 21 (1), pp. 51--86, 2016.

2. {\it Zhang P.} Algebraic Aspects of Compatible Poisson Struc\-tures. PhD Thesis, Loughborough University, 2012.


3. {\it Болсинов А.В.} Интегрируемые по Лиувиллю гамильтоновы системы на алгебрах Ли. Диссертация на соискание ученой степени кандидата физико-математических наук, МГУ им. М.\,В.\,Ломоносова, 1987.

4. {\it Мищенко А.С., Фоменко А.Т.} Уравнения Эйлера на конечномерных группах Ли. Изв. АН СССР. Сер. матем., 42 (2), c. 396–415, 1978.


