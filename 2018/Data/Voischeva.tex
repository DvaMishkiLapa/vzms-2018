\vzmstitle{ \bf  ПОСТРОЕНИЕ НЕЛОКАЛЬНЫХ ФАЗОВЫХ ВЕТВЕЙ В ОДНОМЕРНЫХ КРИСТАЛЛАХ}

\vzmsauthor{{Воищева}}{А.\, Ю.}

\vzmsinfo(Воронеж; {\it nastyabems@yandex.ru} )

\addcontentsline{toc}{section}{Воищева А.Ю.\dotfill}

В данном сообщении представлена процедура нелокального применения
конечномерной редукции Ляпунова--Шми\-д\-та к задаче о фазовых
переходах в кристаллах. Такая редукция реализует переход от
исходного модельного уравнения к (эквивалентной) задаче анализа
(ключевой) функции на конечномерном пространстве ключевых переменных
[1]. Использование ключевой функции даёт возможность использования
при построении и анализе периодических решений уравнений
вариационного исчисления новые аналитические и вычислительные
технологии. Для рассмотренных ниже уравнений получены алгоритмы
приближенного вычислять ключевой функции и её критических точек.


Хорошо известно, что моделирование фазовых переходов в кристаллах
основано на выборе <<подходящего>> термодинамического потенциала
Ландау. В случае одномерного кристалла с двухкомпонентным параметром
порядка часто используется (см. [2], [3]) потенциал
 $$
\Pi(x) = \frac14 \left((x_1^2+x_2^2)^2+ 2a\,x_1^2x_2^2 -
2\lambda(x_1^2+x_2^2)\right), \ \ \  x =(x_1,x_2)\,.
 $$
Ему отвечает функционал энергии  \ $V =
\int\limits_0^{2\pi}L(\frac{d x}{d z},x)dz\,,$
 $$
L\left(\frac{d x}{d z}\,,x\right) = \frac12 \left|\frac{d x}{d
z}\right|^2- \sigma\left(I\,x,\,\frac{d x}{d z}\right) + \Pi(x) +
(\psi\,, x)\,,
 $$
 $$
I= \left(
    \begin{array}{cc}
    0 &- 1\\
   1 &  0\\
    \end{array}
    \right)\,,
\ \ \psi(z) =(\psi_1(z)\,, \psi_2(z))\,,
 $$
рассматриваемый здесь на пространстве $2\pi$-периодических функций.
Сегнетоэлектрические фазы соответствуют экстремалям этого
функционала. Их исследование можно проводить, перейдя к ключевой
функции
 $$
W(\xi):=\inf_{x:p(x)=\xi} V(x),
 $$
$p(x)=(p_1,\dots,p_n),$ где $\{p_j\}_{j=1}^n$ --- некоторый набор
ключевых параметров. Положив сначала, что $\psi_1 = \psi_2 =\sigma =
0$ (в отсутствие внешних полей и анизотропии), перейдём к полярным
координатам в пространстве значений параметра порядка:
 $
x_1=r\cos\varphi,\ x_2=r\sin\varphi.
 $
Выразив через них подинтегральное выражение, получим для $V$
представление
 {\footnotesize
 $$
\int\limits_0^{2\pi} \left(\frac12\left(\left(\frac{d r}{d
z}\right)^2 + r^2\left(\frac{d \varphi}{d z}\right)^2-\lambda r^2
\right) + \frac{r^4}4 +\frac{b\,r^4}4(1-\cos( 4\varphi))\right)dz\,,
 $$}
$a=4b$. В соответствии с принципом наименьшего действия, стабильное
равновесное распределение параметра порядка соответствует
глобальному минимуму этого функ\-ци\-о\-на\-ла. Что означает, в
частности, выполнение равенств
 $
\frac{\partial V}{\partial r}= \frac{\partial V}{\partial \varphi}=0
\,,
 $
или
 $
r^3 + b\,r^3(1-\cos (4\varphi)) + r\left(\frac{d \varphi}{d
z}\right)^2-\frac{d^2 r}{d z^2} -\lambda r = r^2\frac{d^2\varphi}{d
z^2} + 2r\frac{d r}{d z}\frac{d \varphi}{d z} + b\,r^4\sin(
4\varphi) = 0\,.
 $
Подставив решение второго уравнения (при дополнительном условии $
\frac{d \varphi}{d z}=0$) в первое уравнение последней системы,
получим уравнение Дуффинга (осциллятор Дуффинга):
 $$
 \frac{d^2 r}{d z^2} + \lambda r - \gamma  r^3=0, \
 \ \ \ \ \  \gamma = 1+b\left(1-(-1)^n\right)\,.
 $$
При его исследовании также можно применить нелокальные редуцирующие
схемы, позволяющие осуществлять детальный анализ нелокального
ветвления соответствующих фаз кристалла. При наличии внешнего поля
аналогичная редукция приводит к неоднородному уравнению
 $
\frac{d^2 r}{d z^2} + \lambda r - \gamma r^3=q\,,
 $
рассматриваемому при периодических краевых условиях с периодом
$2\,\pi$. Случай периода, отличного от $2\,\pi$, также
<<охватывается>> этой задачей посредством масштабирующих
преобразований вида $s\longmapsto \alpha s$, \ $\lambda\longmapsto
\gamma \lambda$.

Каждое решение рассмотренной периодической краевой задачи являются
экстремали функционала энергии
 $$
V (r, \lambda, q):= \frac1{2\pi}\int\limits_0^{2\,\pi}
\left(\frac12\left(\frac{dr}{dz}\right)^2  +   \lambda  \frac{r^2}2
+ \frac{r^4}4 + qr\right)\,dz\,.
 $$
Положим, с целью упрощения,  $q = q_0e_0 + q_1e_1 + q_2e_2\,,$
получим
 $
f(\varphi):=\frac{d^2 r}{dz^2} + \lambda\,z - z^3 = q_0e_0 + q_1e_1
+ q_2e_2\,,
 $
где $ e_0 = 1\,, \  e_1=\sqrt {2}\sin(z)\,, \  e_2=\sqrt
{2}\sin(2z)$
--- первыве три фазовые моды, отвечающие критическим значениям
параметра $\lambda$: \ $\lambda_0=0$  и $\lambda_1=\lambda_2=1$ (с
фиксированным периодом $2\,\pi$). Известно, что множество всех мод
(с периодом $2\,\pi$) исчерпывается следующим набором функций и
отвечающих им собственных значений
 $
e_{2k-1} = \sqrt2\,\sin(k\,t)\,,  \ e_{2k} = \sqrt2\,\operatorname{cos}(k\,t)\,, \
\lambda_{2k-1} = \lambda_{2k} = k^2\,, \ k=1,2,\dots \,.
 $
Нормирующий множитель $\sqrt2$ выбран из-за удобств вычислительного
характера, связанных с тем, что система таких функций образует
ортонормированный базис в $L_2[0,2\,\pi]$. При $\lambda<(2n)^2$ в
качестве редуцирующей системы ключевых параметров можно взять
совокупность коэффициентов Фурье
 $
p_j(w)=\left<e_j,w\right>,\ j=0,1,\ldots,2(n-1)
 $
в рамках вариационной редуцирующей схемы Ляпунова-Шмидта [1].

Если записать левую часть исходного уравнения в операторном виде
 $
f (w) =q\,, \  w\in E:=C^2(\mathbb{R})\cap\{w(t+2\,\pi)\equiv
w(t)\}\,,
 $
то рассмотренную задачу можно заменить системой двух операторных
уравнений (при $n=2$ )
 $$
 \left.
\begin{array}{l}
A_1 \left( u \right) - \lambda P^3 \left(\sin( u+v) \right)
+q_0e_0+q_1e_1+q_2e_2 = 0\,,
\\
A_2( v ) - \lambda P\,^{\infty-3}
 \left( \sin(u+v)\right) =0,
\end{array}\right\}\,,
  $$
где $P^3,\ P\,^{\infty-3}$ --- пара ортопроекторов (в метрике $H$)
на $E^3:=Lin(e_0,e_1,e_2)$ и
$E\,^{\infty-3}:=E\cap\left(E^3\right)^\perp$, \ \ $A_1 =
A\vert_{E^3}$, \ $A_2 = A\vert_{E\,^{\infty-3}}$ --- соответствующие
сужения оператора $A$, \ $u = u(\xi):=\xi_0\,e_0 + \xi_1\,e_1 +
\xi_2\,e_2 \in E^3$, \ $v\in E\,^{\infty-3}$.

Для второго уравнения последней системы имеет место однозначная
разре\-ши\-мость по $v$ при каждом $u\in E^3$~--- всле\-д\-с\-т\-вие
выпуклости и коэрцитивности функционала энергии на подпространстве
$E^{\infty-3}$ (при условии $\lambda<4$). После построения
(приближенного) аналитического выражения для решения второго
уравнения системы  в виде $v=\Phi(\xi)$ получим приближенное
выражение так называемой глобальной ключевой функции в виде
 $
W(\xi)=V\left(u(\xi)+\Phi(\xi)\right)\,.
 $
Вследствие круговой симметрии функционала действия (относительно
трансляции  $r(z) \longrightarrow r(z+s)$, получим вблизи нуля (и при
$ q=0$) ключевую функцию в виде
 $
W(\xi_0,\xi_1,\xi_2) = \widehat{W}(\xi_0,r^2), \ \
r^2=\xi_1^2+\xi_2^2.
 $


\litlist

1.  {\it Даринский Б.М.} Бифуркации экстремалей фредгольмовых функционалов
/ Б.М. Даринский, Ю.И. Сапронов, С.Л. Царев  // Современная
математика. Фундаментальные направления. Том 12 (2004) --- С. 3-140.

2. {\it Изюмов Ю.А.} Фазовые переходы и симметрия кристаллов /Ю.А. Изюмов,
В.И. Сыромятников // М. : Наука, 1984. --- 247 с.

3. {\it Ж.--К. Толедано}  /  Толедано Ж.--К., Толедано П.  // Теория
Ландау фазовых переходов. М.: Мир. 1994. --- 461 с.
