\begin{center}{ \bf  КРИТЕРИЙ СОГЛАСИЯ НА ОСНОВЕ ПРОИЗВОДЯЩЕЙ ФУНКЦИИ ВЕРОЯТНОСТИ}\\
{\copyright} 2017 {\it  E. В. Акиндинова, А. Г. Бабенко, В. М. Вахтель,
В. А. Работкин, К. С. Рыбак} \\
( Воронеж, ВГУ; {\it vakhtel@phys.vsu.ru} )
\end{center}
\addcontentsline{toc}{section}{Акиндинова E.В., Бабенко А.Г., Вахтель В.М.,
Работкин В.А., Рыбак К.С.}

Исследование характеристик потоков излучений часто ведется в счетном режиме,
то есть измеряя число частиц \textit{K($\Delta$t)}  в последовательные фиксированные интервалы времени \textit{$\Delta$t}. Последовательность из \textit{n} таких случайных значений \textit{K} является случайной выборкой, которая позволяет получить эмпирическое распределение \textit{ЭР(K)}, которое обычно сопоставляют с моделями дискретных распределений Пуассона, биномиального, отрицательного биномиального и их обобщений.

Проверка гипотез согласия \textit{ЭР(K)} основана обычно на критерии хи-квадрат [1], в котором применяется неоднозначная процедура группирования выборочных значений, а асимптотические значения его квантилей могут существенно отличаться от значений при небольших объемах выборок.

В данной работе рассмотрена задача модификации критерия согласия на основе метода эмпирических производящих функций вероятности (ЭПФВ).

Массивы значений \textit{K($\Delta$t)} объемом до \textit{N}=10${}^{8}$ получены методом статистического моделирования и непосредственно при измерении потоков излучений.

Производящая функция вероятности (ПФВ) неотрицательной дискретной случайной величины определяется выражением
\[g(t,\theta )=\sum _{j=1}^{\infty }P(X_{{\rm j}} ,\theta )t^{X_{j} }  ,\]
где $P(X_{{\rm j}} ,\theta )$- распределение вероятностей значений случайной величины $X_{j} =K_{j} =0,_{} 1,_{} 2,_{} ...$, $\theta $ - вектор параметров распределения, $t$ - формальная переменная $\left|t\right|<1$. ЭПФВ определяется выражением [2]
$g_{n} (t)=\frac{1}{n} \sum _{j=1}^{n}t^{K_{j} }  =\frac{1}{n} \sum _{i=0}^{l}n_{i} t^{i}  $, $n=\sum _{i=1}^{l}n_{i}  $,  $n_{i} $- число одинаковых значений $K_{j} $ из \textit{n} имеющихся в выборке \textit{K}${}_{1}$,  {\dots} , \textit{K${}_{n}$}. С вероятностью единица при $n\to \infty $ справедливо $\sup \left|g_{n} (t)-g(t,\theta )\right|\to 0$ при соответствии \textit{ЭР} модели распределения.

В качестве основы для тестовых статистик в работе [2] предложено выражение
\[Q(t,\theta ,n)=\frac{\sqrt{n} \left(g_{n} (t)-g(t,\theta )\right)}{g(t^{2} ,\theta )-g^{2} (t,\theta )} ,\]
которое в асимптотике является гауссовской нормированной случайной функцией
 $N(0,1)$  и соответственно $Q^{2} (t,\theta ,n)$ в асимптотике можно аппроксимировать двухпараметрическим гамма-распределением $\gamma ({1 \mathord{\left/{\vphantom{1 2}}\right.\kern-\nulldelimiterspace} 2} ,{1 \mathord{\left/{\vphantom{1 2}}\right.\kern-\nulldelimiterspace} 2} )$. Непосредственное применение $Q(t,\theta ,n)$ и $Q^{2} (t,\theta ,n)$ затруднено из-за неопределенности выбора значений $t\in [0,1]$. Альтернативный подход на основе тестовой статистики в виде функционала [2]
\[n\int _{0}^{1}\left[g_{n} (t)-g(t,\theta )\right]^{2} dt \]
имеет недостаток  - его предельное распределение зависит от параметров $\theta $.

В нашей работе предложен критерий согласия проверки основной гипотезы для распределений Пуассона, биномиального и  отрицательного биномиального распределений против общей альтернативы, а также против конкретных альтернатив для этих распределений на основе тестовой статистики
\[V(a,b,n)=\int _{a}^{b}Q^{2} (t,\theta ,n)dt .\]
Математическое ожидание в асимптотике $E[V(a,b,n)]=b-a$, так как $E[Q^{2} (t,\theta )]=1$, поскольку $E[Q^{2} (t,\theta )]=[E[Q(t,\theta )]]^{2} +D(Q(t,\theta )=1$,  $E[Q(t,\theta )]=0$, $D(Q(t,\theta )=1$, где $D(Q(t,\theta )$ - дисперсия.

При $b-a<<1$, распределение $V(a,b,n)$ допускает аппроксимацию гамма-распределением.


 Сопоставляя теоретическую и эмпирическую вариации $\delta (V)=D^{1/2} (V)/E[V]$, $\delta  (V)=S^{1/2} (V_{n} )/\bar{V}_{n} $ (где $S(V_{n} )$ и $\bar{V}_{n} $ выборочное среднеквадратичное  и среднее) выбраны оптимальные значения $a=0,9$ и $b=0,98$. Квантили  гамма-распределения статистики $V(a,b,n)$ при оптимальных параметрах $\gamma (p=0,5028;_{} \beta =6,2844)$ составляют при уровнях значимости критерия $\alpha \% =10_{}  ;5_{}  ;1$  соответственно 0,216;   0,307;  0,529.


Из полученных результатов можно сделать следующие выводы.
Для эмпирического распределения Пуассона мощность критерия на основе ЭПФВ выше для альтернативы в виде биномиального распределения,
чем для альтернативы отрицательного биномиального распределения,
при котором мощность достаточно велика (более 50\%), если параметр распределения Пуассона  $\lambda >1$ и $n\ge 10^{3} $.
В случае эмпирического отрицательного биномиального распределения мощность относительно альтернатив
биномиального и пуассоновского распределений достаточно велика, и при малых объемах выборок $n\ge 10$,
и близких значениях средних и дисперсий.

%%%%  ОФОРМЛЕНИЕ СПИСКА ЛИТЕРАТУРЫ %%%
\smallskip \centerline{\bf Литература}\nopagebreak

1. {\it Большев Л. Н., Мирвалиев М.} Критерий согласия хи-квадрат для пуассоновского, биномиального и отрицательного биномиального распределений // Теория вероятности и ее применение, 23:3 (1978), С. 481--494.

2. {\it Rueda R., O' Reilly F.} Tests of fit for discrete distributions based on the probability generating function // Communications in Statistics - Simulation and Computation Vol. 28 , Iss. 1,1999, pp 259-274.
