\begin{center}{ \bf  МОДИФИЦИРОВАННЫЕ ЭКСТРАГРАДИЕНТНЫЕ МЕТОДЫ ДЛЯ ВАРИАЦИОННЫХ НЕРАВЕНСТВ}\\
{\it В.В. Семёнов } \\
(Киев; {\it semenov.volodya@gmail.com})
\end{center}
\addcontentsline{toc}{section}{Семёнов В.В.\dotfill}


Доклад основан на результатах недавних исследований, опубликованных в  [1--6].

Пусть $H$ -- действительное гильбертово пространство со скалярным произведением $\left(\cdot, \cdot \right)$ и порождённой нормой  $\left\| \cdot \right\|$, $C$ -- непустое выпуклое замкнутое подмножество пространства  $H$ и $A:H\to H$ -- некоторый оператор. Рассмотрим вариационное неравенство:
$$
	\mbox{найти} \  x \in C :  \ \left(Ax, y-x \right)\geqslant 0\quad \forall y \in C. \eqno{(1)}
$$
Множество решений вариационного неравенства (1) обозначим  $VI\left(A, C\right)$.

Будем предполагать выполненными следующие условия:
\begin{itemize}
  \item[(A1)] $VI\left(A, C\right)\ne \emptyset $;
  \item[(A2)]  оператор $A:H\to H$ -- монотонный, равномерно непрерывный  на ограниченных множествах и отображает ограниченные множества в ограниченные.
\end{itemize}

Далее  рассмотрим модификацию cубградиентного экстраградиентного алгоритма с динамической регулировкой величины шага, предложенную в [1].

{\it
\textbf{Инициализация.} Задаём числовые параметры  $\sigma >0$, $\tau \in \left(0, 1\right)$, $\theta \in \left(0, 1\right)$ и элемент  $x_{0} \in H$.

\textbf{Итерационный шаг.} Для $x_{n} \in H$ вычисляем
$$
y_{n} = P_{C} \left(x_{n} -\lambda _{n} Ax_{n} \right),
$$
где $\lambda _{n}$ получаем из условия
$$
\left\{
\begin{array}{l}
  j\left(n\right) = \min  \left\{j\geqslant 0: \; \;  \left\| AP_{C} \left(x_{n} -\sigma \tau ^{j} Ax_{n} \right)-Ax_{n} \right\| \right. \leqslant \\ \quad \quad \quad \quad \quad \quad \quad \quad \quad \quad \left.\leqslant \frac{\theta}{\sigma \tau ^{j}} \left\| P_{C} \left(x_{n} -\sigma \tau ^{j} Ax_{n} \right)-x_{n} \right\| \right\},  \\
\lambda_{n} =\sigma \tau ^{j\left(n\right)} .
\end{array}
\right.
$$
Если $y_{n} = x_{n}$ то конец и $x_{n}$ --- решение (1), иначе вычисляем
$$
x_{n+1} =P_{T_{n} } \left(x_{n} -\lambda _{n} Ay_{n} \right),
$$
где
$$
T_{n} =\left\{z\in H:\; \left(x_{n} -\lambda _{n} Ax_{n} -y_{n} ,z-y_{n} \right)\leqslant 0\right\}.
$$
}

\textbf{Лемма~1.} {\it
Правило выбора параметра $\lambda _{n}$ корректно, то есть
$$
j\left(n\right)<+\infty .
$$}

\textbf{Лемма~2.} {\it
Для последовательностей $\left(x_{n} \right)$, $\left(y_{n} \right)$, порождённых алгоритмом, имеет место неравенство
$$
\left\| x_{n+1} -z\right\|^{2} \leqslant \left\| x_{n} -z\right\|^{2} -\left(1-\theta \right)\left\| x_{n} -y_{n} \right\|^{2} -
$$
$$
 \quad \quad \quad \quad \quad \quad \quad \quad \quad \quad \quad -\left(1-\theta \right)\left\| x_{n+1} -y_{n} \right\|^{2} , \eqno{(2)}
$$
где $z\in VI\left(A,C\right)$.}

Из неравенства (2) следует фейеровское свойство последовательности $\left(x_{n} \right)$  относительно множества $VI\left(A,C\right)$ и сходимость к нулю последовательностей $( x_{n} -y_{n} )$, $( x_{n+1} -y_{n} )$. Это позволяет получить следующий результат относительно сходимости предлагаемого итерационного алгоритма.

\textbf{Теорема~1.} {\it Последовательности $\left(x_{n} \right)$, $\left(y_{n} \right)$, порождённые алгоритмом, слабо сходятся к точке $z\in VI(A,C)$.}




Сильно сходящийся вариант   метода можно получить, используя метод итеративной регуляризации или гибридный метод.

Работа выполнена при финансовой поддержке ГФФИУ (проект \No\ F74/24921) и  МОНУ (проект \No\ 0116U004777).


%%%%  ОФОРМЛЕНИЕ СПИСКА ЛИТЕРАТУРЫ %%%
\smallskip \centerline{\bf Литература}\nopagebreak

1. {\it Denisov S.V., Semenov V.V.,  Chabak L.M.} Convergence of the Modified Extragradient Method for Variational Ine\-qua\-li\-ties with Non-Lipschitz Operators, Cybernetics and Systems Analysis, Vol. 51 (2015), p. 757-765.

2. {\it  Верлань Д.А., Семенов В.В., Чабак Л.М.} Сильно сходящийся модифицированный экстраградиентный метод для вариационных неравенств с нелипшицевыми операторами, Проблемы управления и информатики, 2015, \No\ 4, c. 37-50.

3. {\it Lyashko S.I.,   Semenov V.V.,  Voitova T.A.} Low-cost mo\-di\-fi\-cation of Korpelevich’s methods for monotone equilibrium pro\-blems, Cybernetics and Systems Ana\-lysis, Vol. 47 (2011), p. 631-639.

4. {\it Malitsky Yu.V.,  Semenov V.V.} Extragradient Algorithm for Monotone Variational Inequalities, Cybernetics and Systems Analysis, Vol. 50 (2014), p. 271-277.

5. {\it Malitsky Yu.V., Semenov V.V.} A hybrid method without extrapolation step for solving variational inequality problems,  Journal of Global Optimization, Vol. 61 (2015), p. 193-202.

6. {\it Semenov V.V.} A version of the mirror descent method to solve variational inequalities, Cybernetics and Systems Ana\-lysis, Vol. 53 (2017), p. 234-243.
