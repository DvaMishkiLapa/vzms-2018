\begin{center}{ \bf  О  СМЕШАННОЙ ЗАДАЧЕ ДЛЯ ВОЛНОВОГО УРАВНЕНИЯ И  ОПЕРАТОРЕ ШТУРМА-ЛИУВИЛЛЯ НА ГРАФЕ}\\
{\it  Я. П. Павлюк } \\
(Воронеж; {\it ioanna1997@yandex.ru } )
\end{center}
\addcontentsline{toc}{section}{Павлюк  Я. П.}


Пусть $\Gamma$ ---  геометрический   граф, состоящий из двух рёбер,
одно из которых образует цикл. Теория дифференциальных уравнений и
краевых задач на геометрических графах   активно развивается (см,
например,  работы [1-5]). Спектральные задачи, а также смешанные
задачи для уравнений первого порядка с инволютивным отклонением,
заданные на  графе указанной структуры, рассматривались в [6-7].
Смешанные задачи для волнового уравнения на графе, состоящем из двух
колец, также изучались в [8-9].

Рассмотрим   смешанную задачу для волнового уравнения на $\Gamma$,
которая в соответствие с работами [8-9] представляет собой следующую
задачу в пространстве вектор-функций $u(x,t)= (u_1(x,t),u_2(x,t))^T
$ ($T$
--- знак транспонирования):
    $$\begin{array}{c}
    \frac{\partial^2 u_j(x,t)}{\partial t^2}=\frac{\partial^2 u_j(x,t)}{\partial x^2},    \\
    u_1(0,t) = u_1(1,t) = u_2(0,t), \quad u_2(1,t)=0,    \\
    u_{1x}'(0,t)-u_{1x}'(1,t)+u_{2x}'(0,t)=0,   \\
    u_1(x,0)=\varphi_1(x), \quad u_2(x,0)=\varphi_2(x), \quad  x\in [0,1],   \\
    u_{1t}'(x,0)=u_{2t}'(x,0)=0.
    \end{array}\eqno(1)$$

    Используя методы из [9], получено классическое решение задачи (1)   при минимальных требованиях на $\varphi(x)=(\varphi_1(x),\varphi_2(x))^T $, где
    $\varphi_j(x)\in C^2[0,1]$ и комплекснозначные,
    $$\begin{array}{c}
    \varphi_1(0) = \varphi_1(1) = \varphi_2(0), \   \varphi_2(1)=0, \  \varphi_1'(0)-\varphi_1'(1)+\varphi_2'(0)=0, \\
       \varphi_1''(0)=\varphi_1''(1)=\varphi_2''(0), \quad  \varphi_2''(1)=0.  \end{array}$$ \\

    \textbf{Теорема 1.}
\textit{Классическое решение задачи (1)  имеет вид
$$u(x,t)=\frac12\bigl(F(x+t)+F(x-t)\bigr),$$
где $F(x)=\left(F_1(x),F_2(x)\right)^T$, $F(x)$ --- дважды
непрерывно дифференцируемая   функция, причём  при $x\in [0,1]$
$F(x)=\varphi(x)$, а на всю ось $F(x)$ продолжается с помощью
соотношений:
$$\begin{array}{l}
F_1(-t)=\frac{1}{3}[2F_1(1-t)+2F_2(t)-F_1(t)], \\
 F_2(-t)=\frac{1}{3}[2F_1(t)+2F_1(1-t)-F_2(t)],\\
F_1(1+t)=\frac{1}{3}[2F_1(t)+2F_2(t)-F_1(1-t)], \\
F_2(1+t)=-F_2(1-t).
\end{array} $$}

В случае наличия потенциалов (добавлении в уравнения  слагаемых
$q_k(x)u_k(x,t)$, $q_k(x)\in C[0,1]$) и краевого условия для $u_k'$
более общего вида предполагается использование методов из [10]. По
методу Фурье в этом случае получаем спектральную задачу $Ly=\lambda
y$ для оператора Штурма-Лиувилля:
$$
    Ly=\Bigl(y_1''(x) + q_1(x) y_1(x), \,
    y_2'(x) +  q_2(x) y_2(x)\Bigl)^T,\   x\in [0,1], $$
%     $$, с краевыми условиями:
    $$\begin{array}{c}
    y_1(0)=y_1(1)=y_2(0), \quad
    y_2(1)=0, \\
    y_1'(0) - y_1'(1) + y_2'(0) + y_2(0)=0.
    \end{array}
    $$
Получена асимптотика собственных значений   оператора~$L$.

    \textbf{Теорема 2.}
\textit{Собственные значения оператора $L$ образуют три серии:
$\lambda _n^{(k)}=(\rho_n^{(k)})^2$ с асимптотикой $\rho_n^{(1)}=\pi
n + \varepsilon_n^{(1)}$, $\rho_n^{(2)}=b_1 n + 2\pi n +
\varepsilon_n^{(2)}$, $\rho_n^{(3)}=b_2 + 2\pi n +
\varepsilon_n^{(3)}$,
    где $b _{1,2}=-i \ln(\frac{2\pm\sqrt{5}i}{3})$, $\varepsilon_n^{(k)}=\tiny{o}(1)$.} \\


    \smallskip {\bf Литература}\nopagebreak

1.  Покорный Ю.В. Дифференциальные уравнения на геометрических
графах / Ю.В. Покорный [и др.] --- М. : Физматлит, 2004. -- 272 с.

2.  Провоторов В.В. Собственные функции задачи Штурма–Лиувилля на
графе-звезде // Математический сборник. 2008. Т. 199. № 10. С.
105-126.

3.  Волкова А.С., Провоторов В.В. Обобщённые решения и обобщённые
собственные функции краевых задач на геометрическом графе  //
Известия высших учебных заведений. Математика. 2014. № 3. С. 3-18.

4. Завгородний М.Г. Сопряжённые и самосопряжённые краевые задачи на
геометрическом графе   //
 Дифференциальные уравнения. 2014. Т. 50. № 4. С.
446.

5. Головко Н.И., Голованева Ф.В., Зверева М.Б., Шабров С.А. О
возможности применения метода Фурье к разнопорядковой математической
модели // Вестник ВГУ. Сер. :  Физика. Математика. --- Воронеж,
2017. --- № 1. --- С. 91-98 .

6.  Бурлуцкая М. Ш., Хромов  А.П.  О равносходимости разложений по
собственным функциям функционально-диф\-фе\-рен\-циаль\-ного
оператора первого порядка на  графе из двух рёбер,  содержащем  цикл
// Диффер. уравн. Т. 43, №  12,  2007. - С. 1597-1605.

7. Бурлуцкая М.Ш. Смешанная задача с инволюцией на графе  из двух
рёбер с циклом// Докл. РАН. --- 2012. --- Т. 447, № 5. --- С.
479-482.

8. Бурлуцкая М.~Ш.  Явное решение одной смешанной задачи с
 инволюцией на графе  // Вестник ВГУ.  Сер. : Физика. Математика. --- Воронеж, 2014 ---  № 3.
    --- С.~79-88.

9.  Бурлуцкая М.Ш. Метод Фурье в смешанной задаче для волнового
уравнения на графе   // Докл. РАН. --- 2015. --- Т. 465, № 5. --- С.
519–522.

10.  Бурлуцкая М.Ш., Хромов А.П. Резольвентный подход в методе Фурье
  // Докл. РАН. --– 2014. --- Т. 458, №2.
--- С. 138-140.
