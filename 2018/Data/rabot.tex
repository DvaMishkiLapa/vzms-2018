\begin{center}
\textbf{О коммутации одного класса вырождающихся псевдодифференциальных
операторов и оператора дифференцирования}
\end{center}

\begin{center}
\textbf{А.Д. Баев, Н.И. Работинская}
\end{center}
\addcontentsline{toc}{section}{Баев А.Д., Работинская Н.И.}


\begin{center}
Воронежский государственный университет
\end{center}

{\sloppy

Исследование теории вырождающихся псевдодифференциальных уравнений в
настоящее время является актуальной задачей в связи с использованием этих
операторов при доказательстве теорем о существовании решений и получении
коэрцитивных априорных оценок решений краевых задач для вырождающихся
уравнений. Такие краевые задачи возникают, например, при моделировании
процессов гидродинамики с сингулярными особенностями. В настоящей работе
исследуется вопрос об ограниченности одного класса весовых
псевдодифференциальных операторов, построенных по специальному интегральному
преобразованию $F_\alpha $, введенному в [1]. Теорема об ограниченности
доказывается в специальных весовых пространствах типа пространств С.Л.
Соболева.

Рассмотрим функцию $\alpha (t),\,\,t \in R_ + ^1 $, для которой $\alpha ( +
0) = {\alpha }'( + 0) = 0$, $\alpha \mbox{(}t\mbox{) > 0}$ при $t > 0$,
$\alpha \mbox{(}t\mbox{) = const}$ для $t \ge d$ при некотором $d\mbox{ >
0}$.

Следуя [1] введём интегральное преобразование
\begin{equation}
\label{eq4800}
F_\alpha [u(t)](\eta ) = \int\limits_0^{ + \infty } {u(t)\exp (i\eta }
\int\limits_t^d {\frac{d\rho }{\alpha (\rho )}} )\frac{dt}{\sqrt {\alpha
(t)} },
\end{equation}
определенное первоначально, например, на функциях $u(t) \in C_0^\infty (R_ +
^1 )$. Преобразование (\ref{eq4800}) связано с преобразованием Фурье
\[
F_{\tau \to \eta } [u] = \int\limits_{ - \infty }^{ + \infty } {u(\tau )\exp
(i\eta } \tau )d\tau ,\,\,\,\,\eta \in R^1
\]
следующим равенством
\[
F_\alpha [u(t)](\eta ) = F_{\tau \to \eta } [u_\alpha (\tau )],
\]
где $u_\alpha (\tau ) = \left. {\sqrt {\alpha (t)} u(t)} \right|_{t =
\varphi ^{ - 1}(\tau )} ,\,\,\,t = \varphi ^{ - 1}(\tau )$ - функция,
обратная к функции $\tau = \varphi (t) = \int\limits_t^d {\frac{d\rho
}{\alpha (\rho )}} .$

Для преобразования $F_\alpha $ справедлив аналог равенства Парсеваля
\[
\left\| {F_\alpha [u](\eta )} \right\|_{L_2 (R^1)} = \sqrt {2\pi } \left\| u
\right\|_{L_2 (R_ + ^1 )} \quad .
\]



Это равенство позволяет расширить преобразование (\ref{eq4800}) до непрерывного
преобразования, осуществляющего гомеоморфизм пространств $L_{2} (R^1)$
и $L_{2} (R_ + ^1 )$, а также рассмотреть это преобразование на
некоторых классах обобщенных функций. Для расширенного таким образом
преобразования $F_\alpha $ сохраним старое обозначение. Обозначим через
$F_\alpha ^{ - 1} $ обратное к $F_\alpha $ преобразование, отображающее
$L_{2} (R^1)$ на $L_{2} (R_ + ^1 )$. Это преобразование можно
записать в виде
\[
F_\alpha ^{ - 1} [w(\eta )](t) = \left. {\frac{1}{\sqrt {\alpha (t)}
}F_{\eta \to \tau }^{ - 1} [w(\eta )]} \right|_{\tau = \varphi (t)} .
\]



Можно показать, что на функциях $u(t) \in C_0^\infty (\bar {R}_ + ^1 )$
выполняются соотношения
\[
F_\alpha [D_{\alpha ,t}^j u](\eta ) = \eta ^jF_\alpha [u](\eta ),\,\,j =
1,2,...,\mbox{г}\mbox{д}\mbox{е}
\]\[
D_{\alpha ,t} = \frac{1}{i}\sqrt {\alpha (t)} \partial _t \sqrt {\alpha (t)}
,\,\,\,\,\partial _t = \frac{\partial }{\partial t}.
\]



\textbf{Определение 1.} Пространство $H_{s,\alpha } (R_ + ^n )$ ($s$ --
действительное число) состоит из всех функций $v(x,t) \in L_2 (R_ + ^n )$,
для которых конечна норма
\[
\left\| v \right\|_{s,\alpha }^2 = \int\limits_{R^n} {(1 + \left| \xi
\right|^2 + \eta ^2)^s\left| {F_\alpha F_{x \to \xi } [v(x,t)]}
\right|^2d\xi d\eta } .
\]



\textbf{Определение 2.} Пространство $\,H_{s,\alpha ,q} (R_ + ^n )\,\,(s
\ge 0,\,\,q > 1)$ состоит из всех функций $v(x,t) \in H_{s,\alpha } (R_ + ^n
)$, для которых конечна норма
\begin{multline*}
\left\| v \right\|_{s,\alpha ,q} =
\\=
\{\sum\limits_{l = 0}^{[\frac{s}{q}]}
{\left\| {F_{\xi \to x}^{ - 1} F_\alpha ^{ - 1} [(1 + \left| \xi \right|^2 +
\eta ^2)^{\frac{s - ql}{2}}F_\alpha F_{x \to \xi } [\partial _t^l v]]}
\right\|_{L_2 (R_n^ + )}^2 } \}^{\frac{1}{2}}.
\end{multline*}



Здесь $[\frac{s}{q}]$ - целая часть числа $\frac{s}{q}.$



Пусть выполнено следующее условие.

\textbf{Условие 1.} Существует число $\nu \in \mbox{(0,1]}$ такое, что
$\left| {\alpha '(t)\alpha ^{ - \nu }(t)} \right| \le c < \infty $ при всех
$t \in [0, + \infty )$. Кроме того, $\alpha (t) \in C^{s_1 }[0, + \infty )$
для некоторого $s_1 \ge 2N - \left| \sigma \right|$, где $N \ge \mathop
{\max }\limits_{0 \le p_1 \le l} \{2p_1 + \frac{l - p_1 + \frac{3}{2}}{\nu }
+ 1,\,\,\sigma + 1,\,\,\sigma + \frac{l}{2}\},\,\,l = 1,\,\,2..., \quad \sigma $
- некоторое действительное число.

Заметим, что указанное выше число $\nu $ существует, если $\alpha ( + 0) =
\alpha '( + 0) = 0$.

С помощью преобразования (\ref{eq4800}) и преобразования Фурье $F_{x \to \xi } = F_{x_1
\to \xi _1 } F_{x_2 \to \xi _2 } ...F_{x_{n - 1} \to \xi _{n - 1} } $
определим весовой псевдодифференциальный оператор по формуле
\[
P^{(\sigma )}(t,D_x ,D_{\alpha ,t} )v(x,t) = F_\alpha ^{ - 1} F_{\xi \to
x}^{ - 1} [p(t,\xi ,\eta )F_{x \to \xi } F_\alpha [v(x,t)]].
\]



\textbf{Определение 3.} Будем говорить, что символ $p(t,\xi ,\eta )$
весового псевдодифференциального оператора $P^{(\sigma )}(t,D_x ,D_{\alpha
,t} )$ принадлежит классу символов $S_{\alpha ,\rho ,\delta }^\sigma (\Omega
)$, где $\Omega \subset \bar {R}_ + ^1 ,
\sigma \in
\mbox{R}^1, 0 \le \delta < \rho \le 1$, если функция $p(t,\xi ,\eta )$
является бесконечно дифференцируемой функцией по переменной $t \in \Omega $
и по переменной $\eta \in R^1\,$. Причем, при всех $j =
0,\,\,1,\,\,2,...,\,\,\,\,l = 0,\,\,1,\,\,2,...$ справедливы оценки
\[
\vert (\alpha (t)\partial _t )^j\partial _\eta ^l \lambda (t,\xi ,\eta
)\vert \le c_{jl} (1 + \left| \xi \right| + \left| \eta \right|)^{\sigma -
\rho l + \delta j}
\]
с константами $c_{jl} > 0$, не зависящими от $\xi \in R^{n - 1},\,\,\,\eta
\in R^1$, $t \in K$, где $K \subset \Omega $ - произвольный отрезок.

Справедливо следующее утверждение.

\textbf{Теорема 1.} Пусть символ $\lambda (t,\xi ,\eta )\,$ весового
псевдодифференциального оператора $\mbox{K}^{\mbox{(}\sigma \mbox{)}}(t,D_x
,D_{\alpha ,t} )$ принадлежит классу $S_\alpha ^\sigma (\Omega )$, $\Omega
\subset \bar {R}_ + ^1 ,\,\,\,\,\sigma \in R^1$. Пусть $v(x,t) \in H_{s +
\sigma ,\alpha } (R_ + ^n )$, $\partial _t^l v(x,t) \in H_{s + \sigma
,\alpha } (R_ + ^n ),\,\,\,\,\,l = 1,2...$. Пусть выполнено условие 1 (с
заменой $\sigma $ на $s + \sigma )$. Тогда для оператора
\begin{equation}
\label{eq4801}
M_{l,\sigma } = \,\,\partial _t^l K^{(\sigma )}(t,D_x ,D_{\alpha ,t} )\, -
K^{(\sigma )}(t,D_x ,D_{\alpha ,t} )\,\partial _t^l
\end{equation}
справедлива оценка
\[
\left\| {M_{l,\sigma } v} \right\|_{s,\alpha } \le c(\sum\limits_{j = 0}^l
{\left\| {\,\partial _t^j v} \right\|_{s + \sigma - 1,\alpha } } +
\sum\limits_{j = 0}^{l - 1} {\left\| {\,\partial _t^j v} \right\|_{s +
\sigma ,\alpha } } )
\]
с константой $c > 0\,$, не зависящей от $v$.

\textbf{Теорема 2.} Пусть $q > 1,\,\,\,s \ge 0$ - действительные числа
$v(x,t) \in H_{s + (l + 1)q,\alpha ,q} (R_ + ^n )$. Пусть символ $\lambda
(t,\xi ,\eta )$ весового псевдодифференциального оператора $K^{(\sigma
)}(t,D_x ,D_{\alpha ,t} )$ принадлежит классу $S_\alpha ^q (\Omega
),\,\,\,\,\Omega \subset \bar {R}_ + ^1 $. Пусть выполнено условие 1 при
$\sigma \,\, = s + q\,\,$. Тогда для оператора $M_{l,q} $, определенного в
(\ref{eq4801}) при $\sigma = q\,\,$, справедлива оценка $\left\| {M_{l,q} v}
\right\|_{s,\alpha ,q} \le c\left\| v \right\|_{s + (l + 1)q - 1,\alpha ,q}
\,$
с постоянной $\,c > 0$, не зависящей от $v$.



Аналогичные свойства для других классов псевдодифференциальных операторов
доказаны в [1] - [8].

\begin{center}
Литература
\end{center}

1. Баев А. Д. Вырождающиеся эллиптические уравнения высокого порядка и
связанные с ними псевдодифференциальные операторы / А. Д. Баев // Доклады
Академии наук. -- 1982. - Т. 265, № 5. - С. 1044 -- 1046.

2. Баев А.Д. Об общих краевых задачах в полупространстве для вырождающихся
эллиптических уравнений высокого порядка /А.Д. Баев// Доклады Академии наук,
2008, т. 422, №6, с. 727 -- 728.

3. Баев А. Д., Садчиков П. В. Априорные оценки и существование решений
краевых задач в полупространстве для одного класса вырождающихся
псевдодифференциальных уравнений / А. Д. Баев, П. В. Садчиков // Вестник
ВГУ. Серия: Физика. Математика.-- 2010, №1. -- С. 162-168.

4. Баев А.Д. О некоторых свойствах одного класса псевдодифференциальных
операторов с вырождением /А.Д. Баев, П.А. Кобылинский// Вестник ВГУ. Серия:
Физика. Математика.-- 2014, №2. -- С. 66-73.

5. Баев А.Д. О свойствах коммутации одного класса вырождающихся
псевдодифференциальных операторов /А.Д. Баев, П.А. Кобылинский// Вестник
ВГУ. Серия: Физика. Математика.-- 2014, №4. -- С. 102 -- 108.

6. Баев А.Д. О некоторых свойствах одного класса вырождающихся
псевдодифференциальных операторов /А.Д. Баев, П.А. Кобылинский// Доклады
Академии наук. -- 2015. - Т. 460, № 2. - С. 133 -- 135.

7. Баев А.Д. Теоремы об ограниченности и композиции для одного класса
весовых псевдодифференциальных операторов /А.Д. Баев, Р.А. Ковалевский//
Вестник Воронежского государственного университета Серия: Физика.
Математика.-- 2014, №1. -- С. 39- 49.

8. Баев А.Д. О некоторых краевых задачах для псевдодифференциальных
уравнений с вырождением /А.Д. Баев, П.А. Кобылинский// Доклады Академии
наук. -- 2016. - Т. 466, № 4. - С. 385 -- 388.





Работа выполнена при финансовой поддержке гранта Российского научного фонда
№16-11-10125, выполняемого в Воронежском госуниверситете.

}
