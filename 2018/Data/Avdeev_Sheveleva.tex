\begin{center}{
	\bf
	Математическое моделирование распространённых орфографических ошибок
	с помощью регулярных выражений на материале математических образовательных ресурсов в сети <<Интернет>>
} \\
{\it Н.Н.Авдеев } \\  {\it К.В.Шевелева  } \\
(Воронеж; {\it nickkolok@mail.ru};{\it ksyusha.shevelyova@yandex.ru})
\end{center}
\addcontentsline{toc}{section}{Авдеев Н.Н., Шевелева К.В.}

\setcounter{table}{0}


В настоящее время Интернет играет большую роль в образовании и самообразовании современных школьников и студентов [1].
С другой стороны, серьёзную обеспокоенность научного сообщества вызывает грамотность интернет-ресурсов [2].
Нами был проведён анализ грамотности текстов на материале наиболее популярных математических сайтов по двум категориям:
авторские монотексты (АМТ), т.е. тексты, написанные преимущественно одним человеком: справочники, статьи и т.д.,
и пользовательский контент
\linebreak
(UGC, англ. user-generated content), т.е. комментарии и форумы.
Суммарная мощность корпуса составила примерно $3.5\cdot 10^7$ словоупотреблений.
Для анализа использовалась программа для ЭВМ, описанная в [3], с доработками программного кода и словарей, описанными в [4].
Там же приведён рейтинг сайтов по количеству обнаруженных ошибок и их средней плотности.
Исходный код программы опубликован по адресу https://github.com/nickkolok/chas-correct

Моделью орфографической ошибки в нашем исследовании является регулярное выражение [5][6].
Ввиду сложности написания точного регулярного выражения,
описывающего только ошибки, используются регулярные выражения,
описывающие возможные верные и неверные написания слова или словосочетания,
снабжённые точным текстом верного написания.
В случае, если замена регулярного выражения на верное написание привела к изменению текста,
орфографическая ошибка считается найденной.

Например, моделью ошибочного написания слова <<в общем>> является пара из регулярного выражения
\linebreak
<<\verb"/([^А-Яа-яЁёA-Za-z­]|^|$)([вВ])[\s-]*о+[пб]щем" \\ \verb"([^А-Яа-яЁёA-Za-z­]|^|$)/gm">> и строки
<<\verb"в общем">> (с соответствующим регистром первой буквы).
Рейтинг слов и словосочетаний, в которых программа обнаружила орфографические ошибки,
представлен в табл. 1--2.
\begin{table}[h]
\caption{\label{tab:amt_error}Популярные ошибки в АМТ.}
\begin{center}
\begin{tabular}{|c|c|c|c|c|c|}
\hline
{№} & {Ошибка} & {Кол-во} & {№} & {Ошибка} & {Кол-во} \\
\hline
1 & По сути  & 14       &  6 & То есть & 4 \\
2 & В общем & 13        &  7 & Комментарии & 4 \\
3 & В виду  & 11        &  8 & Рассчитывать & 4 \\
4 & Истолковано & 9     &  9 & В общем-то & 3 \\
5 & По порядку & 5      & 10 & Всё равно  & 3 \\
\hline
\end{tabular}
\end{center}
\end{table}
\begin{table}[h]
\caption{\label{tab:ugc_error}Популярные ошибки в UGC.}
\begin{center}
\begin{tabular}{|c|c|c|c|c|c|}
\hline
{№} & {Ошибка} & {Кол-во} & {№} & {Ошибка} & {Кол-во}\\
\hline
1 & В виду & 2238   &  6 & Во-первых & 625 \\
2 & Наверное & 1419 &  7 & Вообще & 403 \\
3 & В общем & 1223  &  8 & Хотя бы & 385 \\
4 & То есть & 901   &  9 & Сейчас & 330 \\
5 & -нибудь & 862   & 10 & Прийти  & 253 \\
\hline
\end{tabular}
\end{center}
\end{table}

Примечательно, что распределение частотности ошибки в зависимости от ранга хорошо описывается
элементарными обобщениями закона Ципфа [7],
а именно сдвинутой гиперболой или степенной зависимостью с отрицательным показателем.

Обозначим через $n$ ранг ошибки, через $x(n)$ --- количество ошибок в корпусе.
Для АМТ имеем приближения
$x(n) \sim y(n) = 1.1339+{17.2396}/{n}$, $R^2 = 0.8420$ или
$x(n) \sim z(n) = 16.6329n^{-0.6599}      $, $R^2 = 0.9247$.

Для UGC имеем приближения
$x(n) \sim y(n) = 2.3594+{2609.7886}/{n}$, $R^2 = 0.9534$ или
$x(n) \sim z(n) = 22360.38n^{-1.4854}         $, $R^2 = 0.9147$.

\smallskip \centerline{\bf Литература}\nopagebreak

1. {\it Каменкова Н. Г. } Использование интернет-технологий при организации изучения курса «Математика и информатика» //Герценовские чтения. Начальное образование. – 2010. – Т. 1. – С. 288-293.

2. {\it Сон Л. П.} Интернет коммуникация и проблема грамотности индивида //Армия и общество. – 2013. – №. 4 (36).

3. {\it Авдеев Н. Н.} Программа анализа грамотности интер\-нет\--СМИ // Культура общения и её формирование. – 2016. – С. 81-83.

4. {\it Авдеев Н. Н., Шевелева К. В.} Анализ орфографической грамотности
математических образовательных ресурсов в сети «Интернет» // Некоторые вопросы анализа, алгебры, геометрии и математического
образования – Воронеж: Издательско-полиграфический центр «Научная книга», 2017. – Вып. 7, Часть I – 244 с.

5. \textit{Гошко В.} Регулярные выражения и поиск текста в Perl //Системный администратор. – 2003. – №. 8. – С. 78-86.

6. \textit{Курбатов С. С., Красовицкий И.}
Генерация регулярных выражений из фраз русского языка с использованием онтологии //Национальная ассоциация учёных. – 2015. – №. 3-6. – С. 14-16.

7. \textit{Маслов В. П., Маслова Т. В.} О законе Ципфа и ранговых распределениях в лингвистике и семиотике //Математические заметки. – 2006. – Т. 80. – №. 5. – С. 718-732.
