\begin{center}{ \bf  ЧЕМ АКСИОМЫ ОТЛИЧАЮТСЯ ОТ ПОСТУЛАТОВ}\\
{\it Г.А. Зверкина } \\
(Москва; {\it zverkina@inbox.ru} )
\end{center}
\addcontentsline{toc}{section}{Зверкина Г.А.\dotfill}


Как известно, современная математика построена по принципам, заложенным учёными Древней Греции, а именно по образу ``Начал'' Евклида.
То есть имеется некоторый набор аксиом (недоказуемых утверждений), из которых по некоторым правилам логики выводятся новые математические факты.
Однако в тексте Евклида кроме аксиом присутствуют также некоторые недоказуемые утверждения, называемые ``пос\-ту\-ла\-та\-ми'' (от лат. {\it pos\-tu\-la\-tus} ``жа\-ло\-ба, иск'',  из {\it po\-s\-tu\-la\-re} ``требовать, просить''; греч. $\alpha\!\!\stackrel{,}{\iota}\!\!\tau\mbox{\'{$\!\!\!\!\eta$}}\mu\alpha\tau\alpha$ -- требование).
В Средние века арабоязычные математики, познакомившиеся в греческой наукой, воспринимали постулаты как некие специфические аксиомы, и именно в это время начинаются попытки доказательства знаменитого пятого постулата.
Как известно, позднее этим вопросом заинтересовались и европейские учёные, результатом чего стали исследования Яно\-ша Бойяи (Bolyai J\'anos, 1802--1860), Карла Фридриха Гаус\-са (Johann Carl Friedrich Gau\ss, 1777--1855) и Николая Ивановича Лобачевского (1792--1856).
При этом всегда постулаты воспринимались как разновидность аксиом, по неизвестной причине выделенных в отдельную группу, сформулированную {\it перед} аксиомами (и сейчас, когда формулируется некая аксиома, математик часто говорит: ``я постулирую\ldots''):

ПОСТУЛАТЫ. Допустим:
\\
П1. {\it Что от всякой точки до всякой точки <можно> провести прямую линию.}
\\
П2. {\it И что ограниченную прямую <можно> непрерывно продолжать по прямой.}
\\
П3. {\it И что из всякого центра и всяким раствором <может быть> описан круг.}
\\
П4. (Акс. 10.) {\it  И что все прямые углы равны между собой.}
\\
П5. (Акс. 11.) {\it  И если прямая, падающая на две прямые, образует внутренние и по одну сторону углы, меньшие двух прямых, то продолженные эти две прямые неограниченно встретятся с той стороны, где углы
меньшие двух прямых.}

Сравните это с аксиомами:
\\
A1. {\it Равные одному и тому же равны и между собой. }
\\
A2. {\it И если к равным прибавляются равные, то и целые будут равны.}
\\
A3. {\it И если от равных отнимаются равные, то остатки будут равны.}
\\
A4. {\it И если к неравным прибавляются равные, то целые будут не равны.}
\\
A5. {\it  И удвоенные одного и того же равны между собой.}
\\
A6. {\it И половины одного и того же равны между собой.}
\\
A7. {\it  И совмещающиеся друг с другом равны между собой.}
\\
A8. {\it И целое больше части.}
\\
A9. {\it И две прямые не содержат пространства.}

Разнородность постулатов и аксиом видна невооруженным глазом!
И чтобы понять, почему Евклид разделил аксиомы и постулаты на две группы утверждений, надо попытаться понять то, в каких условиях он творил.

Мы часто забываем, что условия, в которых работали математики прошлого, и их инструментарий (в т.ч. теоретический) сильно отличались от условий современного математического исследования.
Здесь не имеется в виду современная электронная техника -- мы не задумываемся о том, на чём писали или чертили свои чертежи наши предшественники, как они выполняли вычисления, какими геометрическими и арифметическими инструментами они пользовались.
Вплоть до распространения бумаги из Китая на Ближний Восток геометрические чертежи и арифметические расчёты делались на разнообразных подсобных поверхностях -- у богатых имелись восковые таблички, менее обеспеченные писали (чертили) на черепках или ровных земляных пло\-щад\-ках; папирус был дорог, а временами и недоступен, а сменивший его пергамент был ещё дороже.
Лишь окончательный результат исследования мог быть записан на дорогом материале типа пергамента или папируса, а все предварительные исследования чаще всего  производились весьма неуклюжими и крупными геометрическими инструментами на подготовленных земляных площадках.
Теперь становятся ясными постулаты или ``требования'', предъявляемые к этим основам чертежей:
\\
П1. {\it Площадка для чертежа не имеет ям или иных препятствий для прочерчивания прямой линии.}
\\
П2. {\it Пло\-щад\-ка для чер\-те\-жа до\-ста\-точ\-но ве\-ли\-ка, что\-бы \linebreak вмес\-тить все необходимые детали чертежа.}
\\
П3. {\it Повтор: площадка для чертежа достаточно велика.}
\\
П4. (Акс. 10.) {\it  Площадка не имеет бугров и впадин, искажающих углы между начерченными линиями -- т.е. это площадка с постоянной кривизной.}
\\
П5. (Акс. 11.) {\it  И эта кривизна - нулевая.}

Объём тезисов доклада не позволяет привести достаточно аргументов для подтверждения этой интерпретации постулатов.
Надеемся, время для доклада будет достаточным, чтобы убедить слушателей в обоснованности высказанного.





\smallskip \centerline{\bf Литература}\nopagebreak

1. {\it Евклид.} Начала Евклида. Книги I-VI  / пер. с греч. и комм. Д.Д. Мордухай-Болтовского при редакционном
участии М.Я. Выгодского и И.Н. Веселовского. -- М.-Л.: ГТТИ, 1948. -- 448 с.

1. {\it Зверкина Г.А.} Об аксиомах и постулатах в античной математике// Труды X Международных Колмогоровских Чтений. Ярославль,
2012.  с.160-164.
