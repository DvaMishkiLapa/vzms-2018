\begin{center}{ \bf  КВАЗИОПТИМАЛЬНОЕ ТОРМОЖЕНИЕ ДВИЖЕНИЙ ТВЕРДОГО ТЕЛА С ВНУТРЕННЕЙ СТЕПЕНЬЮ СВОБОДЫ В СРЕДЕ С СОПРОТИВЛЕНИЕМ}\\
{\it Л.Д. Акуленко, Д.Д. Лещенко, Т.А. Козаченко } \\
(Москва, ИПМехРАН; {\it kumak@ipmnet.ru}, Одесса; {\it leshchenko$\_$d@ukr.net, kushpil.t.a@gmail.com} )
\end{center}
\addcontentsline{toc}{section}{Акуленко Л.Д., Лещенко Д.Д., Козаченко Т.А.\dotfill}

Рассматривается задача квазиоптимального торможения вращений динамически симметричного твёрдого тела с подвижной массой, прикреплённой к точке на оси симметрии. Считается, что при движении на тело действует упругая сила и сила сопротивления, пропорциональная квадрату скорости. На тело также действует момент сил линейного сопротивления среды.

Система уравнений управляемого движения в проекциях на главные центральные оси инерции тела имеет вид [1-4]

$$
\begin{array}{c}
A_1\dot p + (A_3 - A_1)qr=-b_1A_1pG^{-1}+FG^2qr+Spr^6\omega_{\perp}-\lambda A_1p,\\
A_1\dot q + (A_1 - A_3)pr=-b_2A_1qG^{-1}-FG^2pr+Sqr^6\omega_{\perp}-\lambda A_1q,\\
A_3\dot r =-b_3A_3rG^{-1}-A_1A_3^{-1}Sr^5\omega_{\perp}^3-\lambda A_3r,\\
0<A_3\leqslant 2A_1,\,\, A_3\neq A_1.
\end{array}
\eqno(1)$$


Здесь $p, q, r$ – проекции вектора угловой скорости тела $ \pmb {\omega}$ на связанные оси, $\mathbf{J}=\text{diag}(A_1,A_1,A_3)$ – тензор инерции невозмущённого тела; кинетический момент тела $ \bf{G}=\bf{J}\pmb{\omega} $, его модуль $ G=\left|\mathbf{G}\right|=\left( A_1^2\omega_\perp^2+A_3^2r^2\right)^{1/2},\,\, \omega_\perp^2=p^2+q^2$.

Считается, что момент сил диссипации пропорционален кинетическому моменту.

Компоненты управляющего момента сил представлены в виде [3]:
$$
M_i^u=b_i u_i,\,\,u_i=-G_iG^{-1},\,\,i=1,2,3,\,\,|{\bf u}|\leqslant 1. \eqno(2)
$$

Отметим, что при $ b_1=b_2=b_3=b,$  управление (2) является оптимальным. Если величины $ b_i $ близки, то указанный закон будет квазиоптимальным.

Введённые в (1) обозначения $ F,\,S $  выражаются  следующим образом [1,2]:
$$
F=m\rho^2\Omega^{-2}A_1^{-3}A_3,\,\,S=m\rho^3\Lambda\Omega^{-3}d|d|A_1^{-4}A_3^4,\,\,d=1-A_3A_1^{-1}.\eqno(3)
$$

Коэффициенты $ F,\,S $ характеризуют моменты сил, обусловленные упругим элементом. Здесь $ m $ – масса подвижной точки,   $ \rho $ – радиус-вектор точки крепления подвижной массы. Постоянные $ \Omega^2=c/m,\,\,\Lambda=\delta/m,\,\,\lambda_1=\mu/m $ определяют частоту колебаний и скорость их затухания; $ c $ – жёсткость, $ \mu $ – коэффициент квадратичного трения.

Рассматривается случай, когда коэффициенты связи $ \lambda_1 $ и $ \Omega $ таковы, что «свободные» движения точки $ m $, вызванные начальными отклонениями, затухают значительно быстрее, чем тело совершит оборот. Предполагается, что
$$
\lambda_1=\Lambda\Omega^3,\,\,\Omega\gg\omega_0,\eqno(4)
$$
где $ \omega_0 $ – модуль начального значения угловой скорости.

Неравенство (4) позволяет ввести малый параметр в (3).

Ставится задача квазиоптимального по быстродействию торможения вращений:
$$
{\pmb \omega }(T)=0, T\to\text{min}_{\bf u},\,\,|{\bf u}|\leqslant 1. \eqno(5)
$$

Параметры $ b_i $ предполагаются близкими ($ b_i \approx b,\,\,|b_i-b|\ll b $).

Для обезразмеривания задачи выбираем моменты инерции тела относительно оси $ x_1-A_1=A_2 $ и величину порядка начальной скорости – $ \omega_0 $ в качестве характерных параметров. Введём безразмерные коэффициенты инерции $ \tilde{A}_i=A_i/A_1 $ и время $ \tau=\omega_0 t$.

Система примет вид
$$
\begin{array}{c}
\dfrac{d\tilde{p}}{d\tau}=-\left(\tilde{A}_3 -1 \right)\tilde{q}\tilde{r}-\varepsilon\dfrac{\tilde{b}_1\tilde{p}}{\tilde{G}}+\varepsilon\tilde{F}\tilde{G}^2\tilde{q}\tilde{r}+\varepsilon\tilde{S}\tilde{p}\tilde{r}^6\left(\tilde{p}^2+\tilde{q}^2\right)^{1/2}-\varepsilon\tilde{\lambda}\tilde{p},\\
\dfrac{d\tilde{q}}{d\tau}=-\left(1-\tilde{A}_3 \right)\tilde{p}\tilde{r}-\varepsilon\dfrac{\tilde{b}_2\tilde{q}}{\tilde{G}}-\varepsilon\tilde{F}\tilde{G}^2\tilde{p}\tilde{r}+\varepsilon\tilde{S}\tilde{q}\tilde{r}^6\left(\tilde{p}^2+\tilde{q}^2\right)^{1/2}-\varepsilon\tilde{\lambda}\tilde{q},\\
\dfrac{d\tilde{r}}{d\tau}=-\varepsilon\dfrac{\tilde{b}_3\tilde{r}}{\tilde{G}}-\varepsilon\tilde{S}\tilde{A}_3^{-2}\tilde{r}^5\left(\tilde{p}^2+\tilde{q}^2\right)^{3/2}-\varepsilon\tilde{\lambda}\tilde{r}.
\end{array}
\eqno(6)
$$
$$
\begin{array}{c}
\varepsilon\tilde{F}=m\rho^2\Omega^{-2}A_1^{-1}\tilde{A}_3\omega_0^2,\,\,\tilde{A}_1=\tilde{A}_2=1,\\
\varepsilon\tilde{b}_i=b_i/A_1\omega_0^2,\,\,\varepsilon\tilde{\lambda}=\lambda/\omega_0,\,\,\tilde{G}=G/A_1\omega_0,\\\varepsilon\tilde{S}=m\rho^3\Lambda\Omega^{-3}\left(1-\tilde{A}_3\right)\left|1-\tilde{A}_3\right|A_1^{-1}\tilde{A}_3^4\omega_0^6.
\end{array}
\eqno(7)
$$

Снова перейдём к обозначению безразмерных выражений в (6), (7) без волн.

Используем порождающее решение системы (6) при $ \varepsilon=0 $:
$$
r=\text{const},\,\, p=a\cos\psi,\,\,q=a\sin\psi,\,\,a>0,\,\,\text{const}\neq 0. \eqno(8)
$$

Здесь $ \psi=\left(A_3-1\right)r\tau+\psi_0 $ - фаза колебаний экваториальной составляющей вектора угловой скорости.

Подставим (8) в третье уравнение системы (6). Усредним полученную систему уравнений для $ a $ и $ r $. После усреднения система примет вид ($ '=d/d\theta $)
$$
\begin{array}{c}
a'=-\dfrac{a}{2}\left[ G^{-1}(b_1+b_2)-2Sr^6a+2\lambda\right],\\
r'=-r\left[ b_3G^{-1}+SA_3^{-2}r^4a^3+\lambda\right].
\end{array}\eqno(9)
$$

Отметим, что при $ b_1=b_2=b_3=b $ уравнения (9) интегрируются полностью [4].

Исследуем частный случай
$$
\dfrac{1}{2}(b_1+b_2)=b_3=b.\eqno(10)
$$

Домножим первое уравнение (6) на $ p $, второе – на $ q $, а третье – на $ A_3^2r $ и сложим (перейдя к обозначению безразмерных выражений в (6) без волн). Проведя усреднение, получим
$$
G'=-b-\lambda G.\eqno(11)
$$

Начальное и конечные условия записываются следующим образом
$$
G(0)=G^0,\,\,G\left(T,\theta_0,G^0\right)=0,\,\,T=T\left(\theta_0,G^0\right). \eqno(12)
$$

Решение уравнения (11) с учётом условий (12) имеет вид
$$
G(\theta)=-\dfrac{b}{\lambda}+\left(G^0+\dfrac{b}{\lambda}\right)\exp(-\lambda\theta),\,\,\Theta=\dfrac{1}{\lambda}\ln\left(G^0\dfrac{\lambda}{b}+1\right). \eqno(13)
$$

Заметим, что величина $ \Theta\to\infty $ при $ G^0/b\to\infty $ для различных $ \lambda $; в свою очередь, $ \Theta\to0 $ при $ G^0\lambda/b\to 0 $ ($ \lambda $ - любое) или при $ \lambda\to\infty $.

Для системы (9) при условии (10) воспользуемся заменой переменных: $ r=\eta G,\,\,a=\alpha G $. В этом случае уравнения системы (9) принимают вид
$$
\alpha'=S\alpha^2\eta^6G^7,\,\,\eta'=-\eta^5SA_3^{-2}\alpha^3G^7. \eqno(14)
$$

Разделим первое уравнение на второе и получим
$$
\dfrac{d\alpha}{d\eta}=-A_3^2\eta/\alpha.
$$

Находим первый интеграл $ C_1 $:
$$
\eta^2=2C_1-A_3^{-2}\alpha^2,\,\,C_1=\dfrac{1}{2}A_3^{-2}.\eqno(15)
$$

Подставим $ \eta^2 $ из (15) в первое уравнение системы (14)
$$
\dfrac{d\alpha}{d\theta}=SA_3^{-6}\left(1-\alpha^2\right)^3G^7\alpha^2,\,\,G_0=1.\eqno(16)
$$

Подставим выражение для $ G $ (13) в уравнение (16) для $ \alpha $ и проинтегрируем это уравнение [5]
$$
\begin{array}{c}
A_3^6\left[-\dfrac{1}{\alpha}+\dfrac{\alpha}{4(1-\alpha^2)^2}+\dfrac{7\alpha}{8(1-\alpha^2)}+\dfrac{15}{16}\ln\left|\dfrac{1+\alpha}{1-\alpha}\right|\right]=\\
=S\Biggl[-\dfrac{b^7}{\lambda^7}\theta-\dfrac{7b^6}{\lambda^7}b_*\exp(-\lambda\theta)+\dfrac{21b^5}{2\lambda^6}b_*^2\exp(-2\lambda\theta)-\\
-\dfrac{35b^4}{3\lambda^5}b_*^3\exp(-3\lambda\theta)+
\dfrac{35b^3}{4\lambda^4}b_*^4\exp(-4\lambda\theta)-\dfrac{21b^2}{5\lambda^3}b_*^5\exp(-5\lambda\theta)+\\
+\dfrac{7b}{6\lambda^2}b_*^6\exp(-6\lambda\theta)-\dfrac{1}{7\lambda}b_*^7\exp(-7\lambda\theta)\Biggr]+C_2,
\end{array}
\eqno(17)
$$
$$
b_*=G_0+\dfrac{b}{\lambda}=1+\dfrac{b}{\lambda}.
$$

Вторая постоянная $ C_2 $ интегрирования определяется из начального условия: при $ \theta=0$ $\alpha=\alpha_0 $. Величины $ \eta $ и $ \alpha $ связаны соотношением (15). Таким образом, получены выражения для параметров оптимального движения $ G(\theta),\,\,\Theta $ (13) и $ a(\theta),\,\,r(\theta) $ (14)-(17). Их качественные свойства достаточно просты.



\smallskip \centerline{\bf Литература}\nopagebreak

1. {\it Акуленко Л.Д., Лещенко Д.Д.} Некоторые задачи движения твёрдого тела с подвижной массой // Изв. АН СССР. МТТ. - 1978. - №5. - С.29-34.

2. {\it Черноусько Ф.Л., Акуленко Л.Д., Лещенко Д.Д.} Эволюция движений твёрдого тела относительно центра масс. М.-Ижевск: Ин-т компьют. исслед., 2015. - 308с.

3. {\it Акуленко Л.Д.} Асимптотические методы оптимального управления. М.: Наука, 1987. - 368с.

4. {\it Акуленко Л.Д., Лещенко Д.Д., Рачинская А. Л., Зинкевич Я.С.} Возмущённые и управляемые вращения твёрдого тела. Одесса: ОНУ им. И.И. Мечникова, 2013. - 288с.

5. {\it Двайт Г.Б.} Таблицы интегралов и другие математические формулы. М.: Наука, 1973. - 228с.

