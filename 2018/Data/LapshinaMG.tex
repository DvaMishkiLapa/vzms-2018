\begin{center}{
	\bf
	ОБРАЩЕНИЕ НЕКОТОРЫХ ВЕСОВЫХ ИНТЕГРАЛЬНЫХ ОПЕРАЦИЙ
} \\
{\it М.\,Г.~Лапшина } \\
(Липецк; {\it marina.lapsh@ya.ru})
\end{center}
\addcontentsline{toc}{section}{Лапшина  М.\,Г.}


Пусть $N$ и $n$~--- фиксированные натуральные числа, $1\le n\le N $,
$x=(x',x'')\in \mathbb{R}_N^+=\mathbb{R}_n^+\times\mathbb{R}_{N-n}$,
$$x'{=}(x_1,\ldots,x_n)\in R_n^+{=}\{x_i{>}0\,,\, i{=}\overline{1,n}\},\,\,
x''{=}(x_{n+1},\ldots,x_N)\in R_{N{-}n},$$
$\gamma{=}(\gamma_1,\ldots,\gamma_n)$~--- мультииндекс, состоящий из
фиксированных положительных чисел, $(x')^\gamma=\prod_{i=1}^n x_i^{\gamma_i}$,
$\langle x\,,\,\xi\rangle{=}\sum_{i=1}^{N}x_i\xi_i$.

Действие обобщенного сдвига определяется  равенством\\
$
(T^y f)(x){=}C(\gamma)\int\limits_{0}^{\pi}
\ldots \int\limits_{0}^{\pi}
f\left(\sqrt{x'^2{+}y'^2{-}2x'y'\cos\alpha},\,
 x''{-}y''\right)sin^{\gamma-1}\alpha' d\alpha',
$ \\
где $\sqrt{x^{'2}+y^{'2}-2x'y'\cos\alpha}$ --- $n$-мерный вектор  с координатами\\
$\sqrt{x_j^{2}+y_j^{2}-2x_jy_j\cos\alpha_j}\,,\,\,\, j=\overline{1,n}$\, и
$
C(\gamma)=\prod\limits_{i=1}^n\frac{\Gamma\left(\frac{\gamma_i+1}{2}\right)}
{\Gamma\left(\frac{1}{2}\right)\Gamma\left(\frac{\gamma_i}{2}\right)}\,.
$

Понятие функции "весовая плоская волна"\, введено в [1] в виде:
${\cal P}^\gamma_{x'} f(\langle x\,,\,\xi\rangle)$,
где ${\cal P}^\gamma$ -- многомерный оператор Пуассона,\\
$
{\cal P}^\gamma_{x'} f(x',x''){=}C(\gamma)
\int\limits_0^\pi\ldots \int\limits_0^\pi f(x_1\cos\alpha_1,\,\ldots\,,x_n\cos\alpha_n,\,x'')
\sin^{\gamma-1}\alpha'\,d\alpha'.
$

\smallskip
\textbf{Теорема~1.}
{\it Пусть функция $f$ удовлетворяет условию Гельдера, носитель функции принадлежит ограниченной области $\Omega_{N}^+$,
тогда если число $N+|\gamma|>2$ -- нечетное и $k=1,3,5,\,\ldots\,$, то

$
\triangle_{B_{\eta}}^{\frac{N+|\gamma|+k}{2}}
\int\limits_{\mathbb{R}_N^+}(T^{-\eta}f)(\xi)
\int\limits_{S^+_1(N)}{\cal P}_{x'}^\gamma |\langle x\,,\,\xi\rangle|^k\,\,
(x')^\gamma\,\,dS(x)(\xi')^{\gamma} d\xi=
$

$
=2^{N+|\gamma|-2n+1}\,\,
\pi^{N-n-1}\,
\prod\limits_{i=1}^{n}\Gamma^2\left(\frac{\gamma_i+1}{2}\right)
\,\,i^{N+|\gamma|-1}\,k!
\, f(\eta)\,;
$\\
а если
число $N+|\gamma|>2$ -- четное и $k=0,2,4,\,\ldots$, то\\
$
\Delta_{B_{\eta}}^{\frac{N+|\gamma|+k}{2}}\int\limits_{\mathbb{R}_N^+}(T^{-\eta}f)(\xi)
\int\limits_{S_1^+(N)}{\cal P}_{x'}^\gamma |\langle x,\xi \rangle|^k\,\ln|\langle x,\xi \rangle|
(x')^{\gamma}\,dS(x)\,(\xi')^{\gamma}\,d\xi=
$

$
=-\pi^{N-n} 2^{N+|\gamma|-2n} i^{N+|\gamma|}
\prod\limits_{i=1}^{n}\Gamma^2\left(\frac{\gamma_i+1}{2}\right)\,k\,!\,f(\eta)\,.
$}

Особенность этих формул состоит в том, что они верны для дробных $\gamma_i$, при условии, что число $|\gamma|$ --- натуральное.


\smallskip \centerline{\bf Литература}\nopagebreak

1. \textit{Киприянов~И.\,А., Кононенко~В.\,И.} Фундаментальные решения $B$-эллиптических уравнений // Дифференциальные уравнения. 1967. Т.~3, №~1.  C.~114--129.}
