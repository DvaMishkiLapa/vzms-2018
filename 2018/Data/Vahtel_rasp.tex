\begin{center}{ \bf  РАСПРЕДЕЛЕНИЕ ИНТЕРВАЛОВ ВРЕМЕНИ МЕЖДУ ИДЕНТИЧНЫМИ ЭМПИРИЧЕСКИМИ РАСПРЕДЕЛЕНИЯМИ}\\
{\copyright} 2017 {\it  E. В. Акиндинова, А. Г. Бабенко, В. М. Вахтель,
В. А. Работкин,  И. В. Муратов, Ю. Н. Горшкова} \\
( Воронеж, ВГУ; {\it vakhtel@phys.vsu.ru} )
\end{center}
\addcontentsline{toc}{section}{Акиндинова E.В., Бабенко А.Г.,
Вахтель В.М., Работкин В.А., Муратов И.В., Горшкова Ю.Н.}

Имеется Пуассоновский стационарный поток событий.
В данной работе реализации потока получены методом статистического моделирования и генерации импульсов квазиточечных событий
источником радиоактивного излучения.
Длительность одной реализации \textit{T} разделена на $N \gg 1$ непересекающихся подынтервалов длительностью $\Delta t$\textit{,} то есть $N=T/\Delta t$. Каждому подынтервалу соответствует случайная величина $K(\Delta t)$ -- число событий.  Соответствующее среднее $\bar{K}(\Delta t)=\nu \cdot \Delta t$, где $\nu $ -- интенсивность потока. Последовательность из \textit{N} случайных величин $K(\Delta t)$ разделена на $M=(N/n)>>1$  подпоследовательностей, каждая объёмом $n\geqslant 1$. Подпоследовательности из \textit{n}  случайных величин $K(\Delta t)$ являются случайными выборками условного распределения Пуассона.


Каждая из \textit{M} выборок имеет свой вариационный ряд $K_{1}  \leqslant _{} K_{2}  \leqslant _{}  ...  \leqslant _{} K{}_{n} $,
которому соответствует эмпирическое распределение {\it ЭР} $ ( n_{0} ,    n_{1} ,  ...  ,   n_{l}  )_{j} $~---
случайный вектор $n=\sum _{i=1}^{l_{j} }n_{ij}  $, где $n_{i} $ и $l_{j} $ --  случайные величины,
$1<j<M$, $j$ -- индекс уникального типа эмпирического распределения.
{\it ЭР}  типа  $j$, очевидно, может $M_{j} \geqslant 1$ кратно повторяться $M=\sum _{j=1}^{m}M_{j}  $, где $m$~--- случайная величина.


Вероятность появления каждого случайного {\it ЭР} $ (K)$ типа $j$ соответствует полиномиальному распределению [2]:
\[P_{j} (( n_{0} ,  n_{1} ,  ...  ,  n_{l}  )_{j} )=\frac{n!}{n_{0j} !\cdot \cdot \cdot n_{lj} !} P_{0}^{n_{0j} } \cdot \cdot \cdot P_{lj}^{n_{lj} } ,\]
где $P_{ij} (\bar{K})=\frac{\bar{K}^{K_{i} } }{K_{i} } e^{-\bar{K}} $, $0\leqslant K_{i} $, $M_{j} =M\cdot P_{j} (\cdot)$. Соответствующая эмпирическая случайная частота появления {\it ЭР}    $\tilde{P}_{j} (( n_{0} ,  n_{1} ,  ...  ,  n_{l}  )_{j} )=\tilde{M}_{j} /M$, где $\tilde{M}_{j} $ --  случайная величина.


Распределение {\it ЭР} $(\cdot )_{j} $ с  $M_{j} >>1$  имеет биномиальное распределение $P(M_{j} \left|P_{j} (\cdot),M)\right. $ с параметрами $P_{j} (\cdot)$ и $M$. При $P_{j} (\cdot )<<1$  и $M>>1$  как следствие асимптотически слабой сходимости возможна аппроксимация распределением Пуассона с параметром $M_{j} =M\cdot P_{j} (\cdot)$.

Поэтому случайный интервал $r_{i} $ между идентичными по типу $j$  {\it ЭР} $ ( n_{0} ,  n_{1} ,  ...  ,  n_{l} )$ подчиняется геометрическому, а в хорошем приближении экспоненциальному  распределению  с параметром $1/P_{j} (\cdot)=\bar{r}_{j} $, где $\bar{r}_{j} $ -- средний интервал.


Этот относительно простой результат значим в практическом применении при исследованиях случайных потоков, в частности, излучений особенно при малых значениях $n<100$. В частности экспоненциальностью распределения $r$ можно объяснить более частое последовательное  появление  каждого из типов {\it ЭР} $ _{j} $ с малыми $r_{j} <\bar{r}_{j} $, названное в работе [3] эффектом «ближней зоны».


Экспоненциальность распределений интервалов $r_{j} $ необходимо учитывать при измерениях и обработке последовательностей эмпирических распределений, особенно на основе вариационных рядов, малых объёмов (\textit{n}$\sim$10) редких событий для пуассоновских потоков событий.


%%%%  ОФОРМЛЕНИЕ СПИСКА ЛИТЕРАТУРЫ %%%
\smallskip \centerline{\bf Литература}\nopagebreak

1. {\it Розанов Ю. А.} Теория вероятностей, случайные процессы и математическая статистика // М.: Наука.  1985, С. 320.

2. {\it Бабенко А. Г., Вахтель В. М., Работкин В. А., Евсеев Н. А., Харитонова Д. Д.} The Multichannel Time Distributions Spectrometer in Real Time // VII International Symposium on Exotic Nuclei (EXON 2014) : Book of Abstracts, Kaliningrad, Russia, Sept. 8-13, 2014. P. 104.

3. {\it Шноль С. Э., Коломбет В. А., Зинденко  Т. А., Пожарский Э. В., Зверева И. М., Конрадов А. А.} О космофизической обусловленности «макроскопических флуктуаций» // Биофизика, Т. 43, 1998, С. 909-915.
