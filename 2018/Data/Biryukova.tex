\vzmstitle{О КРАЕВОЙ ЗАДАЧЕ ДЛЯ РЕЗОЛЬВЕНТЫ   ФУНКЦИОНАЛЬНО-ДИФФЕPЕНЦИАЛЬНОГО ОПЕРАТОРА С ИНВОЛЮТИВНЫМ ОТКЛОНЕНИЕМ }

\vzmsauthor{Бирюкова}{Е.\,И.}

\vzmsinfo{Воронеж; {\it elenabiryukova2010@yandex.ru}}

\vzmscaption


Рассматривается функционально-дифференциальный
\linebreak
оператор с инволюцией:
\vspace{-0.1em}
$$Ly=y'(1-x)+q(x)y(x), \quad y(0)=y(1), \eqno(1)$$
\vspace{-0.1em}
область определения которого состоит из
непрерывно-диф\-фе\-рен\-ци\-руе\-мых функций, удовлетворяющих
периодическому  краевому условию в (1).

Такие операторы и различные задачи, в которых они возникают, активно
исследуются (см., например, [1-6]).

При исследовании  вопросов о разложении произвольной   функции
$f(x)$ в ряд Фурье по системе собственных и присоединённых функций
оператора $L$ может быть использован    метод контурного
интегрирования резольвенты, основанный на представлении частичной
суммы ряда Фурье через   интеграл от резольвенты по расширяющимся
контурам комплексной плоскости (см. [7]). Для этого  требуется
изучение поведения резольвенты $R_\lambda =(L-\lambda E)^{-1}$
оператора $L$ (здесь $\lambda $ --- спектральный параметр, $E$ ---
единичный оператор) при больших $|\lambda|$, и соответственно
решения $y(x,\lambda)$ краевой задачи
\vspace{-0.1em}
$$
  (Ly)(x)=\lambda y(x) +f(x),  \qquad  y(0)=y(1).
$$
 Согласно работе [5], исследование спектральных свойств оператора $L$
проводится с помощью невозмущённого оператора
    $L_0=y'(1-x), \quad y(0)=y(1),$
 резольвента которого  изучается в данной работе.

\textbf{Лемма.} {\it Для резольвенты $R^{0}_\lambda =(L_0-\lambda
E)^{-1}$ оператора $L_0$  имеет место  формула
 $$R^{0}_\lambda f=[\Gamma R_{0\mu} \Phi(x)]_1,$$
где $\Gamma =\begin{pmatrix} \ 1 & \ -i \\ \ -i & \ 1
\end{pmatrix}$, $\Phi(x) =diag(-i,i)\Gamma ^{-1}(f(x),f(1-x))^{T}$,
(\:$T$~--- знак транспонирования), $\mu =-i\lambda $, ~$[~]_1$
означает первую компоненту вектора, а $R_0\mu $ обозначает решение
следующей задачи:
$$u'(x)-\mu Du(x)=\Phi(x),  \eqno(2)$$
$$U_0(u)=M_0\Gamma u(0)+M_1\Gamma u(1)=0, \eqno(3)$$
$D=diag(1,-1)$, $M_0=\begin{pmatrix} \ 1 & \ -1 \\ \ 0 & \ 0
\end{pmatrix}$, $M_1=\begin{pmatrix} \ 0 & \ 0 \\ \ 1 & \ -1
\end{pmatrix}$.}

При доказательстве этой леммы, аналогично работе [5],
устанавливается, что если
 резольвента $R^{0}_\lambda $ оператора $L_0$ существует и $y=R^{0}_\lambda f$, то вектор-функция
$$
	z(x)=(z_1(x),z_2(x))^{T}\!\!, \mbox{~~где~~} z_1=y(x), z_2=y(1-x),
$$
является решением краевой задачи
$$Bz'(x)-\lambda z(x)=F(x), \eqno(4)$$
$$M_0z(0)+M_1z(1/2)=0, \eqno(5)$$
где $B=\begin{pmatrix} \ 0 & \ -1 \\ \ 1 & \ 0 \end{pmatrix}$,
$F(x)=(f(x),f(1-x))^{T}$. Диагонализируя систему (4) с помощью
замены $z(x)=\Gamma u(x)$, получим задачу (2), (3).

Также как в [6], получим следующий результат.

\textbf{Теорема.} {\it Если $\mu $ таково, что обратима матрица
\linebreak $\Delta _0(\mu ){=}U_0(V(x,\mu ))$, где $V(x,\mu
)=diag(e^{\mu x}, e^{-\mu x})$,   то краевая задача (2)-(3)
однозначно разрешима при любой $\Phi (x)$ с компонентами из $L[0,1]$
и её решение $u(x)=u(x,\mu )$
\linebreak
имеет вид
$$
u(x,\mu )=R_{0\mu} \Phi (x)=-V(x,\mu )\Delta ^{-1}_0(\mu )U_0(g_\mu \Phi )+g_\mu \Phi (x).
$$
Здесь
$$
g_\mu \Phi (x)=\int\nolimits_0^1 g(x,t,\mu )dt, ~~U_0(g_\mu
\Phi )=\int\nolimits_0^1 U_{0x}(g(x,t,\mu ))\Phi(t) dt,
$$ ($U_{0x}$
означает, что $U_0$ применяется к $g$ по переменной $x$), $g(x,t,\mu
)=diag(g_1(x,t,\mu ),g_2(x,t,\mu ))$,\\ $g_k(x,t,\mu )=-\varepsilon
(t,x)exp{(-1)^{k-1}\mu (x-t)}$, при $(-1)^{k-1}Re\mu \geqslant 0$,
$g_k(x,t,\mu )=\varepsilon (t,x)exp{(-1)^{k-1}\mu (x-t)}$, при
$(-1)^{k-1}Re\mu \leqslant 0$, $\varepsilon (t,x)=1$, если $x \geqslant t$,
$\varepsilon (t,x)=0$, если $x<t$.}



    \litlist



1. Андреев А.А.Об аналогах классических краевых задач для одного
дифференциального уравнения второго порядка с отклоняющимся
аргументом // Дифферец. ур-ния. 2004. Т. 40. № 5. С. 1126-1128.

2. Платонов С.С. Разложение по собственным функциям для некоторых
функционально-дифференциальных операторов / С.С. Платонов
// Тр. Петрозавод. гос. ун-та. Сер. мат. --- 2004. --- Вып. 11. --- С.
15--35.

3. Линьков А.В. Обоснование метода Фурье для краевых задач с
инволютивным отклонением // Вестник СамГУ. --- 1999. --- \No 2(12).
--- С. 60--65.

4. Бурлуцкая М.\,Ш., Курдюмов В.\,П.,  Луконина А.\,С.,  Хромов А.\,П.
фу\-н\-к\-ци\-о\-нально-дифференциальный оператор с инволюцией  // Докл. РАН.
--- 2007. --- Т. 414, № 4. - С. 443--446.

5.  Бурлуцкая М.Ш., Хромов А.П.  Об одной теореме равносходимости на
всем отрезке для функционально-диф\-фе\-рен\-циаль\-ных операторов
// Изв. Сарат. ун-та.   --- 2009. --- Т.~9, вып. 4. --- С.3--10.

6. Бурлуцкая М.\,Ш. Теорема Жордана--Дирихле для
фу\-н\-к\-ци\-о\-нально-дифференциального оператора с инволюцией // Изв.
Сарат. ун-та.  --- 2013. --- Т.~13, вып. 3. --- С. 9--14.

7. Расулов М.\,Л. Метод контурного интеграла и его применение к
исследованию задач для дифференциальных ура\-в\-нений.   М. : Наука,
1964. --- 464 с.
