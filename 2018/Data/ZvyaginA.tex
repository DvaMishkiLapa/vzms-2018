\begin{center}{ \bf  ЗАДАЧА ТЕРМОВЯЗКОУПРУГОСТИ ДЛЯ МОДЕЛИ КЕЛЬВИНА-ФОЙГТА С ОБЪЕКТИВНОЙ ПРОИЗВОДНОЙ} \footnote{Работа выполнена при финансовой поддержке РНФ (проект 16-11-
10125, выполняемый в Воронежском государственном университете).}\\
{\it А.В. Звягин } \\
(Воронеж; {\it zvyagin.a@mail.ru} )
\end{center}
\addcontentsline{toc}{section}{Звягин А.В.}

Пусть $\Omega \subset \mathbb{R}^n$, $n=2,3$, -- ограниченная область с границей
$\partial\Omega$ класса $C^2$. В $Q_T=[0,T]\times \Omega$ рассматривается начально-краевая задача:
\small{
\begin{gather*}
\label{Eq1}\frac{\partial v}{\partial t} +  \sum_{i=1}^nv_i\frac{\partial v}{\partial x_i}- 2 \mbox{Div }(\nu(\theta)\mathcal{E})-\varkappa\frac{\partial\Delta v}{\partial t} - 2 \varkappa \mbox{Div} (\sum_{i=1}^n v_i \frac{\partial\mathcal{E}}{\partial {x_i}})-\\
- 2\varkappa\mbox{Div} \left(\mathcal{E}(v) W_{\rho}(v)-W_{\rho}(v)\mathcal{E}(v)\right)+\nabla{p}=f;\\
\label{Eq2} \mbox{div }{v}=0 \mbox{ в } Q_T;\qquad v|_{t=0}\,=v_0 \mbox{ в } \Omega; \qquad v|_{[0,T]\times\partial \Omega  }\,=0;\\
\label{Eq3}
\frac{\partial \theta}{\partial t} + \sum_{i=1}^n v_i \frac{\partial \theta}{\partial x_i}  - \chi\Delta\theta=
\\=
2\big{(}{\nu}(\theta)\mathcal {E}+\varkappa \frac{\partial \mathcal{E}}{\partial t} + \varkappa \sum_{i=1}^n v_i \frac{\partial\mathcal{E}}{\partial {x_i}}\big{)}: \mathcal{E}(v)+g;\\
\label{Eq4}
  \theta|_{t=0} =\theta_0 \mbox{ в } \Omega; \qquad \theta|_{[0,T]\times\partial \Omega  }\,=0.
\end{gather*}}
Здесь $v$, $\theta$ и $p$ -- вектор-функция скорости, функции температуры и давления среды соответственно, $f$ -- плотность внешних сил, $g$ -- источник внешнего тепла, $\varkappa >0$ -- время ретардации, $\chi >0$ -- коэффициент теплопроводности, $\nu(
\theta) >0$ -- вязкость жидкости, $\mathcal {E}$ -- тензор скоростей деформаций. Обзор исследуемой математической модели можно найти в [1], а результаты по изучаемым в данном докладе моделям с вязкостью, зависящей от температуры, в [2-4].

Введём пространства:
$
E_{1}=\{v: v\in L_\infty(0,T,V^1), v' \in L_2(0,T;$ $V^{-1})\}
$  и
$E_{2}=\{v: v\in L_p(0,T; W^1_p(\Omega)), v' \in L_1(0,T;$ $W^{-1}_{p}(\Omega)),\ 1<p<+\infty\}$.

\textbf{Определение 1.} \textit{Слабым решением называется пара $(v, \theta) \in E_1 \times E_2,$ удовлетворяющая начальным условиям $v|_{t=0} = v_0$ и $\theta|_{t=0}=\theta_0$ и соотношениям
\begin{gather*}\begin{array}{lcc}
\int_{\Omega} {\partial v}/{\partial t}\, \varphi \,dx -\int_{\Omega} \sum_{i,j=1}^n v_i v_j {\partial
{\varphi_j}}/{\partial
 x_i}\,dx+ \int_{\Omega} (2\nu(\theta)\mathcal{E}(v)+\\+\varkappa\mathcal{E}( {\partial v}/{\partial t})):
 \mathcal{E}(\varphi)\,dx -\varkappa \int_{\Omega} \sum_{i,j,k=1}^n v_k (\partial v_i/\partial x_j-\\-\partial v_j/\partial x_i )\partial^2 \varphi_j/\partial x_i\partial x_k\,dx+2\varkappa\int_{\Omega}(\mathcal{E}(v)W_{\rho}(v)-\\-W_{\rho}(v)\mathcal{E}(v)) : \nabla\varphi\,dx= \langle f, \varphi\rangle, \, \forall\,\varphi \in V \mbox{ и п.в. } t\in [0,T],
\end{array}\\ \begin{array}{lcc}
\label{f4}
 \int_\Omega {\partial\theta}/{\partial t}\phi\,dx-\int_\Omega\sum_{i,j=1}^n
 v_i\theta_j{\partial \phi_j}/{\partial x_i}\,dx+\chi\int_\Omega
\mathcal{E}(\theta):\mathcal{E}( \phi)\,dx=\\= 2\int_\Omega\big{(}({\nu}(\theta)  \mathcal {E}(v)+\varkappa{\partial  \mathcal {E}(v)}/{\partial t}+\varkappa \sum_{i=1}^n v_i {\partial\mathcal{E}(v)}/{\partial {x_i}}): \mathcal {E}(v)\big{)}:\\:\phi \,dx+\langle g,\phi\rangle \quad \mbox{при всех }  \phi \in C^{\infty}_0(\Omega) \mbox{ и п.в. } t\in [0,T].\end{array}
\end{gather*}}

\textbf{Теорема 1.} \textit{Пусть  функция $\nu(\theta)\in C^2(-\infty,+\infty)$ является монотонно возрастающей и $0\leqslant\nu(\theta)\leqslant M$, $f \in L_2(0,T;V^{-1})$, $g \in L_1(0,T; H^{-2(1-1/p)}_p(\Omega))$, $v_0 \in V^1$, $\theta_0 \in W^{1-2/p}_p(\Omega)$. Тогда при $1<p<4/3$ для $n=2$ и для $1<p<5/4$ при $n=3$ существует слабое решение.}

%%%%  ОФОРМЛЕНИЕ СПИСКА ЛИТЕРАТУРЫ %%%
\smallskip \centerline{\bf Литература}\nopagebreak

1. {\it Звягин А.В., Орлов В.П.} Разрешимость задачи термовязкоупругости для одной модели Осколкова // Известия Вузов. Математика. 2014. $\No 9$, стр. 69-74.

2. {\it Zvyagin A.V., Orlov V.P.}
Solvability of the ther\-mo\-vis\-co\-elas\-ti\-ci\-ty problem for linearly elastically retarded Voiht liquids
//
Mathe\-ma\-tical Notes. 2015. V.97. $\No 5$. pp. 38-52.

3. {\it Звягин А.В.} Разрешимость задачи термовязкоупругости для альфа-модели Лере // Известия ВУЗов. Математика. 2016. $\No 10$. стр. 70-75.

4. {\it Звягин А.В.} Слабая разрешимость термовязкоупругой модели Кельвина-Фойгта // Известия ВУЗов. Математика. 2018. $\No 3$. стр. 91-95.
