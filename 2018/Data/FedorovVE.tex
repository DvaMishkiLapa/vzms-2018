\begin{center}{ \bf  ОБРАТНАЯ ЗАДАЧА ДЛЯ  УРАВНЕНИЯ\\ В БАНАХОВОМ ПРОСТРАНСТВЕ,\\ НЕ РАЗРЕШИМОГО ОТНОСИТЕЛЬНО ПРОИЗВОДНОЙ РИМАНА~--- ЛИУВИЛЛЯ}\\
{\it В.Е. Федоров, Р.Р. Нажимов } \\
(Челябинск; {\it kar@csu.ru, goldenboy454@mail.ru} )
\end{center}
\addcontentsline{toc}{section}{Федоров В.Е., Нажимов Р.Р.}

В работе исследованы вопросы  разрешимости обратных задач для линейного дифференциального уравнения в банаховом пространстве, не разрешимого относительно производной Римана~--- Лиувилля, с постоянным неизвестным коэффициентом и с двумя различными наборами начальных условий~--- условиями типа Коши [1] и условиями типа Шоуолтера~--- Сидорова. Найдены достаточные условия разрешимости в первом случае и критерий корректности~--- во втором. Используется условие относительной ограниченности пары операторов в уравнении.

Аналогичные задачи для вырожденного эволюционного уравнения первого порядка при различных условиях на операторы в уравнении рассматривались в [2--5]. Разрешимость различных обратных задач для таких же классов уравнений с неизвестным коэффициентом, зависящим от времени, исследовалась в работах [6--8] для уравнения первого порядка, в [9]~--- для вырожденного уравнения с дробной производной Герасимова~--- Капуто.

Пусть ${\mathcal  X}$, ${\mathcal  Y}$~--- банаховы пространства, ${\mathcal L}(\mathcal X; \mathcal Y)$~--- банахово пространство линейно непрерывных операторов, действующих из $\mathcal X$ в $\mathcal Y$,
${\mathcal C}l({ \mathcal X; \mathcal Y})$~--- множество линейных замкнутых плотно определенных в пространстве ${\mathcal X}$ операторов, действующих в пространство ${\mathcal Y}$, кроме того,
 ${\mathcal L}(\mathcal X; \mathcal X):={\mathcal L}(\mathcal X)$,  ${\mathcal C}l(\mathcal X;\mathcal X):={\mathcal C}l(\mathcal X).$

Пусть
$L\in{\mathcal  L}({\mathcal  X};{\mathcal  Y})$, $\ker L\ne\{0\}$, $M\in{\mathcal  C}l({\mathcal  X;\mathcal Y})$ имеет область определения $D_M$, на которой задана норма графика оператора $M$. Мы также введём обозначения $\rho^L(M)=\{\lambda\in\mathbb C:
(\lambda L-M)^{-1}\in{\mathcal  L(\mathcal Y;\mathcal X)}\},$ $\sigma^L(M)=\mathbb C\setminus\rho^L(M)$,  $R_\lambda^L(M)=(\lambda L-M)^{-1}L$, $L_\lambda^L(M)=L(\lambda L-M)^{-1}.$

Оператор $M$ называется {\it $(L,\sigma)$-ограниченным},
если условие $\sigma^L(M)\subset\{\lambda\in\mathbb  C:|\lambda|\le a\}$ выполняется для некоторого $a>0$. При условии $(L,\sigma)$-ограниченности оператора $M$ существуют  проекторы
$$
P=\frac1{2\pi i}\int\limits_\gamma R_\lambda^L(M)d\lambda\in\mathcal L(\mathcal X),\, Q=\frac1{2\pi i}\int\limits_\gamma L_\lambda^L(M)d\lambda\in\mathcal L(\mathcal Y),
$$
где $\gamma=\{\lambda\in\mathbb C:|\lambda|=a+1\}$.
Положим ${\mathcal  X}^0 = \ker P$, ${\mathcal  Y}^0 = \ker Q$, ${\mathcal
X}^1 = \mbox{\rm im}P$, ${\mathcal  Y}^1 = \mbox{\rm im}Q$. Пусть $M_k$ $(L_k)$~--- сужение оператора $M$ $(L)$ на $D_{M_k}={\mathcal
X}^k \cap D_M$ ${(\mathcal X}^k),$ $k=0,1.$




\textbf{Теорема 1} [10]. {\it Пусть оператор $M$
$(L,\sigma)$-ограничен. Тогда

%
{\rm (i)} ${\mathcal  X}={\mathcal  X}^0 \oplus{\mathcal  X}^1,$
${\mathcal  Y}={\mathcal  Y}^0\oplus {\mathcal  Y}^1;$

%
{\rm (ii)} $L_k\in{\mathcal  L}({\mathcal  X}^k;{\mathcal  Y}^k),$
 $k=0,1,$ $ M_0\in{\mathcal  C}l({\mathcal  X}^0;{\mathcal  Y}^0)$, $ M_1\in{\mathcal  L}({\mathcal  X}^1;{\mathcal  Y}^1);$

%
{\rm (iii)} существуют операторы $M^{-1}_0\in{\mathcal  L}({\mathcal  Y}^0;{\mathcal  X}^0)$,
$L^{-1}_1\in{\mathcal  L}({\mathcal  Y}^1;{\mathcal  X}^1).$
}



При $p\in\mathbb N_0$ оператор $M$ называется {\it $(L,p)$-оганиченным}, если он $(L,\sigma)$-ограничен, $G^p\ne0$, $G^{p+1}=0$, где $G=M^{-1}_0L_0\in{\mathcal  L}({\mathcal  X}^0)$.

Обозначим, $g_{\delta}(t)=\Gamma(\delta)^{-1}t^{\delta-1}$ при $\delta>0$, $t>0$, $J^{\delta}_t z(t)=(g_{\delta}*z)(t)=\int\limits_{0}^{t}g_{\delta}(t-s)z(s)ds.$
	Пусть $\alpha > 0$,  $m$~--- наименьшее целое число,  не превосходимое числом $\alpha$, $D^m _t$~--- обычная производная порядка $m$, $J^0_t$~--- тождественный оператор, $D^{\alpha}_t$~--- дробная производная Римана~--- Лиувилля, т.\,е. $$D^{\alpha}_t z(t)=D^m_tJ^{m-\alpha}_t z(t).$$

Рассмотрим  обратную задачу
$$ \label{degequ}
D^{\alpha}_t Lx(t)=Mx(t)+\varphi(t)u,\quad  t\in [0,T],\eqno(1)
$$
$$ \label{degini}
\lim_{t \rightarrow 0+}D^{\alpha -m+k}_{t}x(t)=x_k , \, \,  k = 0,1,  \ldots ,  m-1 ,\eqno(2)
$$
$$ \label{degove}
\int\limits_0^T x(t)d\mu(t)=x_T,\eqno(3)
$$
где $\varphi\in C([0,T];\mathbb R)$, неизвестный вектор $u\in\mathcal Y$.



Обозначим
$$
\psi(A) \equiv Px_T -\int\limits_0^T\sum\limits_{k=0}^{m-1} t^{\alpha -m+k} E_{\alpha, \alpha-m+k+1}(At^{\alpha})Px_k d\mu(t),
$$
$$ \chi(\lambda)\equiv \int\limits_0^Td\mu(t)\int\limits_0^t (t-s)^{\alpha-1} E_{\alpha , \alpha} (\lambda(t-s)^{\alpha})\varphi(s)ds, \quad \lambda\in\mathbb C,$$
 $$f_n(s)=\int\limits_0^s \left(D^{\alpha}_t \right)^n \varphi (t)d\mu(t),\quad n=0,1,\ldots, p,$$
$$F(s)v=M_0\sum\limits_{k=0}^{p} (-1)^{k+1} \left(\sum\limits_{n=1}^{p}\frac{f_n(s)G^n}{f_0(s)}\right)^k\frac{v}{f_0(s)}.$$
Здесь $E_{\alpha,\,\beta}(z)\!=\!\!\sum\limits_{k=0}^\infty\!\frac{z^k}{\Gamma(\alpha k+\beta)}$~--- функция Миттаг-Леффле\-ра.




\textbf{Теорема 2.} {\it Пусть оператор $M$ $(L,p)$-ограничен,  для $n=0,1,\ldots, p$ $\left(D^{\alpha}_t \right)^n \varphi\in C([0,T];\mathbb R)$, функция $\mu:[0,T]\to\mathbb R$ имеет ограниченную вариацию,   $\chi(z)\neq 0$ для всех ${z\in\sigma^L(M)}$ и $\int\limits_0^T \varphi (t)d\mu(t)\neq 0$. Тогда для всех $x_k\in {\cal X}$,  $x_T\in D_M$, таких, что при $k=0,1,\ldots, m-1$
$$
(I-P)x_k=-D^{\alpha-m+k}_t|_{t=0}\sum\limits_{l=0}^{p}\left(D^{\alpha}_t \right)^l\varphi (t) G^l M_0^{-1}F(T)(I-P)x_T$$
существует единственное решение задачи {\rm(1)--(3)}. Кроме того, оно имеет вид $$u=(\chi(L_1^{-1}M_1))^{-1}\psi(L_1^{-1}M_1)+F(T)(I-P)x_T.\eqno(4)$$
}



Таким образом, задача (1)--(3) переопределена и рассмотрение вопросов корректности для нее не имеет смысла. Рассмотрим обратную задачу (1), (3) с обобщенными начальными условиями типа Шоуолтера~--- Сидорова
$$
\lim_{t \rightarrow 0+}D^{\alpha -m+k}_{t}Px(t)=x_k , \, \,  k = 0,1,  \ldots ,  m-1.\eqno(5)
$$
Задачу (1), (3), (5) назовем корректной, если при любых $x_k\in {\cal X}^1$, $k=0,1,\ldots, m-1$, $x_T \in D_M$ существует ее единственное решение $u\in {\cal Y}$, при этом оно удовлетворяет неравенству $$\left\|u\right\|_{\mathcal Y}\leq C\left(\sum\limits_{k=0}^{m-1}\left\|x_k\right\|_{\mathcal X}+\left\|x_T\right\|_{\mathcal X}+\left\|Mx_T\right\|_{\mathcal Y}\right),$$
где $C>0$ не зависит от $x_k$, $k=0,1,\dots,m-1$, $x_T$.

Заметим, что такое определение корректности не является стандартным, поскольку в нем использована норма графика данных переопределения $x_T$.

\textbf{Теорема.} {\it Пусть оператор $M$ $(L,p)$-ограничен, $\left(D^{\alpha}_t\right)^n\varphi\in C([0,T];\mathbb R)$
	для $n=0,1,\ldots$, $p$, функция $\mu:[0,T]\to\mathbb R$ имеет ограниченную вариацию.
Тогда задача {\rm(1), (3), (5)} имеет единственное решение в том и только в том случае, когда  $\int\limits_0^T \varphi (t)d\mu(t)\neq 0$,   $\chi(z)\neq 0$ для каждого ${z\in\sigma^L(M)}$.
При этом задача корректна, и ее решение имеет вид} (4).







%%%%  ОФОРМЛЕНИЕ СПИСКА ЛИТЕРАТУРЫ %%%
\smallskip \centerline{\bf Литература}\nopagebreak


1. {\it Kilbas A.A., Srivastava H.M., Trujillo J.J.}
Theory and Applications of Fractional Differential Equations. Amsterdam; Boston; Heidelberg: Elsevier Science Publishing, 2006. 541~p.

2.	{\it Fedorov V.E., Urazaeva A.V.} An inverse problem for linear Sobolev type equations // J. of Inverse and Ill-Posed Problems. 2004. V.12, no.4. P.387-395.

3.	{\it Федоров В.Е., Уразаева А.В.} Обратная задача для одного класса сингулярных линейных операторно-диф\-фе\-рен\-ци\-аль\-ных уравнений // Тр. Воронежск. зимн. мат. шк. Воронеж: ВГУ, 2004. С.161-172.

4.	{\it Уразаева А.В., Федоров В.Е.} Задачи прогноз-управ\-ле\-ния для некоторых систем уравнений гидродинамики // Дифференц. уравнения. 2008. Т.44, № 8. С.1111-1119.

5.	{\it Уразаева А.В., Федоров В.Е.} О корректности задачи прогноз-управления для некоторых систем уравнений // Мат. заметки. 2009. Т.85, вып.3. С.440-450.

6.	{\it Федоров В.Е., Уразаева А.В.} Линейная эволюционная обратная задача для уравнений соболевского типа // Неклассические уравнения математической физики. Новосибирск: Изд-во Ин-та математики им. С.Л.Соболева СО РАН, 2010. С.293-310.

7.	{\it Fedorov V.E., Ivanova N.D.} Identification problem for a degenerate evolution equation with overdetermination on the solution semigroup kernel // Discrete and Continuous Dyna\-mi\-cal Systems. Series S. 2016. V.9, no.3. P.687-696.

8.	{\it Fedorov V.E., Ivanova N.D.} Inverse problem for Oskol\-kov’s system of equations // Mathematical Methods in the Applied Sciences. 2017. Vol.40, iss.17. P.6123-6126.

9.	{\it Fedorov V.E., Ivanova N.D.} Identification problem for degenerate evolution equations of fractional order // Fractional Calculus and Applied Analysis. 2017. V.20, no.3. P.706-721.

10.	{\it Sviridyuk G.A., Fedorov V.E.} Linear Sobolev Type Equ\-a\-ti\-ons and Degenerate Semigroups of Operators. Utrecht; Bos\-ton: VSP, 2003. 213~p.
