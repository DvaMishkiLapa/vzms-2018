\vzmstitle{О полноте корневых элементов пучков обыкновенных дифференциальных операторов}

\vzmsauthor{Абуд}{A.\,Х.}

\vzmsinfo{Ирак; {\it jameel@inbox.ru}}

\vzmscaption

Теоремы полноты корневых элементов пучков дифференциальных операторов появились после работы  [1]. Нарушение такой полноты было впервые обнаружено в случае обыкновенных дифференциальных пучков 2-го порядка в работе [2]. По данному вопросу здесь приводятся различные примеры, относящиеся к квадратичным пучкам обыкновенных дифференциальных операторов:

{\bf Пример 1.} Пучок $y''-\lambda^2y$, $y(0)=y(1)$, $y'(0)=y'(1)$, $x\in(0,1)$, связанный с тригонометрическими рядами Фурье. Здесь $\lambda_k=2\pi k$, $k\in Z$ и соответствующая тригонометрическая система собственных функций $(\sin\,2\pi k x,\cos\,2\pi k x)$ порождает двукратный базис в $L_2(0,1)$.

{\bf Пример 2.} $l(y)\equiv y''-2\lambda y'+\lambda^2y$, $y(0)=y(1)=0$, $0<x<1$. В силу кратности корня $-1$ характеристического уравнения $\varphi^2+\varphi+1=0$ устанавливается отсутствие  собственных значений.

{\bf Пример 3.}
$y''=\lambda y'$, $y(0)=y(1)=0$.
Собственные значения $\lambda_k=2k\pi k i$, $k=\pm1,\pm2,\dots$~--- все простые
и им соответствуют системы 2--производных цепочек $\{e^{2k\pi xi}-1,$ $\lambda_k(e^{2k\pi xi}-1)\}$.
Эти системы имеют бесконечное ортогональное дополнение в
$L_2^2(0,1)$: $\{2\pi m\cos2\pi mx,\sin 2\pi mx\}$, $m\in Z$.

{\bf Пример 4.} $y''-5\lambda y'+6\lambda^2y$, $(0)=y(1)=0$. Здесь характеристические корни $\varphi_1=2$, $\varphi_2=3$ уравнения $\varphi^2-5\varphi+6=0$ расположены на одном луче плоскости. Поэтому, система собственных функций $y_k(x)=e^{6\pi r xi}-e^{4\pi k xi}$, $k\in Z$, соответствующая простым собственным значениям $\lambda_k=2\pi k i$, ортогональна в $L_2(0,1)$ системе $e^{2\pi pxi}$ для $\forall p\neq 0(mod\,3)$ где $p$--- нечётно. То есть система $y_k(x)$ обладает бесконечным дефектом в смысле однократной полноты.

{\bf Пример 5.} $y''-3\lambda u'+2\lambda^2y$, $y(0)=y(1)=0$, $\lambda_k=2r\pi i$, $k=0,\pm1,\pm2,\dots$ исчерпывают собственные значения. им соответствуют собственные функции $y_k=e^{2k\pi xi},e^{4k\pi xi}$. Константа $c\bot y_k$ при $\forall k$. Однако если $f(x)\bot y_k$ коэффициенты Фурье $f_k\to0$ при $k\to\infty$, то $k\equiv0$. Таким образом, собственные функции $y_k$ имеют лишь однократную полноту в $L_2(0,1)$, а именно постоянную $c\bot y_k$.

\litlist

1. {\it Келдыш М.В.} // УМН. 1971, Т. 26, N4, С. 15--41.

2. {\it Гасымов М.Г.} // ДАН Азерб.ССР, 1976, Т. 30, N12, С. 9--12.

3. {\it Вагабов А.И.} Разложение в ряды Фурье по главным функциям дифференциальных операторов ...//Дисс. д--ра физ.--мат. наук, М.: 1987.--- 201 с.
