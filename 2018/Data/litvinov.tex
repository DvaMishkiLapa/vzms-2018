\vzmstitle{ \bf  НАХОЖДЕНИЕ МИНИМАЛЬНОГО ОГРАНИЧЕНИЯ ДЛЯ НОРМЫ ФУНКЦИИ УПРАВЛЕНИЯ}

\vzmsauthor{{Литвинов}}{Д.\, А.}

\vzmsinfo{Воронеж; {\it d77013378@yandex.ru}}

\addcontentsline{toc}{section}{Литвинов Д.А.\dotfill}

Дана  динамическая система:

\begin{equation}
\frac{dx(t)}{dt}=A\, x(t)+B\, u(t),
\end{equation}
где $x(t)\in R^{n}$;
$u(t)\in R^{n}$;
$A$, $B$ матрицы размера $n\times n
$,  $t\in [0,T]$.

Ставится задача построения  управляющей функции $u(t)$ и функции состояния $x(t)$, удовлетворяющих системе (1), при этом  функции состояния и управления  принимают в начальной и конечной точках следующие  значения:

\begin{equation}
x(0)=x_{0},x(T)=x_{T},
\end{equation}

\begin{equation}
u(0)=u_{0},u(T)=u_{T}.
\end{equation}

а также таких, что функция управления при $t\in [0,T]$ удовлетворяет следующему ограничению
\begin{equation}
||u(t)||\leqslant d.
\end{equation}
 Построение $x(t)$ и $u(t)$ ведётся методом каскадной декомпозиции, разработанном в работе [1].

Показывается, что разрешимость данной задачи для взятого ограничения $d$ полностью зависит от разрешимости  неравенства[2]

\begin{multline}
\max\limits_{t\in(0,T)}\frac{-F(t)-\sqrt{F^{2}(t)-4\cdot E(t)(G(t)-d^{2})}}{2\cdot E(t)}
\leqslant
\\ \leqslant
\min\limits_{t\in(0,T)}\frac{-F(t)+\sqrt{F^{2}(t)-4\cdot E(t)(G(t)-d^{2})}}{2\cdot E(t)},
\end{multline}
где $$\sum\limits_{l=1}^{n}(v_{l}(t))^{2}=G(t),$$
$$2\cdot\sum\limits_{l=1}^{n}v_{l}(t)w_{l}(t)=F(t),$$
$$\sum\limits_{l=1}^{n}(w_{l}(t))^{2}=E(t),$$
а $w_{l}(t)\;\;l:\overline{1,n}$ ---компоненты управляющей  функции $w(t)$, являющейся решением задачи (1) и принимающей в крайних точках 0 и $T$ нулевые значения, а $v_{l}(t)\;\;l:\overline{1,n}$ ---компоненты предварительной управляющей  функции $v(t)$, являющейся решением задачи (1)-(3), но не обязательно удовлетворяющей ограничению (4).


Во второй части статьи ищется такое $d_{1}>0$, что для всех $d\geqslant d_{1}$ задача из первой части статьи имеет решение, а для всех $d<d_{1}$ --- не имеет. Решение этой задачи проводится путём поиска минимума функции

$\Psi (c)=\max\limits_{t\in[0,T]}(E(t)\cdot c^{2}+F(t)\cdot c+G(t))$.

Приводится пример использования методов данной статьи для ограничения  компонентов управления самолётом <<Боинг - 747>>.[3]

Часть действий автоматизируется с помощью программы, реализованной на языке Java.
Графики функций состояния и уп\-ра\-в\-ле\-ния строятся в среде Matlab.

\litlist

1. {\it Зубова С.П.} О критериях полной управляемости дескрипторной системы. Полиномиальное решение задачи управления при наличии контрольных точек // Автоматика и телемеханика. 2011. Вып. 1. С. 27--41.

2. {\it Литвинов Д.\,А.}
Об ограниченности нормы управления для линейной стационарной динамической системы
//
Актуальные направления научных исследований XXI века: теория и практика. ВГЛТУ. Воронеж, 2017. Т.5. № 8-1 (34-1). С. 257--259.

3.{\it Афанасьев В.Н., Колмановский В.Б., Носов В.Р. } Математическая теория конструирования систем управления. М.: Высшая школа, 1989. 448 с.


