\vzmstitle{
	СПЕКТРАЛЬНЫЕ СВОЙСТВА ИНТЕГРО-ДИФФЕРЕНЦИАЛЬНОГО ОПЕРАТОРА С ВЫРОЖДЕННЫМ ЯДРОМ,
	ОПРЕДЕЛЯЕМОГО НЕЛОКАЛЬНЫМИ КРАЕВЫМИ УСЛОВИЯМИ
}

\vzmsauthor{Шелковой}{А.\,Н.}

\vzmsinfo{Воронеж; {\it shelkovoj.aleksandr@mail.ru}}

\vzmscaption


Пусть $L_{2}[0,1]$ - гильбертово пространство комплексных измеримых (классов) функций, суммируемых с квадратом модуля со скалярным произведением вида \\$(x,y) = \int\limits_0^1{x(\tau)\overline{y(\tau)}}d\tau$. Через $W_2^2{[0,1]}$ обозначим пространство Соболева $\{x\in L_{2}[0,1]: x'~\text {абсолютно непрерывна},~x''\in L_{2}[0,1]\}$. Рассматривается интегро-дифференциальный оператор $\mathcal{L}:D(\mathcal{L})\subset{L_{2}[0,1]}\to{L_{2}[0,1]}$, порождаемый интегро-дифференциальным выражением вида
$$
(\mathcal{L}x)(t) = -\ddot{x}(t) - [\dot{x}(0)a_{0}(t) - \dot{x}(1)a_{1}(t)] - \int\limits_0^1{K(t,s)x(s)}ds\eqno (1)
$$
с вырожденным ядром $K(t,s) = \sum\limits_{i = 1}^k{p_i(t)q_i(s)},~p_i, q_i\in{L_{2}[0,1]}$,
с областью определения $D(\mathcal{L}) = \{x\in{W_2^2[0,1]},~x(0) = x(1) = 0\}$ и краевыми условиями
$$
x(0) = x(1) = 0.\eqno (2)
$$

Такого класса оператор возникает при переходе к сопряжённому при исследовании оператора, действующего в $L_{2}[0,1]$, задаваемого выражением
$$
(\mathcal{L}y)(t) = -\ddot{y}(t) - \int\limits_0^1{\sum\limits_{i = 1}^k{p_i(t)q_i(s)}y(s)}ds
$$
и нелокальными краевыми условиями
$$
y(0) = \int\limits_0^1{a_0(t)y(t)}dt,~
y(1) = \int\limits_0^1{a_1(t)}y(t)dt.
$$
Здесь $a_0$ и $a_1$ - функции из $L_{2}[0,1]$.

Для исследования спектра оператора $\mathcal{L}$ рассмотрим сопряжённый ему оператор $\mathcal{L}^*$ (см. [3]), который задаётся\\интегро-дифференциальным выражением
$$
(\mathcal{L}^*x)(t) = -\ddot{x}(t) - [\dot{x}(0)a_{0}(t) - \dot{x}(1)a_{1}(t)] - \int\limits_0^1{K(t,s)x(s)}ds\eqno (3)
$$
и краевыми условиями (2).

Методом исследования оператора $\mathcal{L}^*$ является метод подобных операторов, рассматриваемый в работах [1-6]. Представим его в виде
$$
\mathcal{L}^*x = Ax - B_1{x} - B_2{x}.\eqno (4)
$$
Оператор $A$ порождается дифференциальным выражением $Ax = -\ddot{x},~x\in{D(A)},~D(A) = \{x\in{L_2[0,1]}:~x, \dot{x}\in{C[0,1]},~\ddot{x}\in{L_2[0,1]},~x(0) = x(1) = 0\}$. Он является самосопряжённым оператором с дискретным спектром, собственные значения которого $\lambda_n = \pi^{2}n^2,~n\in{\mathbb{N}}$, являются простыми, а соответствующие собственные функции $e_n(t) = \sqrt{2}\sin{\pi{nt}},~n\geqslant{1}$, образуют ортонормированный базис в гильбертовом пространстве $L_{2}[0,1]$ (см., например, [1]).\\
Операторы $B_1$ и $B_2$ на области определения $D(A)$ задаются соотношениями
$$(B_1{x})(t) = \dot{x}(0)a_{0}(t) - \dot{x}(1)a_{1}(t);~(B_2{x})(t) = \int\limits_0^1{K(t,s)x(s)}ds,$$
$x\in{d(A)},~t\in{[0,1]}$.

Применяя метод подобных операторов для исследования спектральных свойств оператора (4), мы получим следующие результаты.

\textbf{Теорема.} {\it Пусть для рассматриваемых функций\\$p_i, q_i, a_0, a_{1}\in{L_{2}[0,1]}$, для последовательностей величин $\gamma_1$ и $\gamma_2$, определённых формулами\\
$\gamma_1(n) = \frac{1}{\pi^2}\bigg(\sum\limits_{\substack{m\geqslant 1 \\ m\ne n}}\bigg\{n^2\bigg[\pi(|a_{0m}^{\sin}|+|a_{1m}^{\sin}|)+
\frac{1}{8}\sup\limits_j\bigg|\sum\limits_{i=1}^{k}{\frac{q_{ij}^{\sin}}{j}\cdot{p_{im}^{\sin}}}\bigg|\bigg]^2 +$
$+ m^2\bigg[\pi(|a_{0n}^{\sin}|+|a_{1n}^{\sin}|)+\frac{1}{8}\sup\limits_j\bigg|\sum\limits_{i=1}^k
\frac{q_{ij}^{\sin}}{j}\cdot{p_{in}^{\sin}}\bigg|\bigg]^2\bigg\}\Bigg/{|n^2-m^2|^2}\bigg)^\frac{1}{2}$\\и\\
$\gamma_2(n)= \frac{1}{\pi^2}\max\bigg\{\frac{n\bigg[\pi(|a_{0n}^{\sin}|+|a_{1n}^{\sin}|)+
\frac{1}{8}\sup\limits_j\bigg|\sum\limits_{i=1}^{k}{\frac{q_{ij}^{\sin}}{j}\cdot{p_{in}^{\sin}}}\bigg|\bigg]}{2n-1},$
$$\sum\limits_{\substack{m\geqslant 1 \\ m\ne n}}\frac{m\bigg[\pi(|a_{0m}^{\sin}|+|a_{1m}^{\sin}|)+
\frac{1}{8}\sup\limits_j\bigg|\sum\limits_{i=1}^{k}{\frac{q_{ij}^{\sin}}{j}\cdot{p_{im}^{\sin}}}\bigg|\bigg]}{|n^2-m^2|}\bigg\},$$
где $a_{0j}^{\sin} = 2\int\limits_0^1{a_0(t)\sin\pi{jt}}dt,~a_{1j}^{\sin} = 2\int\limits_0^1{a_1(t)\sin\pi{jt}}dt,\\p_{ij}^{\sin} = 2\int\limits_0^1{p_i(t)\sin\pi{jt}}dt,~q_{ij}^{\sin} = 2\int\limits_0^1{q_i(s)\sin\pi{js}}ds, ~j= 1, 2, \ldots$, - коэффициенты разложения функций $a_0,~a_1,~p_i,~q_i, ~i=\overline{1, k},$ в ряд Фурье по синусам, выполнены условия:\\$\lim\limits_{n\to\infty}\gamma_1(n)= 0,~\lim\limits_{n\to\infty}\gamma_2(n)=0$.
Тогда спектр $\sigma(A - B)$ оператора $A - B (B = B_1 + B_2)$ представим в виде $~\sigma(A-B)= \widetilde{\sigma}_m\bigcup\bigg(\bigcup\limits_{n\geqslant m+1}\widetilde{\sigma}_n\bigg),~$
где $\widetilde{\sigma}_n$, $n\geqslant m+1$,~--- одноточечные множества, а $~\widetilde{\sigma}_m$~--- конечное множество с числом
точек, не превосходящим $m$. Для собственных значений $\widetilde{\lambda}_n$ из $\widetilde{\sigma}_n$ имеет место оценка:

\begin{flalign*}
\bigg|\widetilde{\lambda}_n &- \pi^2n^2 - \pi n(a_{0n}^{\sin}+(-1)^{n+1}a_{1n}^{\sin})-
\frac{1}{2}\sum\limits_{i=1}^k q_{in}^{\sin}p_{in}^{\sin}~+ \\
&+ \frac{1}{\pi^2}\sum\limits_{\substack{m\geqslant 1 \\ m\ne n}}\bigg(\bigg[\pi n(a_{0m}^{\sin}+(-1)^{n+1}a_{1m}^{\sin})+
\frac{1}{2}\sum\limits_{i=1}^k q_{in}^{\sin}p_{im}^{\sin}\bigg]\cdot \\
&\cdot \bigg[\pi m(a_{0n}^{\sin} + (-1)^{m+1}a_{1n}^{\sin}) +
\frac{1}{2}\sum\limits_{i=1}^k q_{im}^{\sin}p_{in}^{\sin}\bigg]\bigg/(n^2-m^2)\bigg)\bigg|\leqslant
\end{flalign*}
$$
\leqslant \mathrm{const}\,\cdot n\cdot\bigg(\pi(|a_{0n}^{\sin}|+|a_{1n}^{\sin}|) +
\frac{1}{8}\sup\limits_j\bigg|\sum\limits_{i=1}^k \frac{q_{ij}^{\sin}}{j}p_{in}^{\sin}\bigg|\bigg)\cdot\gamma_2(n).
$$

Для собственных функций $\widetilde{e}_n$ оператора (4) справедлива оценка:
$$
\bigg(\int\limits_0^1\bigg|\widetilde{e}_{n}(t)-\sqrt{2}\sin{\pi{nt}}~+
$$
$$\frac{\sqrt{2}}{\pi^2}\sum\limits_{\substack{m=1 \\ m\ne n}}^\infty \frac{\pi n(a_{0m}^{\sin} + (-1)^{n+1}a_{1m}^{\sin}) +
\frac{1}{2}\sum\limits_{i=1}^k q_{in}^{\sin}p_{im}^{\sin}}{n^2-m^2}\sin\pi mt\bigg|^2\,dt\bigg)^\frac{1}{2}\leqslant
$$
$$
\leqslant \mathrm{const}\,\cdot n\cdot\bigg(\sum\limits_{\substack{m=1 \\ m\ne n}}^\infty
\frac{\bigg[\pi(|a_{0n}^{\sin}| + |a_{1n}^{\sin}|) + \frac{1}{8}\sup\limits_j\bigg|
\sum\limits_{i=1}^k\frac{q_{ij}^{\sin}}{j}p_{in}^{\sin}\bigg|\bigg]^2}{|n^2-m^2|^2}\bigg)^\frac{1}{2}.
$$
}

\litlist

1. {\it Баскаков А.Г.} Гармонический анализ линейных операторов. Воронеж: Изд-во ВГУ, 1987. 165 с.

2. {\it Баскаков А.Г.} Теорема о расщеплении оператора и некоторые смежные вопросы аналитической теории возмущений / А.Г. Баскаков // Изв. АН СССР. Сер. матем., 1986. Т. 50, № 3. С. 435-1457.

3. {\it Баскаков А.Г.} Спектральный анализ интегро"=дифференциальных операторов с нелокальными краевыми условиями / А.Г. Баскаков, Т.К. Кацаран // Дифференциальные уравнения, 1998. Т. 24, № 8. С. 1424-1433.

4. {\it Ульянова Е.Л.} О некоторых спектральных свойствах одного класса дифференциальных операторов с нелокальными краевыми условиями / Е.Л. Ульянова, А.Н. Шелковой // Вестник ВГУ. Сер. физика, математика, 2002. № 2. С. 106-110.

5. {\it Шелковой А.Н.} Спектральный анализ дифференциальных операторов с нелокальными краевыми условиями: дисс. ... канд. физ.-мат. наук. Воронеж, 2004. 144 с.

6. {\it Шелковой А.Н.} Асимптотика собственных значений дифференциального оператора с нелокальными краевыми условиями / А.Н. Шелковой // Научные ведомости БелГУ.
Сер. Математика. Физика, 2016. № 13. С. 72-80.

