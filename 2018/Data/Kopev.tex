\vzmstitle{ОТРАЖЕНИЕ НА МЕТЕОРНЫХ ПОТОКАХ}

\vzmsauthor{Копьев}{А.\,А.}

\vzmsauthor{Яшагин}{Е.\,И.}

\vzmsinfo{Зеленодольск; {\it eugene.yashagin@gmail.com}}

\vzmscaption


В данной работе найдено аналитическое решение для поверхности отражения на метеорных потоках, когда последние представлены несобственным пучком прямых.

Сначала рассмотрим пучок прямых второго рода. Из фиксированной точки $S$ исходят лучи, и в точке $P$ они собираются, испытывая отражение в точках заданного пучка, где выполняется известное условие равенства падающего и отражённого угла. Определим кривую, состоящую из точек отражения.

Обозначим направляющий вектор пучка прямых $\vec a = \{ 1,l \}$, где $l$ действительное число. Такое задание направляющего вектора $\vec a $ исключает случай, когда $\vec a $ параллелен оси $ OY$, но он тривиален. Из точки $S$ источника (совпадающей с началом координат) проведём векторы $\overrightarrow{OM}$ и $\overrightarrow{MP}$ --- ход луча после отражения от точки $M$ на прямой в точку приёмника $P$.

В выбранных координатах $\overrightarrow{OM}=\{x,y\}$. Точка $P$ находится на положительной части оси $OX$ и имеет координаты $(p;0)$.


$$\overrightarrow{OM}=\overrightarrow{OP}+\overrightarrow{PM}$$
$$\overrightarrow{PM}=\overrightarrow{OM}-\overrightarrow{OP}=\{x,y\}-\{p,0\}=\{x-p,y\}$$

Построим направляющий вектор биссектрисы   угла между $\overrightarrow{OM}$ и $\overrightarrow{MP}$ , для этого каждый из векторов сделаем единичной длины и сложим.


$$ \vec{N}=\frac{1}{|\overrightarrow{OM}|}\cdot\overrightarrow{OM}+\frac{1}{|\overrightarrow{PM}|}\cdot\overrightarrow{PM} $$



Для того, чтобы получить кривую отражения, найдём такие точки на прямых пучках,
в которых выполняется условие перпендикулярности направляющего вектора $\vec a$ и  вектора,
определяющего направление биссектрисы $\vec N$

$$(\vec a,\vec N )=0$$

или

$$\left(\vec a,\frac{1}{|\overrightarrow{OM}|}\cdot\overrightarrow{OM}+\frac{1}{|\overrightarrow{PM}|}\cdot\overrightarrow{PM}\right) =0$$



Используя свойства скалярного произведения получим:

$$\overrightarrow{PM}\cdot(\vec a ,\overrightarrow{OM})=(-1)\cdot\overrightarrow{OM}\cdot(\vec a,\overrightarrow{PM})$$

И  окончательно в координатах:
\[
\sqrt{(x-p)^2+y^2}
\left(x_a x +y_a y\right)
=
\sqrt{x^2+y^2}
\left( x_a (p-x)-y_ay \right)
\]

С учётом выбранного вектора $\vec a = \{ 1,l \}$:
\[
\sqrt{(x-p)^2+y^2}
\left(x+l y\right)
=
\sqrt{x^2+y^2}
\left( (p-x)-l y \right)
\]

Это уравнение кривой отражения на пучке прямых второго рода с заданным вектором $\vec a = \{ 1,l \}$. Оно задаёт на плоскости две полуветви гиперболы, что было известно ранее, но получено из иных соображений.

Мы, используя наш метод, добавим третью координату и получим:

\[
\sqrt{(x-p)^2+y^2+z^2}
\left(x+l y+m z\right)
= \]
\[=
\sqrt{x^2+y^2+z^2}
\left( (p-x)-l y-m z \right)
\]


Здесь $\vec a = \{ 1,l,m\}$ --- уже определяет направление несобственного пучка в пространстве.
Уравнение (2) имеет следующее действительное решение:

$$z=(-p+2x+2ly)/(6m)-(2^\frac{1}{3}(-\frac{1}{4}(-p+2x+2ly)^2 +\frac{3}{2} m(mp^2-$$
$$-2mpx+2mx^2+2my^2)))/(3m(\frac{1}{4}p^3-\frac{9}{4}m^2p^3-\frac{3}{2}p^2x-\frac{9}{2}m^2p^2x+$$
$$+3px^2+27m^2px^2-2x^3-18m^2x^3-\frac{3}{2}lp^2y-9lm^2p^2y+6lpxy+$$
$$+18lm^2pxy-6lx^2y-18lm^2x^2y+3l^2py^2+9m^2py^2-6l^2xy^2-$$
$$-18m^2xy^2-2l^3m^3-18lm^2y^3+((\frac{1}{4}p^3-\frac{9}{4}m^2p^3-\frac{3}{2}p^2x-\frac{9}{2}m^2p^2x+$$
$$+3px^2+27m^2px^2-2x^3-18m^2x^3-\frac{3}{2}lp^2y-9lm^2p^2y+6lpxy+$$
$$+18l^2pxy-6lx^2y-18lm^2x^2y+3l^2py^2+9m^2py^2-6l^2xy^2-$$
$$-18m^2xy-2l^3y^3-18lm^2y^3)^2+4(-\frac{1}{4}(-p_2x+2ly)^2+\frac{3}{2}m(mp^2-$$
$$-2px+2mx^2+2my^2))^3)^\frac{1}{2})^\frac{1}{3}) +
(1)/(32^\frac{1}{3}m) *
(\frac{1}{4}p^3-\frac{9}{4}m^2p^2-$$
$$-\frac{3}{2}p^2x-\frac{9}{2}m^2p^2x+3px^2+27m^2px^2-2x^3-18m^2x^3-\frac{3}{2}lp^2y-$$
$$-9lm^2p^2y+6lpxy+18lm^2xy-6lx^2y-18lm^2x^2y+3l^2py^2+$$
$$+9m^2py^2-6l^2xy^2-18m^2xy^2-2l^3y^3-18lm^2y^3+((\frac{1}{4}p^3-\frac{9}{4}m^2p^2-$$
$$+\frac{3}{2}p^2x-\frac{9}{2}m^2p^2x+3px^2+27m^2px^2-2x^3-18m^2x^3-\frac{3}{2}lp^2y-$$
$$-9lm^2p^2y+6lpxy+18lm^2pxy-6lx^2y-18lm^2x^2y+3l^2py^2+$$
$$+9m^2py^2-6l^2xy^2-18m^2xy^2-2l^3y^3-18lm^2y^3)^2+4(-\frac{1}{4}(-p+$$
$$+2x+2ly)^2+\frac{3}{2}m(mp^2-2mpx+2mx^2+2my^2))^3)^\frac{1}{2})^\frac{1}{3}$$
Эта формула представлена впервые, явное выражение даёт возможность применить методы дифференциальной геометрии в полной мере.

