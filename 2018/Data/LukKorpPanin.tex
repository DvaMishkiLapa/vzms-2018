\begin{center}{ \bf  АНАЛИТИКО-ЧИСЛЕННОЕ ИССЛЕДОВАНИЕ ЯВЛЕНИЯ РАЗРУШЕНИЯ РЕШЕНИЙ ЗАДАЧ МАТЕМАТИЧЕСКОЙ ФИЗИКИ}\\
{\it Д. В. Лукьяненко, М. О. Корпусов, А. А. Панин } \\
(Москва; {\it lukyanenko@physics.msu.ru, a-panin@yandex.ru} )
\end{center}
\addcontentsline{toc}{section}{Лукьяненко Д. В.,Корпусов М. О.,. Панин А. А\dotfill}


Нами рассмотрен [4]--[9] ряд начально-краевых задач математической физики, в которых при некоторых начальных данных происходит разрушение решения за конечное время. Как правило, это означает обращение нормы решения в бесконечность за конечное время:
\begin{equation*}
\lim_{t\to T_0-0}\|u(t)\|=+\infty,
\end{equation*}
но в некоторых случаях решение, существующее на конечном промежутке времени, может быть непродолжаемым и при этом ограниченным по норме~[6].

Установлен факт разрушения решения для некоторых начальных данных и получены оценки сверху на время разрушения. Продемонстрировано, как в каждом конкретном примере данные оценки могут быть численно уточнены методом, основанным на идеях Н.~Н.~Калиткина и соавторов~[1]--[3].





%%%%  ОФОРМЛЕНИЕ СПИСКА ЛИТЕРАТУРЫ %%%
\smallskip \centerline{\bf Литература}\nopagebreak

1. {\it Альшина Е. А., Калиткин Н. Н., Корякин П. В.} Диагностика особенностей точного решения при расчетах с
контролем точности // Журнал вычислительной математики и математической физики, 2005. 45, № 10. 1837–
1847.

2. {\it Калиткин Н. Н., Альшин А. Б., Альшина Е. А., Рогов Б. В.} Вычисления на квазиравномерных сетках. М.: Физматлит, 2005.

3. {\it Al’shin A. B., Al’shina E. A.} Numerical diagnosis of blow-up of solutions of pseudoparabolic equations // Journal of
Mathematical Sciences, 2008. 148, no. 1. 143–162.


4.	{\it А. А. Панин, Г. И. Шляпугин.} О локальной разрешимости и разрушении решений одномерных уравнений типа Ядзимы–Ойкавы–Сацумы. Теоретическая и математическая физика, 2017, 193, № 2, с.~179—192. {\sloppy


}

5.	{\it М. О. Корпусов, А. А. Панин.} О непродолжаемом решении и разрушении решения одномерного уравнения ионно-звуковых волн в плазме // Матем. заметки, том 102, № 3 (2017), 383–395.

6.	{\it Корпусов М. О., Лукьяненко Д. В., Овсянников Е. А., Панин А. А.} Локальная разрешимость и разрушение решения одного уравнения с квадратичной некоэрцитивной нелинейностью. // Вестник Южно-Уральского государственного университета. Серия Математическое моделирование и программирование, 2017,  10, № 2, с. 107—123.

7.	{\it Korpusov M. O., Lukyanenko D. V., Panin A. A., Yushkov E. V.} Blow-up phenomena in the model of a space charge stratification in semiconductors: analytical and numerical analysis // Mathematical Methods in the Applied Sciences. 2017. Vol. 40, no. 7, pp. 2336—2346.{\sloppy

}

8.	{\it Д. В. Лукьяненко, А. А. Панин.} Разрушение решения уравнения стратификации объемного заряда в полупроводниках: численный анализ при сведении исходного уравнения к дифференциально-алгебраической системе. // Вычислительные методы и программирование: Новые вычислительные технологии (Электронный научный журнал), 2016, том 17, № 1, с. 437–446.

9.	{\it Korpusov M. O., Lukyanenko D. V., Panin A. A., Yushkov~E.~V.} Blow-up for one Sobolev problem: theoretical approach and numerical analysis // Journal of Mathematical Analysis and Applications. 2016. Vol. 442. No. 2. P. 451—468. {\sloppy
 