\vzmstitle[
	\footnote{
		Работа выполнена при финансовой поддержке Российского фонда фундаментальных исследований
		(проекты 16-01-00301, 17-08-01279).
	}
]{
	КОНЕЧНО-ЭЛЕМЕНТНАЯ АППРОКСИМАЦИЯ ПРОИЗВОЛЬНОГО ПОРЯДКА ЗАДАЧИ
	О СОБСТВЕННЫХ КОЛЕБАНИЯХ НАГРУЖЕННОЙ БАЛКИ
}

\vzmsauthor{Самсонов}{А.\,А.}

\vzmsauthor{Соловьёв}{С.\,И.}

\vzmsinfo{Казань; {\it anton.samsonov.kpfu@mail.ru}}


\vzmscaption


Пусть ось балки длины $l$
занимает в равновесном горизонтальном положении отрезок $\overline{\Omega}=[0,l]$
оси $Ox$, $\Omega=(0,l)$. Обозначим через $\rho=\rho(x)$ и $E=E(x)$ линейную плотность и
модуль упругости материала балки в точке $x$,
через $S=S(x)$ и $J=J(x)$ -- площадь поперечного сечения балки и момент
инерции сечения в точке $x$ относительно своей горизонтальной оси.
Предположим, что концы балки $x=0$ и $x=l$ заделаны жёстко.
Предположим,
что в точке балки
$x_0\in(0,l)$ упруго присоединён груз (осциллятор) с массой $M$  и
коэффициентом жёсткости подвески $K$,
$\sigma=K/M$,
при этом
$\sqrt{\sigma}$ есть собственная частота осциллятора.

Исследование собственных колебаний механической системы балка-пружина-груз приводит к
задаче на собственные значения с нелинейной зависимостью от спектрального параметра\,[1]:
найти числа $\lambda$ и ненулевые функции $u(x)$, $x\in\Omega$,
удовлетворяющие уравнению

$$
(EJu^{\prime\prime})^{\prime\prime}+
\frac{\lambda}{\lambda-\sigma}
K\delta(x-x_0)u=
\lambda\,\rho S\,u,\quad
x\in\Omega,
$$
и граничным условиям
$
u(0)=u^{\prime}(0)=u(l)=u^{\prime}(l)=0,
$
где $\delta(x)$ -- дельта-функция Дирака.
Число $\sqrt{\lambda}$ определяет частоту собственного колебания системы, а функция
$u(x),$ $x\in\overline{\Omega}$ -- амплитуду собственного колебания каждой точки балки.

В работе установлено существование последовательности положительных
простых собственных значений с предельной точкой
на бесконечности, которой соответствует последовательность нормированных собственных функций задачи.
Задача аппроксимируется сеточной схемой метода конечных
элементов на равномерной сетке с эрмитовыми конечными элементами произвольного порядка.
Исследована сходимость и доказаны оценки погрешности приближённых
собственных значений и собственных функций.
Полученные результаты развивают и обобщают результаты из [1--3].




\litlist

1. {\it Соловьёв С.И.}
Нелинейные задачи на собственные значения. Приближённые методы. --
Saarbr\"ucken: LAP Lambert Academic Publishing, 2011. -- 256 с.

2. {\it Соловьёв С.И.}
Аппроксимация нелинейных спектральных задач в гильбертовом пространстве
//~Дифференциальные уравнения. -- 2015. -- Т. 51,
\No 7. -- С. 937--950.

3. {\it Соловьёв\,С.И.}
Собственные колебания стержня с упруго присоединённым грузом
//~Дифференциальные уравнения. -- 2017. -- Т. 53,
\No 3. -- С. 418--432.

