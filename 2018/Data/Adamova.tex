
\begin{center}{ \bf  К \glqq теореме Ньютона\grqq}\\
	{\it Р. С. Адамова, М.А. Курлыкина } \\
	(Воронеж; {\it adamova\_rs@mail.ru} )
\end{center}
\addcontentsline{toc}{section}{Адамова Р.С., Курлыкина М.А.}


В работе рассмотрен вопрос о существовании точек перегиба у вещественной эллиптической кривой и его отношении к \glqq теореме Ньютона\grqq \, на ту же тему.

Из [1] следует, что существование точек перегиба у эллиптической кривой равносильно возможности приведения её уравнения к форме Вейерштрасса
$$y^2=x^3+px+q.$$

Если кривая рассматривается над замкнутым полем, то такая точка всегда есть. В случае поля рациональных чисел существуют кривые без точек перегиба. Вероятно, такая же ситуация и в случае конечных полей. Обращаясь к полю вещественных чисел, отметим, что простыми преобразованиями координат в аффинной плоскости уравнение эллиптической кривой можно привести к одному из двух видов, один из которых -- форма Вейерштрасса. Работая с другим видом

$$y^2-2yx^2-a_0x^3-a_1x^2-a_2x-a_3=0,$$
приходим к ситуации, что для поиска её точек перегиба следует искать вещественные корни у многочлена 8-ой или 9-ой степени относительно переменной  $x$ [2]. В первом, почти безнадёжном случае, оказалось, что точкой перегиба является бесконечная точка $(2, - a_0, 0).$ Во втором случае вещественный корень уравнения существует, но это не гарантирует вещественности соответствующего значения координаты $y$, о чём также сделано замечание в [1] при подходе к вопросу с другой стороны. Благодаря форме полученного уравнения удаётся найти соотношение координат для точек перегиба в виде линейного уравнения относительно $y$ с коэффициентами из кольца многочленов R[$x$]. В результате получаем справедливость следующей теоремы.

\textbf{Теорема.} {\it Всякая вещественная эллиптическая кривая в подходящей системе проективных координат имеет ура\-в\-не\-ние в форме Вейерштрасса.}

\textbf{Следствие.} {\it Всякая вещественная эллиптическая кривая имеет ровно три точки перегиба.}

Заметим, что в популярной литературе ([3]) эта теорема носит название теоремы Ньютона. Осмелимся сказать, что это ошибочная информация, поскольку в классификации Ньютона к различным классам отнесены кривые с уравнениями в форме Вейерштрасса и уравнениями вида
$$xy^2+ey=ax^3+bx^2+cx+d,$$
в то время как последнее является уравнением эллиптической кривой.



\smallskip \centerline{\bf Литература}\nopagebreak

1. {\it Прасолов В.В, Соловьев С.П. } Эллиптические функции и алгебраические уравнения. -- М.: Изд-во Факториал, 1997, -- 288С.

2. {\it Адамова Р.С., Артёмов Н.М. }
Эллиптические кривые, Современные методы теории функций и смежные проблемы, материалы Международной конференции,
Во\-ро\-не\-ж\-с\-кая зимняя математическая школа (27 января -- 2 февраля 2015г.), стр. 167-168.

3. {\it Острик В.В., Цфасман М.А.   } Алгебраическая геометрия и теория чисел: рациональные и эллиптические кривые. (Серия: "<Библиотека \glqq Математическое просвещение\grqq "> ), М.: МЦНМО, 2001, -- 48С.:ил.
