\vzmstitle{ \bf  Априорные оценки решений краевых задач для вырождающихся
эллиптических уравнений высокого порядка}

\vzmsauthor{{Ковалевский}}{Р.\, А.}

\vzmsinfo{Воронеж; {\it vlkostin@mail.ru}}

\vzmscaption


Рассмотрим функцию $\alpha (t),\,\,t \in R_ + ^1 $, для которой $\alpha ( +
0) = {\alpha }'( + 0) = 0$, $\alpha \mbox{(}t\mbox{) > 0}$при$t > 0$,
$\alpha \mbox{(}t\mbox{) = const}$ для $t \geqslant d$ при некотором $d\mbox{ >
0}$. Рассмотрим интегральное преобразование
\[
F_\alpha [u(t)](\eta ) = \int\limits_0^{ + \infty } {u(t)\exp (i\eta }
\int\limits_t^d {\frac{d\rho }{\alpha (\rho )}} )\frac{dt}{\sqrt {\alpha
(t)} },
\]
определённое, например,
на функциях $u(t) \in C_0^\infty (R_ + ^1
)$.
Свойства этого преобразования доказаны в
[1]--[5].
Для преобразования $F_\alpha $ справедлив аналог равенства Парсеваля
$\left\| {F_\alpha [u](\eta )} \right\|_{L_2 (R^1)} = \sqrt {2\pi } \left\|
u \right\|_{L_2 (R_ + ^1 )} ,$ что даёт возможность расширить это
преобразование до непрерывного преобразования, осуществляющего гомеоморфизм
пространств $L_{2} (R^1)$ и $L_{2} (R_ + ^1 )$. Это равенство
позволяет также рассмотреть преобразование $F_\alpha $ на некоторых классах
обобщённых функций. Для расширенного таким образом
\linebreak
преобразования $F_\alpha
$ сохраним старое обозначение. Обозначим через $F_\alpha ^{ - 1} $ обратное
к $F_\alpha $ преобразование. Это преобразование можно записать в виде
$F_\alpha ^{ - 1} [w(\eta )](t) = \left. {\frac{1}{\sqrt {\alpha (t)}
}F_{\eta \to \tau }^{ - 1} [w(\eta )]} \right|_{\tau = \varphi (t)} $, где
$F_{\eta \to \tau }^{ - 1} $~--- обратное преобразование Фурье. В [1]
показано, что на функциях $u(t) \in C_0^\infty (\bar {R}_ + ^1 )$
выполняются соотношения
\linebreak
$F_\alpha [D_{\alpha ,t}^j u](\eta ) = \eta
^jF_\alpha [u](\eta ),\,\,j = 1,2,...$, где $D_{\alpha ,t} =
\frac{1}{i}\sqrt {\alpha (t)} \partial _t \sqrt {\alpha (t)}
,$ $\partial _t = \frac{\partial }{\partial t}.$

\textbf{Определение 1.} Пространство$H_{s,\alpha } (R_ + ^n )$ ($s$ --
действительное число) состоит из всех функций, для которых конечна норма
\[
\left\| {v,\left| p \right|} \right\|_{s,\alpha }^2 = \int\limits_{R^n}
{(\left| p \right|^2 + \left| \xi \right|^2 + \eta ^2)^s\left| {F_\alpha
F_{x \to \xi } [v(x,y)]} \right|^2d\xi d\eta } ,
\]
зависящая от комплексного параметра $p \in Q = \{p \in C,\,\,\left| {\arg p}
\right| < \frac{\pi }{2},\,\,\left| p \right| > 0\}$.

\textbf{Определение 2.} Пространство$\,H_{s,\alpha ,q} (R_ + ^n )\,\,(s
\geqslant 0,\,\,q > 1)$ состоит из всех функций $v(x,y) \in H_{s,\alpha } (R_ + ^n
)$, для которых конечна норма
\begin{multline*}
\left\| {v,\left| p \right|} \right\|_{s,\alpha ,q} =
\\=
\{\sum\limits_{l =
0}^{[\frac{s}{q}]} {\left\| {F_{\xi \to x}^{ - 1} F_\alpha ^{ - 1} [(\left|
p \right|^2 + \left| \xi \right|^2 + \eta ^2)^{\frac{s - ql}{2}}F_\alpha
F_{x \to \xi } [\partial _y^l v]]} \right\|_{L_2 (R_ + ^n )}^2 }
\}^{\frac{1}{2}},
\end{multline*}
зависящая от комплексного параметра. Здесь $[\frac{s}{q}]$ - целая часть
числа $\frac{s}{q}.$

\textbf{Условие 1.} Существует число $\nu \in \left( {0,1} \right]$ такое,
что
\linebreak
$\left| {\alpha ^ / (t)\alpha ^{ - \nu }(t)} \right| \leqslant c < \infty $
при всех $t \in \left[ {0,\, + \infty } \right)$. Кроме того, $\alpha \left(
t \right) \in C^{s_1 }\left[ {0, + \infty } \right)$ для некоторого $s_1 =
s_1 (\nu )$.

В $R_ + ^n $ рассматривается линейное дифференциальное уравнение вида
\begin{equation}
\label{eq4600}
A(p,y,D_x ,D_{\alpha ,y} ,\partial _y )v(x,y) = F(p,x,y),
\end{equation}
где
\begin{equation}
\label{eq4601}
A(p,y,D_x ,D_{\alpha ,y} ,\partial _y )v = \sum\limits_{\left| \tau \right|
+ j_1 + qj_2 + rj_3 \leqslant 2m} {a_{\tau j_1 j_2 j_3 } (y)p^{j_3 }D_x^\tau
D_{\alpha ,y}^{j_1 } } \partial _y^{j_2 } v.
\end{equation}



Здесь $m,k,l$~--- натуральные числа, $q = \frac{2m}{k} > 1,\,\,r = \frac{2m}{l} >
1,\,\,a_{\tau j_1 j_2 j_3 } (y)$~--- некоторые ограниченные на $\bar {R}_ + ^1
$ функции,
\linebreak
$a_{00k0} (y) \ne 0$ при всех $y \in \bar {R}_ + ^1 .$ Без
ограничения общности будем считать, что $a_{00k0} (y) = 1$ при всех $y \in
\bar {R}_ + ^1 .$

На границе $y = 0$ полупространства $R_ + ^n $задаются граничные условия вид
\begin{equation}
\label{eq4602}
B_j (p,D_x ,\partial _y )\left. v \right|_{y = 0} = \sum\limits_{\left| \tau
\right| + rj_3 + qj_2 \leqslant m_j }^ {b_{\tau j_2 j_3 } p^{j_3 }D_x^\tau
\partial _y^l } \left. v \right|_{y = 0} = G_j (p,x),
j = 1,2,...,\mu .
\end{equation}



Пусть выполнены следующие условия.

\textbf{Условие 2.} Уравнение
\[
\sum\limits_{\left| \tau \right| + j_1 + qj_2 + rj_3 = 2m}^ {a_{\tau j_1 j_2
j_3 } (y)\xi ^\tau \eta ^{j_1 }} z^{j_2 }p^{j_3 } = 0.
\]
не имеет $z$ -- корней, лежащих на мнимой оси при всех $y \geqslant 0\,\,(\xi
,\eta ) \in R^n\,\,$, $p \in Q = \{p \in C,\,\,\left| {\arg p} \right| <
\frac{\pi }{2},\,\,\left| p \right| > 0\}$, $\left| p \right| + \left| \eta
\right| + \left| \xi \right| > 0$.

Пусть $z_1 (p,y,\xi ,\eta ),...,z_{r_1 } (p,y,\xi ,\eta )\,\,\,(1 \leqslant r_1
\leqslant k)$ - корни, лежащие в левой полуплоскости, а $z_{r_1 + 1} (p,y,\xi
,\eta ),...,z_k (p,y,\xi ,\eta )$ лежат в правой полуплоскости.

\textbf{Условие 3.} Функции $z_j (p,y,\xi ,\eta )$, $j = 1,\,2,...,\,k,$ при
всех $\xi \in R^{n - 1}$ являются бесконечно дифференцируемыми функциями по
переменным $y \in \Omega \subset \bar {R}_ + ^1 $ и $\eta \in R^1$. Причём,
при всех $p \in Q = \{p \in C,\,\,\left| {\arg p} \right| < \frac{\pi
}{2},\,\,\left| p \right| > 0\}$, $j_1 = 0,\,\,1,\,\,2,\,...,l =
0,\,\,1,\,\,2,\,...,\xi \in R^{n - 1}$, $y \in \Omega \subset \bar {R}_ +
^1 $, $\eta \in R^1$ справедливы оценки
\[
\left| {(\alpha (y)\partial _y )^{j_1 }\partial _\eta ^{j_2 } z_j (y,\xi
,\eta )} \right| \leqslant c_{j_1 ,l} (\left| p \right| + \left| \xi \right| +
\left| \eta \right|)^{q - j_2 },\,\,\,\,\,\left| p \right| + \left| \xi
\right| + \left| \eta \right| > 0,
\]
с константами $c_{j_1 ,l} > 0$, не зависящими от$p,y,\,\,\xi ,\,\,\eta .$

Из условия 2 следует, что при всех $p \in Q$, $\xi \in R^{n - 1}$, $y \in
\Omega \subset \bar {R}_ + ^1 $, $\eta \in R^1$ справедливы оценки
\[
Rez_j (p,y,\xi ,\eta ) \leqslant - c_1 (\left| p \right| + \left| \xi \right| +
\,\left| \eta \right|)^q,\,\,\,j = 1,...,r_1 ;
\]
\[
Rez_j (p,y,\xi ,\eta ) \geqslant c_2 (\left| p \right| + \left| \xi \right| +
\,\left| \eta \right|)^q,\,\,\,j = r_1 + 1,...,k,
\]
с некоторыми константами $c_1 >0$
и $c_2 > 0$, не зависящими
от $p,\,\,y,\,\,\xi ,\,\,\eta $.

\textbf{Условие 4.} Число граничных условий (\ref{eq4602}) равно числу $z$ - корней
уравнения (\ref{eq4601}), лежащих в левой полуплоскости, и при всех $\xi \in R^{n -
1},\,\,\,\,\left| \xi \right| > 0$ многочлены
$$
	B_j^0 (\xi ,z) =
	\sum\limits_{\left| \tau \right| + qj_2 + rj_3 = m_j } {b_{\tau j_2 j_3 }
	p^{j_3 }} \xi ^\tau z^{j_2 }
$$
линейно независимы по модулю многочлена
$$
P(\xi ,z) = \prod\limits_{j_1 = 1}^{r_1 } {(z - z_{j_1 } (0,\xi ,0))} .
$$

Доказано следующее утверждение.

\textbf{Теорема 1.} Пусть $s \geqslant \max \{2m,\,\,\mathop {\max }\limits_{1 \leqslant
j \leqslant r_1 } m_j + q\}$ - действительное число и выполнены условия $1$, 2, 3,
4. Тогда для любого решения $v(x,t) \in H_{s,\alpha ,q} (R_ + ^n )\,\,\,$
задачи (20), (22) справедлива априорная оценка
\begin{multline*}
\left\| {v,\left| p \right|} \right\|_{s,\alpha ,q} \leqslant
\\ \leqslant
c(\left\| {F,\left|
p \right|} \right\|_{s - 2m,\alpha ,q} + \left\| {v,\left| p \right|}
\right\|_{s - 1,\alpha ,q} + \sum\limits_{j = 1}^{r_1 } {\left\| {G_j
,\left| p \right|} \right\|} _{s - m_j - \frac{1}{2}q}
)
\end{multline*}
с постоянной $c > 0$, не зависящей от
$p, v, F, G_j , j = 1,2,\ldots, r_1$.



Некоторые свойства весовых псевдодифференциальных операторов с символом из
класса $S_{\alpha ,p}^m (\Omega )$ содержатся в [4] - [8].

\litlist

1. {\bf Баев А. Д.} Вырождающиеся эллиптические уравнения высокого порядка и
связанные с ними псевдодифференциальные операторы / А. Д. Баев // Доклады
Академии наук. -- 1982. - Т. 265, № 5. - С. 1044 -- 1046.

2. {\bf Баев А.Д.} Об общих краевых задачах в полупространстве для вырождающихся
эллиптических уравнений высокого порядка /А.Д. Баев// Доклады Академии наук,
2008, т. 422, №6, с. 727 -- 728.

3. {\bf Баев А.Д.} Об одном классе псевдодифференциальных операторов с вырождением
/А.Д. Баев, Р.А. Ковалевский// Доклады академии наук. -- 2014. - T. 454.- №
1. - С. 7-10.

4. {\bf Баев А.Д.} Теоремы об ограниченности и композиции для одного класса весовых
псе\-в\-до\-диф\-фе\-ре\-н\-ци\-а\-ль\-ных операторов / А.Д. Баев, Р.А. Ковалевский// Вестник
Воронежского государственного университета Серия: Физика. Математика.--
2014, №1. -- С. 39- 49.

5. {\bf Баев А.Д.} Краевые задачи для одного класса вырождающихся
псевдодифференциальных уравнений /А.Д. Баев, Р.А. Ковалевский// Доклады
академии наук. -- 2015. - T. 461.- № 1. - С. 1-3.



Работа выполнена при финансовой поддержке Министерства образования и науки
РФ (проект 14.Z50.31.0037).

