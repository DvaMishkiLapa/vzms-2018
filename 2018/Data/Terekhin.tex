\vzmstitle{ \bf  СИСТЕМЫ СЖАТИЙ И СДВИГОВ В ФУНКЦИОНАЛЬНЫХ ПРОСТРАНСТВАХ}

\vzmsauthor{{Терехин}}{П.\, А.}

\vzmsinfo{Саратов; {\it terekhinpa@mail.ru}}

\addcontentsline{toc}{section}{Терехин П.А.\dotfill}

Пусть $f\in L^1[0,1]$ и $\int_0^1f(t)\,dt=0$. Системой сжатий и сдвигов функции $f$ называется последовательность
$$
f_n(t):=\begin{cases}
f(2^kt-j), & \text{если}\,\,\, t\in[\frac{j}{2^k},\frac{j+1}{2^k}], \\
0, & \text{иначе},
\end{cases}
$$
где $n=2^k+j$, $k=0,1,\ldots$ и $j=0,\ldots,2^k-1$. В частности, если $f=h=\chi_{(0,\frac12)}-\chi_{(\frac12,1)}$, то получаем
систему Хаара $\{h_n\}_{n=1}^{\infty}$, нормированную в $L^{\infty}$ (без функции $h_0(t)=1$).

Обозначим $T_f$ --- линейный оператор, заданный посредством равенств
$$
T_fh_n=f_n, \qquad n=1,2,\ldots.
$$

Под нормой $\|T_f\|_{X\to Y}$ будем понимать обычную норму оператора $T_f:\overline{\text{span}}_X\{h_n\}_{n=1}^{\infty}\to Y$.

Положим $\mathbb{A}:=\bigcup_{k=0}^{\infty}\{0,1\}^k$ --- множество всех конечных наборов, состоящих из нулей и единиц.
Если $\alpha=(\alpha_1,\ldots,\alpha_k)$ и $\beta=(\beta_1,\ldots,\beta_l)$, то обозначим
$\alpha\beta=(\alpha_1,\ldots,\alpha_k,\beta_1,\ldots,\beta_l)$ конкатенацию наборов $\alpha,\beta\in\mathbb{A}$.

Спектром Хаара функции $f$ называется множество
$$
S(f):=\{\beta\in\mathbb{A}:(f,h_{\beta})\neq0\}.
$$
Скажем, что функция $f$ имеет простой спектр Хаара, если из равенства $\alpha\beta=\alpha'\beta'$, где $\alpha,\alpha'\in\mathbb{A}$ и $\beta,\beta'\in S(f)$, следует, что $\alpha=\alpha'$ и $\beta=\beta'$.

\textbf{Лемма~1.} {\it Если $f\in L^2$, $f\neq0$, и $f$ имеет простой спектр Хаара, то система сжатий и сдвигов функции $f$ является ортогональной.
Следовательно, $\|T_f\|_{L^2\to L^2}=\|f\|_2$.}

Если $x\in L^1[0,1]$ и $x=\sum_{n=0}^{\infty}\xi_nh_n$, то функция
$$
Px(t)=\biggl(\sum_{n=0}^{\infty}|\xi_nh_n(t)|^2\biggr)^{\frac12}
$$
называется функцией Пэли. Пространство $P(L^{\infty})$ состоит из всех функций $x\in L^1[0,1]$, для которых $Px\in L^{\infty}$.
При этом $\|x\|_{P(L^{\infty})}=\|Px\|_{\infty}$.

Двоичное пространство $BMO_d$ состоит из всех функций $x\in L^1[0,1]$, для которых
$$
\sup\frac{1}{|I|}\int_I|x(t)-x_I|\,dt<\infty, \qquad x_I=\frac{1}{|I|}\int_Ix(s)\,ds,
$$
где супремум берётся по всем двоичным интервалам $I\subset[0,1]$. Если $\int_0^1x(t)\,dt=0$, то величина
$$
\|x\|_d=\sup\biggl(\frac{1}{|I|}\int_I|x(t)-x_I|^2\,dt\biggr)^{\frac12}
$$
определяет (эквивалентную) норму пространства $BMO_d$.

\textbf{Лемма~2.} {\it Если $f\in BMO_d$ и $f$ имеет простой спектр Хаара, то $\|T_f\|_{P(L^{\infty})\to BMO_d}=\|f\|_d$.}

Банахово пространство $X$ измеримых функций на $[0,1]$ называется симметричным, если
1) из неравенства $|x(t)|\leqslant|y(t)|$ для всех $t\in[0,1]$,
где функция $x$ измерима и $y\in X$, следует $x\in X$ и $\|x\|_X\leqslant\|y\|_X$;
2) из равенства
$$
|\{t\in[0,1]:|x(t)|>\tau\}|=|\{t\in[0,1]:|y(t)|>\tau\}|
$$
для всех $\tau>0$, т.е. из равноизмеримости $x$ и $y$, где функция $x$ измерима и $y\in X$,
следует $x\in X$ и $\|x\|_X=\|y\|_X$ (здесь $|A|$ --- мера Лебега множества $A\subset\mathbb{R}$).
Оператор растяжения $\sigma_{\tau}$, определяемый равенством
$$
\sigma_{\tau}x(t)=\begin{cases}
x(t/\tau), & 0\leqslant t\leqslant\min(1,\tau),\\
0, & \min(1,\tau)<t\leqslant1,
\end{cases}
\qquad \tau>0,
$$
ограничен в каждом симметричном пространстве $X$.

Нижний и верхний индексы Бойда симметричного пространства $X$ определяются соотношениями
$$
\alpha_X=\lim_{\tau\to0}\frac{\ln\|\sigma_{\tau}\|_{X\to X}}{\ln\tau}, \qquad
\beta_X=\lim_{\tau\to\infty}\frac{\ln\|\sigma_{\tau}\|_{X\to X}}{\ln\tau}.
$$
Всегда $0\leqslant\alpha_X\leqslant\beta_X\leqslant1$. Если $0<\alpha_X\leqslant\beta_X<1$, то говорят, что пространство $X$ имеет нетривиальные индексы Бойда.
Примерами симметричных пространств являются пространства $L^p$, Орлича $L_{\varPhi}$, Лоренца $\varLambda_{\varphi}$, Марцинкевича $M_{\varphi}$ и др. (см. [1]).

\textbf{Теорема 1.} {\it Пусть
$$
f=\sum_{i=1}^{\infty}f_i, \qquad \sum_{i=1}^{\infty}\|f_i\|_{BMO_d}<\infty,
$$
где каждая функция $f_i\in BMO_d$, $\int_0^1f_i(t)\,dt=0$, имеет простой спектр Хаара. Тогда оператор $T_f$ ограничен в каждом симметричном пространстве $X$ с нетривиальными индексами Бойда $0<\alpha_X\leqslant\beta_X<1$.}


\litlist

1. {\it Крейн С.Г., Петунин Ю.И., Семенов Е.М.} Интерполяция линейных операторов. М.: Наука, 1978. 400 с.

2. {\it Асташкин С.В., Терехин П.А.} Об ограниченности оператора, порождённого мультисдвигом Хаара // Доклады Академии наук. - 2017. -
Т. 476, № 2. - С.~133--135.

3. {\it Astashkin S.V., Terekhin P.A.} Sequences of dilations and translations in function spaces // J. Math. Anal. Appl. - 2018. -
V. 457, № 1. - P.~645--671.
