\begin{center}
{\bf Сферическое преобразование \\ обобщённого сдвига  Пуассона }\\
Л.Н.\,Ляхов \\
Воронежский государственный университет, Россия, \\
{\it levnlya@mail.ru} \\
Рощупкин\,С.А. \\
Елецкий государственный университет им. И.А. Бунина, Россия \\
 {\it roshupkinsa@mail.ru}\\
 Санина \,Е.Л. \\
Воронежский государственный университет, Россия \\
{\it sanina08@mail.ru}
\end{center}

%\begin{quote}
% {\small
Получена формула сферического преобразования интегральной операции от обобщённого сдвига функции, если функция зависит от многоосевой сферической симметрии. При этом формула показывает зависимость порядка обобщённого сдвига от размерности сферически симметричной части евклидова пространства.%}
 %\end{quote}

	Через
	$
	\left(^\gamma T^xf(y)\,\star\,g(y)\right)
	$
	будем обозначать одну или несколько арифметических операций (типа умножения, суммы, разности) между функциями $\left(^\gamma T^xf\right)(y)$\,и \,$g(y)$.

	 Рассмотрим случай, когда обобщённый сдвиг действует только по одной переменной

$$
(\,^\gamma T^{x}f)(y){=}(\,^\gamma T_{y_1}^{x_1}f)(y_1,y'{-}x')=$$
$$
=\frac{\Gamma\left(\gamma+1\over2\right)}{\Gamma\left(\frac{\gamma}{2}\right)\,\,\Gamma\left(\frac{1}{2}\right)}\int\limits_0^\pi
f(\sqrt{x_1^2{+}y^2_1{-}2x_1y_1\,\cos\alpha}\,,x'-y')\,\sin\alpha
d\alpha,
$$
где $x,\,\,y\in\overline{\mathbb{R}^+_m},\quad	\mathbb{R}^+_m{=}\{x{=}(x_1,x'){=}(x_1,x_2,\,\ldots\,,x_m),\,\,x_1>0\}\,.$

	{\bf Теорема 1.}  {\it $\lim_{\gamma\to0}\,\,^\gamma T^x f(y)=\,\,^0T^x f(y)={f(x+y)+f(x-y)\over2}$.}

{\bf Теорема 2.}  {\it    Пусть $\gamma{>}0$ и $\gamma=[\gamma]+\{\gamma\}$, где $[\gamma]$ и $\{\gamma\}$ --- соответственно целая и дробная части числа $\gamma$.
Положим $[\gamma]=m-1, \,\,\,m\geqslant1$ и $\{\gamma\}=\mu$. Пусть $f=f(|x|)$ и
$g=g(|x|)$ радиальные функции в $\mathbb{R}^+_m=\{x:\,\,x_1>0\}$ и существует весовой интеграл Лебега от операции $(\,\star\,)$:
$$
[f\,\star\,g]_\mu=\int\limits_{\mathbb{R}^+_m}\,\left(^\mu T^{x_1}_{y_1}f(\sqrt{y_1^2+
|x'-y'|^2}\,)\,\star\,g(|y|)\,\right)\,y_1^\mu\,dy<\infty.
$$ \\
Тогда справедлива формула
\begin{equation}\label{eq2}
\begin{array}{cccc}
[f\,\star\,g]_\mu=|S_1^+(m)|_\mu\, \left[f(r)\,\star\,g(r)\right]_{m+\mu-1} =\\
|S^+_1(m)|_\mu\int\limits_0^\infty \,\biggl(\,^\gamma T^\rho f(r)\,\star\,g(r)
\biggl)\,r^\gamma\,dr,
\end{array}
\end{equation}
где  $\rho=|x|\,, \,S^+_1(m){=}\{|y|{=}1\}^+{=}\{y:|y|{=}1,\,\,y_1>0\}$,

$
|S_1^+(m)|_\mu{=}\int\limits_{S_1^+(m)} y_1^\mu\,dS,$  $\gamma{=}m{-}1{+}\mu$}.

 При $\mu=\{\gamma\}=0$ (т.е. $\gamma=[\gamma]=m$ --- натуральное число >1) соотношение \eqref{eq2} надо понимать так:
 слева роль обобщённого сдвига выполняет обычный, а в правой части этой формулы операция $[\,\star\,]_\gamma$ выполняется для одномерных функций с обобщённым сдвигом Пуассона целого порядка $[\gamma]=m-1$. Соответствующее равенство для обобщённых свёрток известно   (см., например,  в [1] формулу (2.1)).

{\bf Литература}
1. Ляхов Л.\,Н. $B$ - гиперсингулярные интегралы и их приложения к описанию функциональных классов Киприянова и к интегральным уравнениям с $B$ - потенциальными ядрами,  {\it Липецк. гос. пед. ун-т}, 226 --- 232 (2007).

