\begin{center}
{\bf Абуд Ахмед Ханун\\ (Дагестанский государственный университет)\\
Задача с трехкратным корнем характеристического уравнения пучка
третьего порядка}
\end{center}
\addcontentsline{toc}{section}{Абуд А.\,Х.}

Рассматривается пучок третьего (нечетного) порядка с регулярными в
классическом понимании граничными условиями
$$l(y)\equiv\left(\frac d{dx}-\lambda\right)^3y(x),\hspace{10mm}0<x<1,\eqno{(1)}$$
$$
U_1(y)\equiv y(0)-y(1)=0,\,U_2(y)\equiv
y'(0)-y'(1)=0,
$$
$$
U_3(y)\equiv y''(0)-y''(1)=0. \eqno{(2)}
$$

Справедлива теорема о трехкратном разложении по собственным функциям
задачи (1)--(2).

{\bf Теорема.} Пусть $f_0(x)$, $f_1(x)$, $f_2(x)$--- трижды
непрерывно дифференцируемые на $(0,1)$ функции и
$\left.\frac{d^kf_s(x)}{dx^k}\right|_{k=0,1}=0$, $k=0,1$, $s=0,1,2$.

Тогда справедлива формула трехкратного разложения по собственным
элементам задачи (1)--(2):
$$\lim_{\begin{array}{c}\nu\to\infty\\ s=0,1,2\end{array}}\frac1{2\pi
i}\int_{C_\nu}\lambda^sd\lambda\int_0^1G(x,\xi,\lambda)F(\xi,f,\lambda)d\xi=f_s(x),\eqno{(3)}$$
где $F(\xi,f,\lambda)=-\lambda^2f_0(\xi)+3\lambda
f'_0(\xi)-3f''_0(\xi)-\lambda f_1(\xi)+3f''(\xi)-f_2(\xi)$,
$C_\nu$--- последовательность окружностей с центром в начале
$\lambda$ плоскости и радиусами $R_\nu=(2\nu+1)\pi$,
$\nu=1,2,3,\dots$, $G(x,\xi,\lambda)$--- функция Грина.

{\bf Литература.}

1. Абуд, А.Х. Спектральная задача с трехкратными корнями основного
характеристического уравнения дифференциального пучка третьего
порядка /А.Х. Абуд //Успехи современной науки. Т.1, № 2, 2016. С.
145-147.

2. Хромов, А.П. Разложение по собственным функциям обыкновенных
дифференциальных операторов с нерегулярными распадающимися краевыми
условиями /А.П. Хромов //Матем. сборник. Т. 70, № 3. 1966. С.
310-329.
