\begin{center}{ \bf ОБ ОСНОВНОМ ПЕРИОДЕ ПЕРИОДИЧЕСКОЙ ФУНКЦИИ НЕСКОЛЬКИХ ПЕРЕМЕННЫХ}\\
{\it Г.К. Соколова, С.С. Орлов } \\
(Иркутск; {\it 98gal@mail.ru, orlov{\_}sergey@inbox.ru} )
\end{center}
\addcontentsline{toc}{section}{Соколова Г.К.,. Орлов С.С\dotfill}

Математическое моделирование самоподобных объектов и их свойств, а также различных процессов, повторяющихся во времени и пространстве, естественным образом приводит к понятию  периодической функции нескольких переменных. Например, оно возникает при изучении  зонной структуры кристаллов [1], когда на волновую функцию
$\psi$, задающую их состояние, накладывают условия Борна--Кармана
$$
\psi(\bar{r}+N_{i}\bar{a}_{i})=\psi(\bar{r}),\,i=1,\ldots,d,
$$
где $d$~--- размерность решётки Брав{\'e},
$\bar{a}_{i}$~--- её элементарные трансляционные векторы,
$N_{i}$~--- целые числа.
Периодическая функция $\psi:\,{\mathbb R}^{3}\to{\mathbb R}$ имеет {\it основные} векторные периоды $\bar{a}_{i}$,
т.~е. векторные периоды наименьших модулей,
и любой её векторный период $\bar{T}$ определяется линейной комбинацией основных векторных периодов с целыми коэффициентами
как вектор трансляций.
Эти факты следуют из структуры множества, на котором рассматривается функция $\psi$,
нежели из самого нелокального свойства периодичности.

Известно, что у произвольной периодической функции $f:\,{\mathbb R}^{n}\to{\mathbb R}$ не всегда существует основной период. Этот факт имеет место даже в одномерном случае, когда $n=1$. Примеры таких функций  доставляют  постоянная функция, у которой любое действительное число является периодом, характеристическая функция множества ${\mathbb Q}$ рациональных чисел или функция Дирихле, множество периодов которой совпадает с ${\mathbb Q}$, и многие другие. Критериев существования у периодической функции $f:\,{\mathbb R}\to {\mathbb R}$ основного периода авторам неизвестно. Достаточные условия даёт теорема, которая приведена в книгах [2, с.~8] и [3, с.~450].

\textbf{Теорема~1.} {\it Если периодическая функция $f:\,{\mathbb R}\to {\mathbb R}$ является непрерывной и отлична от постоянной, то она имеет основной период.}

В представляемой работе аналогичная теорема доказана для функции  $f:\,{\mathbb R}^{n}\to {\mathbb R}$, определенной всюду на ${\mathbb R}^{n}$.

\textbf{Определение~1.} {\it Функция $f:\,{\mathbb R}^{n}\to {\mathbb R}$ называется периодической с периодом $\bar{T}$, если существует вектор $\bar{T}\neq\bar{0}$, что для всех $\bar{r}\in{\mathbb R}^{n}$ выполняется $f(\bar{r}+\bar{T})=f(\bar{r})$.}

Из определения~1 следует, что, если $\bar{T}$~--- период функции $f$, то для любого  $k\in{\mathbb Z}\setminus\lbrace 0\rbrace$ вектор $k\cdot\bar{T}$ также является периодом этой функции.

Рассмотрим далее множество $n$-мерных прямых $\ell_{\bar{T}}(\bar{a})$ с направляющим вектором $\bar{T}$. Здесь $\bar{a}\in{\mathbb R}^{n}$~---  радиус-вектор некоторой точки, принадлежащей данной прямой $\ell_{\bar{T}}(\bar{a})$. Эту точку можно выбирать, например, в линейном многообразии $\langle\bar{r},\,\bar{T}\rangle=0$, тогда соответствие $\bar{a}\to\ell_{\bar{T}}(\bar{a})$ оказывается взаимно однозначным. Параметрическое уравнение прямой $\ell_{\bar{T}}(\bar{a})$ имеет вид  $\bar{r}=\bar{a}+t\bar{{\cal T}}$, в котором $t$~--- действительная переменная, $\bar{T}=\vert\bar{T}\vert\cdot\bar{{\cal T}}$. Вдоль каждой прямой функция $f:\,{\mathbb R}^{n}\to {\mathbb R}$ принимает значения $\bigl.f(\bar{r})\bigr|_{\bar{r}\in\ell_{\bar{T}}(\bar{a})}=f(\bar{a}+t\bar{{\cal T}})$, т.~е. является функцией $g_{\bar{a}}(t)=f(\bar{a}+t\bar{{\cal T}})$ одной переменной.

\textbf{Лемма.} {\it Всякая функция $f:\,{\mathbb R}^{n}\to {\mathbb R}$,  периодическая с периодом $\bar{T}$, является периодической с периодом $\vert\bar{T}\vert$ вдоль каждой прямой $\ell_{\bar{T}}(\bar{a})$ с направляющим вектором $\bar{T}$.}

Для доказательства следует показать периодичность с периодом $\vert\bar{T}\vert$  функций $g_{\bar{a}}:\,{\mathbb R}\to{\mathbb R}$.

\textbf{Определение~2.} {\it Пусть $f:\,{\mathbb R}^{n}\to {\mathbb R}$  периодическая с периодом $\bar{T}$. Период $\bar{T}_{0}$ наименьшего модуля, коллинеарный $\bar{T}$, называется основным (базисным) периодом функции $f$ в данном направлении $\bar{{\cal T}}$.}

Рассмотрим некоторые примеры. Функция $f:\,{\mathbb R}^{2}\to{\mathbb R}$ вида $(x,\,y)\to x\sin y+(x+1)\sin 2y$ является периодической в направлении орта $\bar{j}$ с основным периодом $\bar{T}_{0}\lbrace0;\,2\pi\rbrace$. Вдоль прямых $(x,\,y)=(a,\,t)$ данная функция имеет следующий вид: $g_{\lbrace a;\,0\rbrace}(t)=a\sin t+(a+1)\sin 2t$, и при $a=-1$ является $\pi$-периодической, а при всех $a\neq-1$~--- $2\pi$-периодической. Таким образом, вдоль разных прямых $\ell_{\bar{T}}(\bar{a})$ периодическая функция $f:\,{\mathbb R}^{n}\to{\mathbb R}$ может иметь различные основные периоды, в т.~ч. меньшие модуля $\vert\bar{T}_{0}\vert$ её базисного векторного периода. Очевидно, что вдоль некоторой прямой функция $f:\,{\mathbb R}^{n}\to{\mathbb R}$ может и совсем не иметь основного периода. Например, $f:\,{\mathbb R}^{2}\to{\mathbb R}$ вида $(x,\,y)\to y\sin x$, периодическая в направлении орта $\bar{i}$ с основным периодом $\bar{T}_{0}\lbrace 2\pi;\,0\rbrace$, но вдоль прямой $(x,\,y)=(t,\,0)$ она постоянна: $g_{\lbrace 0;\,0\rbrace}(t)=0$. Пример отсутствия базисных периодов доставляет функция $f:\,{\mathbb R}^{2}\to{\mathbb R}$ вида $(x,\,y)\to(x-y)\sin(x-y)$. Множество её периодов континуально и состоит из векторов $\bar{T}\lbrace a;\,a\rbrace$, где $a\in{\mathbb R}$, среди которых нет вектора наименьшей длины.

\textbf{Теорема~2.} {\it Если периодическая с периодом $\bar{T}$ функция $f:\,{\mathbb R}^{n}\to {\mathbb R}$ вдоль хотя бы одной прямой $\ell_{\bar{T}}(\bar{a})$ непрерывна и отлична от постоянной, то она имеет основной период в данном направлении $\bar{{\cal T}}$.}

Предполагая, что периодическая функция $f:\,{\mathbb R}^{n}\to {\mathbb R}$ не имеет основного периода в направлении $\bar{{\cal T}}$, заключаем существование её периода $\bar{S}$, модуль которого как угодно мал. Согласно лемме, $\vert\bar{S}\vert$~--- период функций $g_{\bar{a}}:\,{\mathbb R}\to {\mathbb R}$, при этом хотя бы одна из них по теореме~1 имеет основной период, что противоречит малости $\vert\bar{S}\vert$.

\smallskip \centerline{\bf Литература}\nopagebreak

1. {\it Ашкрофт Н., Мермин Н.} Физика твёрдого тела. Том~1. М.: Мир, 1979.~-- 400~с.

2. {\it Ахиезер Н.И.} Элементы теории эллиптических функций. М.:  Наука, 1970.~-- 304~с.

3. {\it Будак Б.М., Фомин С.В.} Курс высшей математики и математической физики. Кратные интегралы и ряды. М.: Наука, 1965.~-- 608~с.
