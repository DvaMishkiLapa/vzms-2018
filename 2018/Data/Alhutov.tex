\vzmstitle{
	Поведение в граничной точке решения
	задачи Дирихле для p(x)--Лапласиана
}
\vzmsauthor{Алхутов}{Ю.\,А.}
\vzmsinfo{Россия, г. Владимир, ВлГУ им. А.Г. и Н.Г. Столетовых, \textit{yurij-alkhutov@yandex.ru}}

\vzmsauthor{Сурначёв}{М.\,Д.}
\vzmsinfo{
	Россия,
	г. Москва, ИПМ им. М.В. Келдыша РАН, \textit{peitsche@yandex.ru}
}

\vzmsinfo{доклад будет представлен в двух пленарных сообщениях}

\vzmscaption


Рассмотрим в ограниченной области $\Omega$ евклидова пространства $\mathbb{R}^n$, $n\geqslant 2$, уравнение
$$
Lu=\mathrm{div}\, \left(|\nabla u|^{p(x)-2}\nabla u \right)=0 \eqno{(1)}
$$
с измеримым в $\mathbb{R}^n$ показателем $p(x)$, удовлетворяющим ус\-ло\-вию
$$
1<\alpha\leqslant p(x)\leqslant \beta<\infty, \quad \text{для почти всех}\quad x\in \mathbb{R}^n %\eqno{(2)}
$$

Введём класс функций
$$
W(\Omega)=\left\{u\,: \, u\in W^{1,1}(\Omega),\, |\nabla u|^{p(x)}\in L^1(\Omega) \right\},
$$
где $W^{1,1}(\Omega)$ --- соболевское пространство функций, суммируемых в $\Omega$ вместе с обобщёнными производными первого порядка.
Будем говорить, что функция $u\in W_0(\Omega)$, если $u\in W(\Omega)$, и существует последовательность функций $u_j \in W(\Omega)$ с компактным носителем в $\Omega$, такая, что
$$
\lim_{j\to \infty} \int_{\Omega} |\nabla u_j-\nabla u|^{p(x)}\, dx =0.
$$
Скажем, что функция $u\in W(\Omega)$ является решением уравнения (1) в $\Omega$, если интегральное тождество
$$
\int_\Omega |\nabla u|^{p(x)-2} \nabla u\cdot\nabla \varphi\, dx=0 %\eqno{(3)}
$$
выполнено для любой пробной функции $\varphi \in W_0(\Omega)$.

Для простоты сделаем дополнительное предположение: для любой $u\in W(\Omega)$ существует последовательность бесконечно дифференцируемых в $\Omega$ функций $u_j$, такая, что для любой подобласти $\Omega'\Subset \Omega$ выполняется $u_j\to u$ в $L^1(\Omega')$ и
$$
\lim_{j\to \infty} \int_{\Omega'} |\nabla u_j-\nabla u|^{p(x)}\, dx =0.
$$

Рассмотрим задачу Дирихле
$$
Lu=0 \quad \text{в}\quad \Omega, \quad u\in W(\Omega), \quad h\in W(\Omega),\quad (u-h)\in W_0(\Omega). %\eqno{(4)}
$$
Решение данной задачи совпадает с минимизантом решения вариационной задачи
$$
\min _{w\in W_0(\Omega) } F(w+h), \quad F(u)=\int_\Omega \frac{|\nabla u|^{p(x)}}{p(x)}\, dx.%,\quad h\in W(\Omega).
$$
Настоящее сообщение посвящено граничным свойствам решения задачи Дирихле
$$
Lu_f=0 \quad \text{в}\quad \Omega, \quad u_f=f\quad \text{на}\quad \partial \Omega \eqno{(2)}
$$
с непрерывной на $\partial \Omega$ функцией $f$, в точке границы $x_0\in \partial \Omega$ в предположении, что в этой точке выполнено условие
$$
|p(x)-p(x_0)|\leqslant \frac{C}{ \ln \frac{1}{|x-x_0|}},\quad \text{при} \quad |x-x_0|\leqslant \frac{1}{2}. %\eqno{(6)}
$$

Решение задачи (3) определяется следующим образом. Продолжим граничную функция функцию $f\in C(\partial \Omega)$ по непрерывности на $\overline{\Omega}$, сохранив за продолжением то же обозначение. Возьмём последовательность бесконечно дифференцируемых в $\mathbb{R}^n$ функций $f_k$, которые равномерно на $\overline{\Omega}$ сходятся к $f$. Решим задачи Дирихле
$$
Lu_k=0 \quad\text{в}\quad \Omega, \quad u_k \in W(\Omega), \quad (u_k-f_k)\in W_0(\Omega).
$$
Последовательность $u_k$ сходится равномерно  на компактных подмножествах $\Omega$ к функции $u\in W(\Omega')$ для всех $\Omega'\Subset \Omega$, которая удовлетворяет интегральному тождеству (2) на пробных функциях $\varphi \in W(\Omega)$ с компактным носителем в $\Omega$. Предельная функция не зависит от способа продолжения и аппроксимации граничной функции $f$ и называется обобщённым решением задачи Дирихле (2).

Граничная точка $x_0\in \partial \Omega$ называется регулярной, если
$$
\mathop{\mathrm{ess\,lim}}\limits_{\Omega \ni x\to x_0} u_f(x)= f(x_0)
$$
для любой непрерывной на $\partial \Omega$ функции $f$.

Для формулировки результата введём понятие ёмкости. Ёмкостью компакта $K\subset B$ относительно относительно шара $B\subset \mathbb{R}^n$ назовём число
$$
C_p (K, B)=\inf \left\{ \int_B \frac{|\nabla \varphi|^{p(x)}}{p(x)}\,dx:\ \varphi\in C_0^\infty(B),\ \,\varphi\geqslant 1\ \text{на}\ K \right\}.
$$
Далее, $B(x_0,r)$ --- шар в $\mathbb{R}^n$ с центром в точке $x_0$ радиуса $r$, $\overline{B(x_0,r)}$ --- его замыкание.

{\bf Теорема.} {\it Для регулярности граничной точки $x_0\in \partial \Omega$ необходимо и достаточно выполнения условия
$$
\int_0 \left( \frac{C_p \left(\overline{B(x_0,r)}\setminus \Omega,\, B(x_0,2r) \right)}{r^{n-p(x_0)}} \right)^\frac{1}{p(x_0)-1}\frac{dr}{r}=\infty.
$$
}

Если $p(x)=\mathrm{const}=2$, то это утверждение является классическим результатом Н.\,Винера~[1].
Случай
%\linebreak
$p=$
\linebreak
$= \mathrm{const}\neq2$ доказан В.Г. Мазьёй [2], позже доказательство было распространено на более общие уравнения Р. Гарипи и В. Цимером [3]. В случае, когда показатель $p(x)$ обладает логарифмическим модулем непрерывности в замыкании области, утверждение теоремы  доказано в работе Ю.А. Алхутова и О.В. Крашенинниковой [4].

Работа выполнена при поддержке РФФИ, проект № 15-01-00471. %Также  работа поддержана Министерством образования РФ (задание № 1.3270.2017/ПЧ).

\litlist

\selectlanguage{english}
1. Wiener N. Certain notions in potential theory // J. Math. Phys. 1924. V. 3. P. 24--51.

\selectlanguage{russian}

2. Мазья В. Г. О непрерывности в граничной точке решений квазилинейных эллиптических
уравнений // Вестн. ЛГУ. Сер. матем. 1970. Т. 25. №13. С. 42--55.

\selectlanguage{english}

3. Gariepy R., Ziemer W.P. A regularity condition at the boundary for solutions of quasilinear
elliptic equations // Arch. Rational Mech. Anal. 1977. V. 67. P. 25--39.

\selectlanguage{russian}

4. Алхутов Ю.\,А., Крашенинникова О.\,В.
Непрерывность в граничных точках решений квазилинейных эл\-ли\-п\-ти\-че\-с\-ких уравнений с нестандартным условием роста
//
Изв.
\linebreak
РАН. Сер. матем. 2004. Т. 68. № 6. С. 3–60.
