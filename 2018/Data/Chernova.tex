
\begin{center}{ \bf  О ЗАДАЧЕ РИМАНА--ГИЛЬБЕРТА ДЛЯ ЭЛЛИПТИЧЕСКИХ СИСТЕМ}\\
{\it О. В. Чернова } \\
(Белгород; {\it volga@mail.ru} )
\end{center}
\addcontentsline{toc}{section}{Чернова О. В.\dotfill}


В конечной области $D \in \mathbb{C}$, ограниченной гладким контуром $\Gamma$,
рассматривается эллиптическая система первого порядка
$$
A_1\frac{\partial U}{\partial x}+A_2\frac{\partial
U}{\partial x}+a(z)U(z)+b(z)\overline{U(z)}=F(z),~ z\in D, \eqno(1)
$$
где  постоянные матрицы $A_1, A_2 \in \mathbb{C}^{l\times l} \in C^{\mu}(D)$ и $l\times l-$ матричные коэффициенты
$a,b \in C^{\mu}(D), 0 < \mu < 1$. Пусть $l_1\; (l_2)$ -- число корней характеристического уравнения этой системы (взятое с учетом кратности)  в верхней (нижней) полуплоскости, так что $l=l_1+l_2$. Очевидно, эти числа совпадают с соответствующими числами собственных значений матрицы $A=-A_2^{-1}A_1$.

Для системы (1) ставится  краевая задача Римана -- Гильберта
$$
\mathrm{Re}\, CU^+=f, \eqno(2)
$$
где комплексная  $l\times l-$матрица--функция $C\in C^\mu(\Gamma)$ и $+$ означает граничное значение функции $U$ на $\Gamma$. Эта задача
рассматривается  в классе классических решений
$U\in C^\mu(\overline{D})\cap C^1(D)$ системы (1), для которых
$$
A_1\frac{\partial U}{\partial x}+A_2\frac{\partial
U}{\partial x}\in C^\mu(\overline{D}).\eqno(3)
$$


\textbf{Теорема~1.} {\it Пусть контур $\Gamma\in C^{1,\nu}$ и матрица $C\in C^\nu(\Gamma)$, $\;\mu<\nu<1$. Пусть обратимая матрица $B\in \mathbb{C}^{l\times l}$, записанная в блочном виде
$(B_1,\overline{B}_2)$ с некоторыми  $B_k\in \mathbb{C}^{l\times l_k}$
такова, что $B^{-1}AB=\mathrm{diag}\,(J_1,\overline{J}_2)$, где собственные значения матриц $J_k\in \mathbb{C}^{l_k\times l_k}, \;k=1,2,$ лежат в верхней полуплоскости.

Тогда условие обратимости матрицы--функции $G$, записанной в блочном виде $(CB_1,\overline{C}B_2),$ необходимо и достаточно для фредгольмовости задачи
(1), (2) в рассматриваемом классе (3)
и ее индекс дается формулой
$$
\ae= -\sum_{j=1}^m\frac{1}{\pi} [\arg\det G]_{\Gamma_j}+(2-m)l,
$$
где $\Gamma_1,\ldots,\Gamma_m$ -- простые контура, составляющие $\Gamma$, и приращение
$[\;]_{\Gamma_j}$ вдоль $\Gamma_j$ берется в направлении, оставляющем область $D$ слева.

Если дополнительно $C\in C^{1,\nu}(\Gamma)$, то любое решение  $U\in C^\mu(\overline{D})\cap C^1(D)$ задачи  из класса (3)
с правой частью $f\in C^{1,\mu}(\Gamma)$ в действительности принадлежит $C^{1,\mu}(\overline{D})$.}


В случае простого контура $\Gamma$  задача (1)--(2) изучена в [1].

\smallskip \centerline{\bf Литература}\nopagebreak

1. {\it Солдатов А.П., Чернова О.В.} Задача Римана- Гильберта для эллиптической системы первого порядка
в классах Гельдера, Научные ведомости БелГУ, 2009, № 13(68, вып. 17/2, С. 115- 121.
