\vzmstitle{Оптимизация
тригонометрического импульса по коэффициенту асимметрии}

\vzmsauthor{{Костин}}{Д.\, В.}

\vzmsinfo{Воронеж; {\it dvk605@mail.ru}}

\vzmscaption


 В докладе будет изложена допускающая алгоритмизацию методика приближенного вычисления и оптимизации по коэффициенту асимметрии бифурцирующего резонансного колебания. В качестве общего модельного уравнения рассмотрено скалярное ОДУ высокого порядка. Изложена  процедура оптимизации (в силу коэффициента асимметрии) полигармонического импульса, являющегося главной асимптотической частью (по ведущим амплитудам) бифурцирующего резонансного периодического решения. Вычислены оптимальные значения базисных амплитуд. Приведено графическое изображение примера оптимального полигармонического импульсов. В примерах, с использованием полученных результатов, предложено решение задачи построения теоретически допустимой диаграммы направленности антенны, отличной от интерполяционного многочлена.


Рассмотрено скалярное ОДУ порядка
$2n\,, \ n\in \mathbb{N}$
{\small
\begin{multline}\label{DS-e0}
\frac{d^{2n}w}{dt^{2n}}+ a_n\frac{d^{2(n-1)}w}{dt^{2(n-1)}}+ \dots +
a_2\frac{d^2w}{dt^2}+a_1w +
\\+
U \left(w,\frac{dw}{dt},
\frac{d^2w}{dt^2},\, \dots\,, \frac{d^{2n-1}w}{dt^{2n-1}}\right)
= 0,
\end{multline}}
$U(w,w_1, w_2,\, \dots\,, w_{2n-1})=O(w^2 + w_1^2+\, \dots\,
+w_{2n-1}^2)$, и изложена процедура оптимизации по коэффициенту
асимметрии полигармонического импульса, являющегося главной
асимптотической частью бифурцирующего периодического решения этого
уравнения в условиях $n$-кратного резонанса.

Под $n$-кратным резонансом (типа $p_1:p_2: \dots \,:p_n$) уравнения
(\ref{DS-e0}) подразумевается случай одновременного существования
(для соответствующего линеаризованного ОДУ) набора из $n$
периодических решений $\exp(\frac{2\pi\; i \;p_k}{T}\;t)$, \ $T >
0,$ $p_k\in \mathbb{Z},$ \ $k=1,2,\dots ,n$ \ $1\leqslant p_1 \leqslant p_2
\leqslant\dots \leqslant p_n,$ \ $HOD(p_1,p_2, \dots , p_n)=1$. Резонанс
$p_1:p_2:\dots , p_n$ называется сильным, если существует такой
ненулевой набор целых чисел $m_1,m_2,\dots , m_n$, что \
$m_1p_1+m_2p_2+ \dots + m_np_n=0$ \ и \ $|m_1|+|m_2|+ \dots +
|m_n|\leqslant 4.$ \ Число $|m_1|+|m_2|+ \dots + |m_n|$ называется
порядком резонансного соотношения. Число, наименьшее из порядков
резонансных соотношений, называется порядком данного резонанса.
Резонансные соотношения порядка $\geqslant 5$ называются слабыми.
Резонанс, для которого существует сильное резонансное соотношение,
называется сильным, и слабым --- в противном случае.

%Напомним, что под простым резонансом (типа $p_1:p_2$, \
%$HOD(p_1,p_2)=1$, \ $1\le p_1 \le p_2,$) уравнения (\ref{DS-e0})
%подразумевается случай одновременного существования (для
%линеаризованного ОДУ) двух периодических решений $\exp(\frac{2\pi\;
%i \;p_k}{T}\;t)$, \ $T > 0,$ $p_k\in \mathbb{Z},$ \ $k=1,2,$.

Ниже предполагается, что $T=2\pi$ \ и \ $1\leqslant p_1 < p_2<\dots <
p_n$. Базовое предположение --- условие потенциальности, означающее,
что уравнение (\ref{DS-e0}) служит уравнением Эйлера-Пуассона
экстремалей функционала
 $$
V(w,a)=\frac1{2\pi}\int\limits_0^{2\pi}\left( \frac12\left
((-1)^n\,\left(\frac{d^n w}{dt^n}\right)^2 + (-1)^{n-1}\,
a_n\left(\frac{d^{n-1} w}{dt^{n-1}}\right)^2 + \right.\right.
 $$
\begin{equation}\label{DS-e1}
\dots \left.\left. - a_2 \left(\frac{d w}{dt}\right)^2 + a_1\;
w^2\right) + \mathcal{U}\right)\,dt,
\end{equation}
$\mathcal{U}=\mathcal{U}\left(w,\frac{dw}{dt}, \ \frac{d^2w}{dt^2},
\ \dots \ \frac{d^{n-1} w}{dt^{n-1}}\right)$, при этом
$$
\mathcal{U}\;(w,w_1, w_2,\dots , w_{n-1}) = o(w^2+ w_1^2+ w_2^2+
\dots + w_{n-1}^2).
$$

Функционал $V$ рассмотрен на пространстве $E$
$2\pi-$пе\-ри\-о\-ди\-чес\-ких вещественнозначных  функций класса
$C^{2n}$.

Одна из центральных конструкций статьи --- редукция вычисления
решения к задаче о бифуркации критических точек полинома от $2n$
переменных, обладающего круговой симметрией [1]. Представленная методика позволяет сводить анализ
бифуркации циклов к анализу ключевой бифуркации критических точек
функции
\begin{equation}\label{DS-7.1}
W(\xi,a) = \inf_{w: \ \langle w,e_k \rangle =\xi_k} V(w,a) =
V\left(\sum\limits^{2n}_{i=1} \xi_i\,e_i + \Phi( \xi,a),a\right)\,,
\end{equation}
зависящей от ключевых переменных \ $\xi_1,\xi_2,\dots ,\xi_{2n}$\,
%\ \ $\xi_2$ --- коэффициент фурье
: \ $\xi_j = \langle
w,e_j\rangle$\,, $e_1,e_2,\dots ,e_{2n}$
--- моды бифуркации, $\langle \cdot,\cdot \rangle$ --- скалярное
произведение в пространстве $ L_2[0,2\pi]$.

Вычисление критической точки $\overline{\xi}(\delta) =
(\overline{\xi}_1(\delta),
\overline{\xi}_2(\delta),\dots,
$\linebreak $
\overline{\xi}_{2n}(\delta))$ ключевой функции приводит к
асимптотической формуле для ветви бифурцирующей экстремали
\begin{equation}\label{asimpt}
w = f(t,\delta) + o(\delta) \, , \ \ \ \  f(t,\delta):=
\sum\limits^{2n}_{i=1} \overline{\xi}_i(\delta)\,e_i.
\end{equation}

При изучении бифуркаций, связанных с проектированием и созданием
некоторых технических устройств, возникает дополнительная задача
оптимизации бифурцирующей ветви --- подбора таких значений $\delta$,
при которых заданный критерий качества принимает максимальное
значение. Например, в задаче отыскания многомодового закритического
прогиба упругой системы возникает вопрос о минимальной величине
относительного прогиба [2], а в задаче
повышения эффективности зубчатой передачи ставится вопрос
максимального увеличения коэффициента асимметрии силового импульса
на выходе [3]. Аналогичные вопросы возникают при
оптимизации антенных устройств, в некоторых задачах нелинейной
оптики и других задачах современной физики. Ниже
рассмотрена задача отыскания таких значений $\delta$, при которых
достигает своего максимума коэффициент асимметрии
  \begin{equation}\label{KN}
K := \frac{f_{max}}{|f_{min}|}, \ \ \ \ \ \ \mathop{\textrm{где}} \
\ \ f_{max} \ \ := \max\limits_t\,f(t,\delta), \ \  f_{min} :=
\min\limits_t\,f(t,\delta).
  \end{equation}

В общем случае математической моделью полигармонического импульса
служит тригонометрический полином
  \begin{equation}\label{imp-gen}
\sum\limits_{k=1}^n\,f_k\cos(kt+\omega_k).
   \end{equation}
Несложные рассуждения приводят к выводу, что для оптимального по
коэффициенту асимметрии полинома можно положить
$\omega_1=\omega_2=\dots = \omega_n = 0$, что позволяет считать
основной моделью полигармонического импульса полином

  \begin{equation}\label{imp}
f_n(t)=\sum\limits_{k=1}^n\,f_k\cos(kt).
   \end{equation}
Ниже в основу анализа этого полинома положено понятие <<минимальный
страт Максвелла>> [4], под которым
подразумевается модмножество многочленов с максимально возможным
количеством минимумов при условии, что все минимумы расположены на
одном уровне (значение многочлена во всех точках минимума одно и то
же). Многочлен (\ref{imp}) при выполнении последнего условия
называется максвелловским.

Оптимальный набор значений коэффициентов $f_k$ для (произвольного
$n$) первоначально был найден экспериментально. Затем появилось
доказательство единственности оптимума для многочлена (\ref{imp}) с
максимально возможным количеством минимумов на отрезке $[0,\pi]$). И была найдена для
произвольного $n$ общая формула (связанная с ядром Фейера
[4]) масквелловского многочлена (с коэффициентом
асимметрии, равным $n$). Таким образом, была доказана

{\bf Теорема} {\it
Многочлен (\ref{imp}) является оптимальным тогда и только тогда, когда он с точностью до постоянного множителя имеет вид суммы Фейера
$$
f_n(t)=\sum\limits_{k=1}^n (n+1-k) \cos(kt).
$$
При этом имеет место равенство
$$
\max\limits_{\lambda} K_n(\lambda) = n.
$$}

Полученные формулы оптимального многочлена дают также прямое
объяснение феномену симметричного расположения точек минимума
оптимального многочлена (обнаруженному авторами ещё в стадии
компьютерных экспериментов).



\litlist

1.  Б.\,М.~Даринский, Ю.\,И.~Сапронов, С.\,Л.~Царев {\it Бифуркации экстремалей фредгольмовых функционалов}
//
Современная
%\linebreak
математика. Фундаментальные направления Т. 12 Москва, МАИ, 2004


2. Д.\,В.~Костин {\it Анализ изгибов упругой лопатки турбины с учётом
неоднородности материала},  Насосы турбины системы, 2013, Т.3 (8), С. 56-61

3. В.А. Костин, Д.В. Костин, Ю.И. Сапронов {\it Многочлены Максвелла-Фейера и оптимизация
полигармонических импульсов},  ДАН, 2012,  445:3 ,С.271--273

4. Т.~Постон, И.~Стюарт {\it Теория  катастроф  и  её  приложения} /  М., Мир, 1980

5. Г.~Сегё {\it Ортогональные многочлены} / М., Физматгиз, 1962
