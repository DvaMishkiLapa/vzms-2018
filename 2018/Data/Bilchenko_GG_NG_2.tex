\vzmstitle{
	ЭКСТРЕМАЛЬНЫЕ  И  НЕЭКСТРЕМАЛЬНЫЕ
	ОБРАТНЫЕ  ЗАДАЧИ  НА  ФРАГМЕНТАХ
}


\vzmsauthor{Бильченко}{Г.\,Г.}
\vzmsauthor{Бильченко}{Н.\,Г.}

\vzmsinfo{Казань; {\it ggbil2@gmail.com, bilchnat@gmail.com}}

\vzmscaption


%%  {Экстремальные  и  неэкстремальные
%%   обратные  задачи  на  фрагментах}
%%% ===========================================================



В  работе  используются
обозначения  и  сокращения  из
[1].
%%% ===========================================================


\textbf{1.}\;%
Пусть
(как  в  [2])
в  сетке
$X_{1}$
выделено  $r$
\textit{фрагментов}:
%% =============================== \\
\[
X_{1,\ell}
=\left(x_{j}^{\wedge}\right)_{j=i_{\ell-1},\ldots,i_{\ell}}
\;\;
\text{для}
\;\;
\ell=1,\ldots,r;
\;\;
{i_{0}} =0,
\;\;
{i_{r}} =n_{1}
\,{.}
\eqno(1)
\]
%% =============================== //
Для  каждого
$\ell=1,\ldots,r\,$
в  зависимости  от  того,
какие  два  из  четырёх  параметров
$m$,  $\tau$,  $q$,  $f$~{\textbf{--}}
свободны  (``0''),
а  какие~{\textbf{--}}
заданы  (``1''),
реализуются
[2]
перечисленные  в  табл.\,1
\textit{неэкстремальные
задачи  на  фрагменте}
$X_{1,\ell}\,$,
где
$\text{ОЗ}_{\tau}^{q}$
и
$\text{ОЗ}_{\tau}^{f}$~{\textbf{--}}
обратные  задачи  по
$\tau$
[2, 3].
%%%
Варианты  сочетаний
$\text{ПЗ}$
и
$\text{ОЗ}$
на  фрагментах
(\textit{смешанные   задачи})
описаны  в
[2].


%% ============================================= \
{\begin{table}[h]\begin{center}
\hfill\text{Таблица~1}\qquad\,

\begin{tabular}
{|  c  || c || c | c | c | c | c || c |}
%%% =====================================
\hline
%%
& $\text{ПЗ}$
& $\text{ОЗ}_{m}^{q}$
& $\text{ОЗ}_{m}^{f}$
& $\text{ОЗ}_{\tau}^{q}$
& $\text{ОЗ}_{\tau}^{f}$
&  $\rule{0pt}{4.5mm}%
\rule[-2.1mm]{0pt}{0pt}$%
$\text{ОЗ}_{\left(m,\tau\right)}^{\left(q,f\right)}$
&  \\
%%% =====================================
%\hline
\hhline{|=#=#=|=|=|=|=#=|}
%%
$q$    & 0  & 1 & 0  & 1 & 0  & 1
& $\rule{0pt}{4.0mm}%
\rule[-1.5mm]{0pt}{0pt}$%
$\delta_{\ell}^{q}$ \\
%%% =====================================
\hline
%%
$f$    & 0  & 0  & 1 & 0 & 1  & 1
& $\rule{0pt}{4.3mm}%
\rule[-1.5mm]{0pt}{0pt}$%
$\delta_{\ell}^{f}$ \\
%%% =====================================
%\hline
\hhline{|=#=#=|=|=|=|=#=|}
%%
$m$    & 1  & 0  & 0  & 1 & 1 & 0 & \\
\hline
$\tau$ & 1  & 1  & 1 & 0  & 0 & 0 & \\
%%% =====================================
\hline
\end{tabular}\end{center}\end{table}}
%% ============================================= /



Для  указанных
$\text{ОЗ}$
на
$X_{1}^{\left(r\right)}
\!\!=\!
\left(X_{1,\ell}\right)_{\ell=1,\ldots,r}$
введём
%% =============================== \\
\begingroup\belowdisplayskip=\belowdisplayshortskip
\[
\hfill
R_{\infty}
 \!
 \left ( \!
    \left (q^{\sim}
    \!{,}\:\!
    f^{\sim}\;\!\!\right )
    \!{;}\!
    \left (q^{\vee}
    \!{,}\:\!
    f^{\vee}\;\!\! \right )\!
  \right )
  \!\;\!\!=\;\!\!\!\!\!
\max \limits_{\ell=1,\ldots,r
}\!\:\!\!\max
\!\;\!\!\left \{
\!
\delta_{\ell}^{q} R_{\infty}
  \!
  \left ( q^{\sim}
  \:\!\!{;}\:\!
  q^{\vee}\;\!\! \right )\!
{,}\;\!
\delta_{\ell}^{f} R_{\infty}
 \!
 \left (
  f^{\sim}
  \:\!\!{;}\:\!
  f^{\vee}\;\!\! \right )
\!
\right \}
\!\!{,}\;\!
\hfill
\text{(2)}
\]
\endgroup
%% =============================== //
%% =============================== \\
\[
R_{p}
\!
\left ( \!
    \left (q^{\sim}
    \!{,}\:\!
    f^{\sim}\;\!\!\right )
    \!{;}\!
    \left (q^{\vee}
    \!{,}\:\!
    f^{\vee}\;\!\!\right ) \!
  \right )
  \!=\!
\Bigl (
\sum\limits_{\ell=1}^{r}
\delta_{\ell}^{q} R_{p}^{p}
    \!
    \left ( q^{\sim}
    \:\!\!{;}\:\!
    q^{\vee} \;\!\!\right )
\:\!\!{+}\:\!
\delta_{\ell}^{f} R_{p}^{p}
    \!
    \left ( f^{\sim}
    \:\!\!{;}\:\!
    f^{\vee} \;\!\!\right )
    \!
    \Bigr )
    ^{\!\!1/p}
\eqno(3)
\]
%% =============================== //
при  $p\!\in\! \left [1;+\infty\right)$,
где
$\delta_{\ell}^{q}$
и
$\delta_{\ell}^{f}$%
~{\textbf{--}}
из  табл.\,1,
а
$R_{\infty}$
и
$R_{p}$  справа~{\textbf{--}}
из  [3].
%%%
Формулы  (2)  и  (3)
позволяют
свести
различные ОЗ
%%% =====================================
%%% page 2 start here
%%% =====================================
с
совпадающими
$p\!\in\! \left [1;+\infty\right]$
к
$\text{ОЗ}_{\left(m,\tau\right)}^{\left(q,f\right)}$.
%%%
Если  на
$X_{1}^{\left(r\right)}$
заданы  разные  типы  задач
(с  общим   $p$)
из  табл.\,1,
и  присутствует  аппроксимационная  постановка
[3, 4],
то
(в  отличие  от  \textit{простой  смешанной  задачи}
[2])
сведённую
(для  ПЗ
$\delta_{\ell}^{q}
\;\!\!=\;\!\!
\delta_{\ell}^{f}
\;\!\!=\;\!\!
0$)
с  помощью
(2)  или  (3)
задачу  отыскания  управлений
$m^{\sim}$
и/или
$\tau^{\sim}$
на  тех  фрагментах
$X_{1,\ell}$,
где
$m$
и/или
$\tau$
не  заданы,
как  приближённых  решений
экстремальной  задачи
%% =============================== \\
\[
R_{p}^{*}\left ( q^{\vee},f^{\vee} \right )
=
\inf\limits_{m^{\sim},\tau^{\sim}}
R_{p}
\!
\left ( \!
    \left (q^{\sim}
    \!{,}\:\!
    f^{\sim}\;\!\!\right )
    \!{;}\!
    \left (q^{\vee}
    \!{,}\:\!
    f^{\vee}\;\!\!\right ) \!
  \right )
{,}
\eqno(4)
\]
%% =============================== //
назовём  \textit{сложной  смешанной  задачей}.
%%% =====================================



\textbf{2.}\;%
Дополняя  фрагмент  таблицы  из
[5]
тремя  ОЭЗ  из
[1],
получим  табл.\,2
\textit{экстремальных  задач  на  фрагменте}
$X_{1,\ell}\,$.
%%%
В
[5, 6]
описаны
варианты
сочетаний
$\text{ПЭЗ}$
и
$\text{ОЗ}$.
%%%
Сочетание
$\text{ПЗ}$
и
$\text{ОЭЗ}$
по  аналогии
с
[5, 6]
назовём
\textit{комбинированной}  ОЗ.


%% ============================================= \
{
\begin{table}[h]
\begin{center}
\hfill\text{Таблица~2}\qquad\,

\begin{tabular}
{|  c  || c | c || c | c | c || c |}
%%% =====================================
\hline
%%
& \!$\text{ПЭЗ}_{m}^{Q}$\!
& \!$\text{ПЭЗ}_{m}^{F}$\!
& \!$\text{ОЭЗ}_{m}^{q}$\!
& \!$\text{ОЭЗ}_{m}^{f}$\!
&  $\rule{0pt}{4.5mm}%
\rule[-2.1mm]{0pt}{0pt}$%
\! $\text{ОЭЗ}_{\left(m,\tau\right)}^{\left(q,f\right)}$\!
& \\
%%% =====================================
% \hline
\hhline{|=#=|=#=|=|=#=|}
%%
$q$
& 0  & 0  & 1 &  0
& 1
& $\rule{0pt}{4.0mm}%
\rule[-1.5mm]{0pt}{0pt}$%
$\delta_{\ell}^{q}$ \\
%%% =====================================
\hline
%%
$f$
& 0 & 0 & 0 & 1
& 1
& $\rule{0pt}{4.3mm}%
\rule[-1.5mm]{0pt}{0pt}$%
$\delta_{\ell}^{f}$ \\
%%% =====================================
% \hline
\hhline{|=#=|=#=|=|=#=|}
%%
$\tau$
& 1 & 1 & 1 & 1
& 0
& \\
%%% =====================================
\hline
%%
\!$N_{\max}\!$
& 1 & 1 & 1  &  1
& 1
& \\
%%% =====================================
\hline
\end{tabular}\end{center}\end{table}}
%% ============================================= /
%%% ===========================================================



\textbf{3.}\;%
Для
некоторых
смешанных  и  комбинированных  задач
обсуждаются  результаты
вычислительных  экспериментов.
%%% ===========================================================



%%%%  ОФОРМЛЕНИЕ СПИСКА ЛИТЕРАТУРЫ %%%
\litlist



%% ==============================
1.~%
\textit{Бильченко~Г.~Г.,  Бильченко~Н.~Г.}\;
{%
  {Обратные  экстремальные  задачи
   тепломассообмена
   на  проницаемых  поверхностях
   при  гиперзвуковых
   режимах  полёта}~/$\!$/
  <<Воронежская  зимняя  математическая  школа
  С.~Г.~Крейна~{\textbf{--}}  2018>>:
  %%% Overfull start
  Материалы\,  международной\,  конференции\,
  (26{\textbf{--}}31\,  января
  %%% Overfull end
  2018~г.).~{\textbf{---}}
  Воронеж:  ИПЦ  <<Научная  книга>>,
  2018.%~{\textbf{---}}
  %С.~???{\textbf{--}}???.
  }
%% ==============================
%%% =====================================
%%% page 3 start here
%%% =====================================

%% ==============================
2.~%
\textit{Бильченко~Г.~Г.,  Бильченко~Н.~Г.}\;
{%
  {Смешанные  обратные  задачи
   тепломассообмена  на  проницаемых
   поверхностях  при  гиперзвуковых
   режимах  полёта}~/$\!$/
   Международная  конференция,
   посвящённая  100-летию
   со  дня  рождения
   С.~Г.~Крейна
  (Воронеж,
  13{\textbf{--}}19  ноября  2017~г.):
  сборник  материалов.~{\textbf{---}}
  Воронеж:  Изд.  дом  ВГУ,
  2017.~{\textbf{---}}
  С.~52{\textbf{--}}54.
}
%% ==============================

%% ==============================
3.~%
\textit{Бильченко~Г.~Г.,  Бильченко~Н.~Г.}\;
{%
  {Обратные  задачи  тепломассообмена
    на  проницаемых  поверхностях
    гиперзвуковых  летательных  аппаратов.
    I.  О  некоторых  постановках
    и  возможности  восстановления  управления}~/$\!$/
  Вестник  Воронеж.  гос.  ун-та.
  Сер.  Системный  анализ
  и  информационные  технологии.~{\textbf{---}}
  2016.~{\textbf{---}}
  \No~4.~{\textbf{---}}
  С.~5{\textbf{--}}12.
  }
%% ==============================

%% ==============================
4.~%
\textit{Бильченко~Г.~Г.,  Бильченко~Н.~Г.}\;
{%
  {Обратные  задачи  тепломассообмена
    на  проницаемых  поверхностях
    гиперзвуковых  летательных  аппаратов.
    III.  О  постановке  двумерных  задач  и  областях
    допустимых  значений
    <<тепло~{\textbf{--}}  трение>>}~/$\!$/
  Вестник  Воронеж.  гос.  ун-та.
  Сер.  Системный  анализ
  и  информационные  технологии.~{\textbf{---}}
  2017.~{\textbf{---}}
  \No~1.~{\textbf{---}}
  С.~18{\textbf{--}}25.
  }
%% ==============================

%% ==============================
5.~%
\textit{Бильченко~Г.~Г.,  Бильченко~Н.~Г.}\;
{%
  {Комбинированные  обратные  задачи
   тепломассообмена  на  проницаемых
   поверхностях  при  гиперзвуковых
   режимах  полёта}~/$\!$/
  Международная  конференция,
  посвящённая  100-летию
  со  дня  рождения
  С.~Г.~Крейна
  (Воронеж,
  13{\textbf{--}}19  ноября  2017~г.):
  сборник  материалов.~{\textbf{---}}
  Воронеж:  Изд.  дом  ВГУ,
  2017.~{\textbf{---}}
  С.~50{\textbf{--}}51.
}
%% ==============================

%% ==============================
6.~%
\textit{Бильченко~Г.~Г.,  Бильченко~Н.~Г.}\;
{%
  {Об  одной  обратной  задаче  тепломассообмена}~/$\!$/
  ``Герценовские  чтения~{\textbf{--}} 2016.
  Некоторые  актуальные  проблемы
  современной  математики
  и  математического  образования''
  в  электронном  журнале
  ``Дифференциальные  уравнения
  и  процессы  управления''.~{\textbf{---}}
  2016.~{\textbf{---}}
  \No~2.~{\textbf{---}}
  Ч.~2.~{\textbf{---}}
  С.~50{\textbf{--}}56.
  [\,\texttt{%
       {http:}/$\!$/www.math.spbu.ru
       /diffjournal/pdf/herzen2016.pdf}\,]
  }
%% ==============================

%%	\newpage
%%	\thispagestyle{empty}
%%	\text{}
%%	\newpage
