\vzmstitle{О НЕОБХОДИМЫХ И ДОСТАТОЧНЫХ УСЛОВИЯХ СУЩЕСТВОВАНИЯ КЛАССИЧЕСКОГО РЕШЕНИЯ НЕОДНОРОДНОГО БИГАРМОНИЧЕСКОГО УРАВНЕНИЯ}

\vzmsauthor{Гришанина}{Г.\,Э.}
\vzmsauthor{Мухамадиев}{Э.\,М.}

\vzmsinfo{Дубна; {\it anora66@mail.ru}; Вологда; {\it emuhamadiev@rambler.ru}}

\vzmscaption


Рассмотрим задачу о существовании классического решения [1] неоднородного бигармонического уравнения
$$
  \Delta^{2}u(x,y)=f(x,y),\,(x,y)\in G,
\eqno(1)
$$
 где  $\Delta$- оператор Лапласа,  $G$- ограниченная область в $R^{2}$,
 $f(x,y)-$ непрерывная в  $G$ функция. Возникает вопрос о том, каким дополнительным
условиям, кроме непрерывности, должна удовлетворять функция $f(x,y)$ для
того, чтобы уравнение (1) имело классическое решение. Подобная задача для уравнения Пуассона
была изучена авторами в работе [2].

Определим множество
$$
\widetilde{G}=\{(x,y,r): \,M=(x,y)\in G,\,0\leqslant r<\rho(M,\partial G)\},
$$
где $ \rho(M,\partial G)$ - расстояние от точки $M$ до границы $\partial G$
области $G$, и функцию
$$
g(x,y,r,\varphi)=f(x+r\cos\varphi,\,y+r\sin\varphi)
$$
на $\widetilde{G}\times[0,2\pi]$. Очевидно, эта функция непрерывна по
совокупности переменных.
Комплекснозначная функция
$$
F_{k}(x,y,r)=\int_{0}^{2\pi}g(x,y,r,\varphi)\exp(ik\varphi)d\varphi,\quad
k=\pm 1, \pm 2, ...
$$
определена и непрерывна на множестве $\widetilde{G}$ и
$$
F_{k}(x,y,0)=0, \quad(x,y)\in G.
$$
Определим функции
$$
f_{k}(x,y,r)=\int_{r}^{r_{1}}F_{k}(x,y,s)\,\frac{ds}{s},
$$$$
0<r\leqslant r_{1}(x,y)\equiv\frac{1}{2}\rho (M, \partial G).
$$
Назовём непрерывную функцию $ f(x,y)$  $k$-усиленно непрерывной,
если $f_{k}(x,y,r)$ имеет
непрерывное продолжение на подмножестве $\{(x,y,0):(x,y)\in G\} $
множества $\widetilde{G}$.
Очевидно, непрерывная по Гёльдеру функция является $k$-усиленно
непрерывной для любого $k$. Более общее достаточное условие
$k$-усиленной непрерывности функции $f(x,y)$ даёт наличие оценки
для функции $F_{k}(x,y,r)$:
$$
|F_{k}(x,y,r)|\leqslant C(1+|\ln r|)^{\nu},
$$
$$
\nu < - 1,\quad C > 0, \quad 0<r\leqslant r_{1}(x,y).
$$


\textbf{Теорема~1.}
  {\it Если уравнение (1) имеет в области G классическое решение,
то функция $f(x,y)$ $k$-усиленно непрерывна при $ k=2$ и $k=4$ в этой области.}

Cформулируем несколько вспомогательных утвержде-\linebreak
ний, необходимых для доказательства этой теоремы.
Пусть функция $u(x,y),\quad (x,y)\in G$, имеет непрерывные частные производные до
4-го порядка включительно в области $G$.  Положим
$$
  f(x,y)=\Delta^{2}u(x,y),\,(x,y)\in G.
\eqno(2)
$$
По функциям $f(x,y)$ и $u(x,y)$ определим функции
$$
g(x,y,r,\varphi )= f(x+r \cos \varphi, y+r \sin \varphi),
$$
$$
v(x,y,r,\varphi )= u(x+r \cos \varphi, y+r \sin \varphi)
$$
в области $\widetilde{G}\times[0,2\pi]$ и комплекснозначные функции
$$
F_{k}(x,y,r)=\int_{0}^{2\pi}g(x,y,r,\varphi)\exp(ik\varphi)d\varphi,
$$
$$
U_{k}(x,y,r)=\int_{0}^{2\pi}v(x,y,r,\varphi)\exp(ik\varphi)d\varphi,
\quad(x,y,r)\in \widetilde{G},
$$
$$
f_{k}(x,y,r)=\int_{r}^{r_{1}}F_{k}(x,y,s)\,\frac{ds}{s},
$$
$$
u_{k}(x,y,r)=\int_{r}^{r_{1}}U_{k}(x,y,s)\,\frac{ds}{s},\quad
$$
$$
(x,y,r)\in \widetilde{G}, \quad
r_{1}(x,y)\equiv\frac{1}{2}\rho (M, \partial G),  \quad M=(x,y).
$$
Отметим, что функции $ F_{k}(x,y,r)$ непрерывны по совокупности переменных,
а $ U_{k}(x,y,r)$ имеют непрерывные частные производные по всем переменным
до 4-го порядка включительно.
Наша задача - установить сходимость несобственного интеграла
$$
f_{k}(x,y,0)=\lim_{r\to 0} f_{k}(x,y,r)=
\lim_{r\to 0} \int_{r}^{r_{1}}F_{k}(x,y,s)\,\frac{ds}{s}=
$$
$$
=\int_{0}^{r_{1}}F_{k}(x,y,s)\,\frac{ds}{s},
$$
равномерно относительно $(x,y)$ из компактного подмножества области $G$,
при $ k=2$ и $k=4$ , исходя из связи (2)
 между функциями $f(x,y)$ и $u(x,y)$.

\textbf{Лемма~1.}
{\it Справедливо тождество
$$
  F_{k}(x,y,r)=
  $$
  $$
  =r \frac{\partial}{\partial r}\left(\frac{1}{r}
  \frac{\partial^{3}U_{k}}{\partial r^{3}}+\frac{3}{r^{2}}
  \frac{\partial^{2}U_{k}}{\partial r^{2}}+\frac{5-2k^{2}}{r^{3}}
  \frac{\partial U_{k}}{\partial r}+\frac{4(4-k^{2})}{r^{4}} U_{k}\right)+
  $$
  $$
  +\frac{(4-k^{2})(16-k^{2})}{r^{4}}U_{k},
     $$
  где
  $$
  \quad U_{k}=U_{k}(x,y,r),\quad (x,y,r)\in \widetilde{G}.
  $$
}
\textbf{Лемма~2.}
{\it Функции $U_{2}(x,y,r)$ и $U_{4}(x,y,r)$ удовлетворяют следующим условиям:
\begin{gather*}
  U_{2}(x,y,0)=\frac{\partial U_{2}}{\partial r}(x,y,0)=0,
\\
  U_{4}(x,y,0)=\frac{\partial U_{4}}{\partial r}(x,y,0)=
\frac{\partial^{2}U_{4}}{\partial r^{2}}(x,y,0)=
  \frac{\partial^{3}U_{4}}{\partial r^{3}}(x,y,0)=0.
\end{gather*}
Существуют пределы
$$
  \lim_{r\to 0}\frac{2}{r^{2}} U_{2}(x,y,r)=
  \lim_{r\to 0}\frac{1}{r}\frac{\partial U_{2}}{\partial r}(x,y,r)=
  $$
  $$
  =\lim_{r\to 0}\frac{\partial^{2} U_{2}}{\partial r^{2}}(x,y,r)=
  \frac{\partial^{2} U_{2}}{\partial r^{2}}(x,y,0),
$$
$$
  \lim_{r\to 0}\frac{4!}{r^{4}}U_{4}(x,y,r)=
  \lim_{r\to 0}\frac{3!}{r^{3}}\frac{\partial U_{4}}{\partial r}(x,y,r)
  =\lim_{r\to 0}\frac{2}{r^{2}}\frac{\partial^{2}U_{4}}{\partial r^{2}}(x,y,r)=
  $$
  $$
  =\lim_{r\to 0}\frac{1}{r}
  \frac{\partial^{3}U_{4}}{\partial r^{3}}(x,y,r)=
  \lim_{r\to 0}\frac{\partial^{4}U_{4}}{\partial r^{4}}(x,y,r)
  =\frac{\partial^{4}U_{4}}{\partial r^{4}}(x,y,0)
$$
равномерно на каждом компактном подмножестве области G.}

\textbf{Лемма~3.}
{\it Пусть функция $u(x,y)$ непрерывна в области $G$ вместе со
всеми частными производными до $k$-го порядка включительно.
Тогда функция
$$
U_{k}(x,y,r)=\int_{0}^{2\pi}v(x,y,r,\varphi)\exp(ik\varphi) d\varphi,
$$
$$
 v(x,y,r,\varphi)=u(x+\rho \cos \varphi,y+\rho \sin \varphi)
$$
в области $G$ имеет непрерывные частные производные по
совокупности переменных $(x,y,r)$, причём
$$
U_{k}(x,y,0)=\frac{\partial U_{k}}{\partial r}(x,y,0)=
...=\frac{\partial^{k-1} U_{k}}{\partial r^{k-1}}(x,y,0)=0,
$$
и равномерно на каждом компактном подмножестве
области $G$ имеют место предельные соотношения
$$
\lim_{r\to 0}\frac{U_{k}(x,y,r)}{r^{k}}\cdot k!=
\lim_{r\to 0}\frac{(k-1)!}{r^{k-1}}\frac{\partial U_{k}(x,y,r)}{\partial r}=...=
$$
$$
=\lim_{r\to 0}\frac{1}{r}\frac{\partial^{k-1} U_{k}(x,y,r)}{\partial r^{k-1}}=
\frac{\partial^{k} U_{k}}{\partial r^{k}}(x,y,0).
$$
}

При некоторых дополнительных условиях необходимое условие
существования классического решения является и достаточным,
а именно справедлива следующая теорема.

\textbf{Теорема~2.}
  {\it Пусть функция $f(x,y)$ интегрируема и $k$-усиленно непрерывна
при $k=2$ и $k=4$ в области $G$. Тогда функция
$$
u_{0}(x,y)=\frac{1}{8\pi}\int\int_{G}f(\xi,\eta)r^{2}\ln r d\xi d\eta,\quad
r^{2}=(x-\xi)^{2}+(y-\eta)^{2},
$$
является классическим решением уравнения (1).}


%%%%  ОФОРМЛЕНИЕ СПИСКА ЛИТЕРАТУРЫ %%%
\litlist

1. {\it А. Н. Тихонов, А. А. Самарский } Уравнения математической физики. М: Наука, 1977. — 735 с.

2. {\it Э. М. Мухамадиев, Г. Э. Гришанина, А.А.Гришанин} О применении метода регуляризации к построению классического решения уравнения Пуассона. Труды института матетатики и механики УрО РАН, т. 21, №1, 2015. С. 196-211.
