%\documentclass[12pt,openbib]{article}
%
%\usepackage{amsmath}
%\usepackage[cp1251]{inputenc}
%\usepackage[english,russian]{babel}
%\usepackage{amsfonts}
%\usepackage{amsfonts,amssymb}
%\usepackage{amssymb}
%\usepackage{latexsym}
%\usepackage{euscript}
%\usepackage{enumerate}
%\usepackage{graphics}
%\usepackage[dvips]{graphicx}
%\usepackage{geometry}
%
%\geometry{verbose,a4paper,tmargin=2.2cm,bmargin=3.2cm,lmargin=2.4cm,rmargin=2.4cm}

\documentclass[a5paper,12pt,openbib]{report}
\usepackage{amsmath}
\usepackage[utf8]{inputenc}
\usepackage[T2A,T1]{fontenc}
\usepackage[english,russian]{babel}
\usepackage{amsfonts}
\usepackage{amsfonts,amssymb}
\usepackage{amssymb}
\usepackage{latexsym}
\usepackage{euscript}
\usepackage{enumerate}
\usepackage{graphics}
\usepackage[dvips]{graphicx}
\graphicspath{{Data/pic/}}
\usepackage{geometry}
\usepackage{wrapfig}
\usepackage{ifthen}
\usepackage{etoolbox}
\usepackage[overload]{textcase}

\geometry{verbose,a5paper,tmargin=1.75cm,bmargin=2.1cm,lmargin=1.75cm,rmargin=1.75cm}

%Специально для тов. Грюнвальд
\usepackage{mathrsfs}
%\setmathfont{XITS Math}
%\setmathfont[version=setB,StylisticSet=1]{XITS Math}

%Специально для тов. Бильченко
\usepackage{hhline}

\righthyphenmin=2


%\topmargin -1.0cm \setlength{\textheight}{23,7cm}
%\setlength{\textwidth}{15cm} \hoffset=-10mm \voffset=-0mm

\newtheorem{theor}{Теорема}
\newcommand{\ImUn}{\mathfrak{i}}
\newtheorem{teo}{\bf Теорема}
\newtheorem{rem}{\bf Замечание}
%\newtheorem{proposition}{Предложение}
\newtheorem{col}{\bf Следствие}
%\newtheorem{theorem}{Теорема}


\DeclareMathOperator{\diag}{diag}\DeclareMathOperator{\re}{Re}\DeclareMathOperator{\im}{Im}
\DeclareMathOperator*{\M}{M}




\newcommand{\WpX}[2]{W_{#1,\omega}^{1}(\mathbb{R},#2)}
\newcommand{\WpR}[1]{W_{#1,\omega}^{1}(\mathbb{R})}
\newcommand\Lpw[2]{L_{\omega}^{#1}(\mathbb{R},#2)}
\newcommand{\Fw}[1]{\mathcal{F}_{\omega}(\mathbb{R},#1)}
\newcommand{\Cw}[1]{C_{\omega}(\mathbb{R},#1)}
\def\LU{{\mathcal{L}_{\mathcal{U}}}}
\renewcommand{\U}[2]{\mathcal{U}(#1,#2)}
\newcommand{\z}{\ensuremath{\mathbb{Z}}}
\newcommand{\I}{\mathbb{I}}
\newcommand{\Z}{\mathbb{Z}}
\newcommand{\R}{\mathbb{R}}
\newcommand{\Bf}{\boldsymbol}
\newcommand{\bR}{{\mathbb R}}
\newcommand{\bC}{{\mathbb C}}
\newcommand\dom{\operatorname{dom}}
\newcommand{\arr}{\rightarrow}
\newcommand{\bd}{\partial}
\newcommand{\img}{\mathrm{Im}\ }
\newcommand{\Ker}{\mathrm{Ker}\ }
\newcommand{\Lin}{\mathrm{Lin}\ }
\newcommand{\Doms}{{\cal D}\left({\Omega}_{s}\right)}
\newcommand{\Dpoms}{{\cal D}_+\left({\Omega}_{s}\right)}
\newcommand{\Dpsoms}{{\cal D}'_+\left({\Omega}_{s}\right)}
\newcommand{\il}{\int\limits}
\newcommand{\lb}{\linebreak}
\newcommand{\eqdef}{\stackrel{\rm def}{=}}
\newcommand{\Dp}{{\cal D}_+}
\newcommand{\Dxsas}{D_{x'}^{\alpha'}}
\newcommand{\DB}{D_{x'}^{\alpha'}B_{x_n}^{\alpha_n}}
\newcommand{\PDB}{P\left(D_{x'},B_{x_n}\right)}
\newcommand{\nequiv}{\not\equiv}
\newcommand{\po}{P_1}
\newcommand{\Dps}{{\cal D}'_+}
\newcommand{\ds}{\displaystyle}
\newcommand{\Dsoms}{{\cal D}'\left({\Omega}_{s}\right)}
\newcommand{\oms}{{\Omega_s}}
\newcommand{\omsb}{\left(\Omega_s\right)}


\def\Z{{\mathbb{Z}}}
\def\T{{\mathbb{T}}}

\def\a{\alpha}
\def\Nat{\mathfrak{N}}
\def\C{\mathbb{C}}
\def\PP{\mathcal{P}}
\def\P{\mathfrak{H}}
\def\N{\mathfrak{N}}
\def\J{\mathfrak{J}}
\def\K{\mathcal{K}}
\def\NN{\mathcal{N}}
\def\M{\mathfrak{M}}
\def\Im{\mathop{\rm Im}}
\def\Ker{\mathop{\rm Ker}}
\def\Hol{\mathop{\rm Hol}}
\def\Ind{\mathop{\rm Ind}}
\def\al{\mathfrak A}
\def\dim{\mathop{\rm dim}}
\def\E{{\cal E}}
\def\cW{{\cal W}}
\def\cA{{\cal A}}
\def\cD{{\cal D}}
\def\R{{\Bbb R}}

\newcommand{\adto}{{\stackrel{\mathcal P^\alpha}{\longrightarrow}}}
\newcommand{\dto}{{\stackrel{\mathcal P}{\longrightarrow}}}
\newcommand{\dPPto}{{\stackrel{\mathcal P \mathcal P}{\longrightarrow}}}
\newcommand{\rank}{\operatorname{rank}}
\newcommand{\fg}{{\mathfrak{g}}}
\newcommand{\dd}{\operatorname{d}}


\newcommand{\No}{№}


\newtheorem{theorem}{Теорема}
\newtheorem{corollary}{Следствие}
\newtheorem{lemma}{Лемма}
\newtheorem{proposition}{Предложение}
\newtheorem{definition}{Определение}
\newtheorem{remark}{Замечание}
\newtheorem{fact}{Утверждение}
\newtheorem{example}{Пример}
\newtheorem{lem}{Лемма}
\newtheorem{defintion}{Определение}
\newtheorem{Th}{Теорема}
\newtheorem{ths}{Теорема}
\newtheorem{state}{Утверждение}
\newtheorem{ass}{Утверждение}
\newtheorem{cor}{Следствие}
\newtheorem{thm}{Теорема}
\newtheorem{opr}{Определение}
\newtheorem{Thm}{Теорема}
\newtheorem{rk}[thm]{Замечание}
\newtheorem{agree}[thm]{Соглашение}


\makeatletter
\renewcommand{\@makecaption}[2]{%
{\begin{center}#1. #2\end{center}}%
} \makeatother




\def\cz{{\Bbb C}}
\def\rz{{\Bbb R}}
\def\gz{{\Bbb Z}}
\def\nz{{\Bbb N}}
\def\Exp{{\Bbb E}}
\def\Pro{{\Bbb P}}


\newcommand{\prepareforthesis}{
	\setcounter{equation}{0}
	\setcounter{figure}{0}
	\setcounter{table}{0}
	\setcounter{footnote}{0}
	\renewcommand{\theenumi}{\arabic{enumi}}
	\renewcommand{\labelenumi}{\theenumi)} % Не правда ли, Виктор Григорьевич?
	\selectlanguage{russian} % Дабы невозбранно переключаться на английский в конце библиографии, если очень надо
}

\newcommand{\thesis}[1]{
	\prepareforthesis
	\typeout{*}
	\typeout{******* Processing thesis #1 *******}
	\typeout{*}
	\input{Data/#1}
	\typeout{*}
	\typeout{******* End of thesis #1 *******}
	\typeout{*}
	\prepareforthesis
}



\def\pretovar#1#2{\edef#1{\unexpanded{#2}\unexpanded\expandafter{#1}}}
\def\addtovar#1#2{\edef#1{\unexpanded\expandafter{#1#2}}}





\def\varvzmsauthor{}
\def\varvzmsauthortoc{}
\def\varvzmstitle{}
\def\varvzmsflushedauthors{}

\newcommand{\vzmsauthor}[2]{%
	\ifthenelse{
		\equal{\varvzmsauthor}{}
	}
	{
		\def\varvzmsauthor{#2~#1}
		\def\varvzmsauthortoc{#1~#2}
	}{
		\addtovar\varvzmsauthor{, #2~#1}
		\addtovar\varvzmsauthortoc{, #1~#2}
	}
}

\newcommand{\vzmsflushauthors}{
	\ifthenelse{
		\equal{\varvzmsflushedauthors}{}
	}{
		\edef\varvzmsflushedauthors{\noexpand\textit{\varvzmsauthor}}
	}{
		\edef\varvzmsflushedauthors{\varvzmsflushedauthors \\ \noexpand\textit{\varvzmsauthor}}
	}
	\def\varvzmsauthor{}
}

\newcommand{\vzmstitle}[1]{
	\def\varvzmstitle{#1}
}



\newcommand{\vzmscaption}{
	\vzmsflushauthors
	\begin{center}
		\textbf{\varvzmstitle} \\
		\varvzmsflushedauthors \\
	\end{center}
	\addcontentsline{toc}{section}{\varvzmsauthortoc}


	\def\varvzmsauthor{}
	\def\varvzmsauthortoc{}
	\def\varvzmstitle{}
}




\begin{document}





\begin{titlepage}
   \begin{center}
   ВОРОНЕЖСКИЙ ГОСУДАРСТВЕННЫЙ УНИВЕРСИТЕТ
   \end{center}
\vspace{60mm}

    \begin{center}
      {\LARGE
      \bf

{\bf Материалы работы международной конференции}

 Воронежская зимняя математическая школа С.Г.Крейна --- 2016}

\vspace{10mm}


  \end{center}

 \begin{center}
   \vspace{20mm} Воронеж 2016
 \end{center}
\end{titlepage}


\setcounter{page}{2}

\noindent УДК 517.5 517.9

\begin{flushright}
{\it Напечатано по решению Ученого \\ совета математического
факультета \\
\vspace{5mm} Издано при поддержке \\
гранта РФФИ № 16-31-10005-моб\_г }
 \end{flushright}
\vspace{15mm}{\bf Материалы работы международной конференции <<Воронежская зимняя математическая школа С.Г. Крейна - 2016>>. Воронеж: ВГУ, 2016 -  с.}

\vspace{5mm}

\noindent{\bf Под редакцией:} \\
В.А.Костин

\vspace{5mm}

\noindent{\bf Редакционная коллегия:} \\
А.Д. Баев, \  А.В. Глушко, \ В.Г. Звягин, \ М.И. Каменский, \ Ю.И. Сапронов, Е.М. Семенов

\vspace{5mm}

В сборнике представлены статьи участников международной конференции <<Воронежская зимняя математическая школа С.Г. Крейна 2016>>, содержащие новые
результаты по функциональному анализу, дифференциальным уравнениям,
краевым задачам математической физики и другим разделам современной
математики.

Предназначен для научных работников, аспирантов и студентов. \\

\vspace{5mm}

\begin{flushright}
{\bf \copyright Воронежский госуниверситет, 2016}
\end{flushright}

\newpage


%%%%%%%%%%%%%% HISTORICAL PART %%%%%%%%%%%%%%%%%%%%

% a) About VSU

% b) About Kreyn

% c) About Bobylev

%\thesis{Sapronov}

% d) About Saint-Petersburg

%\thesis{hist/begin}

\thesis{hist/vsu}

\thesis{hist/khatset}

\thesis{hist/kostryukova}

\thesis{hist/krein_mark}

\thesis{hist/krein_history}

\thesis{hist/sapronov_bobilev}
%TODO: which is right?
\thesis{Sapronov}

\thesis{hist/obukhovski}

\thesis{hist/krasnoselski}

\thesis{hist/skaliga}

\thesis{hist/bobyleva}

\thesis{Abud}

\thesis{Abud2}

\thesis{Avdeev_Semenov}

\thesis{Avdeev_Sheveleva}

\thesis{Adamova}

\thesis{Vahtel_rasp}%Акиндинова

\thesis{Vahtel_krit}%Акиндинова

\thesis{Akinshina} % naive

\thesis{Akulenko}

\thesis{Alhutov}

\thesis{Atanov}

\thesis{Baev}%Баев

\thesis{Pankov}%Баев

\thesis{rabot}%Баев, Работинская

\thesis{Bilchenko_GG_NG_1}

\thesis{Bilchenko_GG_NG_2}

\thesis{Biryukova}

\thesis{vasilyev}

\thesis{vakhito}

\thesis{Vedushkina}

\thesis{Vladimirova}

\thesis{Voischeva}

\thesis{Vorushilov}

\thesis{Galdin}

\thesis{GLA}

\thesis{Grishanina}

\thesis{Grunwald_L_A}

\thesis{Eleckih}

\thesis{Zverkina}

\thesis{ZvyaginA}

\thesis{Zvyagin_Turbin}

\thesis{Zubova}

\thesis{Zukin}

\thesis{Ignatev}

\thesis{IkonnikovaEV}

\thesis{kaverina}

\thesis{kalitvin}

\thesis{Karginova}

\thesis{Kibkalo}

\thesis{KOBCEV}

\thesis{Kovalevsky}

\thesis{Kovalevsky2}

\thesis{Kopev}

\thesis{korchagin}

\thesis{Kostin}

\thesis{KostinVA}

\thesis{KostinDV}

\thesis{Kostina}

\thesis{KukushkinMV}

\thesis{LapshinaMG}

\thesis{litvinov}

\thesis{LukKorpPanin}

\thesis{LyahovRoschSanina}

\thesis{Makarova}

\thesis{Malutina}

\thesis{Moskvin}

\thesis{Muhamediev}

\thesis{Nikolaienko}

\thesis{Novikov}

\thesis{Orlov}

%\thesis{Pavljuk}
%TODO: which is right?
\thesis{Pavluk}

\thesis{Pepa}

\thesis{Peshkova}

\thesis{Podoprikhin}

\thesis{Pustovoytov}

\thesis{Pchelova}

\thesis{Repina}

\thesis{Roshupkin}

\thesis{Rudenko}

\thesis{SAMSONOV}

\thesis{Semenov}

\thesis{Scoromnik}

\thesis{Sokolova}

\thesis{Solodskih}

\thesis{Stenuhin}

\thesis{Sterlikova}

\thesis{Subbotin}

\thesis{Terekhin}

\thesis{tokareva}

\thesis{Turashova}

\thesis{Uskov}

\thesis{Fahat}

\thesis{FedorovVE}

\thesis{fedoseev}

\thesis{Fomenko}

\thesis{Fomenko_Vedushkina}

\thesis{Frolkina}

\thesis{Khartseva}

\thesis{Khatskevich}

\thesis{Chernova}

\thesis{ShamolinMV}

\thesis{Shananin}

\thesis{Shelkovoi}

\thesis{Shemetova_V_V}

\thesis{Khabibulin}

\thesis{KornevSV}

\thesis{Piskarev}

\thesis{Reinov}

\thesis{ZvyaginV}

\newpage

{\normalsize \tableofcontents }

\end{document}
