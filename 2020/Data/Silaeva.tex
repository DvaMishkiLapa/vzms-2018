
\begin{center}
    {\bf КОРРЕКТНАЯ РАЗРЕШИМОСТЬ ЗАДАЧИ КОШИ ДЛЯ УРАВНЕНИЯ С ПРОИЗВОДНОЙ ПО КАПУТО В БАНАХОВОМ ПРОСТРАНСТВЕ}

    {\it  М.Н.Силаева, Алкади Хамса Мохамад}

    (Воронеж; {\it  marinanebolsina@yandex.ru})
\end{center}

\addcontentsline{toc}{section}{ Силаева М.Н., Алкади Хамса Мохамад}
В [6] для уравнения $$\frac{\partial^{\alpha}u(t,x)}{\partial t^{\alpha}}=\frac{\partial^{2}u(t,x)}{\partial x^{2}}, 0<\alpha<1, t>0, x\in (-\infty, \infty), $$
где
$$\frac{\partial^{\alpha}u(t,x)}{\partial t^{\alpha}}=\frac{1}{\Gamma(1-\alpha)}\int_{0}^{t}(t-s)^{-\alpha}\frac{\partial u(s,x)}{\partial s}ds-$$
 производная в смысле Капуто по переменной $t$, рассматривается задача Коши с условиями
 $$u(0, x,\alpha)=g(x),$$
 $$ u(t,\pm\infty,\alpha)=0,$$
 и приводится решение этой задачи в виде
 $$u(t,x,\alpha)=\int_{-\infty}^{\infty}G(t,\xi,\alpha)g(x-\xi)d\xi$$
 с указанием функции Грина $G(t,\xi,\alpha)$.

 Однако, корректная разрешимость этой задачи, с точки зрения устойчивости решения к погрешностям, в [6] не обсуждается.

Решению этой проблемы для общего случая дифференциальных уравнений в банаховом пространстве посвящена настоящая заметка.

В банаховом пространстве $E$ с нормой $\|\cdot\|_{E}=\|\cdot\|$ рассматривается уравнение
 $$\frac{d^{\alpha}u(t)}{dt^{\alpha}}=Au(t), t\geq 0,\eqno {(1)}$$
 где $A$-линейный замкнутый оператор с областью определения $D(A)\subset E$ и областью значений $R(A)$,
 $$\frac{d^{\alpha}u(t)}{dt^{\alpha}}=\frac{1}{\Gamma(1-\alpha)}\int_{0}^{t}(t-s)^{-\alpha}u'(s)ds -\eqno {(2)}$$
 производная по Капуто порядка $0<\alpha<1$.

 \textbf{Определение 1.} {\it Решением уравнения  (1) называется вектор-функция $u(t)$ со значениями в $D(A)$, для которой определена производная (2) и, удовлетворяющая уравнению (1).}


\textbf{Определение 2.} {\it Решение уравнения  (1)  удовлетворяющее условию
$$u(0)=u_{0}\in D(A)\eqno {(3)}$$ будем называть решением задачи Коши (1)-(2).}

\textbf{Определение 3.} {\it Задачу (1)-(2) будем называть равномерно корректной, если для её решений выполняется оценка
$$\|u(t)\|_{E}\leq M\|u_{0}\|_{E},\eqno {(4)}$$где константа $M$ от $t$ и $u_{0}$ не зависит.}



 \textbf{ Теорема } {\it Если оператор $A$ является производящим оператором сильно непрерывной полугруппы линейных  преобразований $U(t,A)$, то задача Коши (1)-(3) равномерно корректна и её решение имеет вид
 $$u(t)=\int_{0}^{\infty}I_{t}^{(1-\alpha)}(h_{\alpha}(t,\xi))U(\xi,A)u_{0}d\xi,$$
 где $I^{(1-\alpha)}f(t)$-дробный интеграл Римана-Лиувилля,

 $h_{\alpha}(t,s)=\frac{1}{2\pi i}\int_{\sigma-i\infty}^{\sigma+i \infty}e^{tp-\xi p^{\alpha}}dp$-функция Иосиды

 и справедлива оценка (4).}


\


\smallskip \centerline {\bf Литература} \nopagebreak

1. {\it Иосида К.} Функциональный анализ : [учебник] / К. Иосида ; пер. с англ. В.М. Волосова.— М. : Мир, 1967 .— 624 с.

2. {\it Крейн С.Г.} Линейные дифференциальные уравнения в банаховом
пространстве/С.Г. Крейн.-- М.: Наука, 1967.--464 с.

3. {\it Костин В.А.} Элементарные полугруппы преобразований и их производящие уравнения / В.А. Костин, А.В. Костин, Д.В. Костин // Доклады Академии Наук.— Москва, 2014 .— Т. 455, № 2. - С. 142-146

4. {\it Костин А.В.} О корректной разрешимости задач без начальных условий для некоторых сингулярных уравнений / А. В. Костин, Д. В. Костин, М. Н. Небольсина // Вестник Воронежского государственного университета. Серия Физика. Математика.— Воронеж, 2018 .— № 1. - С. 87-93

5. {\it Самко С.Г.} Интегралы и производные дробного порядка и некоторые их приложения / С.Г. Самко, А.А. Килбас, О.И. Маричев.— Минск : Наука и техника, 1987 .— 687 с.

6. {\it Майнарди Ф.} Временное уравнение дробной диффузионно-волновой функции /Майнарди Ф. // Радиофизика и квантовая электроника. Выпуск 38, №1-2, 1995.-С.20-36


