\vzmstitle[\footnote{Работа выполнена при финансовой поддержке МОНУ (проект 0219U008403) и НАНУ (проект 0119U101608).}]{О ДИНАМИЧЕСКИХ СИСТЕМАХ КРАСНОСЕЛЬСКОГО--МАННА}
\vzmsauthor{Семёнов}{В.\,В.}
\vzmsinfo{Киев; {\it semenov.volodya@gmail.com}}
\vzmscaption

Построение и исследование приближенных методов ме\-трической теории неподвижных точек -- интересная, имеющая много приложений и активно развивающаяся область нелинейного анализа.

Классические теоремы Брауэра, Шаудера и Какутани о неподвижной точке имеют неконструктивный характер. Но для более узких классов операторов существует развитая алгоритмика аппроксимации неподвижных точек. Одним из таких классов является класс нерастягивающих операторов. Хорошее введение в их теорию с описанием основных методов содержится в [1] (см. также книгу [2]).

Ещё в 1955 г. М.А. Красносельский [3] (близкие идеи были высказаны в работе W.R. Mann [4]) предложил для поиска неподвижных точек действующего в банаховом пространстве нерастягивающего оператора $T$ процесс
$$
x_{n+1} = \frac{x_{n}+ Tx_{n}}{2}. \eqno{(1)}
$$
В [3] также доказана сильная сходимость (1) для компактных нерастягивающих операторов $T$, действующих в равномерно выпуклых банаховых пространствах.

Классические работы [3, 4] положили начало большому количеству исследований по сходимости метода Красно\-сельского--Манна, т.е. процесса вида
$$
	x_{n+1} = x_{n} + \lambda_{n}( T x_{n} - x_n ), \eqno{(2)}
	$$
где $\lambda_n \in (0, 1]$ (см. [1]). А в статье [5] для нерастягивающих операторов в гильбертовом пространстве $H$ был предложен и изучен непрерывный аналог метода (2)
$$
\left\{\begin{array}{l}
\dot{x}(t) = \lambda(t) \left(T x(t)- x(t)\right), \\\
x(0) = x_0 \in H,
\end{array}\right. \eqno{(3)}
$$
где $\lambda: [0, +\infty) \to (0, 1]$. В частности, при определённых условиях доказана слабая сходимость при $t \to +\infty$ решения (3) к неподвижной точке $T$. Регуляризованный по Тихонову вариант динамики (3) изучен в [6].

Мы отталкиваемся от упомянутых работ [5, 6]. Пусть $T: H \to H$ "--- нерастягивающий оператор, действующий в гильбертовом пространстве $H$, причём $F(T) \neq \emptyset$. В лекции мы рассмотрим асимптотическое поведение траекторий динамических систем
$$
\left\{\begin{array}{l}
\dot{x}(t) = \lambda(t) \left(T x(t)- x(t)\right) - \varepsilon (t) A x(t), \\
x(0) = x_0 ,
\end{array}\right. \eqno{(4)}
$$
где $x_0 \in H$, оператор $A: H \to H$ сильно монотонный и липшицевый, а функции $\lambda: [0, +\infty) \to (0, 1]$, $\varepsilon: [0, +\infty) \to [0, +\infty)$ удовлетворяют определённым условиям. Доказана сильная сходимость $x(t)$ (при $t \to +\infty$) к решению вариационного неравенства
$$
x \in F(T) : \, \, (Ax, y-x ) \geq 0 \, \, \, \forall y \in F(T).
$$
Также рассмотрим распределённый вариант динамики (4) и конкретные методы, порождаемые при применении (4) к задачам математического программирования, игровым задачам и вариационным неравенствам. Например, с задачей
$$
f (x) \to \min , \, \, \, x \in \mbox{argmin} \, g , \eqno{(5)}
$$
где функция $g$ выпукла и полунепрерывна снизу, а функция $f$ сильно выпукла и имеет липшицев градиент, можно связать динамическую систему
$$
\left\{\begin{array}{l}
\dot{x}(t) = \left(\mbox{\rm prox}_g x(t)- x(t)\right) - \frac{1}{\sqrt{t+1}} \nabla f (x(t)), \\
x(0) = x_0 \in H,
\end{array}\right.
$$
где $$\mbox{\rm prox}_{g} x = \mbox{argmin}_{y \in H} \left\{g(y) + \frac{1}{2} \left\| y-x \right\|^2\right\}.$$
 Если $\mbox{argmin} \, g \neq \emptyset$, то $x(t)$ при $t \to +\infty$ сильно схо\-дит\-ся к ре\-шению (5).



\litlist

1. {\it Bauschke H.H., Combettes P.L.} Convex Analysis and Monotone Operator Theory in Hilbert Spaces. NY: Springer, 2011. 408 p.

\selectlanguage{russian}

2. {\it Васин В.В., Еремин И.И.} Операторы и ите\-ра\-ци\-онные процессы фейеровского типа. Москва-Ижевск: Регулярная и хаотическая динамика, 2005. 200 c.



3. {\it Красносельский М.А.} Два замечания о методе последовательных приближений. УМН. 1955. Том 10. Выпуск 1. С. 123--127.

\selectlanguage{english}

4. {\it Mann W.R.} Mean value methods in iteration. Proc. Amer. Math. Soc. 1953. Vol. 4 P. 506--510.


5. {\it Bot R.I., Csetnek E.R.} A dynamical system associated with the fixed points set of a
nonexpansive operator. Journal of Dynamics and Differential Equations. 2017. Vol. 29. Iss. 1. P. 155--168.

6. {\it Vilches P., Perez-Aros E.} Tikhonov regularization of dy\-na\-mi\-cal systems associated with nonexpansive operators defined in closed and convex sets. arXiv. 1904.05718. 2019.
