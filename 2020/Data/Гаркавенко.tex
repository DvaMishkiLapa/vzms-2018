\begin{center}
    {\bf О СОБСТВЕННЫХ ЗНАЧЕНИЯХ ОДНОЙ БЕСКОНЕЧНОЙ МАТРИЦЫ}

    {\it Г.В. Гаркавенко, Д.Г. Усков}

    (Воронеж; {\it g.garkavenko@mail.ru, uskov.dan@mail.ru})
\end{center}

\addcontentsline{toc}{section}{Гаркавенко Г.В., Усков Д.Г.}

Рассматривается бесконечная матрица $\mathcal{A}: \mathbb{N}\times\mathbb{N}\to\mathbb{R}$, элементы которой задаются
формулами $\mathcal{A}(n, n)=n^2$, $n\in\mathbb{N}$, $\mathcal{A}(n, m)=1/(n-m)^2$, $n\ne m$, $n$, $m\in\mathbb{N}$.
Рассматриваемую матрицу
можно представить в виде $\mathcal{A}=\mathcal{A}_0-B$, где $\mathcal{A}_0: \mathbb{N}\times\mathbb{N}\to\mathbb{R}$~---
диагональная матрица, т.~е. $\mathcal{A}_0(n, n)=n^2$, $n\in\mathbb{N}$, и $\mathcal{A}_0(n, m)=0$, $n\ne m$. Матрица
$B$ будет иметь нулевую главную диагональ, остальные её элементы $B(n, m)=-1/(n-m)^2$, $n\ne m$, $n$, $m\in\mathbb{N}$.
Отметим, что матрица $B$ является теплицевой матрицей с суммируемыми диагоналями. Рассматривается вопрос приближенного
нахождения собственных значений матрицы $\mathcal{A}$.

При применении проекционных методов в вопросах приближенного нахождения собственных значений [1] обычно вместо
матрицы $\mathcal{A}$ рассматривают последовательность конечномерных матриц $\mathcal{A}_n$, собственные значения которых \linebreak
считают численно и далее полученные последовательности собственных значений используются в качестве приближений к искомым
собственным значениям. Но здесь возникает вопрос обоснования проекционного метода (метода Галеркина). Важным является тот факт, что
собственные значения $\lambda_n$, $n\in\mathbb{N}$, диагональной матрицы $\mathcal{A}_0$ стремятся к бесконечности, поэтому
при конечном $n\in\mathbb{N}$ имеем $\|\mathcal{A}-\mathcal{A}_n\|\to\infty$. Таким образом, если рассматривать матрицу $\mathcal{A}$
как возмущение конечномерной матрицы $\mathcal{A}_n$, то возмущение не является матрицей ограниченного оператора.

Подойдём к методу Галеркина с точки зрения метода подобных операторов.

Пусть задано семейство матриц $P_n$, $n\in\mathbb{N}$, таких, что каждая из матриц $P_i$ имеет только один ненулевой элемент
$P_i(i, i)=1$, а остальные элементы равны нулю и $P_{(n)}=\sum_{i\leqslant n}P_i$.

Согласно [2, Теорема~19.2] матрицы $\mathcal{A}_0-B$ и
$$
\mathcal{A}_0 - P_{(n)}\mathcal{X}P_{(n)}-\sum_{i>n}P_i\mathcal{X}P_i=\mathcal{A}_0-J_n\mathcal{X}
$$
$$
= \mathcal{A}_0P_{(n)}-P_{(n)}\mathcal{X}P_{(n)}+\sum_{i>n}(\mathcal{A}_0P_i-P_i\mathcal{X}P_i),
$$
где $\mathcal{X}$~--- решение уравнения метода подобных
операторов, подобны, если выполнено условие $\pi^2/(3(2n+1))<1/4$. В этом случае $\sigma(P_{(n)}(\mathcal{A}_0-B)P_{(n)})=
\sigma(\mathcal{A}_0P_{(n)}-P_{(n)}\mathcal{X}P_{(n)})$. Но нам известна не сама
матрица $\mathcal{X}$, а последовательные приближения к ней, причём первым приближением является матрица $B$, а также известна
оценка $\|J_n\mathcal{X}-J_nB\|=\mathcal{O}((2n+1)^{-1})$. Таким образом, для собственных значений $\lambda_k$, $k\in\mathbb{N}$,
матрицы $\mathcal{A}$ и соответствующих собственных значений $\lambda_{kn}$ последовательности матриц $A_n$, $n\geqslant k$,
имеет место оценка $|\lambda_k-\lambda_{kn}|=\mathcal{O}((2n+1)^{-1})$, $n\geqslant k$.

Проведён вычислительный эксперимент, результаты которого полностью согласуются с вышеизложенным.

Отметим, что вопросы асимптотической оценки элементов бесконечных матриц с помощью метода подобных операторов рассматривались,
например, в [3 - 6].


% Оформление списка литературы
\smallskip \centerline {\bf Литература} \nopagebreak

1. {\it Красносельский М. А. и др.} Приближенное решение операторных уравнений. М.: Наука, 1967. 456 с.

2. {\it Баскаков А. Г.} Гармонический анализ линейных операторов. Воронеж: Изд-во ВГУ, 1987. - 165 с.

3. {\it Бройтигам И. Н., Поляков Д. М.} Асимптотика собственных значений бесконечных блочных матриц~--- Уфимск. матем. журн. 11:3
(2019), С. 10-29.

4. {\it Гаркавенко Г. В., Ускова Н. Б.} О диагонализации некоторых классов матриц~--- В сборнике: Информационные технологии в науке, образовании и производстве (ИТНОП-2018). VII Международная научно-техническая конференция. Сборник трудов конференции. 2018. С. 539-542.

5. {\it Гаркавенко Г. В., Ускова Н. Б.} Асимптотика собственных значений операторов с ленточной матрицей и полугруппыоператоров~---
Вопросы науки. -- 2016. -- Т. 1. -- С. 27-30.

6. {\it Гаркавенко Г. В., Ускова Н. Б.} О собственных значениях одного разностного оператора~---
В сборнике: Современные методы прикладной математики, теории управления и компьютерных технологий (ПМТУКТ-2015). Сборник трудов VIII международной конференции. -- 2015. -- С. 369-371.
