\vzmstitle{МАРК АЛЕКСАНДРОВИЧ КРАСНОСЕЛЬСКИЙ~--- ВЫДАЮЩИЙСЯ ПРОФЕССОР И ЛЮБИМЫЙ ПЕДАГОГ}
\vzmsauthor{Ускова}{О.\,Ф.}
\vzmsinfo{Воронеж; {\it sunny.uskova@list.ru}}
\vzmscaption

Знаменитому учёному и педагогу, профессору Красносельскому Марку Александровичу принадлежат многочисленные работы в различных
областях математики:
\begin{itemize}
\item теории функций вещественных переменных;
\item теории дифференциальных уравнений;
\item теории интегральных уравнений;
\item функциональному анализу;
\item топологии;
\item численным методам;
\item приближенным методам.
\end{itemize}

Марк Александрович родился на Украине 27 апреля 1920 года в Староконстантиновке Хмельницкой области. В 1942 году он успешно окончил
Объединённый украинский университет, в 1951 защитил докторскую диссертацию физико"=математических наук. В Воронежском государственном
университете М.А. Красносельский плодотворно работал с 1952 по 1969 год заведующим кафедрой функционального анализа
математико"=механического факультета, где успешно руководил защитой кандидатских диссертаций своих аспирантов. Вместе с профессорами
ВГУ В.\,И.\,Соболевым и С.\,Г.\,Крейном Марк Александрович создал ставшую хорошо известной не только в нашей стране, но и за рубежом,
школу функционального анализа [1].

До настоящего времени пользуются популярностью монографии и учебники М.А. Красносельского, изданные в центральных издательствах:
\begin{itemize}
\item <<Приближенное решение операторных уравнений>>\linebreak(Москва, 1969, совместно с Г.М. Вайникко, П.П. Забрейко, Я.Б. Рутицким,
В.Я. Стеценко);
\item <<Топологические методы в теории нелинейных интегральных уравнений>>\, (Москва, 1956);
\item <<Выпуклые функции и пространства Орлича>>\, (Мос\-к\-ва, 1958, совместно с Я.Б. Рутицким);
\item <<Оператор сдвига по траекториям дифференциальных уравнений>>\, (Москва, 1965);
\item <<Положительные решения операторных уравнений>> \linebreak(Москва, 1962);
\item <<Векторные поля на плоскости>>\, (Москва, 1963, совместно с коллективом авторов);
\item <<Интегральные операторы в пространствах суммируемых функций>>\, (Москва, 1966, совместно с коллективом авторов).
\end{itemize}

Марк Александрович лично в моей жизни сыграл важную, никем не превзойдённую, роль. Моя первая встреча с Красносельским М.А.
состоялась в сентябре 1957 года, когда он начал читать лекции по математическому анализу для студентов 2 курса
физико"=математического факультета Воронежского государственного педагогического института, где я училась. Его лекции стали для нас
образцом лекций по математическим дисциплинам, эталоном тщательной подготовленности, доступности, идейной ясности. Марк
Александрович проводил в течение года консультации один раз в неделю и коллоквиумы 2-3 раза в семестр. Благодаря такой
организации учебного процесса существенно повысилась успеваемость, ответственность и интерес студентов к
математическому анализу.

Марк Александрович был для нас яркой личностью, талантливым организатором и очень заботливым человеком, не раз выручавшим
своих студентов. Он несколько раз посещал студенческие вечера и обратил внимание, что его студентки не умеют танцевать
вальс. Чтобы помочь таким студенткам, Красносельский М.А. договорился с директором Воронежского музыкального театра и нам выделили
(бесплатного для нас) тренера, который приезжал в пединститут еженедельно в течение двух месяцев и мы стали танцевать на
студенческих вечерах.

Марк Александрович всегда интересовался жизнью своих учеников, помогал решать возникшие проблемы. Он способствовал
переводу студентов нашей группы после окончания 2 курса пединститута на математико"=механический факультет ВГУ. Чтобы
перевести нас не на второй курс (с потерей года обучения), а на третий, Марк Александрович организовал наше обучение
в течение июня"=августа 1958 года, когда нам читались лекции, проводились практические занятия, зачёты, экзамены.

Марк Александрович Красносельский был нашим настоящим кумиром. Он один из тех преподавателей, с которыми любой студент
мог быть самим собой. В течение всех лет обучения мы чувствовали его поддержку. Он разрешал любому студенту нашей
группы заниматься в своей профессорской домашней библиотеке, а после окончания Первомайской демонстрации приглашал
к себе домой на чай. Несколько раз Марк Александрович вместе со своими аспирантами приходил на встречи вузовских
баскетбольных команд. В составе сборной ВГУ по баскетболу я тогда играла и хорошо помню, когда он наградил нашу
команду за победу большим тортом.

Только благодаря Красносельскому М.А. студенты нашей группы проходили в 1960 году производственную практику в Московском
государственном университете, когда в ВГУ не было современной вычислительной техники.


% Оформление списка литературы
\smallskip \centerline {\bf Литература} \nopagebreak

1. {\it Ускова О.Ф.} Учитель в науке и жизни. Некоторые вопросы анализа, алгебры, геометрии и математического образования.
Материалы второй международной молодёжной научной школы <<Актуальные направления математического анализа и смежные вопросы>>.
Воронеж: Научная книга, 2018, № 8. - 364 с. С.~324--325.
