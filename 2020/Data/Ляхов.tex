\vzmstitle{ЗАДАЧА ДИРИХЛЕ ДЛЯ В-ГАРМОНИЧЕСКОГО УРАВНЕНИЯ В ШАРЕ}
\vzmsauthor{Ляхов}{Л.\,Н.}
\vzmsauthor{Санина}{Е.\,Л.}
\vzmsauthor{Рощупкин}{С.\,А.}
\vzmsinfo{Воронеж; {\it levnlya@mail.ru}; {\it sanina08@mail.ru}; Елец; {\it roshupkinsa@mail.ru}}
\vzmscaption


Рассматривается сингулярный дифференциальный оператор Лапласа\--Бесселя $\Delta_B$ и $B$-гармоническое уравнение $\Delta_B u=0$ в шаре. Решение граничной задачи Дирихле представлено в виде ряда Лапласа по весовым сферическим функциям. Полученные решения совпадают с решениями классического гармонического уравнения при равенстве нулю всех размерностей операторов Бесселя, входящих в оператор $\Delta_B$.


%Библиография: 13 названий.

%{\bf Ключевые слова}.Весовая сферическая функция, оператор Бесселя, оператор Лапласа---Бесселя на сфере, ряды Лапласа.


  1. {\bf Весовые сферические функции (В"=гармоники)}.
Пусть $n$ и $N$ натуральные числа и $1{\le}n{\le}N$ и пусть\\
$\mathbb{R}^+_N \{x{=}(x',x'')
{=}(x_1\,\ldots\,,x_n,x_{n+1}\,\ldots\,,x_N),\,\,\, x_1{>}0,\ldots,x_n{>}0\}.$ \\
 В $\mathbb{R}_N^+$ рассматривается сингулярный
дифференциальный оператор
$\Delta_B=\sum_{j=1}^nB_j{+}\sum_{i=n+1}^N{\partial^2\over\partial x_i^2},
\,\, B_{x_i}{=}{{\partial^2}\over {\partial x^2}}{+}{\gamma_i\over {x_i}}
{\partial\over{\partial x_i}}\,,\,\, \gamma_i>0.$ Отметим, что обозначение $\Delta_B$ (в настоящее время принятое в мировой литературе) введено И.А. Киприяновым в 60-х годах прошлого столетия. Из [1] %\cite{Keld}
и книги [2]
%\cite{Kipr}
вытекает, что ограниченные решения уравнений, содержащие оператор $\Delta_B$, надо искать в классе функций $C^2$, чётных по каждой из переменных $x_1,\,\ldots\,,x_n$. Такие функции будем называть $x'$-чётными.

%2. {\bf Весовые сферические функции (В-гармоники)}.

{\it Однородный x'-чётный многочлен $P_m^\gamma(x)$ порядка $m$,  удовлетворяющий
уравнению $ \Delta_BP_m^\gamma(x)= 0$\,, называется В"=гармоничес\-ким.}
{\it Весовой сферической функцией (В"=гармоникой)} (далее используется сок\-ра\-ще\-ние в.с.ф.)
{\it называется сужение В"=гармонического многочлена на сферу:}
$$Y_m^\gamma(\Theta)={P_m^\gamma(x)\over|x|^m}= P_m^\gamma\left(x\over|x|
\right).$$

    Для наших исследований потребуются следующие свойства в.с.ф., полученные в [3], [4], [5]
%\cite{L1}, \cite{L2}, \cite{L3},
 (см. также книгу [6]).
%\cite{Lya}

 Пусть $S^+_1=\{x:\,|x|=1\}\cap\mathbb{R}_N^+$.

{\bf Ортогональность в.с.ф.}, отвечающих различным порядкам $m$ и $k$
(m,\,k=0,1,2,\ldots),
определяется равенством
$$\int\limits_{S^+_1}Y^\gamma_m(\Theta)\,Y^\gamma_k(\Theta)\,
(\Theta')^\gamma\,dS{=}0, \quad m{\neq}k,\quad Y_0^\gamma(\Theta)=1\,. \eqno (1)
$$
\label{posm(1)}
где $(\Theta')^\gamma=\prod\limits_{i=1}^n\Theta_i^{\gamma_i}$, ${\cal P}^\gamma_{\Theta'}$ -- многомерный
оператор Пуассона, $C_m^\nu$ "--- многочлен Гегенбауэра и коэффициент $|S^+_1|_\gamma$ определён по формуле
''площади нагруженной сферы'' (см. [6],
с. 20).
Среди в.с. функций данного порядка $m$ в свою очередь можно выбрать
ортогональный базис $Y_{m,k}^\gamma(\Theta)\,,\,\,1{\le}k{\le}d_\gamma(m)$.
В результате получим систему ортогональных в.с. функций
$$\{Y_{m,k}^\gamma(\Theta)\},\quad m{=}0,1,2,\,\ldots\,,\quad k{=}1,2,\,
\ldots\,d_\gamma(m),
 $$
которая плотна в пространстве непрерывных на $S^+_1$ функций, чётных
по каждому аргументу $x_1,\,\ldots,x_n$ и плотна в $L_2^\gamma(S^+_1)$
и полна в $L_2^\gamma(S^+_1)$.


\textbf{Оценки $D_B$-производных от в.с. функций $Y^\gamma_m(\Theta)$
при $m\to\infty$.}

   Введём обозначение
$$
\left(D_B\right)^\beta_{x'}=
\left\{
\begin{array}{lll}
B^{\beta_i\over2}_{x_i} & , & \beta_i=2k - \hbox{ чётное число,} \\
{\partial\over\partial x_i}B^{\beta_i-1\over2}_{x_i} & , & \beta_i=2k{+}1 -
\hbox{ нечётное число,}.
\end{array}\right.
$$
   Имеет место весовая
среднеквадратическая оценка
$$\int\limits_{S^+_N}\left|\left((D_B)^\beta_{x'}\,\,D^\alpha_{x''}\,
P^\gamma_m\right)(x)\right|^2(x')^\gamma\,dS\leqslant C_1 m^{2|\alpha+\beta|}
\Vert Y^\gamma_m\Vert_{L_2^\gamma(S^+_1)}
$$
и равномерная оценка
$$\left|\left(B^\beta_{x'}D^\alpha_{x''}\,P_m^\gamma\right)(x)\right|^2\leqslant C_2m^{2|\alpha+2\beta|{+}N{+}|\gamma|{-}2}
\,|x|^{2m{-}2|\alpha{+}2\beta|}\,\,\Vert
Y^\gamma_m\vert_{L_2^\gamma(S^+_1)}\,\,,
$$% \eqno (1.6.3)
где постоянные $C_1$ и $C_2$ зависят от $n,k,\alpha$ и $\beta$, но не от $m$.


 {\bf Дифференциальное уравнение в.с. функций}:
$$\left(\Delta_B(\Theta)\,Y_m^\gamma\right)(\Theta){=}
m(m{+}N{+}|\gamma|-2)Y_m^\gamma (\Theta)\,.
$$
 Здесь через $\Delta_B(\Theta)$ обозначено сужение $\Delta_B$ на сферу $S_1^+$ (оператор Бельтрами): $\Delta_B{=}{\partial^2\over\partial r^2}{+}{N{+}|\gamma|{-}1\over r}
{\partial\over\partial r}{+}{1\over r^2}\,\Delta_B(\Theta)$.


  {\bf Ряды  Лапласа по в.с. функциям}. Справедливы следующие утверждения.

  {\it Пусть $f{\in}C^{2l}_{ev}(S^+_1)$ и $a_{m,k}$ и
$b_{m,k}$ коэффициенты Фурье-Лапласа функций $f(\Theta)$ и
$\left(\Delta_B^l(\Theta)\,f\right)(\Theta)$ соответственно по системе в.с.функций. Тогда}
$$a_{m,k}=[m(m+N+|\gamma|-2)]^l\,b_{m,k}.$$

 {\it Если $f{\in}C^{2l}_{ev}(S^+_1)$, то}
$$|a_{m,k}|\le M\,m^{-2l}\,,\quad M{=}\int_{S^+_1}\left|\left(\Delta_B
(\Theta)f\right)(\Theta)\right|^2(\Theta')^\gamma\,dS\,.
$$




\begin{center}
{\bf 2. Задача Дирихле для В"=гармонического уравнения в шаре}
\end{center}
%\vskip\baselineskip


Пусть $U=\{x:\,|x|<1\}\cap\mathbb{R_N^+}$. Рассмотрим задачу Дирихле
$$\Delta_B u{=}0,\,\,\, u\left|_{|x|=1}=f(\Theta)\right.\,, u\in C^2(U)\cap C(\overline{U}),\eqno(4)$$
где $\Theta_j=\Theta(\varphi^i),\,\,\,\varphi^i=\left(\varphi_1,...,\varphi_j\right)$ "--- сферические координаты точки на замкнутой $n$-полусфере в $\mathbb{R}_N^+$:
 $$
 \left. \begin{array} {l}
\Theta_1=\cos\varphi_1\\
\Theta_2=\sin\varphi_1 \cos\varphi_2\\
\ldots\,\ldots\ldots\ldots\ldots\ldots\ldots\ldots\ldots\ldots\ldots,\,\\
\Theta_{N-1}=\sin\varphi_1 \sin\varphi_2...\sin\varphi_{n-1}\cos\varphi_{n-1},\\
\Theta_N=\sin\varphi_1\sin\varphi_2...\sin\varphi_{n-2}\sin\varphi_{n-1},
\end{array}\right|
\quad\begin{array}{l}
0\leq\varphi_i\leq\pi/2,\\
i=\overline{1,n},\\\\
0\leq\varphi_j\leq\pi,\\
j=\overline{n,N-1},\\
0\leq\varphi_{N-1}\leq2\pi.
\end{array}
$$
  Решение задачи (4) в сферических координатах обозначим $\widetilde{u}=u(r,Q)$. Это приведёт к задаче:
$$\Delta_B u{=}\Delta B_{n_r \left|\gamma\right|-1,r}\widetilde{u}\left(r,\Theta\right){+}\frac{1}{r^2}\Delta_{B,\Theta}\widetilde{u}(r,\Theta),\,\,\, \widetilde{u}\bigl|_{r=1}{=}f(\Theta)\,.
\eqno(5) $$
\label{eq8}

  Предположим существование решения (5) в виде $\widetilde{u}=R(r)\cdot Y(\Theta)$. Тогда оно должно удовлетворять уравнению
$$r^2Y(\Theta)B_{N+\left|\gamma\right|-1,r}R(r)+R(r)\Delta_{B,\Theta}Y\left(\Theta\right)=0.\eqno(6)$$
\label{eq9}
Отсюда имеем два обыкновенных дифференциальных уравнения:
$$\left\{
\begin{array}{l}
r^2 B_{N+\left|\gamma\right|-1,r}R(r)-\lambda R(r)=0, \,\,\,\,\,\,\,\,\,\,\,\,\,\,\,\,\,\\
\Delta_{B,\Theta}Y\left(\Theta\right)+\lambda Y(\Theta)=0. \,\,\,\,\,\,\,\,\,\,\,\,\,\,\,\,\,\,\,\,\,\,\,\,\,\,\,\,\,\,\,\,
\end{array}\right.$$
Из условия Дирихле следует, что по каждой переменной $\varphi_j$ решение задачи (6) периодично с периодом $2\pi$. Такому условию удовлетворяет и $B$-гармоника $Y_m^\gamma\left(\Theta\right)$. Поэтому за нетривиальное решение уравнения \label{net form ()} примем спектральный набор
$$\lambda=m\left(m+N+\left|\gamma\right|-2\right),\,\,\,\,Y\left(\Theta\right)=Y_m^\gamma\left(\Theta\right),\,\,\,m=0,1,2,...$$

Решение $R(r)$ ищем в виде
$R(r)=r^k.$ Имеем \\
$k\left(k{+}N{+}\left|\gamma\right|{-}2\right){-}k\left(m{+}N{+}\left|\gamma\right|{-}2\right){=}0
 \,\,\,\Longrightarrow\,\,\, k^2-km=0$.
  Таким образом, число $k$ может принимать два значения $k_1=0$ и $k_2=m$. В первом случае $R(r)=1$ и мы получили решение, не зависящее от $r$. Т.е. оно постоянно на лучах, выходящих из центра шара и, следовательно, представляет собой однородную функцию, которая, очевидно, не определена в начале координат. Итак решение уравнения (6) имеет вид  $\widetilde{u}_m(r,\Theta)=r^m\cdot a_m\cdot Y_m^\gamma (\Theta)$. Предположим, что это решение представлено
рядом Лапласа
$$\widetilde{u}(r,\Theta)=\sum_{m=0}^\infty r^m a_m Y_m^\gamma (\Theta).$$
При выполнении условия задачи и при $r<1$ он сходится абсолютно и равномерно, поэтому функция $\widetilde{u}(r,\Theta)$ "--- решение (5). Если это решение удовлетворяет условию Дирихле, то это условие представлено рядом Фурье\--Бесселя
$f(\varphi)=\sum_{m=0}^\infty a_m Y_m^\gamma (\Theta(\varphi)),$
следовательно
$$a_m=\int_{S^+_N} f(\varphi)Y_m^\gamma \left(\Theta(\varphi)\right)\left(\Theta'\right)^\gamma \,\, dS(\Theta).$$


% Оформление списка литературы
\smallskip \centerline {\bf Литература} \nopagebreak

%\bibitem{Keld}
1. {\it Келдыш М.В}. О некоторых случаях вырождения уравнений эллиптического типа на границе области. // ДАН СССР. 1951.
Т.77. № 1.С.181-183.

%\bibitem{Kipr}
2. {\it Киприянов И. А.} Сингулярные эллиптические задачи / И. А. Киприянов.
{\bf --} М. : Наука, 1997. {\bf --} 199 с.

%\bibitem{L1}
3. {\it Ляхов Л.Н.} Об одном классе сферических функций и сингулярных псевдодифференциальных операторов. // ДАН. 1983. Т.272.№ 4. С.781-784

%\bibitem{L2}
4. {\it Ляхов Л.Н.} О рядах по весовым сферическим функциям. //
Новосибирскю: СО АН СССР. 1984
Кн. Корректные краевые задачи для неклассических уравнений математ. физики. С. 102-109.

%\bibitem{L3}
5. {\it Ляхов Л.Н.} Весовые сферические функции и сингулярные псевдодифференциальные операторы.
// Дифференц. уравн. 1985. Т.21. № 6. С. 1020-1032.

%\bibitem{Lya}
6. {\it Ляхов Л.Н. }
В"=гиперсингулярные интегралы и их приложение к функциональным классам Киприянова и интегральным уравнениям с В"=потенциальными ядрами.
/ Липецк: Редакционно издательский центр ЛГПУ. 2007. С. 232.
