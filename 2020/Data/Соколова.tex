\vzmstitle{ПЕРИОДИЧЕСКИЕ РЕШЕНИЯ ВПОЛНЕ ИНТЕГРИРУЕМОГО УРАВНЕНИЯ ПФАФФА}
\vzmsauthor{Соколова}{Г.\,К.}
\vzmsinfo{Иркутск; {\it 98gal@mail.ru}}
\vzmscaption

Заметка посвящена изучению проблемы существования периодических решений уравнения Пфаффа и построению множества периодов этого решения. Рассмотрим уравнение
$$
f_{1}(\bar{r})dx_{1}+f_{2}(\bar{r})dx_{2}+\ldots+f_{n}(\bar{r})dx_{n}=0,\eqno(1)
$$
где функции $f_{i}:\,{\mathbb R}^{n}\to{\mathbb R}$, $i=1,2,\ldots,n$, непрерывны и имеют непрерывные частные производные по совокупности переменных $x_{1},\,x_{2},\ldots,\,x_{n}$. Известно, что, если уравнение (1) удовлетворяет условиям {\it полной интегрируемости}
$$
\partial_{x_{j}}f_{i}(\bar{r})=\partial_{x_{i}}f_{j}(\bar{r}),\,\,\,\,\,i,j=1,2,\ldots,\,n,\eqno(2)
$$
то оно является уравнением в полных дифференциалах, и его общий интеграл имеет вид
$$
u(\bar{r})=\sum\limits_{i=1}^{n}\int\limits_{0}^{x_{i}}f_{i}(0,\,0,\ldots,\,t,\,x_{i+1},\ldots,\,x_{n})dt+C.\,\eqno(3)
$$

\textbf{Определение.} {\it Функцию $f:\,{\mathbb R}^{n}\to{\mathbb R}$ будем называть {\it периодической} с периодом $\bar{T}$, если существует ненулевой вектор $\bar{T}\in{\mathbb R}^{n}$ такой, что для всех $\bar{r}\in{\mathbb R}^{n}$ выполняется равенство $f(\bar{r}+\bar{T})=f(\bar{r})$. Период $\bar{T}_{0}$, наименьшего модуля, сонаправленный с вектором $\bar{T}$, назовём основным периодом функции $f$ в данном направлении $\bar{\cal T}$, где $\bar{T}=\vert\bar{T}\vert\cdot\bar{\cal T}$.}

В статье [1] показано, что неособенной линейной заменой аргумента $\bar{r}\in{\mathbb R}^{n}$ периодическую функцию $f:\,{\mathbb R}^{n}\to{\mathbb R}$ с решёткой периодов, порождённой векторами $\bar{T}_{1},\bar{T}_{2},\ldots,\bar{T}_{m}$, можно преобразовать к периодической функции по любым $m$ переменным с основными периодами $\vert\bar{T}_{1}\vert,\vert\bar{T}_{2}\vert,\ldots,\vert\bar{T}_{m}\vert$. Т.~е. всякую периодическую функцию $f:\,{\mathbb R}^{n}\to{\mathbb R}$ можно считать периодической по первым $m$ переменным.

Предположим, что общим интегралом~(3) уравнения~(1) является периодическая по всем переменным $x_{1},\,x_{2},\ldots,\,x_{n}$ функция $u:\,{\mathbb R}^{n}\to{\mathbb R}$. В этом случае, с учётом условия~(2), функции $f_{i}:\,{\mathbb R}^{n}\to{\mathbb R}$ также являются периодическими, а множествами ${\rm P}_{f_{i}}$ их периодов"--- решётки, порождённые векторами $\bar{T}_{i1},\,\bar{T}_{i2},\ldots,\,\bar{T}_{in}$ и имеющие непустое пересечение. Здесь и далее $\bar{T}_{ij}=T_{ij}\bar{\rm e}_{j}$, $i,\,j=1,2,\ldots,n$, система векторов $\lbrace\bar{\rm e}_{j}\rbrace_{j=1}^{n}$ задаёт классический базис Гамеля в ${\mathbb R}^{n}$.

Ранее проблема существования периодических решений уравнения (1) в условиях (2) его полной интегрируемости изучалась в статье [2], где доказан критерий периодичности общего интеграла~(3) с периодом
$\bar{T}=T_{1}\bar{\rm e}_{1}+T_{2}\bar{\rm e}_{2}+\ldots+
T_{n}\bar{\rm e}_{n}$. В данной работе сформулирован критерий периодичности общего интеграла по переменным $x_{1},\,x_{2},\ldots,\,x_{n}$ и указано множество периодов. Приведём некоторые вспомогательные утверждения, доказанные в [3].

\textbf{Теорема~1.} {\it Пусть функция $f:\,{\mathbb R}^{n}\to{\mathbb R}$ непрерывна по переменной $x_{i}$ и $T_{i}$-периодическая по этой переменной, тогда справедливо равенство
$$
\int\limits_{0}^{x_{i}}f(x_{1},\ldots,\,x_{i-1},\,t,\ldots,\,x_{n})dt=S_{T_{i}}[f]x_{i}+\varepsilon_{i}(\bar{r}),\eqno(4)
$$
где $S_{T_{i}}[f]=\frac{1}{T_{i}}\int\limits_{0}^{T_{i}}f(x_{1},\ldots,\,x_{i-1},\,t,\ldots,\,x_{n})dt$ означает среднее значение функции $f$ по переменной $x_{i}$ на отрезке $[0,\,T_{i}]$, функция $\varepsilon_{i}:\,{\mathbb R}^{n}\to{\mathbb R}$~--- $T_{i}$-периодическая по переменной $x_{i}$.}

Заметим, что если функция $f:\,{\mathbb R}^{n}\to{\mathbb R}$ непрерывна по переменной $x_{i}$, то $\varepsilon:\,{\mathbb R}^{n}\to{\mathbb R}$ имеет непрерывную частную производную по этой переменной и удовлетворяет равенству $\partial_{x_{i}}\varepsilon=f(\bar{r})-S_{T_{i}}[f]$, с условием $\varepsilon(\bar{r})\vert_{x_{i}=0}=0$. Однозначная разрешимость данной задачи гарантирует единственность представления (4). Функция $\varepsilon:\,{\mathbb R}^{n}\to{\mathbb R}$ наследует основной период функции $f:\,{\mathbb R}^{n}\to{\mathbb R}$ по переменной $x_{i}$. Справедлива следующая теорема.

\textbf{Теорема~3.} {\it Пусть функции $f_{i}:\,{\mathbb R}^{n}\to{\mathbb R}$ непрерывны и имеют непрерывные частные производные $\partial_{x_{j}}f_{i}:\,{\mathbb R}^{n}\to{\mathbb R}$ первого порядка такие, что выполнено условие (2); пусть также функции $f_{i}:\,{\mathbb R}^{n}\to{\mathbb R}$ являются периодическими с решётками периодов ${\rm P}_{f_{i}}$, которые порождены векторами вида
$\bar{T}_{ij}=T_{ij}\bar{\rm e}_{j}$, где $i,\,j=1,2,\ldots,\,n$. Тогда для того, чтобы общий интеграл (3) уравнения Пфаффа (1) был функцией, периодической по переменным $x_{1},\,x_{2},\ldots,\,x_{n}$, необходимо и достаточно, чтобы ${\rm P}_{f_{1}}\cap{\rm P}_{f_{2}}\cap\ldots\cap{\rm P}_{f_{n}}\neq\varnothing$, и для всех $i=1,2,\ldots,\,n$ выполнялись равенства
$$
\int\limits_{0}^{T_{i}}f_{i}(0,\ldots,\,0,\,t,x_{i+1},\ldots,\,x_{n})dt=0.
$$
При этом множеством периодов этой функции является ${\rm P}_{u}={\rm P}_{f_{1}}\cap{\rm P}_{f_{2}}\cap\ldots\cap{\rm P}_{f_{n}}$.}

\litlist

1. {\it Orlov~S.~S., Sokolova~G.~K.} Periodic function of several real variables~// Обозрение прикладной и промышленной математики. 2018. Т.~25, №~1. С.~50--51.

2. {\it Тажимуратов~И.} О периодических решениях одной системы уравнений в частных производных первого порядка. Математические заметки. 1981. Т.~30, №~3. С.~363--369.

3. {\it Соколова~Г.~К.} Периодические функции нескольких переменных: элементы теории и приложения // Сборник трудов XVI Междунар. конф. студентов, аспирантов и молодых учёных <<Перспективы развития фундаментальных наук>>. В 7 томах. Том 3. Математика. Томск: Изд-во ТПУ, 2019. С.~28--30.
