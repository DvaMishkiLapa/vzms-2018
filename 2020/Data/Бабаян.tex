\vzmstitle{О ЗАДАЧЕ ДИРИХЛЕ ДЛЯ НЕЭЛЛИПТИЧЕСКОГО УРАВНЕНИЯ В ЕДИНИЧНОМ КРУГЕ}
\vzmsauthor{Бабаян}{А.\,О.}
\vzmsinfo{Ереван; {\it barmenak@gmail.com}}
\vzmscaption

Пусть $D=\{z:\vert z\vert<1\}$ "--- единичный круг комплексной плоскости, а $\Gamma=\partial D$ "--- его граница. В области $D$ рассмотрим уравнение
$$\left(\frac{\partial}{\partial y}-\lambda_1\frac{\partial}{\partial x}\right)^2\left(\frac{\partial}{\partial y}-\lambda_2\frac{\partial}{\partial x}\right)^2V(x,y)=0, (x,y)\in D, \eqno(1)$$
где $\lambda_1$, $\lambda_2$ "--- различные действительные числа. Решение уравнения (1) ищем в классе функций $C^4(D)\cap C^{(1,\alpha)}(D\cup\Gamma)$. На границе $\Gamma$ решение удовлетворяет условиям Дирихле:
$$V\big\vert_{\Gamma}=f_0(x,y),\ \ V_r\big\vert_{\Gamma}=f_1(x,y), (x,y)\in\Gamma. \eqno(2)$$
Здесь $f_k\in C^{(1-k,\alpha)}(\Gamma)$, $k=0,1$ "--- заданные функции, $\frac{\partial}{\partial r}$, $\frac{\partial}{\partial\theta}$ "--- производные, соответственно, по радиусу и аргументу комплексного числа $z=x+iy=re^{i\theta}$.
\par Как известно (см. [1]), задача Дирихле для неэллиптического уравнения не является корректно поставленной, однако в этом случае также удаётся получить содержательные результаты.
В работе [2] рассмотрено уравнение $u_{xy}=0$ в произвольной области. Были получены условия на границу области, при которых задача Дирихле имеет решение. Далее, в [3] было рассмотрено уравнение $(1+\lambda)u_{xx}-(1-\lambda)u_{yy}=0$ в единичном круге и получены условия на $\lambda$, при которых однородная задача Дирихле для этого уравнения имеет нетривиальные решения. В работах [4,5] исследована задача Дирихле для строго гиперболической системы первого порядка в произвольной области. Были получены условия на геометрию границы области, при которых задача Дирихле имеет решение.
\par В работе предлагается схема решения краевых задач в единичном круге, основанная на представлении искомого решения в виде ряда по полиномам Чебышева. Приведём краткое описание метода и сформулируем полученный результат.
Общее решение уравнения (1) представим в виде:
$$V=\Phi_0(x+\lambda_1y)+\frac{\partial}{\partial\theta}\Phi_1(x+\lambda_1y)+\Psi_0(x+\lambda_2y)+\frac{\partial}{\partial\theta}\Psi_1(x+\lambda_2y)$$
Здесь $\Phi_j$, $\Psi_j$ "--- функции, подлежащие определению. Граничные условия (2) представим в эквивалентной форме:
$$V_x\vert_\Gamma=F; V_x\vert_\Gamma=G; V(1,0)=f_0(1,0). \eqno(3)$$
Здесь $F=\cos\theta f_1-r^{-1}\sin\theta f^{'}_0$ и $G=\sin\theta f_1+r^{-1}\cos\theta f^{'}_0$. Подставим общее решение в равенства (3) и используем операторные тождества: $\frac{\partial}{\partial x}\frac{\partial}{\partial\theta}=\frac{\partial}{\partial\theta}\frac{\partial}{\partial x}+\frac{\partial}{\partial y}$, $\frac{\partial}{\partial y}\frac{\partial}{\partial\theta}=\frac{\partial}{\partial\theta}\frac{\partial}{\partial y}-\frac{\partial}{\partial x}$. Получим два уравнения для определения функций $\Phi^{'}_j$, $\Psi^{'}_j$.
$$\Phi^{'}_0(A_1)+\left(\frac{\partial}{\partial\theta}+\lambda_1I\right)\Phi^{'}_1(A_1)+\Psi^{'}_0(A_2)+\left(\frac{\partial}{\partial\theta}+\lambda_2I\right)\Psi^{'}_1(A_2)=$$
$$=F(\theta); \ \ \lambda_1\Phi^{'}_0(A_1)+\left(\lambda_1\frac{\partial}{\partial\theta}-I\right)\Phi^{'}_1(A_2)+\lambda_2\Psi^{'}_0(A_2)+$$
$$+\left(\lambda_2\frac{\partial}{\partial\theta}-I\right)\Psi^{'}_1(A_2)=G(\theta), \ A_j=\cos\theta+\lambda_j\sin\theta, \eqno(4)$$
где $j=1,2$. Aргументы $A_j$ представим в виде $$A_j=\sqrt{1+\lambda^2_j}\cos(\theta-\alpha_j);\ \cos\alpha_j=\frac{1}{\sqrt{1+\lambda^2_j}},\sin\alpha_j=\frac{\lambda_j}{\sqrt{1+\lambda^2_j}}.$$ Тогда неизвестные функции будут зависеть от $\cos(\theta-\alpha_j)$ и, следовательно, могут быть представлены рядами по многочленам Чебышева первого рода. Далее, приравнивая в уравнениях (4) коэффициенты при $\cos k\theta$ и $\sin k\theta$, получим систему линeйных уравнений четвёртого порядка для определения коэффициентов этих рядов. Определитель этой системы, с точностью до ненулевого сомножителя, равен $\Delta_k(\gamma)=k^2-U^2_{k-1}(\cos\gamma)$, где $U_{k-1}$ "--- многочлен Чебышева второго рода степени $k-1$, $\gamma=\alpha_2-\alpha_1$. Учитывая, что при $\gamma\neq 0,\pi$ имеем оценку $\vert U_{k-1}(\cos\gamma)\vert<k$ (см.[6], пункт 2.2) искомые коэффициенты разложений определяем однозначно для произвольных граничных функций. Итак, получена следующая теорема.

\textbf{Теорема~1.} {\it Задача (1),(2) однозначно разрешима.}


% Оформление списка литературы
\litlist

1. {\it Courant R., Hilbert D.} Methods of Mathematical Physics. New York, Chichester etc.: John Wiley and Sons, 1989. 831p.

2. {\it John F.} The Dirichlet problem for a Hyperbolic Equation. Amer. J. of Math. 1941.— V.63(1) - P.141-155.

3. {\it Александрян Р.А.} Спектральные свойства операторов, порождённых системами дифференциальных уравнений типа С.Л. Соболева. Труды ММО, 1960. — Т.9. - С.455-505.

4. {\it Жура Н.А., Солдатов А.П.} Граничная задача для гиперболической системы первого порядка в двухмерной области. Известия РАН, сер. Математика, 2017. — Т.81, No.3. - С. 83-108

5. {\it Солдатов А. П.} Характеристически замкнутые области для строго гиперболических систем первого порядка на плоскости. Проблемы матем. анализа, 2018. — No. 93. - С. 133-135 с.

6. {\it Mason J.S., Handscomb D.C.} Chebyshev Polynomials. New York etc.: CRC Press, 2003. 335p.
