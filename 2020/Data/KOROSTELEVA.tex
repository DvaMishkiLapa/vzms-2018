\begin{center}
    {\bf АППРОКСИМАЦИЯ СОБСТВЕННЫХ КОЛЕБАНИЙ ПЛАСТИНЫ\\ С ПРИСОЕДИНЁННЫМ ОСЦИЛЛЯТОРОМ\footnote{Работа выполнена при финансовой поддержке РФФИ и Правительства Республики Татарстан в рамках научного проекта  № 18-41-160029.
Работа поддержана РФФИ (проект № 19-31-90063).}}\\

    {\it Д.М. Коростелева, А.А. Самсонов, П.С. Соловьёв, С.И. Соловьёв}

    (Казань; {\it diana.korosteleva.kpfu@mail.ru})
\end{center}

\addcontentsline{toc}{section}{Коростелева Д.М., Самсонов А.А., Соловьёв П.С., Соловьёв С.И.}


Исследуется нелинейная дифференциальная задача на собственные значения в частных производных четвёртого порядка,
описывающая поперечные собственные колебания квадратной упругой изотропной пластины с присоединённым осциллятором.
Предполагаются выполненными граничные условия шарнирного опирания.
Эта задача имеет неубывающую последовательность положительных конечнократных собственных значений
с предельной точкой на бесконечности.
Последовательности собственных значений соответствует нормированная система
собственных функций.
В настоящей работе изучаются свойства собственных значений и собственных функций
при изменении параметров присоединённого осциллятора.
Исходная дифференциальная задача на собственные значения аппроксимируется
сеточной схемой метода конечных разностей на равномерной сетке.
Исследуется точность приближённых собственных значений
и собственных функций в зависимости от шага сетки.
Полученные результаты развивают и обобщают результаты работ~[1--3].



% Оформление списка литературы
\smallskip \centerline {\bf Литература} \nopagebreak

1. {\it Соловьёв~С.И.}
Нелинейные задачи на собственные значения. Приближённые методы.
Saarbr\"ucken: LAP Lambert Academic Publishing, 2011. 256 с.

2. {\it Соловьёв~С.И.}
Аппроксимация нелинейных спектральных задач в гильбертовом пространстве
// Дифференциальные уравнения. 2015. Т. 51,
№ 7. С. 937--950.

3. {\it Соловьёв~С.И.}
Собственные колебания стержня с упруго присоединённым грузом
// Дифференциальные уравнения. 2017. Т. 53,
№ 3. С. 418--432.





