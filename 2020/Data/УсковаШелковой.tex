\vzmstitle{МЕТОД ПОДОБНЫХ ОПЕРАТОРОВ В ЗАДАЧЕ О ПРОДОЛЬНОМ ИЗГИБЕ ТЯЖЕЛЫХ СТЕРЖНЕЙ}
\vzmsauthor{Ускова}{Н.\,Б.}
\vzmsauthor{Шелковой}{А.\,Н.}
\vzmsinfo{Воронеж; {\it nat-uskova@mail.ru}; {\it shelkovoj.aleksandr@mail.ru}}
\vzmscaption

Пусть $L_{2}[0,1]$ - гильбертово пространство комплексных измеримых (классов) функций, суммируемых с квадратом модуля со скалярным произведением вида $$(x,y) = \int\limits_0^1{x(\tau)\overline{y(\tau)}}d\tau.$$ Через $W_2^2{[0,1]}$ обозначим пространство Соболева $$\{x\in L_{2}[0,1]: x'~\mbox{абсолютно непрерывна},~x''\in L_{2}[0,1]\}.$$ В диссертации [1] рассматривались спектральные свойства интегро-дифференциального оператора $$\mathcal{L}:D(\mathcal{L})\subset{L_{2}[0,1]}\to{L_{2}[0,1]},$$ порождаемого интегро-дифференциальным выражением вида
$$
(\mathcal{L}x)(t) = -\ddot{x}(t) - [\dot{x}(0)a_{0}(t) - \dot{x}(1)a_{1}(t)] - \int\limits_0^1{K(t,s)x(s)}ds\eqno (1)
$$
с вырожденным ядром $K(t,s) = \sum\limits_{i = 1}^k{p_i(t)q_i(s)},~p_i, q_i\in{L_{2}[0,1]}$,
с областью определения $D(\mathcal{L}) = \{x\in{W_2^2[0,1]},~x(0) = x(1) = 0\}$ и краевыми условиями
$
x(0) = x(1) = 0.
$
Методом исследования является метод подобных операторов, развиваемый в работах Баскакова А.Г. (см. [2]) и используемый в спектральном анализе дифференциальных операторов [3-6] и смежных вопросах [7].
Одним из примеров, где возникают операторы типа (1), является задача о продольном изгибе тяжёлых стержней (см. [8]). Определение критической нагрузки $P$ при продольном изгибе шарнирно опёртого с обоих концов, вертикально расположенного стержня длины $l$ постоянного сечения при учёте его собственного веса приводит к задаче на собственные значения
$$
y^{IV}-\varepsilon(xy')'=-{\lambda}y'',~
y(0)=y''(0)=y(l)=y''(l)=0.\eqno (2)
$$

В пространстве $L_{2}[0,l]$ введём оператор $Ay=y''$ с областью определения $D(A)$, определяемой краевыми условиями $y(0)=y(l)=0$, тогда $A^{2}y=y^{IV}$. Дифференциальное уравнение в задаче (2) приобретёт вид: $A^{2}y-\varepsilon(xy')'=-{\lambda}Ay$. Применив к обеим частям оператор $A^{-1}$, получим краевую задачу $Ay-{\varepsilon}A^{-1}xAy-{\varepsilon}A^{-1}xAy'=-{\lambda}y, ~ y(0)=y(l)=0$.
Оператор $A^{-1}$ имеет вид: $(A^{-1}y)(x)=\int\limits_0^l{K(x,s)y(s)}ds$, где $K(x,s)=G(x,s)$ - функция Грина для краевой задачи $y''=0,~y(0)=y(l)=0$.
Известно, (см., например, [9]), что в данном случае $G(x,s)=x(s-l)/l$,~если $0\le{s}<x$, и $G(x,s)=s(x-l)/l$,~если $x\le{s}\le{l}$.
Непосредственные вычисления приводят исходное дифференциальное уравнение в задаче (5) к операторному уравнению $Ly=-{\lambda}y$, где
$$(Ly)(x)=y''(x)+{\varepsilon}xy(x)-{\varepsilon}l(x-l)y'(l)+$$
$$+\varepsilon\left(\int\limits_x^l{y(s)}ds-x\int\limits_0^l{y(s)}ds\right)/l.$$
К данному оператору применим метод подобных операторов, то есть оператор $L$ можно представить в виде $A-B$, где $Ay)(x)=y''(x)$ --- невозмущённый оператор, а
$$(By)(x)={\varepsilon}\left(l(x-l)y'(l)-xy-\left(\int\limits_x^l{y(s)}ds-x\int\limits_0^l{y(s)}ds\right)/l\right)$$
- возмущение.



% Оформление списка литературы
\litlist
1. {\it Шелковой А.Н.} Спектральный анализ дифференциальных операторов с нелокальными краевыми условиями: дисс. канд. физ.-мат. наук. Воронеж, 2004. 144 с.

2. {\it Баскаков А.Г.} Гармонический анализ линейных операторов. Воронеж: Изд-во ВГУ, 1987. 165 с.

3. {\it Шелковой А.Н.} Оценки собственных значений и собственных функций одного дифференциального оператора с нелокальными краевыми условиями // Вестник факультета прикладной математики и механики. - 2000. - Вып. 2. - С. 226-235.

4. {\it Шелковой А.Н.} Метод подобных операторов в исследовании интегро-дифференциальных операторов с квадратично суммируемым ядром // Вопросы науки. - 2016. - Т. 2. - С. 68-80.

5. {\it Шелковой А.Н.} Спектральные свойства дифференциальных операторов, определяемых нелокальными краевыми условиями // Вопросы науки. - 2016. - T. 3. - С. 83-90.

6. {\it Шелковой А.Н.} Спектральные свойства дифференциального оператора второго порядка, определяемого нелокальными краевыми условиями // Математическая физика и компьютерное моделирование. - 2018. - Т. 21. - № 3. - С. 18-33.

7. {\it Ускова Н.Б., Шелковой А.Н.} Об одной задаче о продольном изгибе тяжёлых стержней // Физико-математическ---ое моделирование систем: материалы XVIII Междунар. семинара. - 2018. - Ч. 2. - С. 159-164.

8. {\it Коллатц Л.} Задачи на собственные значения (с техническими приложениями. М.: Наука, 1968. 504 с.

9. {\it Камке Э.} Справочник по обыкновенным дифференциальным уравнениям. СПб.: Лань, 2003. 576 с.
