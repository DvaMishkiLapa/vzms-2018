\vzmstitle{ДОСТАТОЧНЫЕ УСЛОВИЯ СОГЛАСОВАНИЯ ХАРАКТЕРИСТИЧЕСКИХ ВТОРЫХ ПРОИЗВОДНЫХ ГРАНИЧНОГО РЕЖИМА С НАЧАЛЬНЫМИ УСЛОВИЯМИ И ОДНОМЕРНЫМ ВОЛНОВЫМ УРАВНЕНИЕМ ДЛЯ ГЛАДКИХ РЕШЕНИЙ}
\vzmsauthor{Ломовцев}{Ф.\,Е.}
\vzmsauthor{Спесивцева}{К.\,А.}
\vzmsinfo{Минск; {\it lomovcev@bsu.by}; {\it ksenia.spesivtseva@gmail.com}}
\vzmscaption

Для первой четверти плоскости ~$\dot{G}_\infty=]0,+\infty[ \times
]0,+\infty[$ в задаче
$$
u_{tt}(x,t) + (a_1 - a_2)u_{xt}(x,t) - a_1a_2 u_{xx}(x,t) =
f(x,t), \, (x,t) \in \dot{G}_{\infty}, \eqno(1)
$$
$$
u(x,t)\lvert_{t=0} = \varphi(x), \, u_t(x,t)\lvert_{t=0} =
\psi(x), \, x>0, \eqno(2)
$$
$$[\zeta(t)u_{tt}+\xi(t)u_{xt}+\theta(t)u_{xx}+\alpha(t)u_t+\beta(t)u_x+\gamma(t)u]\lvert_{x=0}=\mu(t),
\,t>0, \eqno(3)
$$
нижними индексами функции $u$ обозначены её частные производные
соответствующих порядков по указанным в индексах переменным, ~$a_1
> 0$, ~$a_2 > 0,$ ~$\zeta,\, \xi,\, \theta,\,
\alpha,\, \beta,\, \gamma$ -- заданные функции от ~$t$ и ~$ f, \,
\varphi,\, \psi, \, \mu$ -- заданные функции своих переменных $x$
и $t$. Впервые зависящие от времени $t$ коэффициенты в граничных
режимах (3) для уравнения колебаний струны (1) при $a_1=a_2$
появились в [1].

Уравнение (1) имеет характеристики $x-a_1t=C_1$, $x+a_2t=C_2,$
$\forall$ $C_1,\,C_2\in \mathbb{R}=]-\infty,+\infty[$.
Характеристика $x=a_1t$ является критической для уравнения (1)
[2]. Она делит первую четверть плоскости $G_{\infty}=[0,+\infty[
\times [0,+\infty[$ на два множества $G_{-}$ и $G_{+}$ [2]. Под
$C^{k}(\Omega)$ понимается множество всех $k$ раз непрерывно
дифференцируемых функций на подмножестве $\Omega\subset
\mathbb{R}^2$ и $C^0(\Omega)=C(\Omega)$. В случае
нехарактеристических вторых производных в граничном режиме (3)
критерий корректности задачи (1)--(3) для классических решений
$u\in C^{2}(G_{\infty})$ найден в [2]. Для характеристических
вторых производных в (3) для ограниченной струны нужны критерии
корректности этой задачи для более гладких решений $u\in
C^{m}(G_{\infty}),\,m\geq 2.$

{\bf\textit{Определение.} } Гладким $m$ раз непрерывно
дифференцируемым решением начально"=граничной задачи (1)--(3)
называется функция ~$u\in C^{m}(G_{\infty})$, где $m=2,3,4,\,...$,
удовлетворяющая уравнению (1) в обычном смысле, а начальным (2) и
граничному (3) условиям в смысле пределов соответствующих
выражений от её значений $u(\dot{x},\dot{t})$ во внутренних точках
$(\dot{x},\dot{t})\in \dot{G}_\infty$ для всех указанных в них
граничных точек $x$ и $t$.

Для решений $u \in C^{m+1}({G}_\infty)$ задачи (1)--(3) на
<<единицу>> большей гладкости верны следующие условия
согласования.

\textbf{Теорема.} {\it Пусть в граничном режиме (3) коэффициенты
имеют гладкость порядка $m$:
$\zeta,\,\xi,\,\theta,\,\alpha,\,\beta,$ $\gamma\in
C^m[0,+\infty[,$ первые производные берутся не вдоль: $a_1
\alpha(t)\neq\beta(t),\, t\in[0,+\infty[,$ а вторые производные --
вдоль критической характеристики уравнения (1):
$a_1^2\zeta(t)-a_1\xi(t)+\theta(t) \equiv 0,\,t\in[0,+\infty[.$
Если начально"=граничная задача (1)--(3) имеет решение $u\in
C^{m+1}(G_\infty)$, то для правой части $f \in C^{m-1}(G_\infty)$,
начальных $\varphi \in C^{m+1}[0,+\infty[,$ $\psi \in
C^{m}[0,+\infty[$ и граничного
 $\mu \in C^{m-1}[0,+\infty[$ данных верны условия согласования}
$$
Y_{k+1} \equiv \sum_{i=0}^{k} Z_{i\,k} = \mu^{(k)}(0), \,\, k \in
[0, m-1],\,\, m\ge2, \eqno(4)
$$
{\it где слагаемые этой суммы соответственно равны}
 $$
Z_{0\,k} \equiv \zeta^{(k)}(0)
\big[f(0,0)+(a_2-a_1)\{\psi^{(1)}(0)+a_1\varphi^{(2)}(0)\}\big]+\xi^{(k)}(0)\times
$$
$$
\times\big[\psi^{(1)}(0) + a_1\varphi^{(2)}(0) \big] +
\alpha^{(k)}(0)\psi(0) + \beta^{(k)}(0)\varphi^{(1)}(0) +
\gamma^{(k)}(0)\varphi(0),
$$
$$
Z_{1\,k} \equiv k \Big\lbrace \zeta^{(k-1)}(0)
[f^{(0,1)}(0,0)+(a_2-a_1)f^{(1,0)}(0,0)+a_2(a_2-a_1)\times
$$
$$
\times\{\psi^{(2)}(0)+a_1\varphi^{(3)}(0)\}] +\xi^{(k-1)}(0)
[f^{(1,0)}(0,0)+a_2\{\psi^{(2)}(0)+a_1\varphi^{(3)}(0)\}]+
$$
$$ +
\alpha^{(k-1)}(0) [f(0,0)+(a_2-a_1)\psi^{(1)}(0) + a_1
a_2\varphi^{(2)}(0)] +
$$
$$
+
\beta^{(k-1)}(0)\psi^{(1)}(0)+\gamma^{(k-1)}(0)\psi(0)\Big\rbrace,
$$
$$
Z_{i\,k} \equiv \frac{k\,!}{i\,!\,(k-i)\,!}\Bigg< \zeta^{(k-i)}(0)
\Bigg\lbrace \sum _{s=1}^{i+1} \rho_{s-1} f^{(s-1,i-s+1)}(0,0)-
$$
$$
- a_1^2\sum _{s=1}^{i-1} \rho_{s-1}
f^{(s+1,i-s-1)}(0,0)+\Big[\eta_{i+1}-a_1^2\eta_{i-1}\Big] \varphi
^{(i+2)}(0) +
$$
$$
+\Big[\rho_{i+1}-a_1^2\rho_{i-1}\Big] \psi ^{(i+1)}(0)
\Bigg\rbrace +\xi^{(k-i)}(0) \Bigg\lbrace \sum _{s=1}^i \rho_{s-1}
f^{(s,i-s)}(0,0)+
$$
$$ + a_1\sum _{s=1}^{i-1}
\rho_{s-1} f^{(s+1,i-s-1)}(0,0)+\Big[\eta_{i}+a_1\eta_{i-1}\Big]
\varphi ^{(i+2)}(0)+
$$
$$ +
\Big[\rho_{i}+a_1\rho_{i-1}\Big] \psi ^{(i+1)}(0)\Bigg\rbrace +
\alpha^{(k-i)}(0) \sum _{s=1}^i \rho_{s-1} f^{(s-1,i-s)}(0,0)+
$$
$$
+\beta^{(k-i)} (0) \sum _{s=1}^{i-1}\rho_{s-1}
f^{(s,i-s-1)}(0,0)+\Big[\eta_i\alpha^{(k-i)}(0)+
\eta_{i-1}\beta^{(k-i)}(0) \Big]\times
$$
$$
\times\varphi ^{(i+1)}(0) + \Big[\rho_i\alpha^{(k-i)}(0)+
\rho_{i-1}\beta^{(k-i)}(0) \Big]\psi ^{(i+1)}(0)+
$$
$$
 + \gamma^{(k-i)}(0)\Bigg\lbrace \sum
_{s=1}^{i-1} \rho_{s-1} f^{(s-1,i-s-1)}(0,0)+\eta_{i-1} \varphi
^{(i)}(0)+
 \rho_{i-1} \psi ^{(i-1)}(0) \Bigg\rbrace \Bigg>,
 $$
 {\it где $i \in [2, k], \, k \in [2, m-1],$ и
$\rho_{j}$ и $\eta_{j}$ находятся реккурентно}
 $$
 \rho_{j}=(a_2-a_1)\rho_{j-1}+a_1 a_2\rho_{j-2}, \,j\geq 2,\, \rho_{0}=1, \, \rho_{1}=a_2-a_1,
 $$
 $$
 \eta_{j} = (a_2-a_1)\eta_{j-1}+a_1 a_2\eta_{j-2}, \,j\geq 2,\,\, \eta_0 = 0, \, \eta_{1} = a_1 a_2.
 $$


 Здесь справа сверху над коэффициентами ~$\zeta,\, \xi,\,
\theta,\, \alpha,\, \beta,\, \gamma$, начальными $\varphi,\,\psi$
и граничным $\mu$ данными задачи цифрой в круглых скобках
обозначены порядки производных по $x$ или $t.$ Аналогично
обозначены в круглых скобках через запятую соответственно порядки
частных производных по $x$ и $t$ от правой части $f.$

{\bf Идея доказательства.} Доказательство осуществляется методом
математической индукции. Чтобы получить условие (4) при $k=0$ в
равенстве (3) полагаем $t=0$ и используем правую часть уравнения,
начальные данные и характеристичность вторых производных. Для
получения условий (4) при $0<k\leq m-1$ равенство (3)
дифференцируется $k$ раз по $t$, вычисляются значения производных
от решения $u$ при $x=0, \, t=0$ с помощью начальных условий (2),
уравнения (1) и характеристичности вторых производных.


\smallskip \centerline{\bf Литература}\nopagebreak

1. \textit{Ломовцев Ф.Е.} {О необходимых и достаточных условиях
однозначной разрешимости задачи Коши для гиперболических
дифференциальных уравнений второго порядка с переменной областью
определения операторных коэффициентов./ Ф.Е.~Ломовцев~//
Дифференц. уравнения.~--- 1992.~--- Т. 28, № 5.~--- С. 873--886.}


2. \textit{Ломовцев Ф.Е.} {Нехарактеристическая смешанная задача
для одномерного волнового уравнения в первой четверти плоскости
при нестационарных граничных вторых производных. / Ф.Е.~Ломовцев,
В.В~Лысенко~// Веснiк Вiцебскага дзяржаунага унiверсiтэта, № 3
(104),~--- 2019.~--- С.~5--17.}
