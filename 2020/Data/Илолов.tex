\begin{center}
    {\bf ДРОБНЫЕ ДИФФЕРЕНЦИАЛЬНЫЕ ВКЛЮЧЕНИЯ С ИМПУЛЬСНЫМИ ВОЗДЕЙСТВИЯМИ}

    {\it М. Илолов}

    (Душанбе, Таджикистан; {\it ilolov.mamadsho@gmail.com})
\end{center}

\addcontentsline{toc}{section}{Илолов М.}

В работе доказана теорема существования решений задачи Коши для нелинейного дробного по времени импульсного дифференциального включения с почти секториальным оператором в линейной части. Соответствующий результат для линейных и полулинейных дробных дифференциальных уравнений в банаховом пространстве без импульсных воздействий установлен в работе [1].

1. Рассмотрим задачу Коши для дробных дифференциальных включений с импульсными воздействиями

$$^{c}D^{\alpha}u(t)\in Au(t)+F(t,u(t)), t\in J=[0,T],t\neq t_{k}, k=1,...,m, \eqno{(1)}$$

$$\triangle u\bigg|_{t=t_{k}}=I_{k}(u(t_{k}^{-})), k=1, ..., m, \eqno {(2)}$$

$$u(0)=u_{0},\eqno{(3)}$$

где $^{c}D^{\alpha}$ дробная производная Капуто, $0<\alpha<1, A$ почти секториальный оператор порождающий сингулярную полугруппу порядка $1+\gamma(-1<\gamma<0)$ при $t=0,F:J\times E\rightarrow P(E)$ многозначное отображение, $P(E)$ семейство непустых подмножеств банахово пространство $E,I_{K}:E\rightarrow E,k=1,...,m$ и $u_{0}\in E$, $0=t_{0}<t_{1}<...<t_{m}<t_{m+1}=T, \triangle u\bigg|_{t=t_{k}}=u(t_{k}^{+})-u(t_{k}^{-})$, $u(t_{k}^{+})=\lim\limits_{h\rightarrow 0+}u(t_{k}+h), u(t_{k}^{-})=\lim\limits_{h\rightarrow 0-}u(t_{k}+h)$ являются правыми и левыми пределами $u(t)$ при $t=t_{k}, k=1, ...,m$.

Аналогичные дифференциальные уравнения целого порядка с импульсными воздействиями и со секториальными операторами были предметом анализа работы [2]. В работе [3] рассматривались нелокальные дробные дифференциальные включения со секториальным оператором. В [1] приводится пример эллиптического дифференциального оператора $A'$ действующего в пространстве функций Гёльдера $C^{l}(\bar{\Omega}), 0<l<1$, где $\Omega$-ограниченная область в $R^{n}(N\geq1)$ с достаточно гладкой границей $\partial\Omega$, такой что $A'+\nu, \nu>0$ не является секториальным, но является почти секториальным оператором. Почти секториальный оператор А генерирует полугруппу $T(t)$ порядка роста $1+\gamma$ со сингулярным поведением при $t=0$.

Обозначим через $E_{\alpha,\beta}$ обобщённую функцию типа Миттаг-Лефлера в виде

$$E_{\alpha,\beta}(z)=\sum\limits_{n=0}^{\infty}\frac{z^{n}}{\Gamma(\alpha n+\beta)}=\frac{1}{\pi i}\int\limits_{L}\frac{\lambda^{\alpha\mp\beta}l^{\lambda}}{\lambda^{\alpha-z}}d\lambda,\alpha,\beta>0,z\in C, \eqno{(4)}$$
где контур $L$ начинается и заканчивается в $-\infty$ и обходит диск $|\lambda|<|z|^{1/\alpha}$ против часовой стрелки. Определим также функцию

$$e_{\alpha,\beta}(z)=-\sum\limits_{n=1}^{N-1}\frac{z^{-n}}{\Gamma(\beta-\alpha n)}+O(|z|^{-N}),\;z\rightarrow\infty. \eqno{(5)}$$

С помощью функций (4),(5) и соответствующей резольвенты оператора А вводим семейство операторозначных функций

$$S_{\alpha}(t)=E_{\alpha}(-zt^{\alpha})(A)=\frac{1}{2\pi i}\int\limits_{\Gamma_{\theta}}E_{\alpha}(-zt^{\alpha})R(z,A)dz, E_{\alpha}(z)=E_{\alpha,\alpha}(z)$$
и

$$T_{\alpha}(t)=e_{\alpha}(-zt^{\alpha})(A)=\frac{1}{2\pi i}\int\limits_{\Gamma_{\theta}}e_{\alpha}(-zt^{\alpha})R(z,A)dz, e_{\alpha}(z)=e_{\alpha,\alpha}(z),$$
где интегральный контур $\Gamma_{\theta}=\{R_{+}l^{i\theta}\}\cup\{R_{+}l^{-i\theta}\}$ направлен против часовой стрелки и $\omega<\theta<\pi/2-|arg t|$.

\textbf{Лемма 1.} Для каждого $t>0, S_{\alpha}(t)$ и $T_{\alpha}(t)$ являются линейными ограниченными операторами на $E$. Более того существуют постоянные $C_{S}=C(\alpha,\gamma)>0$, $C_{T}=C(\alpha,\gamma)>0$ такие, что

$$\|S_{\alpha}(t)\|\leq C_{S}t^{-\alpha(1+\gamma)},\|T_{\alpha}(t)\|\leq C_{T}t^{-\alpha(1+\gamma)}.$$


2. Пусть $F$ в (1) является компактным многозначным отображением.

Рассмотрим пространство кусочно непрерывных функций

$$PC(J,E)=\{u:J\rightarrow E, u\in C((t_{k},t_{k+1}), E), k=0, ..., m+1 $$
$$u(t_{k}^{-}) , u(t_{k}^{+}) , u(t_{k}^{-})=u(t_{k}), k=0, ..., m\}$$
с нормой

$$\|u\|_{PC}=\sup\limits_{t\in J} \|u(t)\|.$$

Через $AC((t_{k},t_{k+1}), E)$ обозначим пространство абсолютно непрерывных функций. Пусть $J'=J/\{t_{1}, .., t_{m}\}$.

\textbf{Определение 1.} Функция $u\in PC(J,E)\; \cap \cup_{k=o}^{m}AC((t_{k},t_{k+1}),E)$ с дробной производной порядка $\alpha$ называется слабым решением задачи (1)-(3), если существует функция $v\in L^{1}([0,T], E)$ такая, что $v(t)\in Av(t)+F(t,v(t))$ для п.в. $t\in [0,T]$ и

$$u(t)=S_{\alpha}(t)x_{0}+\sum\limits_{k=1}^{m}s_{\alpha}(t-t_{k})I_{k}(u(t_{k}^{-}))+\int\limits_{0}^{t}T_{\alpha}(t-s)v(s)ds,$$
где $v$ является селектором для многозначного отображения $F(t,u(t))$.

Приводим основной результат для почти секториального оператора А.

\textbf{Теорема 1.} Пусть А почти секториальный оператор и пусть $F:J\times E\rightarrow P(E)$ непустое выпуклое компактное многозначное отображение. Пусть выполнены следующие предположения:

(HF1) Для каждого $u\in E$, функция $t\rightarrow F(t,u)$ измеримая и для п.в. $t\in J$ полунепрерывная сверху;

(HF2) Существует функция $\varphi\in L^{1/q}(J,R_{+}), q\in(0,\alpha)$ и неубывающая непрерывная функция $\psi:R_{+}\rightarrow R_{+}$ такая, что для каждой $u\in E$,

$$\|F(t,u)\|\leq\varphi(t)\psi(\|u\|), t\in J;$$

(HF3) Существует функция $\beta\in L^{1/q}(J,R_{+}), qt(0,\alpha)$ удовлетворяющей неравенству

$$2\eta C_{T}\|\beta\|_{L^{1/q}(J,R_{+})}<1,$$
где $\eta=\frac{T^{\alpha-q}}{\bar{\omega}^{1-q}}$ и $\bar{\omega}=\frac{\alpha-q}{1-q}$, и для каждого ограниченного подмножества $D\subset E$

$$\chi(F(t,D))\leq\beta(t)\chi(D)$$
для п.в. $t\in J$, где $\chi$ мера некомпактности Хаусдорфа в $E$.

(HI) Для каждого $k=1, ..., m, I_{k}$ непрерывный и компактный оператор и существуют положительные постоянные $h_{k}$ такие, что

$$\|I_{k}(u)\|\leq h_{k}\|u\|, u\in E.$$


Тогда задача (1)-(3) имеет слабое решение при условии, что

$$C_{S}(\|u_{0}\|+hr)+C_{T}n\psi(r)\|\varphi\|_{L^{1/2}(J,R_{+})}\leq r,$$
где $r>0$ и $h=\sum\limits_{k=i}^{m}h_{k}.$

Доказательство теоремы состоит в сведении задачи (1),(2),(3) к задаче о неподвижной точке многозначного отображения $R:P(C(J,E))\rightarrow 2^{P(C(J,E))}$. Для $u\in P(C(J,E))$ $R(u)$ является множеством всех функций $v\in R(u)$ такие, что

$$v(t)=S_{\alpha}(t)u_{0}+\sum\limits_{k=1}^{n}S_{\alpha}(t-t_{k})I_{k}(u(t_{k}^{-}))+\int\limits_{0}^{t}(t-s)f(s)ds,$$
где $f$ является селектором многозначного отображения $F(t,u(t))$. Легко видеть, что любая неподвижная точка отображения $R(u)$ является слабым решением задачи (1)-(3).

Таким образом задача сводится к доказательству существования неподвижной точки многозначного отображения $R(u)$. Существование неподвижной точки многозначного отображения $R(u)$ подробным образом исследовано в работе [4].










% Оформление списка литературы
\smallskip \centerline {\bf Литература} \nopagebreak

1. {\it Wang R.-N., Chen D.-H., Xiao T.-J.} Abstract fractional Cauchy problems with almost sectorial operators. J. Differential Equations. 2012. v.252. — pp. 202-235

2. {\it Самойленко А.М., Илолов М.} К теории неоднородных по времени эволюционных уравнений с импульсными воздействиями. Докл.АН СССР. 1991. т. 319, №1. — с.63-67

3. {\it Wang J., Ibrahim A.G. and Feckan M.} Nonlocal impulsive fractional differential inclusions with fractional sectorial operators on Banach Spaces. Appl.Math.Comput. 2015. v.257. — pp. 103-118

4. {\it Гельман Б.Д., Обуховский В.В.} О неподвижных точках многозначных отображений ациклического типа. Фундаментальная и прикладная математика. 2015. т.20, №3. — с.47-59
