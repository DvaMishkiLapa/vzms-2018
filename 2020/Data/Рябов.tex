\vzmstitle[\footnote{Работа первого автора поддержана Российским научным фондом (№ 19-71-30012).}]{ТОПОЛОГИЧЕСКИЙ АТЛАС ОДНОЙ ИНТЕГРИРУЕМОЙ СИСТЕМЫ С ТРЕМЯ СТЕПЕНЯМИ СВОБОДЫ}
\vzmsauthor{Рябов}{П.\,Е.}
\vzmsauthor{Каверина}{В.\,К.}
\vzmsinfo{Москва; {\it PERyabov@fa.ru}; {\it VKKaverina@fa.ru}}
\vzmscaption


В докладе представлены задачи и результаты исследования фазовой топологии интегрируемых систем с двумя и тремя степенями свободы из динамики твёрдого тела, допускающих представление Лакса. Основой таких исследований послужило понятие топологического атласа, введённое М.~П.~Харламовым в начале 2000-х гг. для неприводимых интегрируемых систем с тремя степенями свободы [1]. Топологический атлас включает аналитическое описание критических подсистем полного отображения момента, каждая из которых при фиксированных физических параметрах является почти гамильтоновой системой с меньшим числом степеней свободы; классификацию оснащённых изоэнергетических диаграмм Смейла с полным описанием регулярных торов Лиувилля и их бифуркаций; определение типов всех критических точек полного отображения момента и программы"=конструктора построения топологических инвариантов. Как оказалось, к настоящему моменту локальное и полулокальное исследование критических подсистем является эффективным средством для конструирования грубых топологических инвариантов. Для интегрируемых гамильтоновых систем с $n$ степенями свободы с полиномиальными или рациональными интегралами множество критических значений отображения момента $\cal F$ может быть записано в виде $P=0$, где $P$ "--- полином от фазовых переменных. Разложение его на неприводимые сомножители $P=\prod\nolimits_j {L_j}$ приводит к определению критической подсистемы ${\cal M}_j$ как множества критических точек нулевого уровня некоторой функции $L_j$. Оказывается, критическая точка ранга $k$ локально является точкой пересечения $n-k$ подобластей критических подсистем. Интегралы $L_j$ этих подсистем порождают симплектические операторы ${\cal A}_{L_j}$, которые определяют тип критической точки. Бифуркации, которые возникают при пересечении поверхностей ${\cal F}(M_j)$ в точке
${\cal F}(x)$, порождают полулокальный тип критической точки. Такой подход приводит к аналитическому описанию топологических инвариантов исключительно в терминах первых интегралов. Для некоторых интегрируемых задач динамики твёрдого тела (волчок Ковалевской в двойном поле сил, интегрируемый случай Ковалевской-Соколова, интегрируемый случай Ковалевской-Яхья) удалось эффективно реализовать программу построения топологического атласа [2], [3], [4].

%%%%%%%%%%%%%%%%%%%%%%
В докладе также представлены некоторые результаты построения топологического атласа интегрируемой системы с тремя степенями свободы на ко"=алгебре Ли ${e(3,2)}^*$, которая описывает динамику двухполевого обобщённого гиростата при наличии двух силовых полей (случай интегрируемости Соколова-Цыганова) [5]. Это один из наиболее общих найденных на сегодня случаев интегрируемости гиростата в двойном поле с условиями типа Ковалевской и гироскопическими силами с непостоянным гироскопическим моментом. В общем случае не удавалось даже выписать в обозримом виде дополнительные интегралы.

Речь идёт о следующей системе обыкновенных дифференциальных уравнений
\begin{equation*}
\begin{array}{l}
 \displaystyle{\dot{\boldsymbol M}={\boldsymbol M}\times\frac{\partial H}{\partial{\boldsymbol
M}}+ {\boldsymbol\alpha}\times\frac{\partial H}{\partial{\boldsymbol\alpha}}+
{\boldsymbol\beta}\times\frac{\partial H}{\partial{\boldsymbol\beta}},}\\[3mm]
\displaystyle{\dot{\boldsymbol \alpha}={\boldsymbol\alpha}\times\frac{\partial
H}{\partial{\boldsymbol M}},\quad \dot{\boldsymbol
\beta}={\boldsymbol\beta}\times\frac{\partial H}{\partial{\boldsymbol M}}}
\end{array}
\eqno (1)
\end{equation*}
c гамильтонианом
\begin{equation*}
\begin{array}{l}
\displaystyle{H= M_1^2+M_2^2+2M_3^2+2\lambda M_3-2\varepsilon_2(\alpha_1+\beta_2)}\\[3mm]
{\displaystyle\qquad+2\varepsilon_1(M_2\alpha_3-M_3\alpha_2+M_3\beta_1-M_1\beta_3).}
\end{array}
\eqno (2)
\end{equation*}

Здесь трёхмерные векторы ${\boldsymbol M}, {\boldsymbol\alpha}, {\boldsymbol\beta}$
представляют собой проекции кинетического момента и двух силовых полей на оси, жёстко
связанные с твёрдым телом; $\lambda$ "--- параметр гиростатического момента, направленного
вдоль оси динамической симметрии; $\varepsilon_1$ и $\varepsilon_2$ "--- параметры деформации. Система с таким гамильтонианом, существенно зависящими от
${\boldsymbol\alpha}$ и ${\boldsymbol\beta}$, не допускает непрерывной группы симметрии,
и поэтому неприводима глобально к семейству систем с двумя степенями свободы.

Соответствующая скобка Ли--Пуассона задаётся формулами
\begin{equation*}
\begin{array}{l}\{M_i,M_j\}=\varepsilon_{ijk}M_k, \{M_i,\alpha_j\}=\varepsilon_{ijk}\alpha_k,\\[3mm]
\{M_i,\beta_j\}=\varepsilon_{ijk}\beta_k, \{\alpha_i,\alpha_j\}=0,\\[3mm]
\{\alpha_i,\beta_j\}=0, \{\beta_i,\beta_j\}=0, \\[3mm]
\varepsilon_{ijk}=\frac{1}{2}(i-j)(j-k)(k-i), 1\leqslant i,j,k\leqslant 3.
\end{array}
\eqno (3)
\end{equation*}

Функциями Казимира являются выражения ${\boldsymbol\alpha}^2$,
${\boldsymbol\alpha}\cdot{\boldsymbol\beta}$ и ${\boldsymbol\beta}^2$.

Относительно
скобки Ли--Пуассона, заданной соотношениями $(3)$, систему $(1)$ можно
представить в гамильтоновом виде
\begin{equation*}
\dot x=\{H,x\},
\end{equation*}
где через $x$ обозначена любая из координат.

Фазовое пространство $\cal P$ системы уравнений $(1)$ задаётся общим уровнем функций Казимира
\begin{equation*}
\boldsymbol\alpha^2=a^2,\quad \boldsymbol\beta^2=b^2,\quad
{\boldsymbol\alpha}\cdot{\boldsymbol\beta}=c, \quad (0<b<a, |c|<ab).
\end{equation*}

Для гамильтониана $(2)$ необходимые для интегрируемости по Лиувиллю два дополнительных интеграла $K$ и $G$ имеют следующий явный вид [6]:
\begin{equation*}
\begin{array}{l}
K=Z_1^2+Z_2^2-\lambda[(M_3+\lambda)(M_1^2+M_2^2)+2\varepsilon_2(\alpha_3M_1+\beta_3M_2)]\\[3mm]
\qquad+\lambda\varepsilon_1^2({\boldsymbol\alpha}^2+{\boldsymbol\beta}^2)M_3+\\[3mm]
+2\lambda\varepsilon_1[\alpha_2M_1^2-\beta_1M_2^2-(\alpha_1-\beta_2)M_1M_2]
-2\lambda\varepsilon_1^2\omega_\gamma,\\[5mm]
G=\omega_\alpha^2+\omega_\beta^2+2(M_3+\lambda)\omega_\gamma-
2\varepsilon_2({\boldsymbol\alpha}^2\beta_2+{\boldsymbol\beta}^2\alpha_1)+\\[3mm]
+2\varepsilon_1[{\boldsymbol\beta}^2(M_2\alpha_3-M_3\alpha_2)-
{\boldsymbol\alpha}^2(M_1\beta_3-M_3\beta_1)]\\[3mm]
+2({\boldsymbol\alpha}\cdot{\boldsymbol\beta})[\varepsilon_2(\alpha_2+\beta_1)+\varepsilon_1(\alpha_3M_1-\alpha_1M_3+\beta_2M_3-\beta_3M_2)],
\end{array}
\end{equation*}
где
\begin{equation*}
\begin{array}{l}
Z_1=\frac{1}{2}(M_1^2-M_2^2)+\varepsilon_2(\alpha_1-\beta_2)+\\[3mm]
\qquad+\varepsilon_1[M_3(\alpha_2+\beta_1)-M_2\alpha_3-M_1\beta_3]+
\frac{1}{2}\varepsilon_1^2({\boldsymbol\beta}^2-{\boldsymbol\alpha}^2),\\[3mm]
Z_2=M_1M_2+\varepsilon_2(\alpha_2+\beta_1)-\\[3mm]
\qquad-\varepsilon_1[M_3(\alpha_1-\beta_2)+\beta_3M_2-\alpha_3M_1]-\varepsilon_1^2(
{\boldsymbol\alpha}\cdot{\boldsymbol\beta}),\\[3mm]
\omega_\alpha=M_1\alpha_1+M_2\alpha_2+M_3\alpha_3,\\[3mm]
\omega_\beta=M_1\beta_1+M_2\beta_2+M_3\beta_3,\\[3mm]
\omega_\gamma=M_1(\alpha_2\beta_3-\alpha_3\beta_2)+\\[3mm]
\qquad +M_2(\alpha_3\beta_1-\alpha_1\beta_3)+
M_3(\alpha_1\beta_2-\alpha_2\beta_1).
\end{array}
\end{equation*}

%%%%%%%%%%%%%%%%%%%%%%%%

В докладе предложен подход к описанию фазовой топологии такой системы. В явном виде описываются некоторые критические подсистемы с указанием бифуркаций торов Лиувиля [6], [7], [8], а также некоторые оснащённые изоэнергетические диаграммы. На сегодняшний день для указанной системы определены порядка 150 оснащённых изоэнергетических диаграмм полного отображения момента с указанием всех камер, семейств регулярных 3-мерных торов и их 4-мерных бифуркаций.



% Оформление списка литературы
\smallskip \centerline {\bf Литература} \nopagebreak

1. {\it Kharlamov M.P.} Bifurcation diagrams of the Kowalevski top in two constant fields // Regular and Chaotic Dynamics. 2005. Vol.~10, \No~4. P.~381--398.

2. {\it Kharlamov M.P., Ryabov P.E.} Topological atlas of the Kovallevskaya
top in a double field // Journal of Mathematical Sciences (United
States). 2017. Vol.~223, \No.~6. P.~775--809.

3. {\it Kharlamov M.P., Ryabov P.E., Savushkin A.Y.} Topologi\-cal Atlas of the
Kowalevski-Sokolov~// Regular and Chaotic Dynamics. 2016. Vol.~21, \No.~1. P.~24--65.

4. {\it Kharlamov M.P., Ryabov P.E., Kharlamova I.I.} Topo\-lo\-gical At\-las of the Kovalevskaya-Yehia Gyrostat // Journal of Mathe\-ma\-tical Sciences (United States). 2017. Vol.~227, \No.~3. P.~241--386.

5. {\it Sokolov V.V., Tsiganov A.V.} Lax Pairs for the Deformed Kowalevski and Goryachev-Chaplygin Tops // Theoretical and Mathematical Physics. 2002. Vol.~131, \No.~1. P.~543--549.

6. {\it Ryabov P.E.} Phase topology of one irreducible integrable problem in the dynamics of a rigid body // Theoretical and Mathematical Physics. 2013. Vol.~176, \No.~2. P.~1000--1015.

7. {\it Ryabov P.E.} New invariant relations for the generalized two-field gyrostat // Journal of Geometry and Physics. 2015. Vol.~87. P.~415--421.

8. {\it Sokolov S.V.} New invariant relations for one critical sub\-system of a generalized two-field gyrostat // Doklady Physics. 2017. Vol.~62, \No.~12. P.~567--570.
