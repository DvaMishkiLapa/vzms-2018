\begin{center}
    {\bf КЛАССИФИКАЦИЯ ЧАСТИЧНО СИММЕТРИЧНЫХ АТОМОВ\footnote{Исследование выполнено в рамках Программы Президента Российской Федерации для государственной поддержки ведущих научных школ РФ (грант НШ-2554.2020.1).}}\\

    {\it В.А.Трифонова}

    (Москва; {\it trifonovaviktoriya2012@yandex.ru})
\end{center}

\addcontentsline{toc}{section}{Трифонова В.А.}

Понятие атома для целей гамильтоновой и симплектической геометрии и топологии было введено А.Т. Фоменко и использовалось для лиувиллевой классификации интегрируемых гамильтоновых систем.


Симметрии атомов отражают дискретные симметрии соответствующих динамических систем, так что для анализа важной является задача описания классов атомов, обладающих заданной группой симметрии. Ранее был получен ряд классификационных результатов, в которых входит описание максимально симметричных атомов, имеющих максимально возможный набор симметрий. Задача классификации максимально симметричных атомов является довольно сложной и может быть решена только для отдельных семейств атомов (атомы малой сложности, атомы малого рода), либо атомов, обладающих некоторым специальным свойством.

В настоящей работе рассматриваются высотные атомы с группой симметрий, транзитивной на кольцах одного цвета (белого). Для таких атомов удалось получить полное описание: предъявлено 22 бесконечные серии.




Пусть $M^2$ - гладкое компактное двумерное многообразие, $f:M^2 \to \mathbb{R}$ - функция Морса	на $M^2$ и \{$x\in M^2 \colon f(x)=k $\}, $k \in \mathbb{R}$, - её особый уровень. Тогда существует ${\varepsilon > 0}$ такое, что ${f^{-1}([k-\varepsilon, k+\varepsilon ])}$ не содержит особых точек, кроме лежащих на особом уровне $\{f=k\}$.



		{\bf Определение.}
		{\em Атомом} называется пара ${(f^{-1}([k-\varepsilon, k+\varepsilon ]), f^{-1}(k))}$ с указанием вложения графа ${f^{-1}(k)}$ в поверхность ${f^{-1}([k-\varepsilon, k+\varepsilon ])}$, где граф ${f^{-1}(k)}$ предполагается конечным и связным. Если на поверхности ${f^{-1}([k-\varepsilon, k+\varepsilon ])}$ фиксирована ориентация, то соответствующий атом называется {\em ориентированным.} Граф ${f^{-1}(k)}$ называется остовом атома. Два атома (соответственно ориентированных атома) считаются {\em изоморфными}, если существует гомеоморфизм пар, который переводит поверхность в поверхность (сохраняя ориентацию, если поверхность ориентирована), остов в остов, а функцию переводит в функцию. Будем говорить, что атом
${(f^{-1}([k-\varepsilon, k+\varepsilon ]),f^{-1}(k))}$ {\em порождён} функцией $f$.




\textbf{Определение. }
	%
	Назовём атом, порождённый функцией $f$, {\em высотным}, если существует такое вложение $g
	\colon f^{-1}([k-\varepsilon, k+\varepsilon ]) \to \mathbb{R}^3$, что $f(p) = z(g(p))$ для каждой точки $p \in f^{-1}([k-\varepsilon, k+\varepsilon ])$, где $z$
	--- стандартная координата в пространстве $\mathbb R^3$, т.е. $z$ --- функция высоты на $g(f^{-1}([k-\varepsilon, k+\varepsilon ]))$.

			%Замечание. Все высотные атомы являются ориентируемыми (см. [2]). Далее под атомом будем понимать ориентированный атом.



\textbf{Определение.} Эквивалентным образом атом можно задать как <<оснащённую пару>> $(P^2, K)^\#$, где $P^2$ --- компактная   поверхность с краем, $K$ -- непустой конечный связный граф, вложенный в $P^2$ и имеющий вершины  степени $4$, причём множество $P^2 \setminus K$ является несвязным объединением колец $S^1 \times (0, 1]$, $ S^1 \times \{1\} \subset\partial P^2$, и множество колец разбито на два подмножества (белые и чёрные кольца) таким образом, что  к каждому ребру графа $K$ примыкают ровно одно белое кольцо и ровно одно чёрное кольцо. Указанное разбиение колец на белые и чёрные называется {\em  оснащением} пары $(P^2, K)$, и оснащённая пара обозначается через $(P^2, K)^\#$. Две оснащённые пары считаются {\em изоморфными}, если существует гомеоморфизм пар, сохраняющий ориентацию поверхностей и раскраску колец.


Далее под атомом будем понимать оснащённую пару $(P^2, K)^\#$ с фиксированной ориентацией на поверхности $P^2$.



Атом может быть определён также как $f$-граф, что в свою очередь даёт нам возможность работать с атомами как с комбинаторными объектами.



\textbf{Определение.} %
Конечный связный граф $G$, некоторые ребра которого ориентированы, назовём {\em
$f$-графом}, если все его вершины имеют степень $3$, причём к каждой его
вершине примыкают ровно два ориентированных полуребра, из которых одно входит в вершину,
а другое выходит из неё. Отметим, что вершина может быть началом и концом одного и того
же ориентированного ребра. Каждое неориентированное ребро в $f$-графе, концы которого лежат на одном ориентированном цикле, будем называть { \em хордой}.



{\bf Определение.}
	%
	{\em Симметрией атома} называется изоморфизм атома на себя, рассматриваемый с точностью до изотопии. Будем говорить, что высотный атом является {\em частично симметричным}, если для любых двух колец белого цвета $u$, $v$ указанного оснащения найдётся симметрия атома $\phi $, такая, что $\phi$($u$) = $v$.



	Обозначим через $P$ множество, состоящее из следующих графов: одна вершина, одно ребро, простой цикл, сети правильных призм и антипризм, Платоновых и Архимедовых тел (исключая ромбоусеченный икосододекаэдр и усечённый кубоктаэдр).



 Оказывается, что любой частично симметричный атом изоморфен ровно одному из атомов, чьи $f$-графы получаются из графов множества $P$ заменой вершины графа на ориентированный цикл и добавлением кратных неориентированных рёбер и хорд <<особым образом>>.






% Оформление списка литературы
\smallskip \centerline {\bf Литература} \nopagebreak

1. {\it Болсинов А. В., Фоменко А. Т.} Интегрируемые гамильтоновы системы, т. 1, // Ижевск: Изд. дом <<Удмуртский университет>> 444 с., (1999).

2. {\it Herbert Fleischner, Wilfried Inrih.} Transitive planar graphs, Mathematica slovaca, 1979, Vol.29
