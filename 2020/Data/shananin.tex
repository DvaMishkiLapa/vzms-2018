\begin{center}
    {\bf К ПРОДОЛЖЕНИЮ РОСТКОВ РЕШЕНИЙ КВАЗИЛИНЕЙНЫХ КВАЗИЭЛЛИПТИЧЕСКИХ УРАВНЕНИЙ ВТОРОГО ПОРЯДКА\footnote{Публикация была подготовлена по проекту № 2 в рамках договора пожертвования
от 01  марта 2019 г. № 1154}}\\


    {\it Н.А. Шананин}

    (Москва; {\it nashananin@inbox.ru})
\end{center}

\addcontentsline{toc}{section}{Шананин Н.А.}



Статья содержит описание некоторых свойств ростков гладких решений квазилинейных уравнений вида
$$
(P(u)=)u_t+
\sum_{|\alpha|=2}
~a_{\alpha}(t,x,u,u_x)~D^{\alpha}u=f(t,x,u,u_x),\eqno(1)
$$
где
$(t,x)=( t, x_1,\dots,x_n)\in \Omega~\subseteq~{\mathcal R}^{n+1}$,
$D_{j}=\frac{1}{i}\frac{\partial}{\partial x_{j}},\, j=1,\dots,n$,
$u_x=( u_{x_1},\dots,  u_{x_1})$,
$a_{\alpha}(t,x,\zeta)\in C^{\infty}(\Omega\times{\mathcal C}^{n+1})$  и
$f_{\alpha}(t,x,\zeta)\in C^{\infty}(\Omega\times{\mathcal C}^{n+1})$.
Операторам дифференцирования $D_j$ поставим в соответствие вес 1, а оператору
$D_t$\,--\,вес~2. Пусть $v(t,x)\in C^{\infty}(V)$, где $V\subset\Omega$.
Тогда в обозначениях и терминах статьи~[1] взвешенный главный символ на функции $v$ имеет вид:
$$
p_{v,2}(t,x;\tau,\xi)=i\tau+\sum_{|\alpha|=2}
~a_{\alpha}(t,x,v(t,x),v_x(t,x))~\xi^{\alpha},
$$
где $\tau\in{\mathcal R}$ и $\xi\in {\mathcal R}^{n}$,
и, поскольку минимальный вес оператора однократного дифференцирования равен 1, совпадает с пучком старших символов ${\mathcal H}_{v}(t,x;\tau,\xi,h)$,
определённым на $v(x)$.
Мы говорим, что росток $u_{(t^0,x^0)}$ в точке $x^0\in \Omega$ удовлетворяет уравнению (1) и писать $(P(u)-f(u))_{(t^0,x^0)}\cong 0$, если для любой бесконечно дифференцируемой функции $u(t,x)$, представляющей росток, найдётся такая окрестность  точки $(t^0,x^0)$, в которой функция $u(t,x)$ является локальным решением уравнения.  Мы говорим, что уравнение (1) является квазиэллиптическим на ростке  $u_{(t^0,x^0)}$, если для любой представляющей росток функции $u(t,x)$ найдётся такая окрестность $U$ точки $(t^0,x^0)$, в которой из равенства $p_{u,2}(t,x;\tau,\xi)=0$ при любых $(t,x)\in U$
следует, что $\tau=0$ и $\xi=0$.
Две гиперповерхности $\Gamma_1$ и $\Gamma_2$ называют эквивалентными в точке $(t^0,x^0)$, если найдётся окрестность $V$ этой точки, такая,
что $\Gamma_1\cap V=\Gamma_2\cap V$. Класс эквивалентных в точке $(t^0,x^0)$
гиперповерхностей называют ростком гиперповерхности.
Росток гиперповерхности $\Gamma_{(t^0,x^0)}$ назовём нехарактеристическим
для оператора $P$ на функции $v$, если
$p_{v,2}(t^0,x^0;0,\varphi_{x}(t^0,x^0))\not=0$
для некоторой (а значит и любой) гиперповерхности
$\Gamma= \{~(t,x) \in U \vert~\varphi (t,x) =0, d\varphi\not=0\}$, представляющей росток $\Gamma_{(t^0,x^0)}$.
Если уравнение (1) является квазиэллиптическим на функции $v$ в точке $(t^0,x^0)$, то росток гиперповерхности в точке $(t^0,x^0)$ является нехарактеристическим, если и только если вектор конормали $(\tau,\xi)$ в этой точке к представителям ростка удовлетворяет условию: $\xi\not=0$.
Говорят, что сужения ростков функций $u_{(t^0,x^0)}$ и $w_{(t^0,x^0)}$
на росток гиперповерхности $\Gamma_{(t^0,x^0)}$  равны и писать
$u_{\Gamma_{(t^0,x^0)}}\cong w_{\Gamma_{(t^0,x^0)}}$,
если для некоторых представителей ростков $u(t,x)$ и $w(t,x)$   и некотрого представителя $\Gamma$ ростка гиперповерхности (а, следовательно, и для любых)
найдётся такая окрестность $V$ точки $(t^0,x^0)$,
что $(u-w)|_{\Gamma\cap V}=0$.\linebreak Ростки решений квазиэллиптических на "фоновом решении"\linebreak уравнений вида (1) однозначно определяются сужениями на нехарактеристические гиперповерности:


\textbf{Теорема~1.} {\it
Предположим, что ростки функций $v_{(t^0,x^0)}$ и $w_{(t^0,x^0)}$ удовлетворяют уравнению $(P(u)-f(u))_{(t^0,x^0)}\cong 0$, причём уравнение
является квазиэллиптическим на ростке  $v_{(t^0,x^0)}$ и, кроме того, на нехарактеристическом на $v_{(t^0,x^0)}$ ростке гиперповерхности
 $\Gamma_{(t^0,x^0)}$ выполняются равенства
$v_{\Gamma_{(t^0,x^0)}}\cong w_{\Gamma_{(t^0,x^0)}}$ и
$(D_ju)_{\Gamma_{(t^0,x^0)}}\cong (D_jw)_{\Gamma_{(t^0,x^0)}}$ при
$j=1,\dots,n$. Тогда $v_{(t^0,x^0)}\cong w_{(t^0,x^0)}$.
}

Доказательство. Предположим, что $n>1$ и ковектор $(0,\eta)\in T^{\ast}_{(t^0,x^0)}(\Omega)\setminus 0$. Возьмём произвольную функцию $v(x)$,   представляющую росток $v_{(t^0,x^0)}$. Отметим, что
множество ковекторов
${\mathcal M}_{(0,\eta)}=\{(\tau,\xi)\in T^{\ast}_{(t^0,x^0)}(\Omega)~|~
(\tau,\xi) \not\parallel  (0,\eta)\}$ при $n>1$ является связным. Вследствие квазиэллиптичности на  $v_{(t^0,x^0)}$
 многочлен $p_{v,2}(t^0,x^0;\tau,z\eta+\xi)$ по переменной $z\in {\mathcal C}$ не имеет вещественных корней. Нетрудно проверить, что из того, что
число $z^0$ является коренем многочлена для ковектора $(\tau,\xi)\in {\mathcal M}_{(0,\eta)}$, следует, что число $(-z^0)$ является  коренем многочлена для  ковектора $(\tau,-\xi)\in {\mathcal M}_{(0,\eta)}$. Отсюда и из непрерывной зависимости корней вытекает, что для каждой неколлинеарной пары ковекторов
$(0,\eta)$ и $(\tau,\xi)$ характеристический многочлен имеет два простых корня, мнимые части которых противоположны по знаку. Отсюда вытекает, что для каждой неколлинеарной пары ковекторов $(0,\eta^0)$ и $(\tau^0,\xi^0)$ существует окрестность $W\subset \Omega\times{\mathcal R}^n\times{\mathcal R}^{n+1}$ точки $(t^0,x^0,\eta^0,\tau^0,\xi^0)$, в которой характеристическое уравнение
$$
p_{v,2}(t,x;\tau,z\eta+\xi)=0,\,z\in{\mathcal C},
$$
имеет ровно два простых комплексных корня, причём мнимая часть каждого из корней отлична от нуля. При $n=1$ указанное свойство корней очевидно. Теперь утверждение доказываемой теоремы следует из теоремы 1 статьи [1].

%Теореме 1 нетрудно придать вид теоремы единственности решения задачи Коши.
Рассмотрим вопрос об однозначном продолжении ростков решений уравнений вида (1) вдоль кривых.
Пусть $\gamma=\{(t^0,x(s))~|~s\in(0,1)\}$ \,--\,непрерывный путь, содержащийся в слое $\Omega\cap\{t=t^0\}$.  Мы говорим, что функция $u$ удовлетворяет уравнению (1) вдоль кривой $\gamma$, если функция $u(t,x)$ определена и бесконечно дифференцируема в некоторой окрестности пути $\gamma$  и в каждой точке $(t^0,x)\in\gamma$ удовлетворяет равенству $(P(u)-f(u))_{(t^0,x)}\cong 0$.

\textbf{Теорема~2.} {\it
Предположим, что функции $v$ и $w$ удовлетворяют уравнению {\rm (1)} вдоль кривой $\gamma$ и уравнение
является квазиэллиптическим на ростках  $v_{(t^0,x)}$ для всех $(t^0,x)\in\gamma$. Тогда из $v_{(t^0,x^0)}\cong w_{(t^0,x^0)}$ в точке
$(t^0,x^0)\in\gamma$ следует $v_{(t^0,x)}\cong w_{(t^0,x)}$ во всех точках
$(t^0,x)\in\gamma$.
}

Доказательство. Пусть $(t^0,x^1)$\,--\,произвольная точка пути $\gamma$. Тогда
найдётся окрестность $U$ части пути, соединяющей точки  $(t^0,x^0)$ и  $(t^0,x^1)$, в которой функции $v$ и $w\in C^{\infty}(U)$, удовлетворяют уравнению (1) в каждой точке и уравнение является квазиэллиптическим на $v$.
Теперь  из полученных при доказательстве теоремы 1 свойств корней характеристического уравнения и теоремы 3 статьи [1] следует, что $v_{(t^0,x)}\cong w_{(t^0,x)}$ во всех точках $(t^0,x)$ связной компоненты слоя $U\cap\{t=t^0\}$ и, в частности, $v_{(t^0,x^1)}\cong w_{(t^0,x^1)}$.



Пусть $\Omega_1$ и $\Omega_2$\,--\,два открытых подмножества в $\Omega$.
Будем говорить, что отображение
$g:C^{\infty}(\Omega_1)\to C^{\infty}(\Omega_2)$
сохраняет решения уравнения (1),
если из того, что $u(t,x)\in C^{\infty}(\Omega_1)$ является решением уравнения
на множестве $\Omega_1$, следует, что $g\circ u=g(u)(t,x)\in C^{\infty}(\Omega_2)$
и является решением на $\Omega_2$. Пусть $(t^0,x^0)\in \Omega_1\cap\Omega_2$.
Мы говорим, что функция $v(t,x)\in C^{\infty}(\Omega_1)$ $g$-инвариантна в $(t^0,x^0)$, если  $g\circ v|_{(t^0,x^0)}\cong v|_{(t^0,x^0)}$. Из теоремы 2
вытекает

\textbf{Теорема~3.} {\it
Предположим, что отображение $g$ сохраняет решения уравнения {\rm (1)},
функция $v\in C^{\infty}(\Omega_1)$ удовлетворяют уравнению вдоль пути $\gamma\subset\Omega_1`\cap\Omega_2\cap\{t=t^0\}$, причём  уравнение
является квазиэллиптическим на ростках  $v_{(t^0,x)}$ для всех $(t^0,x)\in\gamma$. Тогда из $g\circ v_{(t^0,x^0)}\cong v_{(t^0,x^0)}$ в точке
$(t^0,x^0)\in\gamma$ следует $g\circ v_{(t^0,x)}\cong v_{(t^0,x)}$ во всех точках
$(t^0,x)\in\gamma$.
}




% Оформление списка литературы
\smallskip \centerline {\bf Литература} \nopagebreak

1. {\it Шананин Н.А.} О слоевой структуре множеств симметрийной инвариантности решений квазилинейных уравнений. Матем. заметки, 88:6, 2010, 924-934.




