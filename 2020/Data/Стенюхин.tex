\begin{center}{ \bf О БИФУРКАЦИЯХ РЕШЕНИЙ В ЗАДАЧЕ КАПИЛЛЯРНОСТИ}\\
{\it Л.В. Стенюхин } \\
(Воронеж; {\it stenyuhin@mail.ru})
\end{center}
\addcontentsline{toc}{section}{Стенюхин Л.В.}


Рассмотрим основной энергетический функционал задачи
$$E(u)=\int\limits_{\Omega}\sqrt{EG-F^2}\,dx+\frac{1}{\sigma}\int\limits_{\Omega}\Upsilon\rho u\,dx+\lambda\int\limits_{\Omega} u\,dx. \eqno (1)$$
Его первое слагаемое является функционалом площади.\\ Пусть $u_0$ --
экстремаль (1). Близкие к $u_0$ поверхности будем задавать в
системе координат нормального расслоения к $u_0.$ Это приведёт к
одному скалярному уравнению для близкой поверхности:
$$\left(\frac{\delta S}{\delta u}(u_0+\eta \bar{n}),\bar{n}\right)=0,$$
или
$$\frac{\delta S}{\delta\eta}(\eta)=0, \eqno(2)$$
где $\frac{\delta S}{\delta u}$ -- функциональная производная
функционала площади, $\bar{n}$ -- нормаль к поверхности $u_0.$ Из
уравнения (2) определяется нормальная координата $\eta=\eta
(x,y).$

\textbf{Теорема~1.} {\it Функционал площади близких к $u_0$
	поверхностей $S(\eta)$ и его оператор Эйлера $\frac{\delta
		S}{\delta\eta}(\eta)$ имеет следующую аналитическую структуру
	$$S(\eta)=\int\limits_{\Omega}\sqrt {EG-F^2}\,dxdy, \eqno(3)$$
	$$\frac{\delta S}{\delta u}(\eta)=E^3(EG-F^2)^{-\frac{3}{2}}(A\eta_{xx}-2B\eta_{xy}+C\eta_{yy}+G). \eqno(4)$$
	Здесь $E,G,F$ -- коэффициенты первой квадратичной формы поверхности,
	$$A=\sum\limits_{p=1}^{6}a_{ijk}\eta_x^i\eta_x^j\eta^k+1,\ B=\sum\limits_{p=1}^{6}b_{ijk}\eta_x^i\eta_y^j\eta^k,$$$$
	C=\sum\limits_{p=1}^{6}c_{ijk}\eta_y^i\eta_y^j\eta^k+1,\ G=\sum\limits_{p=2}^{7}g_{ijk}\eta_x^i\eta_y^j\eta^k+g\eta,$$
	где $i, j, k$ -- целые неотрицательные числа, $p=i+j+k.$ Все
	коэффициенты $a_{ijk}, b_{ijk}, c_{ijk}, g_{ijk}, g$ -- являются
	аналитическими функциями и находятся по формулам, подобным следующей
	$$g=(\bar{n},\bar{n}_{xx}+\bar{n}_{yy})+\frac{4}{E}\left[(\bar{n},u_{xx})^2+(\bar{n},u_{yy})^2\right]. \eqno(5)$$}

Линейная часть оператора $A\eta_{xx}-2B\eta_{xy}+C\eta_{yy}+G$ равна
$$\Delta\eta+g\eta, \eqno(6)$$ где $\Delta$ -- лапласиан.

Линейная часть первой вариации равна
$$L\eta=E^3(EG-F^2)^{-\frac{3}{2}}(\Delta\eta+g\eta)+(\frac{\Upsilon\rho}{\sigma}+\lambda)\eta,$$
где $g$ определена равенством (5).
Соотношение $\frac{\Upsilon\rho}{\sigma}$ определяет число Бонда, $B=\frac{\Upsilon\rho}{\sigma}.$ Поэтому линеаризованная задача имеет вид
$$
\left\{\begin{array}l
\Delta\eta+(g+E^{-3}(EG-F^2)^{\frac{3}{2}}(B+\lambda))\eta-0,\\
\eta\biggm|_{\partial\Omega}=0.
\end{array}\right.\eqno{(7)}
$$

Линейный оператор (7) действует из $W^2_2(\Omega)$ в $L_2(\Omega)$ и самосопряжён в $\mathring{W}^2_2(\Omega)$ относительно скалярного произведения в $L_2(\Omega).$

Пусть поверхность капли $u(x,y)=(u_1(x,y),u_2(x,y),\\u_3(x,y))$ задана в конформных координатах, $E=G, F=0.$ Тогда функция $u(x,y)$ удовлетворяет условиям $u_x^2=u_y^2,\\ u_xu_y=0.$ В этом случае функционал энергии имеет вид
$$E(u)=\int\limits_{\Omega}\frac{E+G}{2}\,dx+\int\limits_{\Omega}Bu\,dx+\lambda\int\limits_{\Omega} u\,dx. \eqno (8)$$
Первая вариация функционала равна
$$\frac{\delta E}{\delta u}(\eta)=\Delta\eta+(B+\lambda)\eta.$$
Получаем задачу
$$
\left\{\begin{array}l
\Delta\eta+(B+\lambda)\eta-0,\\
\eta\biggm|_{\partial\Omega}=0.
\end{array}\right.\eqno{(9)}
$$

\textbf{Теорема~2.} {\it Собственные значения оператора $\Delta+B+\lambda$ задачи (9) являются $\tilde{\lambda}_n=\lambda_n+B+\lambda,$ где $\lambda_n$ -- собственные значения оператора $\Delta$ с нулевым граничным условием.}

%%%% ОФОРМЛЕНИЕ СПИСКА ЛИТЕРАТУРЫ %%%
\smallskip \centerline{\bf Литература}\nopagebreak

1. {\it Стенюхин Л.В.} Бифуркационный анализ задачи капиллярности с круговой симметрией // Вестник \\ Южно-Уральского государственного университета. Серия: Математическое моделирование и программирование. Том 7, №3, 2014. С. 77 – 83.
