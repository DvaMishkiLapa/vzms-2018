\vzmstitle[\footnote{Работа выполнена при финансовой поддержке РФФИ (проект 183100111 мол\_а).}]{ЗАДАЧА КЕЛДЫША ДЛЯ УРАВНЕНИЯ СМЕШАННОГО ТИПА С ХАРАКТЕРИСТИЧЕСКИМ ВЫРОЖДЕНИЕМ}
\vzmsauthor{Сабитова}{Ю.\,К.}
\vzmsinfo{Стерлитамак; {\it sabitovauk@rambler.ru}}
\vzmscaption

Для уравнения смешанного эллиптико"=гиперболического типа
$$
Lu= u_{xx}+({\rm{sgn}}\,y)\,y\,u_{yy}+au_{y}-b^{2}u=0 \eqno{(1)}
$$
в прямоугольной области $D=\{(x,y)|\;0<x<l,-\alpha <y<\beta\}$,
где $l>0,$ $\alpha>0,$ $\beta>0,$ $a>1,$ $b$ "--- заданные
действительные числа, поставим следующую задачу.

{\bf{Задача Келдыша.}}{\emph{ Найти функцию $u(x,y),$
удовлетворяющую следующим условиям$:$}}
$$
u(x,y)\in C(\overline{D})\cap C^{1}_{x}(\overline{D})\cap
C^{2}(D_{+}\cup D_{-});\eqno{(2)}
$$
$$
Lu(x,y)\equiv 0,\;\;\;(x,y)\in D_{+}\cup D_{-};\eqno{(3)}
$$
$$
u(0,y)=u(l,y)=0,\;\;\;\;\;\;\;\;\;-\alpha\leq y \leq
\beta;\eqno{(4)}
$$
$$
u(x,\beta)=f(x),\;\;\;0\leq x \leq l,\eqno{(5)}
$$
{\emph{где $D_{-}=D\cap \{y<0\},$ $D_{+}=D\cap \{y>0\},$ $f(x)$
$-$ заданная достаточно гладкая функция, удовлетворяющая условию
$f(0)=f(l)=0.$}}

М.В. Келдыш [1] впервые исследовал более общее эллиптическое
уравнение, чем уравнение (1) при $y>0,$ второго порядка от двух
переменных с характеристическим вырождением. Он показал, что
корректность первой граничной задачи существенным образом зависит
от показателя степени вырождения и коэффициента при младшей
производной $u_{y}$.

Опираясь на эту работу, И.Л. Кароль [2] исследовал задачу Трикоми
для уравнения смешанного типа (1) при $b=0$ в области $G$, где
$G$ "--- область плоскости $XOY,$ ограниченная простой жордановой
кривой $\Gamma,$ лежащей в полуплоскости $y>0$ с концами в точках
$O(0,0)$ и $A(1,0),$ характеристиками $OC$ и $AC$ уравнения,
расположенными в полуплоскости $y<0.$ И.Л. Кароль доказал, что
характер краевых задач, которые могут быть поставлены для
уравнения (1) в области $G,$ в отличие от уравнений с
нехарактеристическим вырождением, существенно зависит от
коэффициента $a$ и класса решений уравнения (1). Например,
классическая задача Трикоми в случае $a<0$ недоопределена, при
$a>0$ "--- переопределена.

В данной статье опираясь на работы [3 -- 8] решена задача (2) -- (5) в
прямоугольной области для уравнения (1). Решение построено в
виде суммы ряда. Доказаны теоремы единственности и существования
решения этой задачи.

Решение задачи (2) -- (5) построено в виде суммы ряда
$$
u(x,y)=\sqrt{\frac{2}{l}}\sum_{k=1}^{\infty}{u_{k}(y)\sin
\frac{\pi k}{l}x}.\eqno{(6)}
$$
В формуле (6) функции $u_{k}(y)$ определены формулой
$$u_{k}(y)=\left\{\begin{array}{ll}\displaystyle
f_{k}\bigg(\frac{y}{\beta}\bigg)^{\frac{1-a}{2}}\frac{I_{a-1}(p_{k}y^{\frac{1}{2}})}
{I_{a-1}(p_{k}\beta^{\frac{1}{2}})}, & y\geq0,
\\ \displaystyle f_{k}
\bigg(\frac{-y}{\beta}\bigg)^{\frac{1-a}{2}}\frac{J_{a-1}(p_{k}(-y)^{\frac{1}{2}})}
{I_{a-1}(p_{k}\beta^{\frac{1}{2}})}, & y\leq0,
\end{array}\right.
$$
где $\displaystyle
f_{k}=\sqrt{\frac{2}{l}}\int\limits_{0}^{l}{f(x)\sin
\lambda_{k}x}dx,$ $\displaystyle
p_{k}=2\sqrt{b^{2}+\lambda^{2}_{k}},$
$I_{a-1}(p_{k}y^{\frac{1}{2}})$ "--- модифицированная функция
Бесселя первого рода, $J_{a-1}(p_{k}(-y)^{\frac{1}{2}})$ "---
функция Бесселя первого рода.


Доказаны следующие утверждения.

{\bf{Теорема 1}}. \textit {Если существует решение задачи $(2)$
-- $(5),$ то оно единственно.}

{\bf{Теорема 2.}} \textit {Если функция $f(x)\in C^{3}[0,l]$ и
выполнены условия ${f}(0)={f}(l)=0,$ $f''(0)=f''(l)=0,$ то
существует единственное решение задачи $(2) - (5)$ и это
решение определяется рядом $(6).$ }

Если вместо условия $(5)$ задать граничное условие на нижнем
основании прямоугольника
$$
u(x,-\alpha)=g(x),\;\;\;0\leq x \leq l,
$$
тогда функции $u_{k}(y)$ в формуле $(6)$ будут иметь вид
$$u_{k}(y)=\left\{\begin{array}{ll}\displaystyle
-g_{k}\bigg(\frac{y}{\alpha}\bigg)^{\frac{1-a}{2}}\frac{I_{a-1}(p_{k}y^{\frac{1}{2}})}
{J_{a-1}(p_{k}\alpha^{\frac{1}{2}})}, & y\geq0,
\\ \displaystyle g_{k}
\bigg(\frac{-y}{\alpha}\bigg)^{\frac{1-a}{2}}\frac{J_{a-1}(p_{k}(-y)^{\frac{1}{2}})}
{J_{a-1}(p_{k}\alpha^{\frac{1}{2}})}, & y\leq0,
\end{array}\right.
$$
где
$$\displaystyle
g_{k}=\sqrt{\frac{2}{l}}\int\limits_{0}^{l}{g(x)\sin
\lambda_{k}x}dx.$$

Из данной формулы видно, что функции $u_{k}(y)\rightarrow{\infty}$
при $k\rightarrow{\infty}$ и фиксированных $y>0.$ Следовательно,
показать равномерную сходимость ряда $(6)$ не удастся.

\smallskip \centerline {\bf Литература} \nopagebreak

1. {\it Келдыш М.В.} О некоторых случаях вырождения уравнений
эллиптического типа на границе области // ДАН. -- 1951. -- Т.77.
-- №2. -- С. 181 -- 184.

2. {\it Кароль И.Л.} Об одной краевой задаче для уравнения
смешанного типа второго рода
  // ДАН. -- 1953. -- Т.88. -- №2. -- С. 197 -- 200.

3. {\it Сабитов К.Б.} Задача Дирихле для уравнений смешанного
типа в прямоугольной области // Докл. РАН. -- 2007. -- Т. 413. --
№ 1. -- C. 23 -- 26.

4. {\it Сабитов К.Б. Сулейманова А.Х.} Задача Дирихле для уравнения смешанного типа второго рода в прямоугольной области // Изв. вузов. Матем. 2007. № 4. С. 45--53.

5. {\it Сабитов К.Б. Сулейманова А.Х.} Задача Дирихле для уравнения смешанного типа с характеристическим вырождением в прямоугольной области // Изв. вузов. Матем. 2009. № 11. С. 43--52.

6. {\it Хайруллин Р.С.} О существовании решения задачи Дирихле
для уравнения смешанного типа второго рода
// Дифференц. уравнения. -- 2017. -- Т. 53. -- \No~5. -- C.~684 -- 692.

7. {\it Сабитова Ю.К.} Критерий единственности решения
нелокальной задачи для вырождающегося уравнения смешанного типа в
прямоугольной области // Дифференц. уравнения. -- 2010. -- Т.46.
-- № 8. -- С. 1205 -- 1208.

8. {\it Сабитова Ю.К.} Краевая задача с нелокальным интегральным
условием для уравнений смешанного типа с вырождением на переходной
линии // Матем. заметки. -- 2015. -- T.98. -- Вып. 3. -- С. 393 --
406.
