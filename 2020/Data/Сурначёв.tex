\vzmstitle{Гёльдеровская непрерывность и неравенство Харнака для многофазного $p(x)$-лапласиана}
\vzmsauthor{Сурначёв}{М.\,Д.}
\vzmsinfo{125047, Москва, ИПМ им. М.В. Келдыша РАН; {\it peitsche@yandex.ru}}
\vzmscaption

В единичном круге $D \subset \mathbb{R}^2$ с центром в начале координат рассмотрим уравнение
$$
\mathop{\mathrm{div}}(|\nabla u|^{p(x)-2}\nabla u)=0. \eqno{(1)}
$$
Предполагается, что этот круг разделён лучами, выходящими из начала координат, на $N$  упорядоченных против часовой стрелки угловых секторов $A_1$, \ldots, $A_N$, в каждом из которых показатель $p$ принимает постоянное значение:
$$
p(x)=p_k, \quad x\in A_k, \quad k=1,\ldots,N, \eqno{(2)}
$$
причём
$$
1<p_1\leq p_2 \leq \ldots \leq p_N < \infty. \eqno{(3)}
$$
Целью данной работы является доказательство гёльдеровской непрерывности решений уравнения (1), которая устанавливается с помощью подходящего неравенства  Харнака для неотрицательных решений данного уравнения.


Для определения решения уравнения (1) введём класс функций
$$
W(D) = \{ u\in W^{1,1}(D)\,:\, |\nabla u|^{p(x)}\in L^1(D)\},
$$
где $W^{1,1}(D)$ -- соболевское пространство функций, суммируемых в $D$ вместе с обобщёнными производными первого порядка. Последовательность $u_j\in W(D)$ сходится в $W(D)$ к функции $u\in W(D)$, если $u_j\to u$ в $L^1(D)$ и
$$%\label{c}
\lim_{j\to \infty} \int\limits_{D}|\nabla u-\nabla u_j |^{p(x)} \,dx= 0.
$$

Известно [1], что для показателя, заданного соотношениями (2) и (3), гладкие функции плотны в $W(D)$ "--- для любой функции $u\in W(D)$ найдётся последовательность $u_j\in C^\infty(D) \cap W(D)$ такая, что $u_j\to u$ в $W(D)$.

Под решением уравнения (1) понимается функция $u\in W(D)$, для которой интегральное тождество
$$
\int\limits_D |\nabla u|^{p(x)-2} \nabla u \cdot \nabla \varphi\, dx=0 \eqno{(4)}
$$
выполнено для любой пробной функции $\varphi \in C_0^\infty(D)$. %Решения уравнения (1) будем называть $p(x)$-гармоническими функциями.

Для формулировки результата обозначим через $B_R$ открытый круг радиуса $R$ с центром в начале координат, а через $A_0$ "--- угловой сектор с вершиной в начале координат, с той же биссектрисой, что и угол  $A_N$, и в два раза меньшего раствора. Ниже $Q_R$ обозначает пересечение сектора $A_0$ с кольцом $B_R\setminus B_{R/2}$.

Установлен следующий результат.

\textbf{Теорема~1.} {\it Если показатель $p(x)$ определён соотношениями  (2) и (3), то при $R<1/4$ для  неотрицательного решения уравнения (1) в $D$ имеет место неравенство
$$
\sup_{Q_R} u \leq C \inf_{B_R} (u+R),
$$
где постоянная $C$ зависит только от $p_1$, $p_2$, \ldots, $p_N$. }

Из теоремы~1 вытекает гёльдеровская непрерывность решений рассматриваемого уравнения.

\textbf{Теорема~2.} {\it  Если показатель $p(x)$ определён соотношениями  (2) и (3),  то решения уравнения (1) непрерывны по Гёльдеру в $D$. }


Отметим, что до настоящего времени исследовался в основном случай регулярного показателя $p(x)$, удовлетворяющего логарифмическому условию В.В. Жикова [2]. Расссматривался и случай двухфазного показателя (см. [4], [5], [6]), когда границей раздела двух фаз, в каждой из которых показатель регулярен, служит гиперплоскость.

Обратимся к ситуации классического примера В.В. Жикова [3], когда единичный с центром в начале координат разделён на четыре сектора $A_1$--$A_4$ пересечением с квадрантами. В первом и третьем квадрантах показатель $p$ принимает постоянное значение меньше $2$, а во втором и третьем квадрантах "--- больше $2$. В обозначениях настоящей работы $N=4$,
\begin{equation*}
\begin{gathered}
\quad A_k=\{0<r<1,\ \pi (k-1)/2 < \varphi < \pi k/2\}, \\
\quad p_3=p_1<2, \quad p_4=p_2>2.
\end{gathered}\eqno{(5)}
\end{equation*}


%$N=4$,  в полярных координатах $(r,\varphi)$ заданы сектора $A_k=\{0<r<1,\ \pi (k-1)/2 < \varphi < \pi k/2\}$, $p_3=p_1<2$, $p_4=p_2>2$.

В этом случае множество гладких функций не является плотным в $W(D)$ и ниже $H(D)$ означает замыкание множества гладких функций в смысле сходимости в $W(D)$. При этом возникает два типа решений (см. [2], [3]) "--- $W$--решения, принадлежащие пространству $W(D)$, для которых интегральное тождество (4) выполнено для всех пробных функций $\varphi\in W(D)$ с компактным носителем в $D$, и $H$-решения, принадлежащие пространству $H(D)$, для которых интегральное тождество (4) выполнено для всех $\varphi \in C_0^\infty(D)$. Известно ([2], [3]), что для показателя $p$, определённого (5) все $W$-решения, не являющиеся $H$-решениями, разрывны в начале координат. %. Можно привести пример такой гладкой граничной функции, что $W$-решение соответствующей задачи Дирихле в круге для эллиптического $p(x)$-лапласиана разрывно, а $H$-решение непрерывно.


%минимизант по всему естественному энергетическому пространству с нулевым условием на границе круга будет разрывен, а минимизант того же функционала по замыканию гладких финитных функций в этом пространстве будет непрерывен по Гёльдеру. Для этой конфигурации  можно привести пример такой гладкой граничной функции, что $W$ решение соответствующей задачи Дирихле в круге для эллиптического $p(x)$-лапласиана разрывно, а $H$-решение непрерывно.%  В случае, когда $p_2,p_1 > 2$ или $p_2,p_1 < 2$ эффекта Лаврентьева в примере Жикова не возникает.

\textbf{Теорема~3.} {\it Для показателя, определённого (5), все $H$-решения уравнения (1) непрерывны по Гёльдеру в $D$.}

\bigskip

Работа выполнена при поддержке РФФИ, проект № 19-01-00184-а.




% Оформление списка литературы
\smallskip \centerline {\bf Литература} \nopagebreak

\noindent 1. {\it Fan~X., Wang~S., Zhao~D.} Density of $C^\infty(\Omega)$ in $W^{1,p(x)}(\Omega)$ with discontinuous exponent $p(x)$ // Math. Nachr. 2006. V.~279. ${\rm N^o}$\,~1--2. P.~142--149.

\noindent 2.  {\it Zhikov~V.V.} On Lavrentiev's Phenomenon // Russian J. Math. Phys. 1994. V.~3.  ${\rm N^o}$\,~2. P.~249--269.


\noindent 3. {\it Жиков~В.В.} Усреднение нелинейных функционалов вариационного исчисления и теории упругости // Изв. АН СССР.
Сер. матем. 1986. Т.~50. ${\rm N^o}$\,~4, С.~675--711.

\noindent 4. {\it Acerbi E., Fusco N.} A transmission problem in the calculus of variations // Calc. Var. Partial Differ. Equ. 1994. V.~2.  ${\rm N^o}$\,~1. P.~1--16.

\noindent 5. {\it Алхутов Ю.А.} О гёльдеровой непрерывности $p(x)$-гар\-мо\-ни\-ческих функций // Матем. сб. 2005. Т.~196. ${\rm N^o}$\,~2. С.~3--28.

\noindent 6. {\it Алхутов Ю.А., Сурначёв М.Д.} О неравенстве Харнака для $p(x)$-лапласиана с двухфазным показателем $p(x)$ // Труды семинара им. И.Г. Петровского. 2019. Вып.~32. С.~8--56.


