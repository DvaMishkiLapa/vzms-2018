\begin{center}
    {\bf ПОСТРОЕНИЕ ПРОГРАММНОГО  УПРАВЛЕНИЯ ДЛЯ ДИНАМИЧЕСКОЙ СИСТЕМЫ В ЧАСТНЫХ ПРОИЗВОДНЫХ}

    {\it С.П. Зубова, Е.В. Раецкая}

    (Воронеж; {\it spzubova@mail.ru, \it raetskaya@inbox.ru})

\end{center}

\addcontentsline{toc}{section}{Зубова С.П., Раецкая Е.В.}

 Рассматривается система в частных производных
\begin{equation}
\frac{\partial x}{\partial t}=B(\frac{\partial x}{\partial s}+ \alpha  x) + Du,
\end{equation}
 где $\quad t\in [0,T],\quad s\in [0,S]$;     $x(t,s)\in R ^{n}$; $u(t,s)\in R^{m}$; $\alpha(t,s)$ - скалярная функция; $B$, $D$ --- матрицы соответствующих  размеров.

Система $(1)$ называется полностью управляемой, если существует такое управление $u(t,s)$,
под воздействием которого система переводится   из произвольного состояния $x(0,s)$ в произвольное состояние $x(T,s)$.

Для системы $(1)$, описывающей  процессы в вязко-пластической среде, решается задача построения программного управления.


Основным методом исследования является метод каскадной декомпозиции $([1]-[6])$, опирающийся на свойство фредголь-
мовости конечномерного отбражения и заключающийся  в поэтапном переходе к системам в подпространствах, так что на
последнем шаге декомпозиции формируется система
\begin{equation}
\frac{\partial x_{p}}{\partial t}=B(\frac{\partial x_{p}}{\partial s}+ \alpha_{p} x_{p}) + D_{p}(\frac{\partial u_{p}}{\partial s}+ \beta_{p}  u_{p}),
\end{equation}
где $D_{p}=0$ или $D_{p}$ - сюръекция;  $x_{p}(t,s)$ - удовлетворяет условиям
\begin{equation}
\frac{\partial x^{j}_{p}}{\partial s}|_{t=0} = a_{j},      \frac{\partial x^{j}_{p}}{\partial s}|_{t=T} = b_{j},    j=1,...,p,
\end{equation}
где  $a_{j}$ и $b_{j}$ - некоторые функции. Доказывается

\textbf{Теорема~1.} {\it Система $(1)$ полностью управляема в том и только том случае, когда  $D_{p}$ - сюръекция.}


В случае сюръективного $D_{p}$, для любого гладкого $x_{p}$, удовлетворяющего
условиям $(3)$ существует гладкое $u_{p}$, удовле-
творяющее уравнению $(2)$. Обратным ходом декомпозиции  строятся функции состояния
и управления в аналитическом виде   с любой желаемой степенью гладкости.

\smallskip \centerline {\bf Литература} \nopagebreak


1. {\it  Zubova С.П.} Construction of Controls Providing the Desired Output of the Linear Dynamic System /Zubova С.П., Raetskaya Е.V.// Automation and Remote Control. 2018 .Vol.79, No. 5, p. 774--791.

2. {\it  Zubova С.П.}  Algorithm to Solve Of Control By The Method Of Cascade Decomposition /Zubova С.П., Raetskaya Е.V.// Automation and Remote Control. 2017 .Vol.78, No. 7, p. 1189--1202.

3. {\it  Zubova С.П.} A Study Of The Rigidity Of Descriptor Dynamical System In a Banach Space /Zubova С.П., Raetskaya Е.V.// Journal of Mathematical Sciences. 2015. Vol.208, 1, p. 131--138.


4.{\it  Zubova С.П.} Solution Of The Coushy Problem For Two Descriptive Equations With Fredholm Operator /Zubova С.П., Raetskaya Е.V.//
Doklady Mathematics. 2014.  Vol. 90, No. 3.  p. 732--736.

5.{\it  Zubova С.П.}  Invariance Of a Nonstationary  Observability System Under Certain Perturbations /Zubova С.П., Raetskaya Е.V.// Journal of Mathematical Sciences. 2013.  Vol.  188, No. 3. p. 218--226.

6.{\it  Zubova С.П.} On Polinomial  Solution Of The Linear Stationary Control /Zubova С.П., Trung L.H., Raetskaya Е.V.// Automation and Remote Control. 2008 .Vol.69, No. 11, p. 1852--1858.
