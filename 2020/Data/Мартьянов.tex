\vzmstitle[\footnote{Исследование выполнено при финансовой поддержке РФФИ в рамках научного проекта №~19-31-90152.}]{О ТРИГОНОМЕТРИЧЕСКОЙ КОНСТАНТЕ НИКОЛЬСКОГО С ПЕРИОДИЧЕСКИМ ВЕСОМ ГЕГЕНБАУЭРА}
\vzmsauthor{Мартьянов}{И.\,А.}
\vzmsinfo{Тула; {\it martyanow.ivan@yandex.ru}}
\vzmscaption

Пусть $L_{\alpha}^{p}(-\pi,\pi]$~--- комплексное пространство периодических
функций с конечной относительно периодического веса Гегенбауэра нормой
\[
\|f\|_{p,\alpha}=\biggl(\int_{-\pi}^{\pi}|f(x)|^{p}\,|\!\sin
x|^{2\alpha+1}\,dx\biggr)^{1/p},\quad \alpha\ge -1/2,
\]
$\mathcal{T}_{n}$~--- подпространство тригонометрических полиномов порядка $n$
с комплексными коэффициентами.

Через
\[
\mathcal{C}_{p,\alpha}(n)=\sup_{T\in \mathcal{T}_{n}\setminus \{0\}}
\frac{\|T\|_{\infty,\alpha}}{\|T\|_{p,\alpha}}
\]
обозначим точную константу Никольского разных метрик. Задача нахождения
$\mathcal{C}_{p,\alpha}(n)$ имеет долгую историю, особенно в безвесовом случае
$\alpha=-1/2$. Однако даже в нем она вычислена только при $p=2$. Отметим
результаты Я.Л.~Геронимуса (1938), Л.В.~Тайкова (1993), Д.В.~Горбачева (2005),
В.В.~Арестова и М.В.~Дейкаловой (2015), И.Е.~Симонова и П.Ю.~Глазыриной (2015),
E.~Levin и D.S.~Lubinsky (2015), М.И.~Ганзбурга и С.Ю.~Тихонова (2017) и многих
других.

\textbf{Теорема.} {\it Пусть $1\le p<\infty$, $\alpha\ge -1/2$. Тогда
\[
\mathcal{C}_{p,\alpha}(n)=T_{*}(0),
\]
где $T_{*}$~--- экстремальный действительный чётный полином порядка $n$, для
которого $\|T_{*}\|_{p,\alpha}=1$.

Кроме того, $\mathcal{C}_{p,\alpha}(n)$ с точностью до положительной константы
совпадает с соответствующей точной константой Никольского для алгебраических
полиномов степени $n$ в пространстве $L^{p}$ на отрезке $[-1,1]$ с весом
Гегенбауэра $(1-x^{2})^{\alpha}$.}

Отметим в данном направлении результаты В.В.~Арестова и М.В.~Дейкаловой (2015),
В.В.~Арестова, А.Г.~Бабенко, М.В.~Дейкаловой и A.~Хорват (2018), Д.В.~Горбачева
и Н.Н.~Добровольского (2018).

Теорема сводит вычисление тригонометрической константы Никольского к константе
для алгебраических полиномов. В последнем случае она может быть вычислена
методами нелинейной оптимизации на основе доказанных В.В.~Арестовым и
М.В.~Дейкаловой соотношений двойственности.

Для доказательства теоремы при $\alpha>-1/2$ применяется положительный оператор
обобщённого сдвига
\[
T^{t}f(x)=\frac{c_{\alpha}}{2}\int_{0}^{\pi}\bigl(f(\psi)(1+B)+
f(-\psi)(1-B)\bigr)\sin^{2\alpha}\theta\,d\theta,
\]
где $c_{\alpha}=\frac{\Gamma(\alpha+1)}{\Gamma(1/2)\Gamma(\alpha+1/2)}$,
\[
\psi=\arccos (\cos x\cos t+\sin x\sin t\cos \theta),
\]
\[
B=\frac{\sin x\cos t-\cos x\sin t\cos \theta}{\sin \psi},\quad T^{t}1=1.
\]
Он построен и изучен Д.В.~Чертовой (2009). В частности, она доказала, что его
норма в $L_{\alpha}^{p}(-\pi,\pi]$ равна единице. На чётных функциях $T^{t}$
совпадает со сдвигами Лежандра ($\alpha=0$) и Гегенбауэра ($\alpha>-1/2$).
