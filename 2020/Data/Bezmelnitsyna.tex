\selectlanguage{english}

\vzmstitle{ON GUIDING POTENTIALS AND ASYMPTOTIC BEHAVIOR OF TRAJECTORIES FOR RANDOM DIFFERENTIAL INCLUSIONS}
\vzmsauthor{Bezmelnitsyna}{Y.\,E.}
\vzmsinfo{Voronezh; {\it bezmelnicyna@inbox.ru}}
\vzmscaption

In the recent years the sphere of applications of the method of guiding functions which is due to M.A. Krasnoselskii and A.I. Perov (see, e.g., [8] and also [3, 5] and their references) has been extended to the study of qualitative behavior of solutions of differential equations and inclusions of various types, covering, in particular, their asymptotics (see, e.g., [2, 6, 7, 9, 10]).

In the present paper we define a random nonsmooth guiding potential for random differential inclusions and apply it to the study of the asymptotic behavior of solutions for such inclusions.

In what follows we will use some known notions and notati\-ons from the theory of multimaps (see, e.g., [3, 5]).

Let $(X,d_X)$ and $(Y,d_Y)$ be metric spaces. By the symbols $P(Y),$ $C(Y)$ and $K(Y)$ we denote the collections of all nonempty, closed and, respectively, compact subsets of the space $Y.$ If $Y$ is a normed space, $Kv(Y)$ denote the collections of all nonempty convex compact subsets of $Y.$

\textbf{Definition~1.}
A multimap $F:X \to P(Y)$ is called {\it upper semicontinuous (u.s.c.)} at the point $x\in X$ if for each open set $V \subset Y$ such that $F(x) \subset V$ there exists $\delta >0$ such that $d_X(x,x^\prime)<\delta$ implies $F(x^\prime)\subset V.$ A multimap $F:X \to P(Y)$ is called u.s.c. if it is u.s.c. at each point $x\in X.$

Let $I$ be a closed subset of $\mathbb{R}$ with the Lebesgue measure.

\textbf{Definition~2.}
A multifunction $F:I \to K(Y)$ is called {\it measurable} if, for each open subset $W \subset Y,$ its pre-image \linebreak
$F^{-1}(W)=\{t\in I:F(t)\subset W\}$ is the measurable subset of $I$.

Let $(\Omega,\Sigma,\mu)$ be a complete probability space.

\textbf{Definition~3.} (see [1]). Multimap $\mathcal{F}\colon\Omega\times X\to C(Y)$ is called a {\it random multioperator} if it is product-measurable, i.e. measurable w.r.t. $\Sigma\otimes\mathbb{B}(X)$, where $\Sigma\otimes\mathbb{B}(X)$ is the smallest
$\sigma$-algebra on $\Omega\times X$ which contains all the sets $A\times B$, where $A\in\Sigma$ and $B\in\mathbb{B}(X)$ and
$\mathbb{B}(X)$ denotes the Borel $\sigma$-algebra on $X$. If, moreover, $\mathcal{F}(\omega,\cdot)\colon X\to C(Y)$ is u.s.c. for all
$\omega\in\Omega$, then $\mathcal{F}$ is called a {\it random $u$-multioperator}.

We consider the following Cauchy problem for a random differential inclusion of the form:
$$
x'(\omega,t)\in \mathcal{F}(\omega,t,x(\omega,t)), \eqno{(1)}
$$
under assumptions that the random $u$-multioperator $\mathcal{F}:\Omega\times\mathbb{R} \times \mathbb{R}^n \to Kv(\mathbb{R}^n)$ satisfies the sublinear growth condition:

\noindent
there exists a function $\alpha:\Omega\times \mathbb{R}_+\to \mathbb{R}_+$ such that
$(i)$ for each $\omega\in\Omega$ a function $\alpha(\cdot,t)$ is measurable,
$(ii)$ a function $\alpha(\omega,\cdot)$ is locally integrable,
and we have for each $\omega\in\Omega$
$$
\|\mathcal{F}(\omega,t,x)\| \leq \alpha(\omega,t)(1 + \|x\|) \,\,\mbox{for a.e.}\;t \in \mathbb{R}; \,x \in \mathbb{R}^n .
$$
By a {\it solution of inclusion (1) on $\mathbb{R}$} we mean a function $x:\Omega\times\mathbb{R}\to \mathbb {R}^n,$ such that
$(j)$ $x(\cdot, t)$ is measurable for a.e. $t\in \mathbb{R};$
$(jj)$ $x(\omega,\cdot)$ is absolutely continuous for each $\omega\in\Omega$;
satisfying for each $\omega\in\Omega$ inclusion (1) for a.e. $t\in \mathbb{R}$ and the initial condition at almost every point
$$
x(\omega,0)=x_0. \eqno{(2)}
$$
Let us recall some notions of non-smooth analysis (see [4]).

Let $V$ be a locally Lipschitz function on the space $\mathbb{R}^{n}.$ For $x_{0} \in \mathbb{R}^{n}$ and $\nu \in \mathbb{R}^{n}$ {\it the Clarke generalized derivative} $V^{0}(x_{0} ;\nu )$ at $x_{0} $ along the direction $\nu $ is given by the formula
$$
V^{0}(x_{0} ;\nu ) = \mathop{\overline {\lim}}\limits_{{\begin{array}{*{20}c}
 {x \to x_{0}} \hfill \\
 {t \to 0+} \hfill \
 \end{array} }}\frac{{V(x + t\nu ) - V(x)}}{{t}},
$$
where $x \in \mathbb{R}^{n}.$ Then {\it the Clarke generalized gradient} $\partial V(x)$ of the function $V$ at the point $x_{0}$ is defined in the following way:
$$
\partial V(x_{0} ) = \left\{ {x \in \mathbb{R}^{n}:\quad \langle {x,\nu } \rangle \le V^{0}(x_{0} ;\nu )\quad \mbox{for all}\;\;\nu \in \mathbb{R}^{n}} \right\}.
$$
Recall (see, e.g., [4]) that a locally Lipschitz function $V:\mathbb{R}^n\to \mathbb{R}$ is called {\it regular} if for each $x\in \mathbb{R}^{n}$ and $\nu \in \mathbb{R}^{n}$ there exists the derivative along the direction $V'(x,\nu)$ and it coincides with $V^0(x,\nu).$ It is known that convex functions are regular.

\textbf{Definition~4.}
A map $V\colon\Omega\times\mathbb{R}^{n}\to\mathbb{R}$ is called a {\it random nonsmooth potential} if the following two conditions are satisfied:
$(i)$ $V(\cdot,x)\colon\Omega\to\mathbb{R}$ is measurable for every $x\in\mathbb{R}^{n}$; \linebreak
$(ii)$ $V(\omega,\cdot)\colon\mathbb{R}^{n}\to\mathbb{R}$ is a regular function for every $\omega\in\Omega$.

\noindent Denote by $\mathfrak{V}$ the collection of all random nonsmooth potentials $V: \Omega\times\mathbb{R}^n \to \mathbb{R}$ such that for each $\omega\in\Omega$ the coercivity condition
$
\lim_{\|x\|\to +\infty} V(\omega,x) = -\infty
$
holds true.

Notice that, given a function $V \in \mathfrak{V},$ for each $r>0$ and $\omega\in\Omega$ there exists $k_\omega(r) > r$ such that
if
$
\alpha_r(\omega):=\inf\{V(\omega,x),\|x\|\leq r\},
$
then for each $\omega\in\Omega$ we have
$
 V(\omega,x)< \alpha_r(\omega), \;\|x\|\geq k_\omega(r).
$

Now, let $g:\Omega\times\mathbb{R}_+\to \mathbb{R}_+$ be a given function such that
$(j)$ $g(\cdot, t)$ is measurable for a.e. $t\in\mathbb{R}_+$;
$(jj)$ $g(\omega, \cdot)$ is absolutely continuous;
$(jjj)$ $\inf \{g(\omega,t):\; \omega\in\Omega,\; t\in \mathbb{R}\}\geq 1$.

\textbf{Definition~5.} A random nonsmooth potential $V \in \mathfrak{V}$ is called a {\it random nonsmooth guiding potential for inclusion (1) along the function $g$} if for each $\omega\in\Omega$ there exists $r_0(\omega)>0$ such that $g(\omega,t)\|x\|\ge r_0(\omega),$ \,$t \in \mathbb{R}$ implies for each $\omega\in\Omega$
$$
\left< {\upsilon,g'(\omega,t)x+g(\omega,t)y} \right> \geq 0, \quad \mbox {if} \; t>0;
$$
$$
\left< {\upsilon,g'(\omega,t)x+g(\omega,t)y} \right> \leq 0, \quad \mbox {if}\; t<0;
$$
for each $y\in F(\omega,t,x), \; \upsilon \in \partial V(g(\omega,t)x).$

\textbf{Theorem~1.}
If $V \in \mathfrak{V}$ is a random nonsmooth guiding potential for inclusion (1) along the function $g$
then each solution of Cauchy problem (1), (2) satisfies the estimate
$$
\|x(\omega,t)\|\leq k_V(\omega)\cdot \frac{1}{g(\omega,t)},\quad \omega\in\Omega,\;t\in \mathbb{R}, \; k_V(\omega)>0.
$$
% Оформление списка литературы
\litlist

1. {\it Andres J., G\'orniewicz L.} Random topological degree and random differential inclusions. Topol. Meth. Nonl. Anal. 40 (2012), 337--358.

2. {\it Avramescu C.} Asymptotic behavior of solutions of nonli\-near differential equations and generalized guiding functions. Elec\-tronic J. of Qualitive Theory of Differ. Equ. 13 (2003), 1-9.

3. {\it Borisovich Yu.G., Gel'man B.D., Myshkis A.D., Obukho\-vskii V.V.} Introduction to the Theory of Multivalued Maps and Dif\-fe\-ren\-tial Inclusions - 2nd ed. Moscow: Librokom, 2011.

4. {\it Clarke F.H.} Optimization and Nonsmooth Analysis - 2nd ed. Classics in Applied Mathematics, 5. Society for Industrial and Applied Mathematics (SIAM). Philadelphia: PA, 1990.

5. {\it G\'orniewicz L.} Topological Fixed Point Theory of Multi\-valued Mappings - 2nd ed. Berlin: Springer, 2006.

6. {\it Kornev S., Obukhovskii V., Yao J.-C.} On asymptotics of solutions for a class of functional differential inclusions. Discus\-sio\-nes Mathematicae. Differential Inclusions, Control and Opti\-mi\-zation. 34 (2014), 219--227.

7. {\it Kornev S.V., Obukhovskii V.V.} On asymptotic behavior of solutions of differential inclusions and the method of guiding functions. Differential Equations. 51 (2015), 711--716.

8. {\it Krasnosel'skii M.A.} The Operator of Translation Along the Trajectories of Differential Equations. Translations of Mathe\-ma\-tical Monographs - Vol. 19. Providence, R.I.: American Ma\-the\-matical Society, 1968.

9. {\it Obukhovskii V., Kamenskii M., Kornev S., Liou Y.-C.} On asymptotics of solutions for a class of differential inclusions with a regular right-hand part. Journal of Nonlinear and Convex Analysis. 18 (2017), no. 5, 967--975.

10. {\it Kornev S., Obukhovskii V., Yao J.-C.} Nonsmooth integral guiding potentials and asymptotic behavior of solutions for inclusions with causal multioperators. Optimization (2019)
