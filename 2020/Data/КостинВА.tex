
\vzmstitle{О КОРРЕКТНОЙ РАЗРЕШИМОСТИ ЗАДАЧИ КОШИ ДЛЯ ОБОБЩЕННОГО УРАВНЕНИЯ МАЙНАРДИ С ДРОБНОЙ ПРОИЗВОДНОЙ}
\vzmsauthor{Костин}{В.\,А.}
\vzmsauthor{Кочетова}{Е.\,Д.}
\vzmsauthor{Лемешаев}{С.\,С.}
\vzmsinfo{Воронеж; {\it jancd@inbox.ru}}
\vzmscaption


В Банаховом пространстве $E$ с нормой $\|*\|_E$ раccматривается уравнение

\begin{equation}\label{eq_lin_operator}
	\frac{\partial^{1+\alpha} u(t)}{\partial t^{1+\alpha}} = Au(t), \ \ \ t \geqslant 0
\end{equation}

где $A$ "--- линейный замкнутый оператор с областью определения $D(A) \subset E$ и областью значения $R(A)$.

\begin{equation}\label{Green_Eq}
        \frac{\partial^{1+\alpha}u(t,x)}{\partial t^{1+\alpha}} = \frac{1}{\Gamma(1-\alpha)} \int\limits_0^t \frac{u''(s)ds}{(t - s)^{\alpha}}; \ \ \ \alpha \in (0, 1)
\end{equation}
 - дробная производная в смысле Капуто.
\vspace{3mm}


Рассматривается задача о нахождении решения уравнения \ref{eq_lin_operator}, удовлетворяющим условиям

\begin{equation}\label{Koshi_uslovie}
	u(0) = u_0, \ u'(0) = 0
\end{equation}

\begin{definition}
	Решением уравнения \ref{eq_lin_operator} называется оператор"=функция $u(t)$ со значениями в $D(A)$, для которой определяется производная \ref{Green_Eq} и удовлетворяющая уравнению \ref{eq_lin_operator}
\end{definition}

\begin{definition}
	Решением уравнения \ref{eq_lin_operator} удовлетворяющим условиям \ref{Koshi_uslovie}, где $u_0 \in D(A), u_1 \in D(A)$, называется решением задачи Коши.
\end{definition}

\begin{definition}
	Задачу \ref{eq_lin_operator} - \ref{Koshi_uslovie} будем называть равномерно"=корректной, если для её решения справедлива оценка
	\begin{equation}\label{ocenka}
		||u(t)||_E \leq M||u_0||_E,
	\end{equation}
	где константа $M$ от $t$ и $u_0$ не зависит.
\end{definition}
Для оператора $A$, заданного выражением $D=d^2/dx^2$, уравнение \ref{eq_lin_operator} рассматривалось Ф. Майнарди [7].
\vspace{3mm}


В настоящем сообщении доказывается
	\begin{theorem}
		Если оператор $A$ является производящим оператором сильно"=непрерывной косинус функции $C(t,A)$ с оценкой
		\begin{equation}
			||C(t,A)||_E \leq M,
		\end{equation}
		то задача Коши \ref{eq_lin_operator} - \ref{Koshi_uslovie} равномерно"=корректна и её решение имеет вид
		\begin{equation}
			u(t) = \int\limits^{\infty}_0 I^{(1-\alpha)}_t (h_{\alpha}(t, \xi) C(\xi, A)u_0 d\xi,
		\end{equation}
	\end{theorem}
	где $I^{(1-\alpha)}f(t)$ "--- дробный интеграл Риммана-Лиувиля,
	$$
		h_\alpha(t,s) = \frac{1}{2\pi i} \int\limits^{\sigma + i\infty}_{\sigma - i \infty} e^{tp-\xi p^{\alpha}} dp
	$$ есть функция Иосиды
	и справедлива оценка \ref{ocenka}

\litlist

1. {\it Иосида К.} Функциональный анализ // К. Иосида - Издательство Мир, Москва 1967, с. 357.

2. {\it Голдстейн Дж.} Полугруппы линейных операторов и их приложения. Киев: Выща школа, 1989. 347 с.

3. {\it Крейн С.Г.} Линейные дифференциальные уравнения в банаховом пространстве, М.: Наука 1967. -464 с.

4. {\it Маслов В.П.} Операторные методы / Главная редакция физико"=математической литературы изд-ва <<Наука>> -- М., 1973. - 544 с.

5. {\it Kurepa S.} Semigroups and cosine functions. Lecture Notes in Math, vol 948 Berlin, Springs, 1982. - p. 47 - 72.

6. {\it Самко С.Г.} Интегралы и производные дробного порядка и некоторих их приложения // С.Г. Самко / А.А. Килбис / О.И. Маричев

7. {\it Mainardi F.} Временное уравнение диффузии"=волны. Радиофизика и квантовая электроника, вып. 38, №1-2, 1995

8. {\it Костин В.А., Костин А.В., Костин Д.В.} <<Операторные косинус"=функции и граничные задачи>>. ДАН, 2019,т.486 №5, с.531-536.
