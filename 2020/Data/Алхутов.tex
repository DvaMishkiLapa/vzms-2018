\vzmstitle[\footnote{Работа выполнена при поддержке РФФИ, проект № 19-01-00184-а.}]{ПОВЕДЕНИЕ В ГРАНИЧНОЙ ТОЧКЕ РЕШЕНИЯ ЗАДАЧИ ДИРИХЛЕ ДЛЯ Р--ЛАПЛАСИАНА, РАВНОМЕРНО ВЫРОЖДАЮЩЕГОСЯ НА ЧАСТИ ОБЛАСТИ ПО МАЛОМУ ПАРАМЕТРУ}
\vzmsauthor{Алхутов}{Ю.\,А.}
\vzmsauthor{Хрипунова Балджы}{А.\,С.}
\vzmsinfo{Россия, г. Владимир, ВлГУ им. А.Г. и Н.Г. Столетовых; Германия, Билефельдcкий Университет; {\it yurij-alkhutov@yandex.ru}; {\it akhripun@math.uni-bielefeld.de}}
\vzmscaption

Рассмотрим в ограниченной области $D$ евклидова пространства $\mathbb{R}^n$, где $n\geq 2$, уравнение
$$
Lu=\mathrm{div}\, \left(\omega_\varepsilon (x)|\nabla u|^{p-2}\nabla u \right)=0, ~p=const>1 \eqno{(1)}
$$
с положительным весом $\omega_\varepsilon$, содержащим малый параметр, который сейчас определим. Предполагается, что область $D$
разделена гиперплоскостью $\Sigma$ на части
$D^{(1)}$, $D^{(2)}$ и
$$
  \omega_\varepsilon (x)= \left \{\begin{array}{lrr}
  \varepsilon,~\mbox{если}~x\in D^{(1)} \\
  1,~\mbox{если}~x\in D^{(2)}
  \end{array}
  \right.,~\varepsilon\in (0,1].
$$

Ниже $W^{1}_p (D)$ означает соболевское пространство функций, которые $L_p$-суммируемы в $D$ вместе со всеми обобщёнными производными первого
порядка, а $\stackrel{\circ}{W^{1}_p} (D )$ - замыкание
финитных, бесконечно дифференцируемых в $D$ функций
$C^{\infty}_{0} (D )$ по норме $W^{1}_p (D)$. Скажем, что функция $u\in W^{1}_p (D) $ является решением уравнения (1) в $D$, если интегральное тождество
$$
\int\limits_D\omega_\varepsilon (x) |\nabla u|^{p(x)-2} \nabla u\cdot\nabla \varphi\, dx=0 \eqno{(2)}
$$
выполнено для любой пробной функции $\varphi \in \stackrel{\circ}{W^{1}_p} (D ) $.

Рассмотрим задачу Дирихле
$$
Lu=0 \quad \text{в}\quad D, \quad u\in W^1_p (D), \quad h\in W^1_p (D),\quad (u-h)\in \stackrel{\circ}{W^{1}_p} (D ).
$$
Решение данной задачи совпадает с минимизантом решения вариационной задачи
$$
\min _{w\in \stackrel{\circ}{W^{1}_p} (D ) } F(w+h), \quad F(u)=\int\limits_D \omega_\varepsilon (x)|\nabla u|^p\, dx.
$$
Настоящее сообщение посвящено граничным свойствам решения задачи Дирихле
$$
Lu_f=0 \quad \text{в}\quad D, \quad u_f\vert_{\partial D}=f \eqno{(3)}
$$
с непрерывной на $\partial D$ функцией $f$.

Решение задачи (3) определяется следующим образом.
Продолжим граничную функцию $f$ по непрерывности на $\overline{D}$,
сохранив за продолжением то же обозначение.
Возьмём последовательность бесконечно дифференцируемых в $\mathbb{R}^n$ фун\-кций $f_k$,
которые равномерно на $\overline{D}$ сходятся к $f$. Решим задачи Дирихле
$$
Lu_k=0 \quad\text{в}\quad D, \quad u_k \in W^1_p(D), \quad (u_k-f_k)\in \stackrel{\circ}{W^{1}_p} (D ).
$$
Последовательность $u_k$ сходится равномерно на компактных подмножествах $D$ к функции $u$, принадлежащей пространству $ W^1_p(D')$ в произвольной
подобласти $D'\Subset D$, которая удовлетворяет интегральному тождеству (2) на пробных функциях $\varphi \in W^1_p(D)$ с компактным носителем в $D$. Предельная функция не зависит от способов продолжения и аппроксимации граничной функции $f$ и называется обобщённым решением задачи Дирихле (3).

Граничная точка $x_0\in \partial D$ называется регулярной, если
$$
\lim\limits_{D \ni x\to x_0} u_f(x)= f(x_0)
$$
для любой непрерывной на $\partial D$ функции $f$.

Далее нам потребуется понятие ёмкости. Ёмкостью компакта $K\subset B$ относительно относительно шара $B\subset \mathbb{R}^n$ называется число
$$
C_p (K,\, B)=\inf \left\{ \int\limits_B |\nabla \varphi|^p\, dx\,:\quad \varphi\in C_0^\infty(B),\quad \varphi\geq 1 \quad \text{на}\quad K \right\}.
$$

Критерий регулярности граничной точки состоит в выполнении равенства
$$
\int\limits_0 \left( \frac{C_p \left(\overline{B^{x_0}_r}\setminus D,\, B^{x_0}_{2r} \right)}{r^{n-p}} \right)^\frac{1}{p-1}\frac{dr}{r}=\infty, \eqno{(4)}
$$
где $B^{x_0}_r$ "--- открытый шар с центром в точке $x_0$ радиуса $r$, a $\overline{B^{x_0}_r}$ "--- его замыкание.

Для уравнения Лапласа это утверждение является классическим результатом Н. Винера [1]. В случае линейных уравнений, когда $p=2$, критерий получен в [2].
Если $p\neq2$, то достаточное условие регулярности граничной точки найдено В.Г. Мазьёй в [3]. Позже в работе [4] было показано, что полученное в [3] достаточное
условие регулярности является и необходимым.

Для уравнения вида (1), не содержащего малый параметр $\varepsilon$, в статье [3] получена оценка модуля непрерывности решений задачи Дирихле в регулярной
граничной точке. Настоящее сообщение посвящено оценке модуля непрерывности решений задачи Дирихле (3) для уравнения (1) с постоянными, не зависящими от $\varepsilon$.
Предполагается, что в окрестности граничной точки границы $x_0\in \partial D\cap\Sigma$ дополнение области $D$
симметрично относительно гиперплоскости $\Sigma$.


Прежде чем сформулировать полученный результат, положим
$$
\gamma(r)=\left( \frac{C_p \left(\overline{B^{x_0}_r}\setminus D,\, B^{x_0}_{2r}) \right)}{r^{n-p}} \right)^\frac{1}{p-1}
$$

\noindent{\bf Теорема.} {\it Если выполнено условие (4), то при достаточно малом $\rho$ и $r\le\rho/4$ для решения $u_f$ задачи Дирихле (3) справедливы оценки
$$
\esssup_{D\cap B^{x_0}_r} |u_f-f(x_0)|\leq
$$
$$
\leq C \biggl(\osc_{\partial D\cap B^{x_0}_\rho}f +\osc_{\partial D}f \cdot \exp\biggl(-k \int\limits_r^\rho \gamma (t)t^{-1}\, dt \biggr)\biggr),~\mbox{если}~p\le n,
$$
и
$$
\esssup_{D\cap B^{x_0}_r} |u_f-f(x_0)|\leq
$$
$$
\leq C \biggl(\osc_{\partial D\cap B^{x_0}_\rho}f +\osc_{\partial D}f\cdot (r/\rho)^{1-n/p}\biggr ),~\mbox{если}~p>n,
$$
в которых положительные постоянные $C$ и $k$ зависят только от $n$ и $ p$.
}




\litlist

1. {\it Wiener N.} Certain notions in potential theory // J. Math. Phys. 1924. V. 3. P. 24--51.

2. {\it Littman~W., Stampacchia~G., Weinberger~H.~F.} Regular points for elliptic equations with discontinuous coefficients // Ann. Scuola Norm. Sup. Pisa. 1963. V. 3. № 17. P. 43--77.

3. {\it Мазья В. Г.} О непрерывности в граничной точке решений квазилинейных эллиптических
уравнений // Вестн. ЛГУ. Сер. матем. 1970. Т. 25. №13. С. 42--55.

4. {\it Kilpel\"ainen~T., Mal\'y~J.} The Wiener test and potential estimates for quasilinear elliptic equations // Acta Math. 1994. V.172. P. 137--161.
