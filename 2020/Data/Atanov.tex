\begin{center}
    {\bf ПРИМЕРЫ ГОЛОМОРФНО-ОДНОРОДНЫХ ГИПЕРПОВЕРХНОСТЕЙ В $\mathbb{C}^4$}

    {\it А.В. Атанов}

    (Воронеж; {\it atanov.cs@gmail.com})
\end{center}

\addcontentsline{toc}{section}{Атанов А.В.}

Известная в комплексной геометрии задача классификации голоморфно-однородных гиперповерхностей в $\mathbb{C}^n$ решена Э.Картаном для случая $n = 2$ (см. [1]). Решение аналогичной задачи при $n = 3$ также фактически завершено (см., например, [2]). При этом в комплексных пространствах больших размерностей к настоящему моменту отсутствуют нетривиальные примеры голоморфно-однородных гиперповерхностей. В тексте строится пример одной голоморфно-однородной гиперповерхности в $\mathbb{C}^4$.

\textbf{Утверждение~1.} {\it Любая гиперповерхность из семейства
$$\mathrm{Im}(z_4) = |z_1| \left(\mathrm{Im}(z_2)^2 + \mathrm{Im}(z_3)\right)^\alpha , \; \alpha \in \mathbb{R}\setminus\{0\}$$
является голоморфно-однородной в пространстве $\mathbb{C}^4_{z_1, z_2, z_3, z_4}$.}

Для получения уравнения данного семейства гиперповерхностей будем применять подход, использованный в работах [2, 3] для случая $\mathbb{C}^3$. Данный подход основан на построении реализаций абстрактных алгебр Ли в виде алгебр голоморфных векторных полей.

В работе [3] описана процедура перехода от абстрактной алгебры Ли к соответствующей алгебре голоморфных векторных полей, касательных к однородным гиперповерхностям. Базисные векторные поля такой алгебры будем записывать в виде
$$
e_k = a_k\frac{\partial}{\partial z_1} + b_k\frac{\partial}{\partial z_2} + c_k\frac{\partial}{\partial z_3} + d_k\frac{\partial}{\partial z_4}, \; k = 1, \ldots, 7,
$$
или, сокращённо, $e_k = \left( a_k, b_k, c_k, d_k\right)$. Здесь $a_k$, $b_k$, $c_k$, $d_k$ --- голоморфные (в окрестности некоторой точки гиперповерхности) функции четырёх переменных $z_1$, $z_2$, $z_3$, $z_4$.

Как известно, любое гладкое векторное поле в окрестности неособой точки может быть выпрямлено подходящим координатным диффеоморфизмом (под выпрямлением здесь понимается уменьшение количества переменных, участвующих в записи векторного поля --- вплоть до одной переменной). В работе [3] для случая $\mathbb{C}^3$ показано, как такие упрощения могут проводиться сразу для нескольких полей. Очевидно, что выпрямление одного поля с учётом коммутационных соотношений алгебры упрощает и остальные поля. Последовательно рассматривая все коммутаторы и постепенно упрощая базисный набор полей, приходим к некоторому относительно простому виду указанного набора. Интегрируя затем систему уравнений в частных производных, отвечающую рассматриваемым полям, можно получить явный вид уравнения голоморфно-однородной гиперповерхности.

Используя эту же технику, можно получать голоморфно-однородные объекты и в случае $\mathbb{C}^4$.

Рассмотрим 7-мерную (разложимую) алгебру Ли $\mathfrak{g}$, определяемую следующими коммутационными соотношениями:
$$
\begin{array}{c}
[e_1, e_2] = e_1, [e_1, e_3] = 2e_2, [e_2,e_3] = e_3, \\[5pt]
\, [e_4,e_7] = (h + 1)e_4, [e_5,e_6] = e_4, \\[5pt]
\, [e_5,e_7] = e_5, [e_6,e_7] = he_6, \; |h| \leqslant 1.
\end{array}\eqno{(1)}
$$

Построение реализаций алгебры (1) в виде алгебр голоморфных векторных полей требует рассмотрения ряда подслучаев. В простейшем из них можно считать, что три векторных поля имеют простейший вид:
$$e_1 = (0,0,0,1), e_4 = (0,0,1,0), e_5 = (0,1,0,0).$$
Справедливо следующее утверждение.

\textbf{Утверждение~2.} {\it Базис голоморфной реализации алгебры (1) допускает представление в виде
	$$
	\begin{array}{c}
	e_1 = (0,0,0,1), \quad e_2 = (z_1,0,0,z_4), \\[5pt]
	e_3 = \left(2z_1z_4 + A_3z_1^2,0,0,z_4^2 - \frac{1}{4}A_3^2z_1^2\right), \\[5pt]
	e_4 = (0,0,1,0), \quad e_5 = (0,1,0,0), \quad e_6 = (0,B_6,z_2,0), \\[5pt]
	e_7 = \left(A_7z_1, z_2, 2z_3 + C_7, -\frac{1}{2}A_3A_7z_1\right),
	\end{array}
	\eqno{(2)}
	$$
где $A_3, \, A_7, \, B_6, \, B_7 \in \mathbb{C}$.
}

Уравнение однородной гиперповерхности $M$, соответствующей алгебре (2), будем искать в виде $\mathrm{Im}(z_4) = F(z_1,z_2,z_3,z_4)$. Определяющая функция $\Phi = - \mathrm{Im}(z_4) + F$ этой гиперповерхности должна удовлетворять системе семи уравнений в частных производных
$$\mathrm{Re}\left(\left. e_k\left(\Phi\right) \right|_M\right) \equiv 0, \; k = 1, \ldots, 7.$$

Решив эту систему и выполнив элементарные преобразования координат, получим уравнение голоморфно-однородной гиперповерхности вида
$$\mathrm{Im}(z_4) = |z_1| \left(\mathrm{Im}(z_2)^2 + \mathrm{Im}(z_3)\right)^\alpha , \; \alpha \in \mathbb{R}.$$

\smallskip \centerline {\bf Литература} \nopagebreak

1. {\it Cartan~E.} Sur la g\'eom\'etrie pseudoconforme des hy\-per\-sur\-fa\-ces de deux variables complexes: I / E.~Cartan // Ann. Math. Pura Appl. --- 1932. --- V.~11, №~4. --- P.~17--90.

2. {\it Атанов~А.~В.} Разложимые пятимерные алгебры Ли в задаче о голоморфной однородности в $\mathbb{C}^3$  / А.В.~Атанов, А.В.~Лобода // Итоги науки и техн. Сер. Соврем. мат. и её прил. Темат. обз. -- 2019. -- Т.~173. -- С.~86--115.

3. {\it Beloshapka~V.~K.} Homogeneous hy\-per\-sur\-fa\-ces in $\mathbb{C}^3$, as\-so\-ci\-ated with a model CR-cubic / V.~K.Beloshapka, I.~G.~Kos\-sov\-skiy // J. Geom. Anal. --- 2010. --- V.~20, №~3. --- P.~538--564.
