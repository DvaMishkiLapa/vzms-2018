\vzmstitle[\footnote{Работа выполнена при финансовой поддержке РФФИ и Правительства Республики Татарстан в рамках научного проекта № 18-41-160029.}]{МАТЕМАТИЧЕСКОЕ МОДЕЛИРОВАНИЕ СОБСТВЕННЫХ КОЛЕБАНИЙ УПРУГОГО СТЕРЖНЯ С ПРИСОЕДИНЁННЫМ ГРУЗОМ}
\vzmsauthor{Самсонов}{А.\,А.}
\vzmsauthor{Соловьёв}{П.\,С.}
\vzmsauthor{Соловьёв}{С.\,И.}
\vzmsauthor{Коростелева}{Д.\,М.}
\vzmsinfo{Казань; {\it anton.samsonov.kpfu@mail.ru}}
\vzmscaption


Исследуется обыкновенная дифференциальная задача на собственные значения второго порядка,
описывающая продольные собственные колебания упругого стержня с присоединённым к торцу грузом.
Задача имеет возрастающую последовательность положительных простых собственных значений
с предельной точкой на бесконечности.
Последовательности собственных значений соответствует полная ортонормированная система
собственных функций.
В данной работе изучаются асимптотические свойства собственных значений и собственных функций
при неограниченном увеличении
массы присоединённого груза.
Исходная дифференциальная задача на собственные значения аппроксимируется
сеточной схемой метода конечных элементов произвольного порядка с численным интегрированием на неравномерной сетке.
Устанавливаются оценки погрешности приближённых собственных значений
и собственных функций в зависимости от шага сетки.
Полученные результаты развивают и обобщают результаты работ~[1--3].
Выводы работы могут быть перенесены на случаи
более сложных и важных прикладных задач расчёта собственных колебаний балок, пластин и оболочек
с присоединёнными грузами.



% Оформление списка литературы
\litlist

1. {\it Соловьёв~С.И.}
Нелинейные задачи на собственные значения. Приближённые методы.
Saarbr\"ucken: LAP Lambert Academic Publishing, 2011. 256 с.

2. {\it Соловьёв~С.И.}
Аппроксимация нелинейных спектральных задач в гильбертовом пространстве
// Дифференциальные уравнения. 2015. Т. 51,
№ 7. С. 937--950.

3. {\it Соловьёв~С.И.}
Собственные колебания стержня с упруго присоединённым грузом
// Дифференциальные уравнения. 2017. Т. 53,
№ 3. С. 418--432.
