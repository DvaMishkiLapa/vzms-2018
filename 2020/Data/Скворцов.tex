\vzmstitle[\footnote{Исследование выполнено в рамках Программы Президента Российской Федерации для государственной поддержки ведущих научных школ РФ (грант НШ-2554.2020.1).}]{ТОПОЛОГИЧЕСКИЙ АНАЛИЗ БИЛЛИАРДОВ, ОГРАНИЧЕННЫХ ДУГАМИ СОФОКУСНЫХ КВАДРИК НА ПЛОСКОСТИ МИНКОВСКОГО, В ПОЛЕ С ГУКОВСКИМ ПОТЕНЦИАЛОМ}
\vzmsauthor{Скворцов}{А.\,И.}
\vzmsinfo{Москва; {\it anton.skvortsov.1996@yandex.ru}}
\vzmscaption

Математический биллиард - это движение материальной точки на плоскости в области, ограниченной кусочно"=гладкой кривой. Многочисленные результаты в области исследования теории биллиардов были получены в работах В.В. Ведюшкиной и А.Т. Фоменко [2, 3, 4].

В настоящей работе рассматриваются биллиарды, ограниченные дугами софокусных квадрик на плоскости Минковского, в поле с гуковским потенциалом. В частности, в ходе исследований было получено доказательство интегрируемости по Лиувиллю систем такого рода. Также проводится исследование топологии возникающих в данной задаче слоений Лиувилля [1].

В данном случае семейство софокусных квадрик задаётся следующим уравнением:
$$\frac{x^2}{a-\lambda}+\frac{y^2}{b+\lambda}=1,$$
где $a>b>0$, $\lambda$ "--- параметр квадрики. Отметим также, что при $-b<\lambda<a$ квадрика является эллипсом, при $\lambda<-b$ квадрика является гиперболой с действительной осью $Ox$, а при $\lambda>a$ квадрика является гиперболой с действительной осью $Oy$. Далее в систему добавляется гуковский потенциал с коэффициентом $k$, действующий на материальную точку.

Рассмотрим фазовое пространство $M^4=\{x, y, \dot{x}, \dot{y}\}$, где $(x, y)$ "--- декартовы координаты материальной точки в биллиарде, $(\dot{x}, \dot{y})$ "--- соответствующие координаты вектора скорости. При отражении точки от границы биллиарда вектора скорости до и после соответствующего отражения отождествляются по стандартному закону. Далее введём на многообразии $M^4$ симплектическую структуру $\omega$, заданную следующей матрицей $\Omega=(\omega_{ij})$:
$$(\omega_{ij})=\left(\begin{array}{cccc}
0 & 0 & -1 & 0 \\
0 & 0 & 0 & 1 \\
1 & 0 & 0 & 0 \\
0 & -1 & 0 & 0 \\
\end{array}\right)
.$$
Также определим стандартную скобку Пуассона.

\textbf{Теорема~1.} {\it Биллиард, ограниченный дугами софокусных квадрик на плоскости Минковского, в поле с гуковским потенциалом является интегрируемым по Лиувиллю. Интегралами являются следующие функции:
$$H=\frac{k(x^2-y^2)}{2}+\frac{{\dot{x}}^2-{\dot{y}}^2}{2},$$
$$G=\frac{{\dot{x}}^2}{a}+\frac{{\dot{y}}^2}{b}-\frac{{(x\dot{y}-\dot{x}y)}^2}{ab}-k(1-\frac{x^2}{a}-\frac{y^2}{b}).$$ При этом симпликтическая структура задаётся вышеуказанной матрицей $(\omega_{ij})$.}

Для дальнейшего анализа целесообразно перейти к эллиптическим координатам. В таких координатах интеграл H запишется в следующем виде:
$$H=\frac{k}{2}(a-b)-\frac{k}{2}(\lambda_1+\lambda_2)+\frac{2(a-\lambda_1)(b+\lambda_1)}{\lambda_1-\lambda_2}\mu_1^2+\frac{2(a-\lambda_2)(b+\lambda_2)}{\lambda_2-\lambda_1}\mu_2^2,$$
где $\mu_1$, $\mu_2$ "--- сопряжённые импульсы.

Также благодаря методу, описанному В.В. Козловым в [5], был найден дополняющий его интеграл F:
$$F=2(a-\lambda_1)(b+\lambda_1)\mu_1^2-\frac{k\lambda_1^2}{2}-H\lambda_1.$$

\textbf{Теорема~2.} {\it Дополняющий интеграл F также может иметь следующий вид:
$$F=2(a-\lambda_2)(b+\lambda_2)\mu_2^2-\frac{k\lambda_2^2}{2}-H\lambda_2.$$}

Далее проводится анализ образа отображения момента.

В ходе данной работы также были построены бифуркационные диаграммы, были получены соответствующие молекулы.

% Оформление списка литературы
\litlist

1. {\it Болсинов А.В., Фоменко А.Т.} Интегрируемые гамильтоновы системы. Геометрия, топология, классификация. Том I. — Ижевск: РХД, 1999.

2. {\it Фокичева В.В., Фоменко А.Т.} <<Интегрируемые биллиарды моделируют важные интегрируемые случаи динамики твёрдого тела>>. Доклады РАН, серия: математика, {\bf 465}, №2, 2015, 150-153.

3. {\it Fokicheva V.V., Fomenko A.T.} <<Billiard Systems as the Models for the Rigid
Body Dynamics>>. {\it Studies in Systems, Decision and Control}, Advances in Dynamical
Systems and Control, {\bf 69}, 13–32.

4. {\it Ведюшкина (Фокичева) В.В., Фоменко А.Т.} <<Интегрируемые топологические биллиарды и эквивалентные динамические системы>>. Известия РАН, серия: математика, {\bf 81}, №4, 2017, 20-67.

5. {\it Козлов В.В.} Некоторые интегрируемые обобщения задачи Якоби о геодезических на эллипсоиде. //Прикладная математика и механика, том 59, вып. 1 1995.
