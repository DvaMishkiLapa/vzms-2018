\vzmstitle{АНАЛИТИЧЕСКИЕ РЕШЕНИЯ ЗАДАЧИ ДЛЯ ВЫРОЖДЕНИЯ СИСТЕМЫ В ЧАСТНЫХ ПРОИЗВОДНЫХ}
\vzmsauthor{Зубова}{С.\,П.}
\vzmsauthor{Мохамад}{А.\,Х.}
\vzmsinfo{Воронеж, ВГУ; {\it spzubova@mail.ru}; {\it abdulftah.hosni90@gmail.com}}
\vzmscaption

Рассматривается уравнение
\begin{equation}
 A(\frac{\partial U}{\partial t} + \alpha U ) = B (\frac{\partial U}{\partial x} + \beta U) + f(t,x),
\end{equation}
где $A,B: E \to E$, $E$ банахово пространство, $A$-замкнутый линейный оператор с плотной в Е областью определения $D(A)$. имеющий число 0 нормальным собственным числом, $B \in L(E,E)$, $B^{-1}$ не существует, $f(t,x)$ - непрерывная вектор-функция со значениями в $E$, $U=U(t,x)\in E $, $\alpha, \beta$ скалярные функции; $\alpha = \alpha(t,x), \beta=\beta(t,x)$, $(t,x) \in T\times X$, где $T=[0,t_{0}], X=[0,x_{0}]$.

Рассматривается регулярный случай, т.е. при \(\lambda\) достаточно малых по модулю отличных от нуля (\(\lambda \in \dot{U}(0) \)) оператор \((A-\lambda B)\) обратим. В этом случае оператор $(A-\lambda B)^{-1}A$ имеет число $0$ нормальным собственным числом
[1], что позволяет разложить пространство $E$ в прямую сумму подпространств $M^{(1)}$ и $N^{(1)}$ с проекторами на них $Q^{(1)}$ и $P^{(1)}$ соответственно. Уравнение (1) расщепляется на уравнения
в этих подпространствах и решается в них с условиями
$$Q^{(1)}U(0,x)=\varphi (x)\in M^{(1)}, \quad P^{(1)}U(t,0)=\psi (t)\in
N^{(1)}. \eqno{(2)}$$

\textbf{Теорема~1.} {\it Решение $U(t,x)$ задачи $(1)$, $(2)$
существует и единственно.}

Получены формулы для $U(t,x)$. Приводится иллюстрирующий пример.

\smallskip \centerline{\bf Литература}\nopagebreak
1.
{\it Зубова С.П.} Solution of the homogeneous Cauchy problem for an
equation with a Fredholm operator multiplying the derivative
// S.P.~Zubova // Doklady Mathematics. - 2009. - Vol. 80, No. 2.
--- P.~710--712.

2. {\it Нгуен Х.Д.} О моделировании с использованием дифференциально"=алгебраических уравнений в
частных производных / Х.Д.~Нгуен, В.Ф.~Чистяков // Вестник ЮУрГУ. Серия:
Математическое моделирование и программирование. - 2013. - Т. 6, № 1. - С.~98--111.

3. {\it Зубова С.П., Мохамад А.Х.} Решение одной задачи для дескрипторного уравнения в частных производных. Современные методы теории краевых задач. Материалы Международной конференции ВВМШ «ПОНТРЯГИНСКИЕ ЧТЕНИЯ — XXX» (3–9 мая 2019 г.). С. 145.
