\vzmstitle[\footnote{Исследование выполнено за счёт гранта Российского научного фонда (проект № 18-11-00002).}]{ИНТЕГРАЛЫ ОТ ЦЕЛЫХ, МЕРОМОРФНЫХ И СУБГАРМОНИЧЕСКИХ ФУНКЦИЙ ПО ЛУЧУ}
\vzmsauthor{Хабибуллин}{Б.\,Н.}
\vzmsinfo{Уфа; {\it khabib-bulat@mail.ru}}
\vzmscaption

В теории роста целых и мероморфных функций $f$ на комплексной плоскости $\mathbb C$ нередко возникала необходимость в оценках интегралов с подынтегральным выражением
$$
{\sf M}_{\ln^+|f|}(r),\quad \text{где }
\mathsf{M}_{u}(r):=\sup_{|z|=r}u(z), \; u^+:=\max\{0.u\},
$$
возможно, дополненным некоторой весовой функцией"=множителем. Такие оценки относительно множества интегрирования можно отнести к одному из следующих двух типов: по интервалам или дугам на луче или окружности или же по малым подмножествам на таких интервалах или дугах.
Среди исходных результатов первого типа "---

\noindent
\textbf{Теорема Р.~Неванлинны} {\rm ([1], [2; гл. 1, теорема 7.2]).} {\it
Для мероморфной функции $f$ на $\mathbb C$ с характеристикой Неванлинны $T_f$ и для числа $k>1$ справедливо неравенство
$$
\frac{1}{r}\int_1^r{\mathsf{M}}_{\ln^+|f|}(t)\mathrm{d} t\leq C(k)T_f(kr,f),
\quad r\geq 1,
$$
где число $C(k)>1$ зависит только от $k$.
}

Истоки второго типа оценок "--- лемма Эдрея\--Фукса о малых дугах [3; {\bf 2}, лемма III, {\bf 9}], [2; гл. 1, теоремы 7.3, 7.4], а также приведённая в [4] без доказательства

\noindent
{\bf Лемма Гришина\,--\,Содина о малых интервалах}{\rm ([4; ле\-мма 3.1]).}
{\it Пусть в условиях Теоремы Р. Неванлинны
$\lambda$ "--- линейная мера Лебега на вещественной оси $\mathbb R$ и
$E\subset [1,r]\subset \mathbb R$ "---
$\lambda$-измеримое подмножество. Тогда
$$
\frac{1}{r}\int_{E}{\mathsf{M}}_{\ln^+|f|}(t)\mathrm{d} t\leq
A\frac{k}{k-1}\Bigl(\frac{\lambda( E)}{r}\ln
\frac{2r}{\lambda( E)} \Bigr)T_f(kr),
$$
где $A$ "--- абсолютная постоянная.
}

Версия леммы Гришина\--Содина о малых интервалах для {\it субгармонических функций конечного порядка\/} "---

 \noindent
{\bf Теорема Гришина\--Малютиной о малых интервалах} {\rm ([5; теорема 8]). {\it Пусть $v\not\equiv -\infty$ "--- субгармоническая функция уточнённого порядка $\mathbf{\rho}$ и $E\subset [1,R]$ "--- $\lambda$-измеримое множество. Тогда для некоторого числа $M$, не зависящего от $r, \theta, E$, имеет место неравенство
$$
\int_{E} \bigl|v(te^{i\theta})\bigr| \mathrm{d} t\leq
M\frac{\lambda( E)}{r}\ln
\frac{4r}{\lambda( E)}\, r^{\rho(r)+1}.
$$
}

Для $\lambda$-измеримого множества $E\subset \mathbb R$ и $p\in [1,+\infty]$
через $L^p(E)$ обозначаем \textit{$L^p$-пространство\/} функций $f\colon E\to \mathbb R$ с нормой $\|f\|_{L^p(E)}:=\Bigl(\int_E|f|^p\,\mathrm{d}\lambda\Bigr)^{1/p}$ при $p<+\infty$ и с нормой \textit{существенная верхняя грань\/} $\mathrm{ess}\sup\limits_E |f|$ на $E$ при $p=+\infty$.

Существенное обобщение и развитие Теоремы Гришина\--Малютиной о малых интервалах уже при $p=+\infty$ "---

\noindent
\textbf{Теорема~о малых интервалах с весом.} {\it При любом значении $p\in (1,+\infty]$ и при значении $q$, определяемом из равенства $1/p+1/q=1$, для любого числа $k>1$ существует такое число $A_p(k)\geq 1$, что
для любого числа $R> 0$, для любого $\lambda$-измеримого множества $E\subset [0,R]$, для любой функции $g\in L^p(E)$ и для любой субгармонической функции $v$ со значением $v(0)\geq 0$ имеет место неравенство
$$
\int_E\mathsf{M}_{|v|}g\,\mathrm{d} \lambda \leq A_p(k)\mathsf{M}_{v}(kR)
\|g\|_{L^p(E)}\bigl(\lambda(E)\bigr)^{1/q}\ln \frac{e^qR}{\lambda(E)}.
$$
}
Из Теоремы о малых интервалах с весом уже при $p=+\infty$, $g\equiv 1$ и $v:=\ln |f|$ из известных вариантов определения характеристики Неванлинны и её взаимосвязей с другими характеристиками роста мероморфных функций легко получаются Теорема Р.~Неванлинны и Лемма Гришина\,--\,Содина о малых интервалах. Более того, равномерный характер оценки в ней позволяет получить многомерные версии Теоремы о малых интервалах с весом для разностей плюрисубгармонических функций в $\mathbb C^n$ и для мероморфных функций в $\mathbb C^n$.

Версия Теоремы о малых интервалах с весом для случая $p=+\infty$ направлена
в печать [6]. Её варианты с $p\in (1,+\infty]$ и с применениями к плюрисубгармоническим и мероморфным функциям в $\mathbb C^n$, а также
к разностям субгармонических функций в $\mathbb R^n$ готовятся для отправки в печать.

\litlist

\selectlanguage{english}

1. {\it Nevanlinna~R.\/}
 Le th\'eorem\`e de Picard\,--\,Borel et la th\'eorie des fonctions m\'eromorphes
 Paris: Gauthier-Villars, 1929, Pp. 171.

\selectlanguage{russian}

2. {\it Гольдберг А.\,А., Островский И.\,В.\/}
Распределение значений мероморфных функций.
М.: Наука, 1970, 591 с.

\selectlanguage{english}

3. {\it Edrei A., Fuchs W.\,H.\,J.\/}
 Bounds for number of deficient values of certain classes of meromorphic functions~//
 Proc. London Math. Soc. 1962, V. 12, P. 315--344 .


\selectlanguage{russian}

4. {\it Гришин А.\,Ф., Содин М.\,Л.\/}
Рост по лучу, распределение корней по аргументам целой функции конечного порядка и одна теорема единственности~//
Респ. сб. <<Теория функций, функциональный анализ и их приложения>>,
Харьков: Вища школа, 1988, вып. 50, С.~47--61.

5. Гришин А.\,Ф., Малютина~Т.\,И.,
 Новые формулы для индикаторов субгармонических функций~//
 Матем. физ., анал., геом. 2005, Т.~12, вып. 1, С. 25--72.

6. {\it Габдрахманова Л.\,А., Хабибуллин Б.\,Н.\/} Одна теорема о малых интервалах для субгармонических функций~// Известия вузов. Математика, 2020 (направлено в печать).
