\vzmstitle[\footnote{Работа выполнена при поддержке гранта РФФИ (проект 18-29-10085мк).}]{ТОЧНЫЕ РЕШЕНИЯ НЕКЛАССИЧЕСКОГО НЕЛИНЕЙНОГО УРАВНЕНИЯ ЧЕТВЕРТОГО ПОРЯДКА}
\vzmsauthor{Аристов}{А.\,И.}
\vzmsinfo{Москва, МГУ им. М.В. Ломоносова; {\it ai\_aristov@mail.ru}}
\vzmscaption

Работа посвящена точным решениям уравнения
$$
\frac{\partial^4u}{\partial t^2\partial x^2}-\frac{\partial}{\partial t}\left(u\frac{\partial u}{\partial t}\right)+
\frac{\partial^2u}{\partial x^2}=0. \eqno(1)
$$

В статье [1] рассмотрены начально"=краевые задачи на отрезке и на луче для модельного уравнения
теории ионно"=звуковых волн
$$
\frac{\partial^2}{\partial t^2}\left(\frac{\partial^2u}{\partial x^2}-\varepsilon u-\frac{\varepsilon^2u^2}{2}\right)+
\frac{\partial^2u}{\partial x^2}=0. \eqno(2)
$$
Для первой задачи обоснована однозначная разрешимость, по крайней мере, локально по времени, для
второй получена верхняя оценка времени существования слабого решения и, кроме того, указаны начальные
данные, для которых имеет место мгновенное разрушение решения.

Уравнение (2) сводится к (1) с помощью линейной замены. В данной работе построено 6 классов
точных решений для (1), представимых через элементарные и специальные функции.

\textbf{Теорема~1.} {\it Существуют точные решения уравнения (1), имеющие следующие типы качественного
поведения:
\begin{itemize}
\item ограниченность глобально по времени;
\item ограниченность на любом ограниченном промежутке времени, но не глобально;
\item обращение в бесконечность на ограниченных промежутках времени.
\end{itemize}
}

Справедливость этого утверждения вытекает из того, что среди построенных решений имеются решения,
соответствующие всем типам, перечисленным в теореме.

При построении точных решений использовались методы аддитивного и мультипликативного разделения
переменных, метод бегущей волны и поиск решений специального вида.




\smallskip \centerline {\bf Литература} \nopagebreak

1. {\it Корпусов М.О.} О мгновенном разрушении слабого решения одной задачи теории плазмы на полупрямой.
Дифференциальные уравнения. 2019 г. Том 55. Номер 1. С. 59--66.

2. {\it Полянин А.Д., Зайцев В.Ф., Журов А.И.} Методы решения нелинейных уравнений математической физики и
механики. М., Физматлит, 2005.
