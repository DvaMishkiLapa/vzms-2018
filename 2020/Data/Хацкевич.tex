\vzmstitle{ОБ ЭКСТРЕМАЛЬНЫХ СВОЙСТВАХ СРЕДНИХ НЕЧЁТКО-СЛУЧАЙНЫХ ВЕЛИЧИН}
\vzmsauthor{Хацкевич}{В.\,Л.}
\vzmsinfo{Воронеж; {\it vlkhats@mail.ru}}
\vzmscaption

Объектом настоящего исследования являются нечётко-случайные величины. Нечёткие множества как предмет рассмотрения введены в пионерской работе Заде [1]. С этого момента и по настоящее время они активно исследуются и находят важное приложение в различных прикладных областях (финансовая математика, теория принятия решений, мягкие вычисления и др.). В частности, много работ посвящено изучению нечётко-случайных величин, т.е. случайных величин, множествами значений которых являются нечёткие числа. Из последних работ укажем работы [2] и [3]. В литературе рассматриваются различные определения нечётко-случайных чисел и нечётко-случайных величин. Ниже мы используем терминологию, принятую в работах [2] и [3].



Как известно, математическое ожидание $EX$ случайной величины $X$ минимизирует среднеквадратическое отклонение $E(X-a)^2$ по всем действительным $a\in R$, т.е.
\begin{equation}
E(X-E(X))^2\leq E(X-a)^2\,\,\,\,\,(\forall a\in R).
\end{equation}

В данной работе рассматриваются нечётко-случайные
\\величины $\tilde{X}$, для которых областью $J$ возможных значений являются нечёткие числа. Устанавливаются экстремальные свойства вида (1). Кроме того, рассматриваются оптимальные линейные регрессии нечётко-случайных величин. В литературе рассматриваются различные аспекты нечётких линейных регрессий. Мы рассматриваем случай чётких (числовых) коэффициентов (ср. [4], [5]). На этом пути получен результат об оптимальной в среднеквадратичном нечёткой регрессии. Установлено, что оптимальная регрессия обладает максимальной корреляцией с прогнозируемой нечётко-случайной величиной. Приведём необходимые термины и обозначения.

Множество $\tilde{z}\subseteq R^2$, лежащее в полосе $0\leq \eta\leq 1$, называется нечётким числом, если существуют монотонные, непрерывные слева функции $z^L:[0,1]\rightarrow R$ и $z^R:[0,1]\rightarrow R$, где $z^L$ не убывает, $z^R$ не возрастает, причём $z^L(1)\leq z^R(1)$ такие, что для любого $\eta_0\in[0,1]$ пересечение множества $\tilde{z}$ с прямой $\eta=\eta_0$ представляет собой множество
$$
\{(\xi, \eta): z^L(\eta_0)\leq\xi\leq z^R(\eta_0), \eta=\eta_0\}.
$$
Функции $z^L(\eta)$ и $z^R(\eta)$ называются, соответственно, левым и правым индексом нечёткого числа $\tilde{z}$.

Пусть $(\Omega, \Sigma, P)$ - вероятностное пространство, где $\Omega$ - множество элементарных событий, $\Sigma $- $\sigma$-алгебра, состоящая из подмножеств множества $\Omega$, $P$ - вероятностная мера.

Измеримое отображение $\tilde{X}:\Omega\rightarrow J$ называется нечётко-случайной величиной, если при любом $\omega\in\Omega$ множество $\tilde{X}(\omega)$ является нечётким числом.

Индексы нечёткого числа $\tilde{X}(\omega)$ будем обозначать
\\$X^L(\omega, \eta)$ и $X^R(\omega, \eta)$. Функции $X^L(\omega, \eta)$ и $X^R(\omega, \eta)$ называются, соответственно, левым индексом и правым индексом нечётко-случайной величины $\tilde{X}(\omega)$.



Скалярным произведением $\left\langle \tilde{X}, \tilde{Y}\right\rangle_{\omega}$ нечётко-случайных величин $\tilde{X}$ и $\tilde{Y}$ называется величина
$$
\left\langle \tilde{X}, \tilde{Y}\right\rangle_{\omega} = 0.25\int\limits_0^1\int\limits_{\Omega}(X^L(\omega, \eta) +
$$
$$
 + X^R(\omega, \eta))(Y^L(\omega, \eta) + Y^R(\omega, \eta))dPd\eta,
$$
а полунормой $||\tilde{X}||_{\omega} = \left\langle \tilde{X}, \tilde{Y}\right\rangle_{\omega}^{1/2}.$


Положим
\begin{equation}
x^L(\eta) = \int\limits_{\Omega}\tilde{X}^L(\omega, \eta)dP,\,\,\,\,x^R(\eta) = \int\limits_{\Omega}\tilde{X}^R(\omega, \eta)dP.
\end{equation}

Нечётким ожиданием нечётко-случайной величины $\tilde{X}$
\linebreak
называется нечёткое число $\tilde{x}$ с левым индексом $x^L(\eta)$ и правым индексом $x^R(\eta)$, определяемыми формулами (2).
Ожиданием $E(\tilde{X})$ нечётко"=случайной величины $\tilde{X}$ называется число, определяемое формулой
$$
E\tilde{X} = 0.5\int\limits_0^1\int\limits_{\Omega}(X^L(\omega, \eta) + X^R(\omega, \eta))dPd\eta.
$$
Ковариацией $Cov(\tilde{X}, \tilde{Y})$ нечётко-случайных величин $\tilde{X}$, $\tilde{Y}$ называется выражение

$$
Cov(\tilde{X}, \tilde{Y}) = 0.25\int\limits_0^1\int\limits_{\Omega}(X^L(\omega, \eta) + X^R(\omega, \eta) - x^L(\omega, \eta) - $$
$$
-x^R(\omega, \eta))(Y^L(\omega, \eta) + Y^R(\omega, \eta) - y^L(\omega, \eta) - y^R(\omega, \eta))dPd\eta
,
$$
где $x^L(\omega, \eta)$ и $x^R(\omega, \eta)$ определяются формулами (2) и аналогично $y^L(\omega, \eta)$ и $y^R(\omega, \eta)$.


Дисперсией $Var(\tilde{X})$ нечётко-случайной величины $\tilde{X}$ называется $Cov(\tilde{X}, \tilde{X})$.



\textbf{Теорема 1.} \textit{Для заданной нечётко-случайной величины $\tilde{X}(\omega)$ минимум выражения $||\tilde{X} - a||_{\omega} $ по всем действительным числам $a$ достигается при $a_0 = E(\tilde{X})$, т.е.}
\begin{equation}
||\tilde{X}-E(\tilde{X})||_{\omega}\leq ||\tilde{X}-a||_{\omega}\,\,\,\,\,(\forall a\in R).
\end{equation}

\textbf{Теорема 2. }\textit{Для заданной нечётко-случайной величины $\tilde{X}$ минимум выражения $||\tilde{X} - \tilde{a}||_{\omega}$ по всем нечётким числам $\tilde{a}$ достигается при $\tilde{a}_0 = \tilde{x}$ нечётком ожидании случайной величины $\tilde{X}$, т.е.}
\begin{equation}
||\tilde{X}-\tilde{x}||_{\omega}\leq ||\tilde{X}-\tilde{a})||_{\omega}\,\,\,\,\,(\forall \tilde{a}\in J).
\end{equation}

Формулы (3), (4) обобщают свойство (1).

Рассмотрим прогнозируемую нечётко-случайную величину $\tilde{Y}$ и попарно независимые прогнозирующие нечётко-случайные величины $\tilde{X}_1, \tilde{X}_2,..., \tilde{X}_n$. Исследуем вопрос об аппроксимации случайной величины $\tilde{Y}$ линейными комбинациями вида $\tilde{a} + \sum\limits_{i=1}^n\alpha_i\tilde{X}_i$, где $\alpha_i$ - вещественные числа, а $ \tilde{a}$ - нечёткое число. Точнее, задача состоит в подборе вещественных коэффициентов $\alpha_i$ и нечёткого числа $\tilde{a}$ так, чтобы ошибка $||\tilde{Y} - \tilde{a}- \sum\limits_{i=1}^n\alpha_i\tilde{X}_i||_{\omega}^2$ была минимальной.

В дальнейшем будем предполагать выполненными условия:

1) рассматриваемые нечётко-случайные величины $\tilde{Y}$ и $\tilde{X}_i$\,\,\,$(i = 1, 2,...,n)$ имеют ограниченные по абсолютной величине индексы.

2) все нечёткие ожидания нечётко-случайных величин $\tilde{Y}$ и $\tilde{X}_i$\,\,\,$(i = 1, 2,...,n)$ совпадают. $\tilde{y} = \tilde{x}_i$\,\,\,$(i = 1, 2,..., n)$.

\textbf{Теорема 3.} \textit{Пусть для заданной нечётко"=случайной
величины $\tilde{Y}$ и системы попарно независимых
нечётко"=случайных величин $\tilde{X}_1, \tilde{X}_2,..., \tilde{X}_n$ выполнены условия 1), 2). Тогда оптимальной в среднеквадратичном смысле линейной несмещённой оценкой нечётко-случайной величины $\tilde{Y}$ по системе $X_1, X_2,..., X_n$ вида $\tilde{y}+\sum\limits_{i=1}^n\alpha_i(\tilde{X}_i - \tilde{y})$ является оценка }
\begin{equation}
\hat{Y} = \tilde{y} + \sum\limits_{i=1}^n\frac{1}{\sqrt{Var(\tilde{X}_i)}}Cov(\tilde{Y}_i, \tilde{X}_i)(\tilde{X}_i-\tilde{y}),
\end{equation}
\textit{где $\tilde{y}$ "--- общее нечёткое ожидание нечётко-случайных величин $\tilde{Y}$ и $\tilde{X}_i$. На $\hat{Y}$ достигается минимум выражения $||\tilde{Y} - \tilde{y} -\sum\limits_{i=1}^n\alpha_i(\tilde{X}_i-\tilde{y})||_{\omega}^2$ по всем действительным $\alpha_i$.}

Определим коэффициент корреляции между нечётко"=случайными величинами $\tilde{Y}$ $\tilde{Z}$ равенством
$$
\rho[\tilde{X}, \tilde{Z}] = \frac{Cov(\tilde{X}, \tilde{Z})}{\sigma(\tilde{Y})\sigma(\tilde{Z})},
$$
где $\sigma^2(\tilde{Y}) = Var(\tilde{Y})$, $\sigma^2(\tilde{Z}) = Var(\tilde{Z})$.

\textbf{Теорема 4.} \textit{Оценка $\hat{Y}$, определяемая формулой (5),
обладает наибольшим коэффициентом корреляции с $\tilde{Y}$
по сра\-в\-не\-нию с другими оценками вида $\tilde{W}_n = \tilde{y} + \sum\limits_{i=1}^n\alpha_i(\tilde{X}_i - \tilde{y})$
с~произвольными вещественными коэффициентами ~$\alpha_i$.
\linebreak
Т.е. для коэффициентов корреляции выполнено неравенство
\linebreak
$\rho[\tilde{W}_n, \tilde{Y}]\geq \rho[\hat{Y},\tilde{Y}] $. }

% Оформление списка литературы
\litlist

1. {\it Zadeh L.A.} Fuzzy sets. Information and Control, 1965, 8, p. 338 - 353

2. {\it Feng Y,. Hu. L., Shu H.} The variance and covariance of fuzzy random variables. Fuzzy Systems,
2001, 120, p. 487-497

3. {\it Шведов А.С.} Оценивание средних и ковариаций
\\нечётко-случайных величин. Прикладная эконометрика,
\\2016, т. 42. с. 121-138

4. {\it Bargelia A., Pedrycz W., Nakashima T.} Multiple
\\Regression with Fuzzy Data, Fuzzy Sets and Systems, 2007, pp. 2169 - 2188

5. {\it Colubi A.} Statistical inference about the means of fuzzy random variables: Applica analysis of fuzzy-and real-valued data. Fuzzy Sets and sytems, 2009, 344-356
