\vzmstitle{ОБРАТНЫЕ ЗАДАЧИ ДЛЯ УРАВНЕНИЙ СМЕШАННОГО ПАРАБОЛО-ГИПЕРБОЛИЧЕСКОГО ТИПА В ПРЯМОУГОЛЬНОЙ ОБЛАСТИ}
\vzmsauthor{Сидоров}{С.\,Н.}
\vzmsinfo{Стерлитамак; {\it stsid@mail.ru}}
\vzmscaption


Рассмотрим уравнение
$$
 Lu=F(x,t),\eqno{(1)}
$$
здесь
$$
 Lu=
\left\{\begin{array}{l}
      t^nu_{xx}-u_t-bt^nu,\\
      (-t)^mu_{xx}-u_{tt}-b(-t)^mu,
\end{array}\right.
$$
$$
F(x,t)=
\left\{\begin{array}{l}
      f_1(x)g_1(t), \,\ t>0, \\
      f_2(x)g_2(t), \,\ t<0,
\end{array}\right.
$$
в прямоугольной области
$
D=\{(x,t)|\,0<x<l,\,-\alpha<t<\beta\},
$
где $n>0$, $m>0$, $l>0$, $\alpha>0$, $\beta>0$ "--- заданные действительные числа, и $b$ "--- заданное любое действительное число, и поставим следующие задачи.


\textbf{Задача 1.} \emph{Найти функцию $u(x,t)$, удовлетворяющую следующим условиям: }
$$
u(x,t)\in C(\overline{D})\cap C^1_t(D)\cap C^1_x(\overline{D})\cap C^{2}_{x}(D_+)\cap C^2(D_-); \eqno{(2)}
$$
$$
Lu(x,t)\equiv F(x,t),\;\;\; (x,t)\in D_{+}\cup D_{-}; \eqno{(3)}
$$
$$
u(0,t)=u(l,t)=0, \hskip3mm -\alpha\leq t\leq \beta; \eqno{(4)}
$$
$$
u(x,-\alpha)=0,\hskip3mm 0\leq x\leq l, \eqno{(5)}
$$
\emph{где $F(x,t)$ "--- заданная достаточно гладкая функция, $D_+=D\cap\{t>0\}$, $D_-=D\cap\{t<0\}$.}


\textbf{Задача 2.} \emph{Найти функции $u(x,t)$ и $g_{1}(t)$, удовлетворяющие условиям $(2)$ -- $(5)$ и }
$$
g_1(t)\in C[0,\beta]; \eqno{(6)}
$$
$$
u(x_0,t)=h_1(t), \hskip3mm 0<x_0<l, \hskip3mm 0\leq t\leq \beta, \eqno{(7)}
$$
\emph{где $f_i(x)$, $i=1,2$, $g_2(t)$, $h_1(t)$ "--- заданные функции, $x_0$ "--- заданная точка из интервала $(0,l)$, $D_+=D\cap\{t>0\}$, $D_-=D\cap\{t<0\}$.}


\textbf{Задача 3.} \emph{Найти функции $u(x,t)$ и $g_{2}(t)$, удовлетворяющие условиям $(2)$ -- $(5)$ и}
$$
g_2(t)\in C[-\alpha,0], \eqno{(8)}
$$
$$
u(x_0,t)=h_2(t),\hskip3mm 0<x_0<l, \hskip3mm -\alpha\leq t\leq 0, \eqno{(9)}
$$
\emph{где $f_i(x)$, $i=1,2$, $g_1(t)$, $h_2(t)$ "--- известные функции.}


\textbf{Задача 4.} \emph{Найти функции $u(x,t)$, $g_{1}(t)$, $g_{2}(t)$, удовлетворяющие условиям $(1.2)$ -- $(1.9)$, здесь $f_i(x)$, $h_i(t)$, $i=1,2$, -- заданные функции.}


Отметим, что ранее задачи 1 -- 4 впервые поставлены и изучены в работах [1, с. 228--238], [2] для уравнения (1) при $n=m=0$. Начально"=граничная задача (1.2) -- (1.5) для уравнения (1) изучена в работах [3 -- 5], когда $n=0$, $m>0$; $n>0$, $m=0$ и $n>0$, $m>0$.
В работе [6] изучены обратные задачи для уравнения (1) при $n=0$, $m>0$, по отысканию функций $u(x,t)$ и $f_i(x)$, когда $g_i(t)\equiv1$. В статьях [7, 8] рассмотрены задачи 1 -- 4 для уравнения (1), когда $n>0$, $m=0$ и $n=0$, $m>0$ соответственно.


В данной работе изучены обратные задачи 2 -- 4 по отысканию сомножителей правой части уравнения смешанного параболо"=гиперболического типа со степенным вырождением на линии изменения типа, исследование которых проводится на основе решения прямой начально"=граничной задачи 1, изученной в работах [3 -- 5]. Решение обратных задач 2 -- 4 эквивалентно редуцировано к разрешимости нагруженных интегральных уравнений. Используя теорию интегральных уравнений доказаны соответствующие теоремы единственности и существования решений поставленных обратных задач и указаны явные формулы решения.

Работа выполнена при финансовой поддержке РФФИ-Перспектива (№ 19-31-60016).

% Оформление списка литературы
\smallskip \centerline {\bf Литература} \nopagebreak

1. {\it Сабитов К.\,Б.} {Прямые и обратные задачи для уравнений смешанного параболо"=гиперболического типа. М.: Наука, 2016. 272~с.}

2. {\it Сабитов К.\,Б.} {Начально"=граничная и обратные задачи для неоднородного уравнения смешанного параболо"=гиперболического уравнения~// Матем. заметки. 2017. Т.~102. Вып.~3. С.~415--435.}

3. {\it Сабитов К.\,Б., Сидоров С.\,Н.} {Об одной нелокальной задаче для вырождающегося па\-ра\-бо\-ло"=гиперболического уравнения~// Диф. уравнения. 2014. Т.~50, \No~3. С.~356--365.}

4. {\it Сабитов К.\,Б., Сидоров С.\,Н.} {Начально"=граничная задача для неоднородных вырождающихся уравнений смешанного параболо"=гиперболического типа~// Итоги науки и техн. Сер. Соврем. мат. и её прил. Тем. обз. 2017. Т.~137. С.~26--60.}

5. {\it Sabitov K.\,B., Sidorov S.\,N.} {Initial-Boundary-Value Problem for Inhomogeneous Degenerate Equations of Mixed Parabolic-Hyperbolic Type~// Journal of Mathematical Sciences. 2019. V.~236. Issue~6. P.~603--640.}

6. {\it Сабитов К.\,Б., Сидоров С.\,Н.} {Обратная задача для вырождающегося параболо"=гиперболического уравнения с нелокальным граничным условием // Известия Вузов. Математика. 2015. №~1. С.~46--59.}

7. {\it Сидоров С.\,Н.} {Обратные задачи для уравнения смешанного параболо"=гиперболического типа с вырождающейся параболической частью~// Сиб. электрон. матем. изв. 2019. Т.~16. С.~144--157.}

8. {\it Сидоров С.\,Н.} {Обратные задачи для вырождающегося смешанного параболо"=гиперболического уравнения по нахождению сомножителей правых частей, зависящих от времени~// Уфимский матем. журнал. 2019. Т.~11. №~1. С.~72--86.}
