\begin{center}
    {\bf ДЕЙСТВИЕ И НЕПРЕРЫВНОСТЬ МАТРИЧНОГО ОПЕРАТОРА С ЧАСТНЫМИ ИНТЕГРАЛАМИ В ПРОСТРАНСТВЕ НЕПРЕРЫВНЫХ ФУНКЦИЙ\footnote{Работа поддержана РФФИ (проект № 19-41-480002).}}\\

    {\it Н.И. Трусова}

    (Липецк; {\it trusova.nat@gmail.com})
\end{center}

\addcontentsline{toc}{section}{Трусова Н.И.}

Рассмотрим матричный оператор $A= (A_{ij})_{i,j=1}^n,$
где операторы $A_{ij} \ \ (i,j=1,...,n)$ с частными интегралами определяются равенствами
$$
A_{ij}=C_{ij}+L_{ij}+M_{ij}+N_{ij}, \ \ i,j=1,...,n,
$$
а операторы $C_{ij},L_{ij},M_{ij},N_{ij}$ задаются следующим образом
$$
(C_{ij}x_j)(t,s)=c_{ij}(t,s)x_j(t,s),
$$
$$
(L_{ij}x_j)(t,s)=\int\limits_T l_{ij}(t,s,\tau)x_j(\tau,s)d\tau,
$$
$$
(M_{ij}x_j)(t,s)=\int\limits_S m_{ij}(t,s,\sigma)x_j(t,\sigma)d\sigma,
$$
$$
(N_{ij}x_j)(t,s)=\int\int\limits_D n_{ij}(t,s,\tau,\sigma)x_j(\tau,\sigma)d\tau d\sigma,
$$
где $T=[a,b],$ $S=[c,d],$ $t,\tau\in T,$ $s,\sigma\in S,$ $D=T\times S $, $c_{ij}, l_{ij}, m_{ij}$ и $n_{ij} \ \ (i,j=1,...,n)$ --- вещественные функции.

Пусть $C(D)$ --- пространство непрерывных на $D$ функций с супремум нормой, где $x_j\in C(D)$ $(j=1,\dots,n)$ и $S(D)$ --- пространство измеримых и почти всюду конечных на $D$ функций со значениями в $R$.

Отметим, что $C(D)$ -- банахово пространство, а $S(D)$ -- полное
метрическое пространство, сходимость в котором совпадает со
сходимостью по мере. Имеет место непрерывное вложение
$C(D)\subset S(D)$. Справедлива

\textbf{Теорема~1.} {\it Для того, чтобы оператор $A$ действовал в $C(D)$ необходимо и достаточно, чтобы операторы
\\$A_{ij} \ \ (i,j=1,...,n)$ действовали в $C(D)$. При этом оператор $A$ является непрерывным.}

Для операторов с частными интегралами справедлив
\\аналог теоремы С. Банаха о непрерывности интегрального оператора.

\textbf{Теорема~2.} {\it Если оператор $A$ действует из $C(D)$ в
\\$C(D)$, то он непрерывен.}

Рассмотрим достаточные условия действия оператора $A$ в
$C(D)$. Эти условия являются и условиями непрерывности оператора
$A$ в $C(D)$ в силу теоремы 2.

Определённая на $D\times\Omega,$ где $\Omega\in\{[a,b], [c,d], D\},$ измеримая функция $a(t,s,\omega)$ называется $L_1$ --- непрерывной, если для любого $\varepsilon>0$ существует $\delta>0$ такое, что при $|t-t'|<\delta, |s-s'|<\delta$ справедливо равенсто
$$
\int\limits_\Omega|a(t,s,\omega)-a(t',s',\omega)|d\omega<\varepsilon
$$
и $L_1$ --- ограниченной, если найдётся такое число $B$, что для любой точки
$(t,s)\in D$
$$
\int\limits_\Omega|a(t,s,\omega)|d\omega\le B.
$$

\textbf{Теорема~3.}
{\it Пусть функции $l_{ij},m_{ij},n_{ij}$ $L_1$ --- непрерывны и $L_1$ --- ограничены.
Тогда оператор $A$ действует в $C(D)$ и непрерывен.}

Работа поддержана РФФИ (проект № 19-41-480002).


% Оформление списка литературы
\smallskip \centerline {\bf Литература} \nopagebreak

1. {\it Калитвин А.С.} Линейные операторы с частными интегралами. Воронеж: ЦЧКИ, 2000. 252 с.

2. {\it Калитвин А.С., Фролова Е.В.} Линейные уравнения с частными интегралами. С-теория. Липецк: ООО <<Оперативная полиграфия>>, 2015. — 195 с.
