\vzmstitle[\footnote{Работа выполнена при финансовой поддержке Российского фонда фундаментальных исследований (научные проекты \No~18-01-00643~А и \No~18-51-54001~Вьет{\_}а) и Иркутского государственного университета (индивидуальный исследовательский грант \No~091-19-212).}]{ФУНДАМЕНТАЛЬНОЕ РЕШЕНИЕ ФУНКЦИОНАЛЬНО-ДИФФЕРЕНЦИАЛЬНОГО ОПЕРАТОРА С ВЫРОЖДЕНИЕМ В БАНАХОВЫХ ПРОСТРАНСТВАХ}
\vzmsauthor{Шеметова}{В.\,В.}
\vzmsauthor{Орлов}{C.\,C.}
\vzmsinfo{Иркутск; {\it valentina501@mail.ru}; {\it orlov{\_}sergey@inbox.ru}}
\vzmscaption

Пусть $E_{1}$ и $E_{2}$~--- вещественные банаховы пространства, $u:\,{\mathbb R}\to E_{1}$ и $f:\,{\mathbb R}\to E_{2}$~--- искомая и заданная функции. Рассмотрим класс линейных дифференциальных уравнений
$$
Bu'(t)=Au(t)+\alpha Au(t-h)+f(t),\,\,\,t>0, \eqno{(1)}
$$
где $B$ и $A$~--- замкнутые линейные операторы из $E_{1}$ в $E_{2}$ такие, что $\overline{D(B)}=\overline{D(A)}=E_{1}$ и $D(B)\subseteq D(A)$, параметр $\alpha\neq 0$. Предполагается, что оператор $B$ {\it фредгольмов}, т.~е. $\overline{R(B)} = R(B)$ и $\dim N(B) = \dim N(B^{*}) = n < +\infty$.
Зададим естественное для уравнений с отклоняющимся аргументом начальное условие вида
$$
u(t)=\omega(t),\,-h\leq t\leq0, \eqno{(2)}
$$
где $h>0$~--- заданное число, функция $\omega(t)\in C\left(\bigl[-h;0\bigr],E_{1}\right)$ известна и задаёт решение уравнения (1) на промежутке $[-h;0]$.
\\{\it Классическим} решением начальной задачи (1), (2) назовём функцию $u(t)\in C\left(\bigl[-h;+\infty\bigr),E_{1}\right)\cap C^{1}\left(\bigl(0;+\infty\bigr),E_{1}\right)$, удовлетворяющую уравнению (1) и начальному условию (2).

Осуществим продолжение классического решения нулём на интервал $(-\infty;-h)$ следующим образом:
$$
\tilde{u}(t)=\omega(t)\bigl(\theta(t+h)-\theta(t)\bigr)+u(t)\theta(t).
$$
Тогда в классе $K'_{+}(E_{1})$ распределений с ограниченным слева носителем начальная задача (1), (2) имеет вид уравнения
$$
\bigl(B\delta'(t)-A\delta(t)-\alpha A\delta(t-h)\bigr)\ast\tilde{u}(t)=\tilde{g}(t), \eqno{(3)}
$$
с правой частью $\tilde{g}(t)\in K'_{+}(E_{2})$ такой, что
$$
\tilde{g}(t)=f(t)\theta(t)+B\delta'(t)\ast\omega(t)\bigl(\theta(t+h)-\theta(t)\bigr)+
$$
$$
+B\omega(0)\delta(t)-A\omega(t)\bigl(\theta(t+h)-\theta(t)\bigr).
$$
Здесь и далее $\delta$~--- функция Дирака, $\theta$~--- функция Хевисайда. Нетрудно показать, что распределение
$$
\tilde{u}(t)={\cal E}(t)\ast\tilde{g}(t)
$$
является единственным решением уравнения (3) в классе $K'_{+}(E_{1})$ ({\it обобщённым} решением начальной задачи (1), (2)), где обобщённая оператор"=функция ${\cal E}(t)$ при произвольных $v(t)\in K'_{+}(E_{2})$ и $w(t)\in K'_{+}(E_{1})$ удовлетворяет равенствам
$$
\bigl(B\delta'(t)-A\delta(t)-\alpha A\delta(t-h)\bigr)\ast{\cal E}(t)\ast v(t)=v(t),
$$
$$
{\cal E}(t)\ast\bigl(B\delta'(t)-A\delta(t)-\alpha A\delta(t-h)\bigr)\ast w(t)=w(t),
$$
и называется {\it фундаментальной оператор"=функцией} [1] или фундаментальным решением абстрактного функционально"=дифференциального оператора $B\delta'(t)-A\delta(t)-\alpha A\delta(t-h)$.

Пусть $n$~--- размерность $N(B)$, $\left\{\varphi_{i}\right\}_{i=1}^{n}$~--- базис в $N(B)$, $\left\{\psi_{i}\right\}_{i=1}^{n}$~--- базис в $N(B^{\ast})$, а $\left\{\gamma_{i}\right\}_{i=1}^{n}\subset E_{1}^{\ast}$ и $\left\{z_{i}\right\}_{i=1}^{n}\subset E_{2}$~--- биортогональные им системы элементов, т.~е.
$$
\left\langle \varphi_{i},\,\gamma_{j}\right\rangle=\left\langle z_{i},\, \psi_{j}\right\rangle=\delta_{ij},\,i,\,j=1,\ldots,n.
$$
Введём ограниченный оператор $\Gamma:\,E_{2}\to D(B)$ вида
$$
\Gamma=\tilde{B}^{-1}=\biggl(B+\sum\limits_{i=1}^{n}\left\langle\cdot,\, \gamma_{i}\right\rangle z_{i}\biggr)^{-1},
$$
называемый {\it оператором Треногина\--Шмидта}, элементы $\varphi^{(j)}_{i}\in E_{1}$ и $\psi^{(j)}_{i}\in E_{2}^{\ast}$, где $i=1,\ldots,n$, $j=1,\ldots,p_{i}$, которые составляют $A$-{\it жорданов набор} оператора $B$ и $A^{*}$-{\it жорданов набор} оператора $B^{*}$ соответственно [2], проектор $\tilde{Q}$ вида
$$
\tilde{Q}=\sum\limits_{i=1}^{n}\sum\limits_{j=1}^{p_{i}}\langle\cdot,\, \psi_{i}^{(j)}\rangle A\varphi^{(p_{i}+1-j)}_{i}.
$$

\textbf{Теорема.} {\it Пусть линейный оператор $B$ фредгольмов и имеет полный $A$-жорданов набор, тогда фундаментальное решение абстрактного функционально"=дифференциального оператора $B\delta'(t)-A\delta(t)-\alpha A\delta(t-h)$ имеет вид
$$
{\cal E}(t)=\Gamma e^{A\Gamma t}({\mathbb I}_{2}-\tilde{Q})\theta(t)+
$$
$$
+\Gamma\sum\limits_{k=1}^{+\infty}\frac{(t-kh)^{k}}{k!}\alpha^{k}(A\Gamma)^k e^{A\Gamma (t-kh)}({\mathbb I}_{2}-\tilde{Q})\theta(t-kh)-
$$
$$
 -\sum\limits_{i=1}^{n}\sum\limits_{k=1}^{p_{i}}\sum\limits_{j=1}^{p_{i}-k+1}\langle\cdot,\,\psi^{(j)}_{i}\rangle\varphi^{(p_{i}-k-j+2)}_{i}\delta^{(k-1)}(t)\ast\mu^{k}(t),
$$
где ${\mathbb I}_{1}$ и ${\mathbb I}_{2}$~--- тождественные операторы в $E_{1}$ и $E_{2}$, степень обобщённой функции
$$
\mu(t)=\delta(t)+\sum\limits_{l=1}^{+\infty}(-\alpha)^{l}\delta(t-lh)
$$
понимается в смысле операции свёртки.}

Справедливо равенство $(\delta(t)+\alpha\delta(t-h))\ast\mu(t)=\delta(t)$, т.е. распределение $\mu(t)\in{\mathcal D}_{+}$ является обратным элементом к $\bigl(\delta(t)+\alpha\delta(t-h)\bigr)\in{\mathcal D}'_{+}$ в сверточной алгебре ${\mathcal D}'_{+}$.

\litlist

1. {\it Sidorov N. et al.} Lyapunov\--Schmidt Methods in Non\-linear Analysis and Applications. Dordrecht\--Boston\--London: Kluwer Academic Publishers, 2002. 568 p.

2. {\it Вайнберг М.М., Треногин В.А.} Теория ветвления решений нелинейных уравнений. М.: Наука, 1969. 528 с.
