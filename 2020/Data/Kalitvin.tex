

\begin{center}
    {\bf О ЧИСЛЕННОМ РЕШЕНИИ ДВУХ КЛАССОВ УРАВНЕНИЙ ВОЛЬТЕРРА С ЧАСТНЫМИ ИНТЕГРАЛАМИ С ПРИМЕНЕНИЕМ PYTHON\footnote{Работа выполнена при поддержке РФФИ (проект номер 19-41-480002).}}\\

    {\it В.А. Калитвин}

    (Липецк; {\it kalitvin@mail.ru})
\end{center}

\addcontentsline{toc}{section}{Калитвин В.А.}

К уравнениям Вольтерра с частными интегралами приводятся различные задачи механики сплошных сред [1-3].

В пространстве $C(D)$ непрерывных на $D\!=\![a,b]\!\times\![c,d]$ функций рассматривается линейное уравнение с частными интегралами вида
$$
x(t,s)\!=\!
\int\limits_a^t l(t,s,\tau)x(\tau,s)d\tau
\!+\!
\int\limits_c^s m(t,s,\sigma)x(t,\sigma)d\sigma\!+\!f(t,s), \eqno(1)
$$

\noindent где $(t,s)\in D,$  $l,$ $m,$  $f$ --- заданные непрерывные на $D\times T,$ $D\times S,$  $D$ соответственно
функции,
$T=\{\tau:a\le\tau\le t\le b\},$ $S=\{\sigma:c\le\sigma\le s\le d\}.$

Найти решение уравнения (1) в явном виде удаётся в редких случаях, поэтому актуальной задачей является разработка алгоритмов и программ для численного решения этого уравнения.

Отрезки $[a,b]$ и $[c,d]$ разобъем на части точками
$$
t_p=a+ph\  (p=0,1,\dots ,P,\  a+Ph\le b< (P+1)h),
$$
$$
s_q=c+qg \ (q=0,1,\dots ,Q,\  c+Qg\le d< (Q+1)g)
$$
соответственно. Полагая  $t=t_p,$  $s=s_q$ и применяя  формулы
$$
\int\limits_a^{t_p} l(t_p,s_q,\!\tau )x(\tau,\!s_q)d\tau\!=\!h\sum_{i=0}^p\alpha_{pi}l_{pqi}x(t_i,s_q)+r^l_{pq},$$
$$
\int\limits_c^{s_q} m(t_p,s_q,\sigma )x(t_p ,\sigma)d\sigma\! =\!g\sum_{j=0}^q\beta_{jq}m_{pqj}x(t_p,s_j)\!+r^m_{pq},
$$
\noindent  где $l_{pqi}=l(t_p,s_q,t_i),$ $m_{pqj}=m(t_p,s_q,s_j),$  а $r^l_{pq},$  $r^m_{pq}$   --- остатки квадратурных  формул, получим после отбрасывания остатков
систему уравнений для приближенных значений $x_{p0},$ $x_{0q},$ $x_{pq}$ функции $x$ в точках $(t_p,s_0),$ $(t_0,s_q),$ $(t_p,s_q)$  ($p=1,\dots ,P;$ $q=1,\dots ,Q).$ Пусть $\delta_{p0},$ $\delta_{0q},$  $\delta_{pq}$ --- погрешности в уравнениях с $x_{p0},$ $x_{0q},$  $x_{pq}.$
Тогда
$$x_{00}\!=\! f(a,c),$$
$$
x_{p0}\!=\!h\sum_{i=0}^p\alpha_{pi}l_{p0i}x_{i0}+f_{p0}+\delta_{p0},$$
$$
x_{0q}\!=\! g\sum_{j=0}^q\beta_{jq}m_{0qj}x_{0j}+f_{0q}+\delta_{0q},
$$
$$
x_{pq}\!=\!h\sum_{i=0}^p\alpha_{pi}l_{pqi}x_{iq}+g\sum_{j=0}^q\beta_{jq}m_{pqj}x_{p_j}+f_{pq}+\delta_{pq}
$$
$$(p=1,\dots ,P; q=1,\dots ,Q),$$

где
$$
f_{p0}=f(t_p,s_0), f_{0q}=f(t_0,s_q), f_{pq}=f(t_p,s_q).
$$


\textbf{Теорема~1.} {\it Если   $r^l_{pq}$ и $r^m_{pq}$
стремятся к нулю равномерно относительно $p,q$ при $h,g\to 0;$ существуют такие числа $A,B$ что  $|\alpha_{pi}|\le A<\infty,\ |\beta_{jq}|\le B<\infty;$ погрешности $\delta_{p0},$ $\delta_{0q},$ $\delta_{pq}$ стремятся к нулю равномерно относительно $p,q$ при $h,g\to 0,$ то  при всех достаточно малых $h$ и $g$ приближенное решение $x_{pq}$ может быть найдено из последней системы, причём для любого заданного $\epsilon>0$ найдутся такие $h_0$ и $g_0,$ что при $h<h_0$ и $g<g_0$ будут выполняться неравенства $
|x_{pq}-x(t_p,s_q)|<\epsilon\ (p=0,1,\dots,P; q=0,1,\dots,Q),
$ а последовательность функций

$$
x_{pq}(t,s)=h\sum\limits_{i=0}^p\alpha_{pi}l(t,s,t_i)x_{iq}+g\sum\limits_{j=0}^q\beta_{jq}m(t,s,s_j)x_{pj}+f(t,s)
$$
\noindent равномерно  сходится на $D$ к решению $x(t,s)$  при $p,q\to\infty.$}

С использованием данного алгоритма разработана программа на языке Python и проведены численные эксперименты, показывающие достаточно хорошие результаты.

В пространстве $C(D)$ непрерывных на $D\!=\![a,b]\!\times\![c,d]$ функций рассматривается нелинейное уравнение с частными интегралами вида
$$
x(t,s)=
\int\limits_a^t l(t,s,\tau, x(\tau,s))d\tau
+
\int\limits_c^s m(t,s,\sigma, x(t,\sigma))d\sigma+ f(t,s), \eqno(2)
$$

Здесь $(t,s)\in D=[a,b]\times [c,d],$ а $l,m,f$ -- заданные непрерывные функции.

При численном решении уравнения (2) отрезок $[a,b]$ разобъем на равные части точками $t_i$ и $s_j$ $(i,j=0,1,\dots, n).$

При $i=j=0$ $x(t_i,s_j=0).$ Заменяя интегралы в уравнении (2) конечными суммами с помощью какой-либо квадратурной формулы, при фиксированных $j=0$ и $i=0$ получаем значения $x(t_i,s_0)$ $(i=1,2,\dots ,n)$ и $x(t_0,s_j)$ $(j=1,\dots ,n).$ Используя найденные значения, вычисляем значения $x(t_i,s_j)$ при $i=1,2,\dots ,n$ и $j=1,2,\dots , n.$ Например, при использовании квадратурной формулы средних прямоугольников,
$$
x(t_i,s_0)=f(t_i,s_0)+h\sum\limits_{k=0}^{i-1}(l(t_i,s_0,t_k+\frac{h}{2},x(t_k,s_0))),
$$
$$
x(t_0,s_j)=f(t_0,s_j)+h\sum\limits_{k=0}^{j-1}(m(t_0,s_j,s_k+\frac{h}{2},x(t_0,s_k))),
$$
$$
x(t_i,s_j)=f(t_i,s_j)+h\sum\limits_{k=0}^{i-1}(l(t_i,s_j,t_k+\frac{h}{2},x(t_k,s_j)))+
$$
$$
+
h\sum\limits_{k=0}^{j-1}(m(t_i,s_j,s_k+\frac{h}{2},x(t_i,s_k))),
$$
$$
i=1,2,\dots, n; j=1,2,\dots ,n
$$

C использованием данного алгоритма разработана программа на языке программирования Python и проведены численные эксперименты, демонстрирующие достаточно хорошие результаты.


% Оформление списка литературы
\smallskip \centerline {\bf Литература} \nopagebreak

1. {\it Appell J.M., Kalitvin A.S., Zabrejko P.P.} Partial Integral Operators and Integro-Differential Equations. New York: Marcel Dekker, 2000, 560 p.

2. {\it Калитвин А.С.} Линейные операторы с частными интегралами. Воронеж: ЦЧКИ, 2000, 252 с.

3. {\it Калитвин А.С., Калитвин В.А.} Интегральные уравнения Вольтерра и Вольтерра-Фредгольма с частными интегралами. Липецк: ЛГПУ, 2006, 177 с.

