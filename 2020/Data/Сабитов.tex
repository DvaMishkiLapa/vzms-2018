 \vzmstitle{ЗАДАЧА ДИРИХЛЕ ДЛЯ ГИПЕРБОЛИЧЕСКИХ УРАВНЕНИЙ ВЫСОКОГО ПОРЯДКА}
\vzmsauthor{Сабитов}{К.\,Б.}
\vzmsinfo{Стерлитамак; {\it sabitov\_fmf@mail.ru}}
\vzmscaption


Исследование вопросов неустойчивых колебаний (резонансов колебаний в жидкости в тонкостенных баках ракет с собственными колебаниями) тесно связано с задачей Дирихле для волнового уравнения. Эта задача изучалась многими математиками (см. [1 -- 3]).

Если задача Дирихле для одномерного волнового уравнения в прямоугольной области изучена достаточно полно [4], [5, с. 112--118], то эта задача для многомерного волнового уравнения практически не исследована. Только работы [6, 7] посвящены задаче Дирихле для трёхмерного волнового уравнения с ненулевой правой частью и однородными граничными условиями в области $\Omega$, когда $\Omega$ -- эллипсоид, цилиндр с образующими, параллельными оси $t$, и параллелепипед, где установлены критерий единственности и существование решения задачи в пространстве Соболева $W_2^1(\Omega)$ при определённых условиях на правую часть, связанных сходимостью числовых рядов. При этом возникающие малые знаменатели не изучены.

В данной работе в классе регулярных решений гиперболических уравнений высокого порядка установлен критерий единственности решения задачи Дирихле и само решение построено в явном виде как сумма ряда Фурье. При обосновании сходимости ряда возникает проблема малых знаменателей от многих переменных более сложной структуры, чем в ранее известных работах [4, 8, 9]. В связи с чем установлены оценки об отделённости от нуля малых знаменателей, на основании которых доказана сходимость ряда в классе регулярных решений при некоторых условиях относительно граничных функций.

\smallskip \centerline {\bf 1.Задача Дирихле для двумерного уравнения}
\centerline {\bf гиперболического типа высокого порядка}
\nopagebreak


Рассмотрим уравнение в частных производных
$$
Lu=\frac{\partial^{2p}u}{\partial
t^{2p}}-\frac{\partial^{2p}u}{\partial x^{2p}}=f(x,t) \eqno{(1.1)}
$$
в прямоугольной области $D=\{(x,t) |\; 0<x<l,\; 0<t<T\}$, где
$l$, $T$ -- заданные положительные числа, $p\in N$, и
поставим
следующую первую граничную задачу.

\textbf{Задача Дирихле.} \emph{Найти функцию $u(x,t)$, удовлетворяющую условиям:}
$$
u\in C^{2p-1}(\overline{D})\cap C^{2p}(D), \eqno{(1.2)}
$$
$$
Lu(x,t)\equiv f(x,t), \;\;\; (x,t)\in D, \eqno{(1.3)}
$$
$$
\frac{\partial^{2k}u}{\partial x^{2k}}\Big |_{x=0}
=\frac{\partial^{2k}u}{\partial x^{2k}}\Big |_{x=l}=0, \;\;\;0\leq
t \leq T, \eqno{(1.4)}
$$
$$
\frac{\partial^{2k}u}{\partial t^{2k}}\Big |_{t=0}
=0, \;\;\;0\leq x \leq l, \eqno{(1.5)}
$$
$$
\frac{\partial^{2k}u}{\partial t^{2k}}\Big |_{t=T}
=0, \;\;\;0\leq x \leq l, \quad k=\overline{0, p-1}. \eqno{(1.6)}
$$



\textbf{Теорема 1.1.} \emph{Если существует решение
задачи $(1.2)$ -- $(1.6)$, то оно единственно только тогда,
когда отношение сторон $\alpha=T/l$ прямоугольника $D$
является иррациональным числом.}


Решение задачи (1.2) -- (1.6) строится в виде суммы двойного ряда
$$
u(x,t)=\frac{2}{\sqrt{l T}}\sum\limits_{m,
n=1}^{+\infty}u_{mn}\sin\frac{\pi m}{l}x \sin\frac{\pi n}{T}t. \eqno{(1.7)}
$$


\textbf{Теорема 1.2.} \emph{Если число $\alpha>0$ является иррациональным
алгебраическим числом степени $n\geq 2$ и
функция $f(x,t)\in
C^{2p+5}(\overline{D})$,
$$f_{x}^{(i)}(0,t)=f_{x}^{(i)}(l,t)=\left\{\begin{array}{l}
0,\;\;\;i=0,2,...,p+1,\,\ p+3 -
\textrm{чётное};
\\
0,\;\;\;i=0,2,...,p+2,\,\ p+3 -
\textrm{нечётное},
\end{array}\right.$$
$$f_{t}^{(j)}(x,0)=f_{t}^{(j)}(x,T)=\left\{\begin{array}{l}
0,\;\;\;j=0,2,...,p+1,\,\ p+2 -
\textrm{нечётное};
\\
0,\;\;\;j=0,2,...,p,\,\ p+2 -
\textrm{чётное},
\end{array}\right.$$
то существует
единственное решение задачи $(1.2)$ -- $(1.6)$, которое определяется рядом $(1.7)$.}



\smallskip \centerline {\bf 2.Задача Дирихле для трёхмерного уравнения }
\centerline {\bf гиперболического типа высокого порядка и}
\centerline {\bf связь с проблемой Ферма}
\nopagebreak

Далее для трёхмерного аналога уравнения (1.1), т.е.
для уравнения вида
$$
Lu=\frac{\partial^{2p}u}{\partial
t^{2p}}-\frac{\partial^{2p}u}{\partial
x^{2p}}-\frac{\partial^{2p}u}{\partial y^{2p}}=f(x,y,t), \eqno{(2.1)}
$$
в прямоугольном параллелепипеде $Q=\{(x,y,t)|\;(x,y)\in D, t\in
(0,T)\}$, где $D=\{(x,y) |\; 0<x<l,\; 0<y<q\}$, $l$, $q$, $T$ --
заданные положительные числа, $f(x,y,t)$ -- заданная в $Q$
функция, $p\in N$, исследуется следующая

\textbf{Задача Дирихле.} \emph{Найти в области $Q$
функцию $u(x,y,t)$, удовлетворяющую следующим условиям:}
$$
u\in C^{2p-1}(\overline{Q})\cap C^{2p}(Q), \eqno{(2.2)}
$$
$$
Lu(x,y,t)\equiv f(x,y,t), \;\;\; (x,y,t)\in Q, \eqno{(2.3)}
$$
$$
\frac{\partial^{2k}u}{\partial x^{2k}}\Big |_{x=0}
=\frac{\partial^{2k}u}{\partial x^{2k}}\Big
|_{x=l}=\frac{\partial^{2k}u}{\partial y^{2k}}\Big |_{y=0}
=\frac{\partial^{2k}u}{\partial y^{2k}}\Big |_{y=q}=0, \;\;\;0\leq
t \leq T, \eqno{(2.4)}
$$
$$
\frac{\partial^{2k}u}{\partial t^{2k}}\Big |_{t=0}
=\frac{\partial^{2k}u}{\partial t^{2k}}\Big |_{t=T}=0,
\;\;\;(x,y)\in \overline{D}, \quad k=\overline{0, p-1}. \eqno{(2.5)}
$$


Здесь установлен следующий результат.

\textbf{Теорема 2.1.} \emph{Если существует решение задачи $(2.1)$ -- $(2.5)$, то оно единственно только тогда, когда уравнение
$$
\Big (\frac{m}{l}\Big )^{2p}+\Big (\frac{n}{q}\Big )^{2p}=\Big
(\frac{k}{T}\Big )^{2p},\;\;\;m,n,k\in N, \eqno{(2.6)}
$$
не разрешимо во множестве натуральных чисел.}

Если $l=q=T$, т.е. $Q$ является кубом, то (2.6)
переходит в известное уравнение Ферма
$$
m^{2p}+n^{2p}=k^{2p}. \eqno{(2.7)}
$$

Если $q=l\neq T$, $\frac{l}{T}\in Q$ или
$q=T\neq l$, $\frac{T}{l}\in Q$, то уравнение (2.6) также
сводится к уравнению типа (2.7). Следовательно, в этих случаях в
силу полученного результата единственность решения задачи Дирихле
для уравнения (2.1) при любом натуральном $p$ равносильна великой
проблеме Ферма.

Решение задачи (2.1) -- (2.5) строится в виде суммы тройного ряда
$$
u(x,y,t)=\frac{2\sqrt{2}}{\sqrt{l q T}}\sum\limits_{m, n, k=1}^{+\infty}u_{mnk}\sin \frac{\pi m}{l}x \sin\frac{\pi n}{q}y\sin \frac{\pi k}{T}t,
$$
$$
u_{mnk}=\frac{1}{\pi^2}\frac{(-1)^{p}f_{mnk}}{(k/T)^{2p}-(m/l)^{2p}-(n/q)^{2p}},
$$
$$
f_{m n k}=\int\limits_{Q}f(x,y,t)\sin \frac{\pi m}{l}x \sin\frac{\pi n}{q}y\sin \frac{\pi k}{T}t\,dxdydt.
$$



% Оформление списка литературы
\litlist

1. {\it Соболев\,С.\,Л.} Пример корректной краевой задачи для уравнения колебания струны с данными на всей границе // ДАН СССР. 1956. Т.\,109. \No\,4. С.\,707--709.

2. {\it Арнольд\,В.\,И.} Математическое понимание природы. М.: Изд-во МЦНМО, 2010. 144 с.

3. {\it Арнольд\,В.\,И.} Малые знаменатели. I // Известия АН СССР. Серия математическая. 1961. \No\,25. С.\,21--86

4. {\it Сабитов\,К.\,Б.} Задача Дирихле для уравнений с частными производными // Матем. заметки. 2015. Т.\,97. №\,2. С.\,262--276.

5. {\it Сабитов\,К.\,Б.} Уравнения математической физики. М.: Физматлит, 2013. 352 с.

6. {\it Денчев\,Р.} О спектре одного оператора // Докл. АН СССР. 1959. Т.\,126. \No\,2. С.\,259--262.

7. {\it Денчев\,Р.} О задаче Дирихле для волнового уравнения // Докл. АН СССР. 1959. Т.\,127. \No\,3. С.\,501--504.

8. {\it Сабитова\,Ю.\,К.} Задача Дирихле для телеграфного уравнения в прямоугольной области // Изв. вузов. Матем. 2017. \No\,12. С.\,46--56.

9. {\it Сабитова\,Ю.\,К.} Задача Дирихле для уравнения гиперболического типа со степенным вырождением в прямоугольной области // Дифференциальные уравнения. 2018. Т.\,54. \No\,2. С. 228--237.
