\vzmstitle{ПОСТРОЕНИЕ ОПЕРАТОРА ПОЛЗУЧЕСТИ В НЕЛИНЕЙНОЙ НАСЛЕДСТВЕННОЙ ТЕОРИИ УПРУГОСТИ}
\vzmsauthor{Бырдин}{А.\,П.}
\vzmsauthor{Прач}{В.\,С.}
\vzmsauthor{Сидоренко}{А.\,А.}
\vzmsauthor{Соколова}{О.\,А.}
\vzmsinfo{Россия, Воронеж, ВГТУ; Украина, Донецк, ДНУ;}
\vzmscaption



При решении задач о нагружении материалов, проявляющих релаксационные свойства и нелинейную зависимость между напряжениями и деформациями, возникает проблема [1] построения решений  нелинейных интегральных и интегро"=дифференциальных уравнений. В одномерном случае в работе [2] построен рекуррентный алгоритм, позволяющий определять ядра полилинейных операторов, представляющих решения. В настоящей работе обобщаются полученные ранее результаты на случай системы нелинейных интегральных уравнений.

Уравнения закона Гука для нелинейной наследственно"=упругой изотропной среды имеют вид

\begin{equation} \label{GrindEQ__1_} \sigma _{i} =\hat{\lambda }\theta +2\hat{\mu }\varepsilon _{i} ,\quad \tau _{ij} =2\hat{\mu }\varepsilon _{ij} ,\quad \quad \theta =\varepsilon _{1} +\varepsilon _{2} +\varepsilon _{3} ,\quad (i,j=1,2,3), \end{equation}
где $\sigma _{i} ,\, \tau _{ij} ,\, \varepsilon _{i} ,\, \varepsilon _{ij} $ - компоненты тензоров напряжений и деформаций, $\hat{\lambda },\; \hat{\mu }$ - операторы Ламе, выбранные в виде

\begin{equation} \label{GrindEQ__2_} \hat{\lambda }=\lambda _{0} (\hat{\lambda }_{1} +\hat{\lambda }_{3} ),\quad \; \hat{\mu }=\mu _{0} \hat{\mu }_{1} , \end{equation}
где $\lambda _{0} ,\; \mu _{0} $- нерелаксированные значения упругой постоянной Ламе и модуля упругости при сдвиге, а операторы в \eqref{GrindEQ__2_} действуют по правилу

\begin{equation} \label{GrindEQ__3_} \begin{array}{l} {\hat{\lambda }_{1} u(t)=\int _{0}^{t}\lambda _{1}  (\tau )u(t-\tau )d\tau ,\quad \quad \hat{\mu }_{1} u(t)=\int _{0}^{t}\mu _{1}  (\tau )u(t-\tau )d\tau ,} \\ {\hat{\lambda }_{3} u(t)=\mathop{\int \int \int    }\limits_{0}^{t} \lambda _{3} (\tau _{1} ,\tau _{2} ,\tau _{3} )\prod _{k=1}^{3}u(t-\tau _{k} )d\tau _{k}  ,} \end{array} \end{equation}
$\lambda _{1} (\tau ),\; \; \mu _{1} (\tau ),\; \lambda _{3} (\tau _{1} ,\tau _{2} ,\tau _{3} )$ - зависящие от времени ядра наследственности, удовлетворяющие условиям симмтрии, принципу затухающей памятии содержащие аддитивно дельта функции [1]

\begin{equation} \label{GrindEQ__4_} \begin{array}{l} {\lambda _{1} (t)=\delta (t)-k_{1} \Lambda _{1} (t),\quad \; \; \mu _{1} (t)=\delta (t)-k_{2} M_{1} (t),} \\ {k_{1} =\frac{\lambda _{0} -\lambda _{r} }{\lambda _{0} } ,\quad k_{2} =\frac{\mu _{0} -\mu _{r} }{\mu _{0} } ,\quad } \end{array} \end{equation}
$\lambda _{r} ,\; \; \mu _{r} $ - релаксированные значения упругих модулей.

 Рассматривая реологические соотношения \eqref{GrindEQ__1_} как уравнения относительно деформаций, получим решение в виде

\begin{equation} \label{GrindEQ__5_} \begin{array}{l} {\varepsilon _{i} =0.5(\hat{\mu }_{1} ^{-1} \sigma _{i} -\hat{\mu }_{1} ^{-1} \hat{\lambda }\hat{K}^{-1} 3\sigma ),\; \; \; (i=1,2,3)} \\ {\hat{K}=3K_{0} \left(\hat{I}-\hat{G}_{1} +\frac{\lambda _{0} }{K_{0} } \hat{\lambda }_{3} \right),\quad \quad \hat{G}_{1} =\frac{\lambda _{0} k_{1} }{K_{0} } \hat{\Lambda }_{1} +\frac{2\mu _{0} k_{2} }{3K_{0} } \hat{M}_{1} ,} \end{array} \end{equation}
где $K_{0} $ - нерелаксированный модуль объёмного сжатия, $\hat{I}$ - единичный оператор, $\hat{\Lambda }_{1} ,\; \hat{M}_{1} $ - линейные интегральные операторы вида \eqref{GrindEQ__3_} с ядрами $\Lambda _{1} (t)\; \; 8\; \; \; M_{1} (t)$.  Для построения оператора $\hat{K}^{-1} $ используем обобщение методики, развитой в работе [2] для одномерного случая. Представим $\hat{K}^{-1} $ операторным рядом Вольтерра-Фреше

\[\hat{K}^{-1} =\sum _{n=0}^{\infty }\hat{L}_{2n+1}  \]
и для ядер интегральных полилинейных операторов $\hat{L}_{n} $ получим реккурентные соотношения, связывающие трансформанты Лапласа искомых ядер и ядер операторов $\hat{\mu }$ и $\hat{\lambda }$.

 Предполагая сепарабельность ядра оператора $\hat{\lambda }_{3} $

\[\lambda _{3} (t_{1} ,t_{2} ,t_{3} )=a_{3} \prod _{n=1}^{3}G_{1}  (t_{n} ),\]

Разрешая систему рекуррентных уравнений относительно трансформант ядер, получим

\begin{equation} \label{GrindEQ__6_} L^{*} _{2n+1} (p_{1} ,...,p_{2n+1} )=\varphi _{n} L_{1}^{*} (p_{1} +...+p_{2n+1} ),\quad L_{1}^{*} (p)=\left(G_{1}^{*} (p)\right)^{-1} , \end{equation}

\[\varphi _{n} =\frac{3(a_{0}^{(0)} )^{n} }{(2n+1)!} \cdot \frac{d^{2n+1} }{dx^{2n+1} } \cdot \exp \left(\frac{1}{3} \arcsin x\right)\left|\begin{array}{l} {} \\ {x=0} \end{array}\right. ,\quad a_{3}^{(0)} =\frac{27\lambda _{0} a_{3} }{8\pi K_{0} } .\]

Учитывая  \eqref{GrindEQ__6_},  получим

\[(\hat{K})^{-1} (\sigma _{1} +\sigma _{2} +\sigma _{3} )=\sum _{n=0}^{\infty }\varphi _{n}  \int _{0}^{t}L_{1}  (t-\tau )(\sigma _{1} (\tau )+\sigma _{2} (\tau )+\sigma _{3} (\tau ))d\tau .\]

Таким образои, операторв в выражениях для деформаций \eqref{GrindEQ__5_} полностью определены. В качестве примера построены выражения для деформаций при выборе ядер наследственности $\Lambda _{1} (t)$ и $M_{1} (t)$ соответствующих модели стандартного линейного тела  [1] и получены в явном виде выражения для релаксационных параметров в ядрах интегральных операторов ползучести.

\smallskip \centerline {\bf Литература} \nopagebreak

1.
{\it Работнов Ю.Н.} Элементы наследственной механики твёрдых тел. -- М.: Наука, 1977. -- 384 с.

2.
{\it Бырдин А.П., Розовский М.Н.} Метод рядов Вольтерра в динамических задачах  нелинейной наследственной упругости.  -- Изв. АН Арм. ССР, 1985, №5. -- с. 49-56.









