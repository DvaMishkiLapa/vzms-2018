\vzmstitle[\footnote{Исследование выполнено в рамках Программы Президента Российской Федерации для государственной поддержки ведущих научных школ РФ (грант НШ-2554.2020.1).}]{КЛАССИФИКАЦИЯ ТОПОЛОГИЧЕСКИХ БИЛЛИАРДОВ С ВЫПУКЛЫМИ СКЛЕЙКАМИ НА ПЛОСКОСТИ МИНКОВСКОГО}
\vzmsauthor{Каргинова}{Е.\,Е.}
\vzmsinfo{Москва; {\it karginov13@gmail.com}}
\vzmscaption

\textbf{Определение~1.} {\it Плоскостью Минковского называется $\mathbf R^2$ со скалярным произведением $$\langle x,y\rangle=x_1 y_1-x_2 y_2$$.}


Рассмотрим на плоскости Минковского эллипс $E$, задаваемый соотношением:
$$E \colon \frac{x^2}{a}+\frac{y^2}{b}=1$$
Здесь $a>b>0$ и $\lambda \in \mathbf{R}$ - вещественные числа. Софокусное семейство квадрик $C_\lambda$ задаётся уравнением
$$C_\lambda \colon \frac{x^2}{a-\lambda}+\frac{y^2}{b+\lambda}=1\eqno (1)$$


Биллиард в эллипсе на плоскости Минковского был исследован в работе (2) В. Драгович и М. Раднович, а именно описан закон отражения в таком биллиарде, первые интегралы системы и движение в системе.


\textbf{Определение~2.} {\it Простым биллирадом назовём двумерное, связное, плоское, компактное многообразие с кусочно"=гладким краем, состоящим из сегментов квадрик семейства (1), попарно пересекающихся под углами, не превышающими $\frac{\pi}{2}$.}

Определим отражение в простом биллиарде: при отражении сохраняется евклидова длина вектора скорости и угол падения в смысле Минковского равен углу отражения.

\textbf{Определение~3.} {\it Топологическим биллиардом называем двумерное ориентируемое многообразие с кусочно"=гладкой метрикой Минковского, полученное отождествлением (склейкой) простых биллиардов вдоль некоторых выпуклых или прямолинейных сегментов (ребра склейки).}


Вышеописанная конструкция была предложена В. В. Ведюшкиной в работе [3].


Отражение в топологическом биллиарде устроено следующим образом: при попадании на ребро склейки материальная точка продолжает движение по другому листу, а при попадании в угол или ребро, не являющееся ребром склейки, точка продолжает движение так же, как и в случае плоской области.

Такое отражение сохраняет два первых интеграла: квадрат евклидовой длины вектора скорости $v_E$ и каустику траектории $\lambda$. Эти функции находятся в инволюции и функционально независимы, следовательно, топологический биллиард интегрируем по Лиувиллю.

Рассмотрим ограничение фазового пространства системы топологического биллиарда на поверхность уровня интеграла $v_E$ - это трёхмерное многообразие $Q^3$, называемое изоэнергетической поверхностью. Изменяя значения интеграла $\lambda$, получим слоение $Q^3$ на двумерные поверхности - неособые слои (двумерные торы) и особые слои, описываемые 3-атомами.

В своей работе я рассматриваю топологические биллиарды, которые могут быть получены склейкой простых биллиардов на плоскости Минковского вдоль выпуклых сегментов границы. Я получила полную классификацию таких биллиардов, и для каждого класса посчитала меченую молекулу Фоменко-Цишанга - граф с целочисленными метками, являющийся инвариантом Лиувиллевой эквивалентности интегрируемых гамильтоновых систем (подробнее о теории Фоменко-Цишанга см. [1]).





\litlist
1. {\it Болсинов А. В., Фоменко А. Т.} Интегрируемые гамильтоновы системы. Геометрия, топология, классификация. Ижевск НИЦ <<Регулярная и хаотическая динамика>>, 1999. - Т.1


2. {\it Драгович, В., Раднович, М.} Топологические инварианты эллиптических биллиардов и геодезических потоков на эллипсоиде. Фундаментальная и прикладная математика. - 2015. - Т. 20(2), - С. 51-64.

3. {\it Ведюшкина В.В.} Топологическая классификация биллиардов в локально плоских областях,
ограниченных дугами софокусных квадрик, Матем. сб., 206:10 (2015), 127-176
