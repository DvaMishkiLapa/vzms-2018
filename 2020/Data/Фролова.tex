\vzmstitle[\footnote{Работа поддержана РФФИ (проект №19-41-480002).}]{ УРАВНЕНИЯ ВОЛЬТЕРРА-ФРЕДГОЛЬМА СМЕШАННЫХ ЗАДАЧ ЭВОЛЮЦИОННОГО ТИПА}
\vzmsauthor{Фролова}{Е.\,В.}
\vzmsinfo{Липецк; {\it lsnn48@mail.ru}}
\vzmscaption


Контактные задачи теории упругости при учёте износа шероховатых
поверхностей взаимодействующих тел, а также ряд смешанных задач для
многослойных вязкоупругих оснований, когда относительная толщина и
относительная жёсткость верхнего слоя достаточны малы, приводятся к
исследованию интегрального уравнения Вольтерра\--Фредгольма с частными
интегралами
 $$
 \lambda x(t,s) + (K x)(t,s) + (N x)(t,s) = g(t,s).
\eqno(1)
$$
В статье рассматривается уравнение (1) для случая
$$
(Kx)(t,s)\equiv(Lx)(t,s)+(Mx)(t,s)=
$$
$$
=\int_0^tl(t,\tau )x(\tau ,s)d\tau +
\int_{-1}^1 m(s,\sigma )x(t,\sigma )d\sigma,
$$
$$
(Nx)(t,s)=\int_0^t\int_{-1}^1n(t,\tau )m(s-\sigma)x(\tau
,\sigma) d\sigma d\tau ,$$
где $\lambda >0$ -- безразмерный параметр, имеющий механический смысл,
функция $g(t,s)$ имеет вид $g(t,s)=g_1(t)+sg_2(t)+f(s)$,
$g_1,g_2\in C([0,a])$, а $f$ -- заданная функция из $L^2([-1;1])$, причём
 ядро $l(t,\tau)$ -- непрерывная функция, ядро $m(s,\sigma)$ таково, что оператор
$M$ действует из $L^2([-1,1])$ в $C([-1,1])$, а в $L^2([-1,1])$  является самосопряжённым компактным
оператором, $n(t,\tau)=l(t,\tau)$ -- непрерывная функция.

При сделанных предположениях оператор $L \bar\otimes I$ имеет равный нулю спектральный радиус,
т.е. $r(L \bar\otimes I) = r(L) = 0$ и $\sigma(L \bar\otimes I) = \{ 0 \}$. Кроме того,  $\sigma(M)$ состоит из нуля и
не более чем счётного множества собственных чисел, т.е.  $M$ -- оператор с чисто
точечным спектром, причём все собственные функции оператора $M$
непрерывны.

Рассмотрим некоторые свойства
 точечного спектра оператора $K + N$, структуру его собственных
 функций, а также условия разрешимости уравнения (1)
 в случае, когда $\lambda \ne 0$ совпадает с одним из собственных
 чисел оператора  $K + N$.

 Оператор $K + N$ можно записать в виде
$K + N = L \bar\otimes I + I \bar\otimes M
+ L \bar\otimes M $. Известно, что $\sigma(K + N) = \sigma(M)$. Так как $\sigma(L) = \{ 0 \}$, то если
$\sigma_p(L)$ не пуст, имеем: $\sigma_p(K)=\sigma_p(M)$ и
$\sigma_p(N)=\sigma_p(L)= \{ 0 \}$.

Так как  $\sigma( K + N) =
\sigma(M)$, то спектр оператора $ K +  N$ состоит из нуля
 и не более чем счётного числа собственных чисел. Если теперь $0 \in
 \sigma_p(M)$, то $0 \in \sigma_p(K + N)$, если  $0 \notin
 \sigma_p(M)$, то $0 \notin \sigma_p(K + N)$. Таким образом,
 $\sigma_p(K + N)=\sigma_p(M)$. Поэтому справедлива

\textbf{Теорема~1.} {\it
Пусть $l(t,\tau)$ непрерывная функция, $n(t,\tau) =
l(t,\tau)$ и оператор $M$ действует из $L^2([-1,1])$
 в $C([-1,1])$ и в  $L^2([-1,1])$ является компактным самосопряжённым
оператором и пусть $\sigma_p(L)$ в $C([0,a])$
не пуст. Тогда не пуст точечный спектр оператора $K + N$ в $C(D)$
и $\sigma_p(K + N)=\sigma_p(M)$.}

Пусть, далее, $\sigma_p(L)$ не пуст. Тогда для любого
$\lambda_i \ne 0$ и $\lambda_i \in
\sigma_p(K +N) \ (i=1,2,\ldots)$ $\lambda_i = \beta_i$, где $\beta_i \in
\sigma_p(M) \ (i=1,2,\ldots)$ и верна

\textbf{Теорема~2.} {\it
Пусть выполнены условия теоремы~1. Собственные функции оператора $K +N$,
соответствующие собственному числу $\lambda_p = \beta_p \ne 0$,
$\beta_p \neq -1$,
 находятся в множестве $X(\lambda_p) \cap C(D)$, где
$X(\lambda_p)$ -- подпространство, образованное
линейными комбинациями функций $ \varphi (t) \psi(s) $, где
$\varphi \in Ker  L$, $ \psi \in Ker(\lambda_p I - M) $,
 а собственные функции оператора $K +N$,
соответствующие собственному числу $\lambda_p = \beta_p = -1$
 находятся в множестве $X(\lambda_p)$, где
$X(\lambda_p)$ -- подпространство, порождаемое линейными
комбинациями функций $ \varphi (t) \psi(s)$, где  $\varphi \in C([0,a])\backslash \{ 0 \}$,
$  \psi \in Ker(\lambda_p I - M) $.}

Рассмотрим условия разрешимости уравнения (1) в
случае, когда $\lambda \ne 0$ попадает в точечный
спектр оператора $K + N$. Используя представление
$ x^0(t,s) = \sum_{i,j=1}^\infty x_{ij} \varphi_i (t) \psi_j (s) $
элементов пространства  $L^2(D)$,
где $x_{ij}$ -- коэффициенты Фурье, а $x^0$ -- собственная функция оператора $K$,
соответствующая собственному числу $\beta_p$, получим
$$
(\lambda - \beta_j)
 \sum_{i=1}^\infty  x_{ij} \varphi_i
(t) - (1+ \beta_j)\! \! \int_0^t l(t,\tau) \! \!  \sum_{i=1}^\infty
x_{ij} \varphi_i (\tau) \, d\tau =  \sum_{i=1}^\infty  f_{ij} \varphi_i (t).
$$
Используя равенства
$f(t,s)=\sum_{i, j=1}^\infty f_{ij}\varphi_i(t)\psi_j(s),$
где $f_{ij}$ --- коэффициенты Фурье функции $f(t,s),$
$f_j(t)=\sum_{i=1}^\infty f_{ij}\varphi_i(t)$
и вводя обозначения
$
 y_j(t) = \sum_{i=1}^\infty  x_{ij} \varphi_i (t),
$
получим
$$
(\lambda - \beta_j)
y_j(t) - (1+ \beta_j )  \int_0^t l(t,\tau) y_j (\tau) \, d\tau =
  f_j (t)  \  (j=1,2, \ldots ). \eqno(2)
$$

Пусть  $\lambda \ne 0$ совпадает с собственным числом
$\lambda_p$ кратности $n$.
Тогда все уравнения, для которых
$j \notin \{p, p+1, \ldots, p+n \}$ однозначно разрешимы для любых
$f_j(t) \in C([0,a])$. Следовательно, система
(2) и уравнение (1)  имеют решения точно в случае, когда разрешимы уравнения
$$
 (1+ \beta_j )  \int_0^t l(t,\tau) y (\tau) \, d\tau =
  f_j (t), \quad  (j =  p, p+1, \ldots, p+n ).  \eqno(3)$$
Таким образом, доказана

\textbf{Теорема~3.} {\it
Пусть $l$, $m$, $n$ удовлетворяют условиям теоремы {\rm 1} и
$\lambda \ne 0$ совпадает с собственным числом $\lambda_p=\beta_p$
 кратности $n$.
 Тогда уравнение {\rm (1)} разрешимо тогда,
 когда разрешимы уравнения {\rm (3)}.}



