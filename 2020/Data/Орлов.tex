\vzmstitle[\footnote{Работа выполнена при финансовой Российского Научного Фонда (проект №19-11-00146).}]{ОБ ОДНОЙ ДРОБНОЙ НЕЛИНЕЙНО-ВЯЗКОУПРУГОЙ МОДЕЛИ}
\vzmsauthor{Орлов}{В.\,П.}
\vzmsinfo{Воронеж; {\it orlov{\_}vp@mail.ru}}
\vzmscaption

Устанавливается существование и единственность сильного решения начально"=краевой задачи для системы уравнений движения нелинейно"=вязкоупругой жидкости, являющейся дробным аналогом модели вязкоупругости Фойгта, в плоском случае.
%\section*{Введение}

В $Q=[0,T] \times \Omega$, где $ \Omega\in R^2$ "--- ограниченная область с гладкой границей $\partial \Omega$, рассматривается начально"=краевая задача $Z$:
$$
\partial v/\partial t + {\sum_{i=1}^n } v_i \partial v /\partial x_i -
$$
$$
 \mu_0 {\rm Div}\,\mathcal{E}(v) -{\rm Div}\,\mu_1(S(v))\,\mathcal{E}(v)-
$$
$$
-\mu_2\mathrm{Div}\,\int_{0}^t(t-s)^{-\alpha}\,\mathcal{E}(v)(s, x)\,ds =
$$
$$
f(t,x)+{\rm grad} \, p,\ (t,x) \in Q;
$$
$$
{\rm div}\,v \!=\!0,\, (t,x) \in Q; \quad \int_{\Omega} p(t,x)\, dx=0;\, t\in [0,T];
$$
$$
 v(0,x) = v^0 (x), \, x\in \Omega,\quad v(t,x)=0,\, (t,x)\in [0,T]\times \partial\Omega.
$$
Здесь $v(t,x)$=$ (v_1(t,x),v_2(t,x))$ и $p(t,x)$ искомые векто\-рная и скалярная
функции, означающие скорость движе\-ния и давление среды, $f(t,x)$ "---
плотность внешних сил, $\mathcal{E}(v)$ -- тензор
скоростей деформаций, т.е. матрица с коэффициентами $\mathcal{E}_{ij}(v)\! =\! \frac 1
2 (\partial v_i/\partial x_j +\partial v_j/\partial x_i )$).
Дивергенция ${\rm Div}\, \mathcal{E}(v)$ матрицы определяется как вектор с
компонентами "--- дивергенциями строк, $S(v)$=$\sum_{i,j=1}^{2} (
\partial v_i/\partial x_j)^2$, $0<\alpha<1$, $\mu_0>0$, $\mu_2\ge 0$, $\mu_1(s)$ неотрицательная непрерывно
дифференцируемая при $s\ge 0$ функция.


 Гильбертовы пространства $H$ и $V$ определяются обычным образом
 (см., напр. известную монографию Темама).
Обозначим через $\mathcal{P}$ ортопроектор Лерэ в $L_2(\Omega)^2$ на $H$.

Пусть
$$
W\!=\{v:\ v\in L_2(0,T;W_2^2(\Omega)^2\cap L_2(0,T; H) \cap $$$$\stackrel{\circ}{W_2^1}(\Omega))^2,\ \ v'\in L_2^1(0,T;H)\}.
$$
% с нормой $\|\cdot \|_{1,2}$.

Запишем задачу $Z$ в операторной форме.
 Определим в $H$ оператор $A$ формулой $Av=-{\mathcal{P}}\triangle v$ на $D(A)=H \cap \stackrel{\circ}{W_2^1}(\Omega)^2\cap {W_2^2}(\Omega)^2$. Оператор $A$ является положительно определённым самосопряжённым оператором.

Положим
$K(v)=\sum_{i=1}^2 v_i\frac{\partial v}{\partial x_i}$ для $v\in V$ и введём операторы
$$
B(v) = - {\mathcal{P}}{\rm Div} \, (\mu_1\,(S(v(t,x)))\,\mathcal{E}(v)(t,x));
$$
$$ C(v)=\int\limits_{0}^t(t-s)^{-\alpha}\,A v(s, \cdot )\,ds.
$$
Операторы $B$ и
$C$ определены при п.в. $t$ для функций $v\in W_2^{0,2}(Q)$.

Рассмотрим при $ t\in[0,T]$ задачу $ZP$:
$$
v'+\mathcal{P}K_{}( v)+ k v+\mu_0 Av +B(v)+\mu_2C(v)= f,\ v(0)=v^0.
$$


\textbf{Определение.} {\it
Сильным решением задачи $ZP$ называется функция $v\in W$, удовлетворяющая условию и при п.в. $(t,x)$ уравнению $ZP$.}

Сформулируем основной результат.

\textbf{Теорема~1.} {\it Пусть $f\in L_2 (0,T;H)$, $v^0\in V$, а $\mu_1(s)$ такова, что
$$
 \mu_1(s)+2\mu'_1(s)\ge 0, s\ge 0; \ \ s|\mu_1(s)|\le M,\ \ s\ge a,
$$
где $a>0$ - некоторая константа. Тогда задача $ZP$ имеет сильное решение.}


\textbf{Теорема~2.} {\it Пусть выполняются условия теоремы 1.
Тогда сильное решение задачи $ZP$ единственно.}

 Для операторного уравнения
определяется семейство аппроксимационных уравнений $ZPA$ при $k\ge 0$:
$$
v'+\exp(kt)\mathcal{P}K(v)+ k v+\mu_0 Av +B_k(v)=
$$
$$
f+\mu_2 C_k(v),\, t\in[0,T], \quad v(0)=v^0.
$$
Здесь
$$
C_k(v)=\int\limits_{0}^tR_k(t-s)\,A v(s)\,ds,
$$
$$
B_k(v) = - {\mathcal{P}}{\rm Div} \, (\mu_1\,(\exp(2kt)S(v(t)))\,\mathcal{E}(v)(t))
$$
$$
R_k(t)=\exp(-kt)t^{-\alpha} \mbox{ при } t\in[0,T],\ \ R_k(t)=0 \mbox{ при } t\notin[0,T].
$$
%\end{equation}
 Задача $ZPA$ получается формально умножением уравнения $ZP$ на $\exp(-kt)$.
 Существование
решений аппроксимационных уравнений $ZPA$ устанавливается с помощью
некоторого итерационного процесса. Решение основного операторного уравнения получается как
слабый предел решений аппроксимационных уравнений. Затем устанавливается, что слабое решение является сильным.




Результаты получены совместно с В.Г. Звягиным.


% Оформление списка литературы

\litlist

1. {\it Звягин В.Г., Орлов В.П.} О сильных решениях дробной нелинейно"=вязкоупругой модели типа Фойгта. {\it Известия \newline ВУЗов. Математика.} No 12 (1919), c. 106-111.

2. {\it Орлов В.П., Соболевский П.Е.} О гладкости обоб\-щен\-ных решений уравнений движения почти ньютоновской \newline жидкости, {\it Численные методы механики сплошной среды}. \newline No 1, Т. 16 (1985), c. 107--119.
