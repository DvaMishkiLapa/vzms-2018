
\begin{center}
    {\bf О ФРЕДГОЛЬМОВОСТИ ОПЕРАТОРОВ И УРАВНЕНИЙ С МНОГОМЕРНЫМИ ЧАСТНЫМИ ИНТЕГРАЛАМИ В ПРОСТРАНСТВЕ НЕПРЕРЫВНЫХ НА ПАРАЛЛЕЛЕПИПЕДЕ ФУНКЦИЙ\footnote{Работа поддержана РФФИ (проект № 19-41-480002).}}\\

    {\it А.И. Иноземцев}

    (Липецк; {\it inozemcev.a.i@gmail.com})
\end{center}

\addcontentsline{toc}{section}{Иноземцев А.И.}

В работе получен критерий фредгольмовости уравнения с частными интегралами
$$(I-\sum\limits_{\alpha} K_{\alpha})x=f, \eqno (1)$$
в пространстве непрерывных на $D=[a,b]\times [c,d]\times [e,f]$ функций, где
$(K_{\alpha}x)(t_1, t_2, t_3) = \int\limits_{D_{\alpha}} k_{\alpha}(t_1, t_2, t_3, t_{\alpha})x(s_{\alpha})dt_{\alpha},$ $k_{\alpha}\in C(L^1(D_{\alpha})),$ $\alpha=(\alpha_1, \alpha_2, \alpha_3)$ --- мультииндекс, $\alpha_j\in \{0, 1\}$, $D_{\alpha}=\prod\limits_{j=1}^n [a_j, b_j]^{\alpha_j}$. При $\alpha_j=0$ отрезок $[a_j, b_j]$ не входит в декартово произведение. $t_{\alpha}$ --- совокупность элементов $\tau_j$, для которых $\alpha_j=1,$ а $s_{\alpha}$ --- совокупность элементов $\tau_j$, для которых $\alpha_j=1$ и элементов $t_k,$ для которых $\alpha_k=0.$ $dt_{\alpha}=\prod\limits_{j=1}^n d\tau_j^{\alpha_j},$ $\bar\alpha=(\bar\alpha_1, \bar\alpha_2, \bar\alpha_3),$ где $\bar\alpha_j=1-\alpha_j,$ $k_{\alpha}\in C(D_{\alpha}).$

Теория уравнений вида $(I-\sum\limits_{\alpha} K_{\alpha})x=f$ существенно отлична не только от теории интегральный уравнений Фредгольма, но и от теории сингулярных интегральных уравнений [1, 2]. Никакая гладкость ядер $k_{\alpha}$ не обеспечивает ни нетеровость, ни фредгольмовость данного уравнения.

Принадлежность ядер $k_{\alpha}$ операторов $K_{\alpha}$ пространству $C(L^1(D_{\alpha}))$ означает $\sup\limits_{(t_1,t_2,t_3)\in D}\int\limits_{D_{\alpha}}|k_{\alpha}(t_1,t_2,t_3,t_{\alpha})|dt_{\alpha}<\infty$
и для любого $\varepsilon>0$ существует $\delta>0$ такое, что при $|t_i-t_i'|<\delta$ $(i=1,2,3)$
$\int\limits_{D_{\alpha}}|k_{\alpha}(t_1,t_2,t_3,t_{\alpha})-k_{\alpha}(t_1',t_2',t_3',t_{\alpha})|dt_{\alpha}<\varepsilon.$ При сделанных предположениях операторы $K_{\alpha}$ непрерывны в пространстве $C(D)$. Непосредственно проверяется, что ядра их композиций также принадлежат $C(L^1(D_{\alpha}))$.

Пусть операторы $I-K_{(1,0,0)},$ $I-K_{(0,1,0)}$ и  $I-K_{(0,0,1)}$ обратимы, то в силу [2] $(I-K_{(1,0,0)})^{-1}=I+R_{(1,0,0)},$ $(I-K_{(0,1,0)})^{-1}=I+R_{(0,1,0)},$ $(I-K_{(0,0,1)})^{-1}=I+R_{(0,0,1)},$
где $R_i$ --- операторы с частными интегралми. Тогда уравнение (1) эквивалентно уравнению
$$x=(I+R_{(0,0,1)})(I+R_{(0,1,0)})(I+R_{(1,0,0)})\times$$
$$\times(K_{12}^{(1)}+K_{13}^{(1)}+K_{23}^{(1)}+K_{123}^{(1)})x+G_1f,\eqno(2)$$
где  $G_1=(I-K_{(0,0,1)})^{-1}(I-K_{(0,1,0)})^{-1}(I-K_{(1,0,0)})^{-1},$ $K_{12}^{(1)}=K_{(1,1,0)}+K_{(1,0,0)}K_{(0,1,0)},$ $K_{13}^{(1)}=K_{(1,0,1)}+K_{(1,0,0)}K_{(0,0,1)},$ $K_{23}^{(1)}=K_{(0,1,1)}+K_{(0,1,0)}K_{(0,0,1)},$ $K_{123}^{(1)}=K_{(1,1,1)}+K_{(1,0,0)}K_{(0,1,0)}K_{(0,0,1)}.$

Умножая операторы в (2), получим уравнение
$x=(K_{12}^{(2)}+K_{13}^{(2)}+K_{23}^{(2)}+K_{123}^{(2)})x+G_1f,$ эквивалентное уравнению
$(I-K_{12}^{(2)})(I-K_{13}^{(2)})(I-K_{23}^{(2)})x=(K_{12}^{(2)}K_{13}^{(2)}+K_{12}^{(2)}K_{23}^{(2)}+K_{13}^{(2)}K_{23}^{(2)}-K_{12}^{(2)}K_{13}^{(2)}K_{23}^{(2)}+K_{123}^{(2)})x+G_1f.$

Если существуют обратные операторы [2] $(I-K_{23}^{(2)})^{-1},$ $(I-K_{13}^{(2)})^{-1},$ $(I-K_{12}^{(2)})^{-1},$ то $(I-K_{23}^{(2)})^{-1}=I+R_{23}, (I-K_{13}^{(2)})^{-1}=I+R_{13},(I-K_{12}^{(2)})^{-1}=I+R_{12},$ где $R_{ij}$ --- операторы с частными интегралами. После умножения операторов получим уравнение
$$x=H_{123}x+F,\eqno(3)$$
где $F=G_2G_1f,$ $H_{123}=(I+R_{23})(I+R_{13})(I+R_{12})(K_{12}^{(2)}K_{13}^{(2)}+K_{12}^{(2)}K_{23}^{(2)}+K_{13}^{(2)}K_{23}^{(2)}-K_{12}^{(2)}K_{13}^{(2)}K_{23}^{(2)}+K_{123}^{(2)})$ --- интегральный оператор, а уравнение (3) --- интегральное уравнение.

Из приведённых рассуждений следует

\textbf{Теорема.} {\it Линейное интегральное уравнение (1) с частными интегралами  фредгольмово в $C(D)$ тогда и только тогда, когда в $C(D)$ обратимы операторы $(I-K_{(1,0,0)}),$ $(I-K_{(0,1,0)}),$ $(I-K_{(0,0,1)}),$ $(I-K_{12}^{(2)}),$ $(I-K_{13}^{(2)}),$ $(I-K_{23}^{(2)}).$ }



\smallskip \centerline {\bf Литература} \nopagebreak

1. {\it Appel~J.M., Kalitvin~A.S.,  Zabrejko~P.P.} Partial Integral Operators and \ \  Integro-Differential \ \  Equations / J.M.~Appel,\  A.S.~Kalitvin,\  P.P.~Zabrejko.  New York-Basel: Marcel Dekker, 200. ---~560~pp.

2. {\it Калитвин А.С., Фролова Е.В.} Линейные уравнения с частными интегралами. C-теория: второе издание. -- Липецк: ООО ''Оперативная полиграфия'', 2015. --- 195 с.


