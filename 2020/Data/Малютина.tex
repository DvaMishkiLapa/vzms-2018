\vzmstitle{О непрерывности отображений c $s$-усреднённой характеристикой}
\vzmsauthor{Малютина}{А.\,Н.}
\vzmsauthor{Бердалиева}{М.\,А.}
\vzmsauthor{Асанбеков}{У.\,К.}
\vzmsinfo{Томск; {\it nmd@math.tsu.ru}; {\it madina.berdalieva@mail.ru}; {\it urmat\_1396@mail.ru}}
\vzmscaption
По известной теореме вложения С. Л. Соболева, если область $G$ звёздная относительно шара, принадлежащего области $G$ и $f \in W^{1}_{n,loc}(G)$ при $p>n$, то $f$ непрерывна в $G$. В нашей работе доказывается непрерывность при $1<p\leq n$ но при некоторых дополнительных условиях на отображение $f$ с $s$-усреднённой характеристикой.
Ключевые слова: отображение с $s$-усреднённой характеристикой, дифференциальные свойства, непрерывность.

Рассмотрим область $G\subset R^n$, $f:G\to R^n$, $f\in W^{1}_{n}(G)$, $1<s<n$, такое что, для любого $y\in G$ выполняется неравенства
\begin{equation}
\int\limits_{G}(\lambda(x,f))^{s}k(|x-y|)d\sigma_{x}<M
\end{equation}
\begin{equation}
\int\limits_{G}(\lambda^{*}(x,f))^{s^{*}}k(|x-y|)d\sigma_{x}<M^{*}
\end{equation}
где функция $k(t)$, определена при $t>0$, положительна, не возрастает и $\lim\limits_{t\to 0+}k(t)=+\infty$. В случае (1) будем говорить, что отображение $f$– отображение с $(s,k)$ усреднённой характеристикой, а в случае (2) – отображение с $(s^{*},k)$ - усреднённой характеристикой, где функция $$f\in W^{1}_{n}(G,k,M), 1<s<n$$.
Определение 1. Назовём отображение $f$ области $D$ на область $D'$ отображением класса $W^{1}_{n,loc}(D')$, если $f \in W^{1}_{n,loc}(D)$, $f^{-1} \in W^{1}_{n,loc}(D')$ и обладает $N$ и $N^{-1}$-свойствами.\\
\textbf{Теорема~1.} {\it Если $f\in\widetilde{W}_{n,loc}^1(G) $ и для $1<s\leq n$ и любой точки $y\in G$
\begin{equation}
I\left(
\int\limits_D\frac{|\bigtriangledown f|^{n}}{J(x,f)}
\right)^{s}||x-y||^{-\alpha}d\sigma_{x}
\end{equation}
если $\alpha>n-s$, то на любом компакте $K$ из области $G$ функция $f$ эквивалентна некоторой непрерывной функции.} Доказательство теоремы следует из теоремы Арцела. Для этого построим равностепенно непрерывную и равномерно ограниченную на $К$ последовательность функций, сходящихся к функции $f$ почти везде в $G$. Рассмотрим последовательность $\epsilon$-усредненний функции $f$ по С.Л. Соболеву при достаточно малых $\epsilon$. $\epsilon$-усреднением функции $f$ по С.Л. Соболеву [] называется функция
$$
f_{\epsilon}=\epsilon^{-n}\int\limits_{R^n}\phi\left(\frac{x-u}{\epsilon}\right)f(u)du=\epsilon^{-n}$$
$$\int\limits_{B(0,\epsilon)}\phi\left(\frac{u}{\epsilon}\right)f(x-u)du
$$
Из [1, c. 79] $f_{\epsilon}=0$ вне области $G$. Известно [1, c. 34], что функция $f_{\epsilon}$ бесконечно дифференцируема в $R^n$ и $\|f_{\epsilon}-f\|_p$, $R^{n}\to 0$ при $\epsilon\to 0$. Существуют открытые множества $G_1$ и $G_2$ такие, что компакт $K\subset G_{1} \subset G_{2}$, $\overline{G_1}\subset G_{2}$, $\overline{G_2}\subset G$ где $G_i$- замыкание множества $G_i$, $i=1,2$.
 	Покажем, что для достаточно малых $\epsilon$
 	$$I\left(\left(
\int\limits_D\frac{|\bigtriangledown f|^{n}}{J(x,f)}
\right)^{s}G_{2},y\right) d\sigma_{x}<M, y\in G_2$$
Известно [1,c. 172], что $\frac{\partial f_\epsilon}{\partial x_i}=\left(\frac{\partial f}{\partial x_i}\right)_\epsilon$. Используя обобщённое неравенство Минковского [1, c. 27] и условие (1) получим
\begin{multline}
I\left(\int\limits_{G_2}\left(\frac{|\bigtriangledown f_{\epsilon}|^{n}}{J(x,f)}\right)^s, G_{2},y\right)\\
=\left[ \int\limits_{G_2}\epsilon^{-n}\int\limits_{B(0,\epsilon)}\phi\left(\frac{u}{\epsilon}\right)\left(\frac{|\bigtriangledown f(x-y)|^{n}}{J(x,f)}\right)^{s}
du||x-y||^{-\alpha}d\sigma_{x}\right]^{\frac{1}{s}}\\
\leq\epsilon^{-n}\int\limits_{B(0,\epsilon)}\left[\int\limits_{G_2}\phi\left(\frac{u}{\epsilon}\right)\left(\frac{|\bigtriangledown f(x-y)|^{n}}{J(x,f)}\right)^{s}||x-y||^{-\alpha}d\sigma_{x}\right]^{\frac{1}{s}}du
\\\leq\epsilon^{-n}\int\limits_{B(0,\epsilon)}\left[\phi\left(\frac{u}{\epsilon}\right)\int\limits_{G_2}\left(\frac{|\bigtriangledown f(x-y)|^{n}}{J(x,f)}\right)^{s}||x-y||^{-\alpha}d\sigma_{x}\right]^{\frac{1}{s}}du\\
\leq\epsilon^{-n}M\int\limits_{B(0,\epsilon)}\phi\left(\frac{u}{\epsilon}\right)du=M
\end{multline}
если $(y-u)\in G$, т.е. при $\epsilon<\epsilon_0$, где $\epsilon_0$ меньше расстояния от границы множества $G_2$ до границы $G$. Из неравенства (4) следует, что $\forall y\in G_2$ выполнено неравенство:
\begin{equation}
\int\limits_{B(y,r)}\left|\frac{|\bigtriangledown f_{\epsilon}(x)|^{n}}{J(x,f)}\right|^s, d\sigma_{x}<n^{\frac{n}{2}}Mr^\alpha
\end{equation}
если $B(y,r)\subset G_2$.
Из (5) следует, что непрерывные функции $f_?$ при $\epsilon<\epsilon_0$ удовлетворяет условию леммы Ч. Морри [2, c. 11], поэтому для любых точек $x, y$ таких, что шар
$$
B\left(\frac{x+y}{2},\frac{3}{2}|x-y|\right)\subset G_{2}; |f_{\epsilon}(x)-f_{\epsilon}(y)|<N|x-y|^\beta,
$$
где $\beta=\frac{\alpha-n+s}{s}$ и $N$ зависит от $M,n,s,\alpha$.
Таким образом семейство функций $f_\epsilon$ при $\epsilon<\epsilon_0$ на $K$ равностепенно непрерывно. Покажем, что функции $f_\epsilon$ при $\epsilon<\epsilon_0$ на $K$ ограничены одним числом. Существует функция $\eta\in D$ такая, что её носитель лежит в $G_2$ и $\eta(x)=1$ для $x\in G_1$ [3, c. 16]. Функция $f_{\eta}\in W_{p}^{1} (R^n)$. Доопределим $f(x)=0$ вне области $G$. Для $\phi\in D$.
\begin{multline}
\int\limits_{R^n}\left(\frac{\partial\eta}{\partial x_i}f+\eta\frac{\partial f}{\partial x_i}\right)\phi dx
= \int\limits_{G}\left(\frac{\partial\eta}{\partial x_i}f+\eta\frac{\partial f}{\partial x_i}\right)\phi dx\\
=\int\limits_{G}\frac{\partial\eta}{\partial x_i}f\phi dx+\int\limits_{G}\eta\phi\frac{\partial f}{\partial x_i}dx=\\
\int\limits_{G}\frac{\partial\eta}{\partial x_i}f\phi dx-\int\limits_{G}\frac{\partial(\eta\phi)}{\partial x_i}dx=-\int\limits_{G}f\eta\frac{\partial\phi}{\partial x_i}dx\\
\end{multline}


Из (6) следует, что обобщённая производная
\begin{equation}
\frac{\partial(\eta,f)}{\partial x_i}=\frac{\partial\eta}{\partial x_i}f+\eta\frac{\partial f}{\partial x_i}
\end{equation}
Покажем, что для функции $\eta f$ справедлива оценка
$$
I\left[\left|\frac{|\bigtriangledown\eta f|^{n}}{J(x,f)}
\right|^{s}\|x-y\|^{-\alpha}\right]<M
$$
для $\forall y\in K$. В самом деле из (7) следует, что
\begin{multline}
I\left[\left|\frac{|\bigtriangledown\eta f|^{n}}{J(x,\eta,f)}
\right|^{s}||x-y||^{-\alpha}\right]\leq I\left[\left|\frac{|\bigtriangledown\eta|^{n}}{J(x,\eta,f)}f
\right|^{s}||x-y||^{-\alpha}\right]+\\I\left[\left|\frac{|\bigtriangledown f|^{n}}{J(x,\eta,f)}
\right|^{s}||x-y||^{-\alpha}\right]
\\
\leq I\left[\left|\frac{|\bigtriangledown\eta|^{n}}{J(x,\eta,f)}f
\right|^{s}||x-y||^{-\alpha}\right]_{y\in G_1}+I\left[\left|\eta\frac{|\bigtriangledown f|^{n}}{J(x,\eta,f)}
\right|^{s}||x-y||^{-\alpha}\right]_{y\in G_2}\\\leq Ad^{\frac{-\alpha}{s}}+CM=M_1
\end{multline}
Здесь $A=max\frac{\partial\eta}{\partial x_i}$ $x\in G_2$, и $C=max\eta(x)$ $x\in G_2$, a $d$ "--- расстояние между границами $G_1$ и $G_2$. В работе исследуются отображения с $s$-усреднённой характеристикой. Приведены некоторые условия, когда эти свойства представляют интерес и могут найти приложение в теории многомерных квазиконформных отображений и их обобщений.
 Известно [4, c. 147], что поэтому функция $\eta f_\epsilon$ удовлетворяет условиям теоремы, и, применим условие Гёльдера и оценки (8) получаем
 \begin{multline}
 \left|\eta(x)\frac{|\bigtriangledown f_\epsilon|^n}{J(x,f)}\right|=\frac{1}{\omega_{n-1}}\int\limits_{R^n}\frac{|\bigtriangledown\eta f_\epsilon|^n}{J(x,f)}(x-y)\frac{y_i}{||y||^n}d\sigma_{y}\leq
 \\
 \frac{1}{\omega_{n-1}}\int\limits_{G_2}\frac{|\bigtriangledown\eta f_\epsilon|^n}{J(x,f^-1)}\frac{d\sigma_{y}}{||y-x||^{n-1}}\leq
 \\
 \frac{1}{\omega_{n-1}}\left(\int\limits_{G_2}\left(\frac{|\bigtriangledown\eta f_\epsilon|^n}{J(x,f^{-1})}(y)\right)^p |y-x|^{-\alpha}d\sigma_y\right)^{\frac{1}{p}}\\\left(\int\limits_{G_2}|y-x|^{m}d\sigma_y\right)^{\frac{p-1}{p}}\leq M_2
\end{multline}
Для $x\in K$ $\eta(x)\bigtriangledown f_{\epsilon}(x)=f_{\epsilon}(x)$ и $|f_\epsilon|\leq M_2$ равносильна.
Таким образом на $K$ семейство функций $f_\epsilon$, $\epsilon<\epsilon_0$ равностепенно непрерывно, ограниченно, следовательно по т. Арцела из семейства можно выделить последовательность функций $|f_{n}(x)|$ равносильно сходится на $K$ к некоторой непрерывной функций $\psi$ таким образом функции $f$ и $\psi$ эквивалентны.


\litlist

1. {\it Никольский С.М.} Приближение функций многих переменных и теоремы вложения. М.: Наука, 1969.

2. {\it Малютина А. Н., Асанбеков У. К.} О модуле непрерывности отображений с $s$-усреднённой характеристикой // Вестн Том. гос. ун-та. Математика и механика. 2019. № 59. C. 11–15.

3. {\it Владимиров В. С.} Обобщённые функции в математической физике. М., Наука, 1976.

4. {\it Стейн И.} Сингулярные интегралы и дифференциальные свойства функций. М., Мир, 1973.
