\begin{center}
    {\bf НЕКОМПАКТНЫЕ ОСОБЕННОСТИ ИНТЕГРИРУЕМЫХ ГАМИЛЬТОНОВЫХ СИСТЕМ С ДВУМЯ СТЕПЕНЯМИ СВОБОДЫ\footnote{Исследование выполнено за счёт гранта Российского научного фонда (проект №17-11-01303).}}\\

    {\it С.С. Николаенко}

    (Москва; {\it nikostas@mail.ru})
\end{center}

\addcontentsline{toc}{section}{Николаенко С.С.}

Изучается задача топологической классификации 3-мер\-ных бифуркаций (перестроек) лиувиллевых слоений, возникающих в интегрируемых гамильтоновых системах с двумя степенями свободы, ограниченных на неособые невырожденные изоэнергетические многообразия $Q^3$. Такие бифуркации были названы А.Т.~Фоменко {\it 3-атомами} (см. [1]). Как было показано А.Т.~Фоменко [2], в случае компактного многообразия $Q^3$ в некоторой малой инвариантной окрестности бифуркационного слоя определено сохраняющее первые интегралы гамильтоново $S^1$-действие с тривиальными либо изоморфными группе $\mathbb Z_2$ стабилизаторами. Как следствие, каждый компактный 3-атом допускает структуру $S^1$-расслоения, а именно, известного в маломерной топологии расслоения Зейферта с особыми слоями типа $(2,1)$ (см., например, [3]), базой которого является {\it 2-атом} (описывающий бифуркации одномерных слоений, задаваемых функциями Морса на 2-мерных многообразиях). Отметим, что аналогичный результат получен Н.Т.~Зунгом [4] в многомерном случае для вещественно"=аналитических систем. Мы обобщаем теорему Фоменко на случай систем с некомпактными изоэнергетическими многообразиями $Q^3$, удовлетворяющих следующим двум условиям:
\begin{enumerate}
	\item гамильтоновы поля, порождаемые первыми интегралами системы, полны (т.~е.~естественный параметр на их интегральных траекториях определён на всей числовой прямой);
	\item на бифуркационном слое хотя бы одна орбита гамильтонова $\mathbb R^2$-действия (определённого в силу предыдущего условия) является нестягиваемой (т.~е.~гомеоморфна окружности $S^1$ или цилиндру $S^1\times\mathbb R$).
\end{enumerate}
Таким образом, как и в компактном случае, задача классификации некомпактных 3-атомов (для систем, удовлетворяющих двум перечисленным условиям) сводится к задаче классификации некомпактных 2-атомов, полученной ранее в работе [5].



% Оформление списка литературы
\smallskip \centerline {\bf Литература} \nopagebreak

1. {\it Болсинов А.В., Фоменко А.Т.} Интегрируемые гамильтоновы системы. Геометрия, топология, классификация. \\ Том~1. Ижевск: изд. дом ``Удмуртский университет'', 1999. -- 444~с.

2. {\it Фоменко А.Т.} Топология поверхностей постоянной\\ энергии интегрируемых гамильтоновых систем и препятствия к интегрируемости // Изв. АН СССР. Сер. матем. -- 1986. -- Т.~50, №~6. -- С.~1276--1307.

3. {\it Матвеев С.В., Фоменко А.Т.} Алгоритмические и компьютерные методы в трёхмерной топологии. М.: изд-во МГУ, 1991. -- 304 с.

4. {\it Zung N.T.} A note on degenerate corank-one singularities of integrable Hamiltonian systems // Commentarii Mathematici Helvetici. -- 2000. -- Vol.~75, no.~2. -- P.~271--283.

5. {\it Николаенко С.С.} Топологическая классификация гамильтоновых систем на двумерных некомпактных многообразиях // Матем. сборник (в печати).
