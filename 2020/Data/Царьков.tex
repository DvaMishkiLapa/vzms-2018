\begin{center}
    {\bf НЕПРЕРЫВНАЯ ВЫБОРКА ИЗ ОТНОСИТЕЛЬНОГО ЧЕБЫШЕВСКОГО ПРОЕКТОРА В $C(Q)$\footnote{Работа выполнена при финансовой
поддержке РФФИ (проект \No\ 19-01-00332-a}}\\

    {\it И.Г. Царьков}

    (Москва; {\it tsar@mech.math.msu.su})
\end{center}

\addcontentsline{toc}{section}{Царьков И.Г.}


Путь $p:[0,1]\rightarrow X$ $($непрерывное
отображение$)$ в линейном нормированном пространстве
$(X,\|\cdot\|)$ называется монотонным, если для любого функционала
$x^*\in\operatorname{extr}S^*$ функция $x^*(p(t))$ является
монотонной. Геометрически это означает, что поверхности уровня этого функционала $($т.е.
соответствующие гиперплоскости$)$ этот путь пересекает один раз или
по следу некоторого его подпути.
Множество $M$ называется монотонно линейно связным, если любые
две точки этого множества можно соединить монотонным путём, след
которого лежит в $M.$
В пространстве $X$ для непустого множества $V\subset X$ и непустого ограниченного множества $M\subset X$ через $r_V(M)$ обозначим относительный чебышевский радиус, т.е. величину $\inf\{r\geqslant 0\mid M\subset B(x,r), x\in V\}$. Через
 $Z_V^\varepsilon (M)$ обозначим множество почти чебышевских центров: $ \{x\in V\mid M\subset B(x,r_V(M)+\varepsilon)\}$.


\textbf{Теорема~1.} {\it Пусть $X$ -- линейное нормированное пространство, $V\subset X$ -- монотонно линейное связное ограниченно компактное непустое множество. Тогда для каждого $\varepsilon>0$ существует непрерывная выборка из отображения $Z_V^\varepsilon(\cdot)$.


}


А.Р.~Алимов~[1], [2] доказал, что в пространстве $c_0$ всякое солнце является монотонно линейно связным. Также им доказано~[2], что в $C(Q)$ ($Q$ -- метрический компакт) всякое строгое солнце является монотонно линейно связным.




\textbf{Следствие~1.} {\it Пусть $X=C(Q) $ $(Q$ -- метрический компакт$)$, $V\subset X$ -- ограниченно компактное строгое солнце. Тогда для каждого $\varepsilon>0$ существует непрерывная выборка из отображения $Z_V^\varepsilon(\cdot)$.

}

Аналогичный результат верен для случая пространства $c_0$.

\smallskip \centerline {\bf Литература} \nopagebreak


1. {\it Алимов А.Р.} Связность солнц в пространстве $c_0$ // Изв. РАН. Сер. матем.
--- 2005. ---
Т.~69,
№~4. --- С.~ 3--18.

2. {\it Алимов А.Р.} Монотонная линейная связность
чебышёвских множеств в~пространстве~$C(Q)$ // Матем. сб.
--- 2006. ---
Т.~197,
№~9. --- С.~ 3--18.
