\vzmstitle{ДОСТАТОЧНЫЕ УСЛОВИЯ СОГЛАСОВАНИЯ ХАРАКТЕРИСТИЧЕСКОЙ КОСОЙ ПРОИЗВОДНОЙ ГРАНИЧНОГО РЕЖИМА С НАЧАЛЬНЫМИ УСЛОВИЯМИ И ОДНОМЕРНЫМ ВОЛНОВЫМ УРАВНЕНИЕМ ДЛЯ ГЛАДКИХ РЕШЕНИЙ}
\vzmsauthor{Ломовцев}{Ф.\,Е.}
\vzmsauthor{Устилко}{Е.\,В.}
\vzmsinfo{Минск; {\it lomovcev@bsu.by}; {\it ustilko@tut.by}}
\vzmscaption

Выведены достаточные условия согласования для смешанной задачи с
характеристической косой производной в граничном режиме в первой
четверти плоскости ~$\dot{G}_\infty=]0,+\infty[ \times
]0,+\infty[$ :
$$
u_{tt}(x,t) + (a_1 - a_2)u_{xt}(x,t) - a_1a_2 u_{xx}(x,t) =
f(x,t), \, (x,t) \in \dot{G}_{\infty}, \eqno(1)
$$
$$
u(x,t)\lvert_{t=0} = \varphi(x), \, u_t(x,t)\lvert_{t=0} =
\psi(x), \, x>0, \eqno(2)
$$
$$[\alpha(t)u_t+\beta(t)u_x+\gamma(t)u]\lvert_{x=0}=\mu(t),
\,t>0, \eqno(3)
$$
где нижними индексами функции $u$ обозначены её частные
производные соответствующих порядков по указанным в индексах
переменным, постоянные~$a_1
> 0$, ~$a_2 > 0,$ коэффициенты~$\alpha,\, \beta,\, \gamma$ -- заданные функции переменной ~$t,$
$a_1\alpha(t)=\beta(t),\, t>0,$ и исходные данные~$ f,
\,\varphi,\, \psi, \, \mu$ -- заданные функции своих переменных
$x$ и $t$.

Уравнение (1) имеет характеристики $x-a_1t=C_1$, $x+a_2t=C_2,$
$\forall$ $C_1,\,C_2\in \mathbb{R}=]-\infty,+\infty[$.
Характеристика $x=a_1t$ является критической для уравнения (1) и
делит множество $G_{\infty}=[0,+\infty[ \times [0,+\infty[$ на два
подмножества $G_{-}$ и $G_{+}$ [1]. Под $C^{k}(\Omega)$ понимается
множество всех $k$ раз непрерывно дифференцируемых функций на
подмножестве $\Omega\subset \mathbb{R}^2$ и
$C^0(\Omega)=C(\Omega)$. Критерий корректности смешанной задачи
(1)--(3) для простейшего уравнения колебаний полуограниченной
струны (1) при $a_1=a_2$ с характеристической косой производной
граничного режима (3) во множестве классических решений $u\in
C^{2}(G_{\infty})$ установлен в [1]. Работа [2] свидетельствует о
том, что в случае характеристической косой производной граничного
режима (3) для уравнения колебаний ограниченной струны нужны
критерии корректности этой вспомогательной смешанной задачи
(1)--(3) для более гладких решений $u\in C^{m}(G_{\infty}),\,m\geq
2.$

{\bf\textit{Определение.} } Гладким $m$ раз непрерывно
дифференцируемым решением начально"=граничной задачи (1)--(3)
называется функция ~$u\in C^{m}(G_{\infty})$, где $m=2,3,4,\,...$,
удовлетворяющая уравнению (1) в обычном смысле, а начальным
условиям (2) и граничному режиму (3) в смысле пределов
соответствующих выражений от её значений $u(\dot{x},\dot{t})$ во
внутренних точках $(\dot{x},\dot{t})\in \dot{G}_\infty$ для всех
указанных в них граничных точек $x$ и $t$.

Для решений $u \in C^{m+1}({G}_\infty)$ задачи (1)--(3) на
<<единицу>> большей гладкости нами найдены условия согласования.
Для этих решений из (1)--(3) вытекают очевидные требования
гладкости
$$
f \in C^{m-1}(G_\infty), \, \varphi \in C^{m+1}[0, +\infty[, \,
\psi \in C^{m}[0, +\infty[, \, \mu \in C^{m}[0, +\infty[.
 \eqno{(4)}
 $$

Для получения первого условия согласования в равенстве (3)
полагаем $t=0$ и вычисляем значения слагаемых его левой части,
используя условия (2) и характеристичность первых производных
$$
 \alpha(0)[\psi(0)+a_1\varphi^{(1)}(0)]+\gamma(0)\varphi(0) = \mu(0).\eqno(5)
$$
Второе условие согласования граничного режима (3) с начальными
условиями (2) и уравнением (1) находится, полагая $t=0$ в первой
производной по $t$ от равенства (3) и вычисляя значения
соответствующих производных от решения $u$ при $x=0, \, t=0$ с
помощью начальных условий (2) при $x=0$, уравнения (1) при $x=0,
\, t=0$ и характеристичности первых производных
$$
 \alpha^{(1)}(0)[\psi(0)+a_1\varphi^{(1)}(0)]+\gamma^{(1)}(0)\varphi(0)+
$$
$$
 + \alpha(0)\big\langle a_2[\psi^{(1)}(0)+a_1\varphi^{(2)}(0)] + f(0,0)\big\rangle + \gamma(0)\psi(0) = \mu^{(1)}(0).
 \eqno(6)
$$

Аналогичным образом выводятся остальные условия согласования
граничного режима с начальными условиями и уравнением.

\textbf{Теорема.} {\it Пусть в граничном режиме (3) с
коэффициентами $\alpha,\,\beta,$ $\gamma\in C^m[0,+\infty[$ косая
производная является характеристической, т.е. она направлена вдоль
критической характеристики уравнения (1): $a_1 \alpha(t) =
\beta(t),\, t\in[0,+\infty[.$ Если смешанная задача (1)--(3) имеет
решение $u\in C^{m+1}(G_\infty)$, то для исходных данных $f,\,
\varphi, \,\psi,\, \mu$ из (4) справедливы условия согласования
(5), (6) и}
 $$
\alpha^{(q)}(0)[\psi(0)+a_1\varphi^{(1)}(0)]+\gamma^{(q)}(0)\varphi(0)
+
 $$
 $$
+ q\bigg\{\alpha^{(q-1)}(0)\bigg\langle
a_2[\psi^{(1)}(0)+a_1\varphi^{(2)}(0)]+f(0,0)\bigg\rangle+\gamma^{(q-1)}(0)\psi(0)\bigg\}+
 $$
 $$
+\sum _{i=2}^{q} \frac{q\,!}{i\,!\,(q-i)\,!}\Bigg\{
\alpha^{(q-i)}(0)\Bigg\langle
a_2^{i}[\psi^{(i)}(0)+a_1\varphi^{(i+1)}(0)]+
$$
$$
 + \sum _{j=0}^{i-1}
a_2^{j}
f^{(j,\,i-j-1)}(0,0)\Bigg\rangle+\gamma^{(q-i)}(0)\Bigg\langle
 (-a_1)^{i-1} \psi ^{(i-1)}(0)+
$$
$$
+\,a_2\,\frac{a_2^{i-1}-(-a_1)^{i-1}}{a_1+a_2}\,[\psi^{(i-1)}(0)+a_1\varphi^{(i)}(0)]+
 $$
 $$
 +\sum _{k=0}^{i-2}\,
\frac{a_2^{k+1}-(-a_1)^{k+1}}{a_1+a_2}\, f^{(k,\,i-k-2)}(0,0)
\Bigg\rangle\Bigg\}=\mu^{(q)}(0),\,q=\overline{2,m}.\eqno(7)
 $$


 Здесь справа сверху над коэффициентами ~$\alpha,\, \beta,\, \gamma$, начальными $\varphi,\,\psi$
и граничным $\mu$ данными этой задачи цифрой в круглых скобках
обозначены порядки производных по $x$ или $t$. Аналогичным образом
в круглых скобках через запятую обозначены соответственно порядки
частных производных по $x$ и $t$ от правой части $f$.

{\bf Замечание.} В дальнейшем мы ослабим достаточные условия
согласования (сопряжения) (5)--(7) до необходимых условий таких,
как, например, в [1]. Затем мы их применим для выявления критерия
корректности аналогичной смешанной задачи при характеристических
косых производных на двух концах ограниченной струны без
продолжений данных новым методом <<вспомогательных смешанных задач
для волновых уравнений на полупрямой>> из [3].

\litlist

1. \textit{Ломовцев Ф.Е.} {Необходимые и достаточные условия
вынужденных колебаний полуограниченной струны с первой
характеристической косой производной в нестационарном граничном
условии./ Ф.Е.~Ломовцев~// Весцi НАН Беларусi. Сер. фiз.-мат.
навук.~--- 2016.~--- № 1~--- С. 21--27.}


2. \textit{Ломовцев Ф.Е.} {Смешанная задача для неоднородного
уравнения колебаний ограниченной струны при характеристических
нестационарных первых косых производных на концах. /
Ф.Е.~Ломовцев, Т.С.~Точко~// Веснiк Гродзенскага дзяржаунага
унiверсiтэта iмя Я. Купалы. Серыя 2.~--- 2019.~--- Т. 9, № 2.~---
С.~56--75.}


3. \textit{Ломовцев Ф.Е.} {Метод вспомогательных смешанных задач
для полуограниченной струны. / Ф.Е.~Ломовцев~// Шестые
Богдановские чтения по обыкновенным дифференциальным уравнениям:
матер. Междунар. матем. конф. Минск БГУ. 7--10 дек. 2015 г. в 2 ч.
Минск : ИМ НАН Беларуси.~--- 2015.~--- Ч. 2.~--- С. 74--75.}
