\begin{center}
{\bf ИНТЕГРАЛЬНЫЕ УРАВНЕНИЯ РЕШЕТЧАТЫХ МОДЕЛЕЙ СТАТИСТИЧЕСКОЙ МЕХАНИКИ}

{\it Е.Ю. Московченко, Ю.П. Вирченко}

(Белгород; {\it virch@bsu.edu.ru})
\end{center}

\addcontentsline{toc}{section}{Московченко Е.Ю., Вирченко Ю.П.}

Изучается класс решётчатых моделей статистической механики классических систем с суммируемым парным потенциалом взаимодействия. Изучается система уравнений для частных распределений вероятностей гиббсовского
точечного случайного поля. Доказана аналитическая зависимость решений этой системы от спектрального параметра $z$ в области $\{z\,: {\rm Re}z \ge 0\}$ при достаточно больших значений
параметра $T > 0$ в распределении Гиббса.
Пусть $\Lambda = \{{\bf x}\in {\bf Z}^3: {\bf x} = \sum_{j = 1}^3 n_j{\bf e}_j, n_j = 0 \div L-1\}$ --- последовательность множеств в ${\bf Z}^3$, где ${\bf e}_j$ -- орты в ${\bf R}^3$. При $L\to \infty$ трансляцией $\Lambda \Rightarrow \Lambda - (L/2)\sum_{j = 1}^3{\bf e}_j$ определён переход к пределу $\Lambda \to {\bf Z}^3$. Для каждого $\Lambda$ вводится пространство случайных событий $\Omega(\Lambda)$, состоящее из класса всех дихотомических функций $\rho ({\bf x})$, ${\bf x} \in \Lambda$ со значениями 0 и 1. На этом пространстве определён функционал
$$
{\sf H}_\Lambda[\rho] = - \mu \sum_{{\bf x}\in \Lambda} \rho ({\bf x}) + \frac 12 \sum_{{{\bf x} \in \Lambda, {\bf y} \in {\bf Z}^d, {\bf y} \ne {\bf x}} } U({\bf x} - {\bf y})\rho ({\bf x})\rho ({\sf P}_{\Lambda}{\bf y})\,, \eqno (1)
$$
где $\mu \in {\bf R}$ и функция $U({\bf x})$ со значениями в ${\bf R}$ --- центрально-симметрична, $U(-{\bf x}) = U({\bf x})$ и суммируема на ${\bf Z}^3$, $\| U \| \equiv \sum_{{\bf x} \in {\bf Z}^d} | U({\bf x})| < \infty$, причём $U(0) = 0$. Здесь ${\sf P}_\Lambda$ -- оператор проектирования, определяемый ${\sf P}_\Lambda {\bf x} = {\bf z} \in \Lambda$ для каждого вектора ${\bf x} \in {\bf Z}^3$ на основе однозначного представления в виде ${\bf x} = {\bf z} + L\sum_{j = 1}^3n_j{\bf e}_j$, $\langle n_1, n_2, n_3\rangle \in {\bf Z}^3$.

На основе гамильтониана (1) вводится распределение вероятностей Гиббса на $\Omega(\Lambda)$.
$$
{\rm Pr}\{\rho\} = Q_\Lambda^{-1}(z)\, \exp \Big(- {\sf H}_\Lambda[\rho]/T\Big),\ \ T > 0,
$$
$$
Q_\Lambda(z) = \sum_{\rho \in\, \Omega(\Lambda)} \exp \Big(- {\sf H}_\Lambda[\rho]/T\Big)\,.$$
Набор ${\sf p}_\Lambda$ вероятностей $p_\Lambda (X, z) = {\sf E}\prod_{{\bf x} \in X}\rho({\bf x})$, $X \subset \Lambda$. В пределе $\Lambda \to {\bf Z}^d$ он удовлетворяет системе уравнений
$${\sf p}(z) = z(1 + z)^{-1}{\sf e} + {\sf K} {\sf p} (z)\,, \quad z = e^{\mu/T}, \eqno (2)$$
где ${\sf e} = \langle \delta_{1, |X|}: \emptyset \ne X \subset \Lambda\rangle$ и линейный оператор ${\sf K}$, действующий в линейном банаховом пространстве ${\mathcal E}$ с нормой
$\|{\sf p}\|_0 = \sup_{X \subset {\bf Z}^3; |X| < \infty} |p(X, z)|$, определяется формулой
$$
\big({\sf K} {\sf g}\big)(X) = \frac{zW({\bf x}; X)}{1 + zW({\bf x}; X)}\Big[ (1 - \delta_{0, |X\setminus\{{\bf x}\}|})g(X\setminus \{{\bf x}\}) + \phantom{AAAAAAAAAA}$$
$$
\phantom{AAAAAA} + \sum_{\emptyset \ne Y \subset {\bf Z}^d \setminus X } K({\bf x}; Y) \big[g (X\setminus \{{\bf x}\}\cup Y) - g(X \cup Y)\big]\Big]\,, \eqno (3)$$
$$
K({\bf x}; Y) = \Big\{\prod_{{\bf y}\in Y} K({\bf x} - {\bf y}),\ \mbox{при}\ |Y| > 0\,;\ \ 1,\ \mbox{при}\ |Y| = 0\,. \Big\}\,, $$
$K({\bf x}) = \exp(- U({\bf x})/T) - 1$, ${\bf x} \in {\bf Z}^d$, $W({\bf x}; X) = \linebreak \exp\Big(- \sum_{{\bf y}\in X} U ({\bf x} - {\bf y})/T\Big)$.
\vskip 0.2cm

\textbf{Теорема~1.} {\it Уравнение (2) с оператором (3), разрешимы однозначным образом в области $\{z:{\rm Re}\,z \ge 0\}\subset {\bf C}$ и их решения ${\sf p} = \langle p(X, z); X \subset {\bf Z}^3\rangle$ являются аналитическими функциями от $z$ при ${\rm Re}z > 0$, если $T > \|U\|/|\ln \kappa_0|$, где $\kappa_0$ -- единственный корень полинома $\kappa^4 + 4 (\kappa - 1)$ на отрезке $[0, 1]$.}
\smallskip

\centerline {\bf Литература}
\nopagebreak

1. {\it Добрушин Р.Л.} Гиббсовские случайные поля для решётчатых систем с попарным взаимодействием // Функциональный анализ и его приложения.-- 1968.-- 4(1).-- С.31-43.
