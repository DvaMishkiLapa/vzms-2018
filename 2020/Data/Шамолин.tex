\selectlanguage{russian}

\vzmstitle{ИНТЕГРИРУЕМЫЕ ДИНАМИЧЕСКИЕ СИСТЕМЫ НЕЧЕТНОГО ПОРЯДКА С ДИССИПАЦИЕЙ}
\vzmsauthor{Шамолин}{М.\,В.}
\vzmsinfo{Москва; {\it shamolin@rambler.ru}, {\it shamolin.maxim@gmail.com}}
\vzmscaption





Дать общее определение динамической системы с диссипацией довольно
затруднительно. В каждом конкретном случае иногда это может быть
сделано: вносимые в систему определённые коэффициенты в уравнениях
указывают в одних областях фазового пространства на рассеяние
энергии, а в других областях --- на её подкачку. Последнее приводит
к потере известных первых интегралов (законов сохранения),
являющимися гладкими функциями [1--3].

Как только в системе обнаруживаются притягивающие или отталкивающие
предельные множества, необходимо забыть о полном наборе даже
непрерывных во всем фазовом пространстве первых интегралов.

Показана интегрируемость некоторых классов однородных динамических
систем нечётного (третьего, пятого, седьмого и девятого) порядка, в
которых выделяется система на касательном расслоении к четномерным
(соответственно, одномерным, двумерным, трёхмерным и четырёхмерным)
многообразиям. При этом силовые поля обладают диссипацией разного
знака и обобщают ранее рассмотренные.

Приведём примеры систем третьего порядка. Пусть $v$, $\alpha$, $z$
--- фазовые переменные в гладкой динамической системе, правые
части которой %по переменным $v$, $z$
--- однородные полиномы степени по переменным $v$, $z$ с коэффициентами,
зависящими от $\alpha$. Тогда, выбирая в качестве новой независимой
переменной величину $q$ ($dq=vdt$, $d/dq=<'>$, $v\neq 0$), а также
новую фазовую переменную $Z$ по формуле $z=Zv$, рассматриваемую
систему можно переписать в следующем виде:
%\begin{equation}\label{b.0-1}
$$
v'=v\Psi(\alpha,Z),\eqno(1)
$$
%\end{equation}
%\begin{equation}\label{b.0-2}
$$
\begin{array}{c}
\alpha'=g(\alpha)+h(\alpha)Z+i(\alpha)Z^2,\\
Z'=d(\alpha)+e(\alpha)Z+f(\alpha)Z^2-Z\Psi(\alpha,Z),
\end{array}\eqno(2)
$$
%\end{equation}
$$\Psi(\alpha,Z)=a(\alpha)+b(\alpha)Z+c(\alpha)Z^2,$$
при этом уравнение (1) на $v$ отделяется, что даёт возможность
рассматривать два оставшихся уравнения в качестве системы (2) с
одной степенью свободы на двумерном многообразии $N^2\{Z;\alpha\}$.
Особняком стоит случай, когда выполнены тождества
%\begin{equation}\label{b.t1}
$$
d(\alpha)\equiv e(\alpha)\equiv f(\alpha)\equiv 0.\eqno(3)
$$
%\end{equation}
При этом остальные функции $a(\alpha)$, $b(\alpha)$, $c(\alpha)$,
$g(\alpha)$, $h(\alpha)$, $i(\alpha)$, вообще говоря, не равны
тождественно нулю. Тогда система (1), (2) имеет естественный
аналитический первый интеграл
%\begin{equation}\label{b.i1}
$$
\Phi_1(v;Z)=z=vZ=C_1=\textrm{const}.\eqno(4)
%z=vZ=\textrm{const}.
$$
%%\end{equation}

Для полной интегрируемости системы (1), (2) при условии (3) нужно
найти ещё один первый интеграл, независимый с (4).
%\begin{prop}\label{pr0-1}
Если выполнены следующие условия
%\begin{equation}\label{b.0-3}
$$
a(\alpha)=\frac{h^2(\alpha)}{i^2(\alpha)}c(\alpha),\\
b(\alpha)=\frac{h(\alpha)}{i(\alpha)}c(\alpha),~g(\alpha)=\frac{h^2(\alpha)}{i(\alpha)},
$$
%\end{equation}
где $c(\alpha)$, $h(\alpha)$, $i(\alpha)$ --- произвольные гладкие
функции на своей области определения, то система (1), (2) при
условии (3) имеет два гладких первых интеграла, а именно, (4), а
также
%\begin{equation}\label{b.i2}
$$
\Phi_0(v;Z;\alpha)=v^2(\gamma(\alpha)+\epsilon(\alpha)Z)=C_0=\textrm{const},\\
$$
%\end{equation}
где функции $\gamma(\alpha)$ и $\epsilon(\alpha)$ имеют вид
%\begin{equation}\label{b.0-4}
$$
\begin{array}{c}
\gamma(\alpha)=\gamma_0\exp\left[-2\int_{\alpha_0}^{\alpha}\frac{c(\xi)}{i(\xi)}d\xi\right],\\
\epsilon(\alpha)=\epsilon_0\exp\left[-\int_{\alpha_0}^{\alpha}\frac{c(\xi)}{i(\xi)}d\xi\right],~
\gamma_0=\gamma(\alpha_0),~\epsilon_0=\epsilon(\alpha_0).
\end{array}
$$
%\end{equation}
%\end{prop}



Внутреннее силовое поле (зависящее от трёх произвольных гладких
функций $c(\alpha)$, $h(\alpha)$ и $i(\alpha)$) в системе (1), (2)
при условии (3) не нарушает консервативности системы. Ограничимся
важным частным случаем системы (1), (2).

Как представительницу систем вида (1), (2) при условии (3) будем
рассматривать следующую систему третьего порядка
%\begin{equation}\label{b.1-1}
$$
v'=v\Psi(\alpha,Z),\eqno(5)
$$
%\end{equation}
%\begin{equation}\label{b.1-2}
$$
\begin{array}{c}
\alpha'=-Z+b_0Z^2\delta(\alpha),\\
Z'=-Z\Psi(\alpha,Z),
\end{array}\eqno(6)
$$
%\end{equation}
$$\Psi(\alpha,Z)=-b_0Z^2\tilde{\delta}(\alpha),~\tilde{\delta}(\alpha)=\frac{d\delta(\alpha)}{d\alpha},$$
$b_0\ge 0$ --- параметр, $\delta(\alpha)$ --- некоторая гладкая
функция, как систему при отсутствии внешнего поля сил.
%\begin{prop}\label{pr0}
Система (5), (6) имеет два гладких первых интеграла:
%\begin{equation}\nonumber%\label{b.1}
$$
\begin{array}{c}
\Phi_0(v;Z;\alpha)=v^2(1-2b_0Z\delta(\alpha))=C_0=\textrm{const},\\
\Phi_1(v;Z)=vZ=C_1=\textrm{const}.
\end{array}
$$
%\end{equation}
%\end{prop}



Другими словами, независимая подсистема (6) на многообразии
$N^2\{Z;\alpha\}$ имеет рациональный по $Z$ первый интеграл вида
%\begin{equation}\label{b.ir}
$$
\Phi(Z;\alpha)=\frac{1-2b_0Z\delta(\alpha)}{Z^2}=C=\textrm{const},
$$
%\end{equation}
который не имеет существенно особых точек. В силу последнего,
подсистема (6) не имеет притягивающих или отталкивающих предельных
множеств, позволяющих говорить о наличии в системе диссипации того
или иного знака.

Итак, внутреннее силовое поле (зависящее от $b_0>0$) в системе (5),
(6) не нарушает консервативности системы.








Добавляя следующим образом в систему (5), (6) внешнее силовое поле
$F(\alpha)$ при наличии внутреннего ($b_0>0$):
%\begin{equation}\label{b.2-1}
$$
v'=v\Psi(\alpha,Z),\eqno(7)
$$
%\end{equation}
%\begin{equation}\label{b.2-2}
$$
\begin{array}{c}
\alpha'=-Z+b_0Z^2\delta(\alpha),\\
Z'=F(\alpha)-Z\Psi(\alpha,Z),
\end{array}\eqno(8)
$$
%\end{equation}
создаётся впечатление, что система осталась консервативной (что
имеет место при $b_0=0$, т.е. при отсутствии внутреннего поля).
Консервативность ``подтвердилась'' бы наличием в системе двух
гладких первых интегралов.

Действительно, при некотором естественном условии у системы (7), (8)
существует гладкий первый интеграл вида
%\begin{equation}\label{b.in1g}
$$
\Phi_1(v;Z;\alpha)=v^2(Z^2+F_1(\alpha))=C_1=\textrm{const},~\frac{dF_1(\alpha)}{d\alpha}=2F(\alpha),
$$
%\end{equation}
структура которого напоминает интеграл полной энергии. Но
дополнительного гладкого первого интеграла система, вообще говоря,
не имеет.

%\begin{prop}\label{pr00}
Если $F(\alpha)=\delta(\alpha)\tilde{\delta}(\alpha)$, то система
(7), (8) имеет два независимых (один, вообще говоря, трансцендентный
и один гладкий) первых интеграла:
%\begin{equation}\label{b.in0}
$$
\begin{array}{c}
\Phi_0(v;Z;\alpha)=\\
=v^2\left(1-b_0Z\delta(\alpha)-b_0(Z^2+\delta^2(\alpha))\arctan\frac{\delta(\alpha)}{Z}\right)=
\\=C_0=\textrm{const},\\
\end{array}
$$
%\end{equation}
%\begin{equation}\label{b.in1}
$$
\Phi_1(v;Z;\alpha)=v^2(Z^2+\delta^2(\alpha))=C_1=\textrm{const}.
$$
%\end{equation}
%\end{prop}



Более того, как видно из вида предъявленных первых интегралов, притягивающее множество рассматриваемой
\linebreak
системы (7), (8) может быть найдено из системы равенств $Z=\delta(\alpha)=0.$

%В данном случае первый интеграл (\ref{b.in1}) является частным
%случаем интеграла (\ref{b.in1g}).

Модифицируем далее систему (7), (8), при наличии двух ключевых
параметров $b_0, b_1\ge 0$, введя внешнее силовое поле. Получим
систему
%\begin{equation}\label{b.3-1}
$$
v'=v\Psi(\alpha,Z),\eqno(9)
$$
%\end{equation}
%\begin{equation}\label{b.3-2}
$$
\begin{array}{c}
\alpha'=-Z+b_0Z^2\delta(\alpha)+b_1F(\alpha)\tilde{f}(\alpha),\\
Z'=F(\alpha)-Z\Psi(\alpha,Z),
\end{array}\eqno(10)
$$
%\end{equation}
$$\Psi(\alpha,Z)=-b_0Z^2\tilde{\delta}(\alpha)+b_1F(\alpha)\delta(\alpha),~\tilde{f}(\alpha)=\frac{\mu-\delta^2(\alpha)}{\tilde{\delta}(\alpha)},$$
$\mu=\textrm{const}$. Коэффициенты консервативной составляющей
силового поля содержат параметр $b_0$, а неконсервативной
составляющей внешнего поля --- параметр $b_1$.

Только что мы ввели такое поле, добавив коэффициент $F(\alpha)$ в
уравнение на $Z'$ системы (7), (8), и убедились, что полученная
система, вообще говоря, не будет консервативной. Консервативность
будет при дополнительном условии: $b_0=0$. Но мы расширим введение
силового поля, положив $b_1>0$. Рассматриваемая система на прямом
произведении числового луча и касательного расслоения
$TM^1\{Z;\alpha\}$ примет вид (9), (10). Как будет показано далее,
только что было введено диссипативное силовое поле с помощью
унимодулярного преобразования.

%\begin{thm}\label{thm0}
Если выполнено условие
$F(\alpha)=\delta(\alpha)\tilde{\delta}(\alpha)$, то система (9),
(10) обладает полным набором
--- двумя (одним гладким и одним, вообще говоря, трансцендентным)
%первыми
интегралами.
%\end{thm}

%Для нумерации формул используйте, пожалуйста, команду \verb"\eqno".

%\begin{center}
%    \textbf{Электронную версию тезисов необходимо выслать по электронному адресу vzms@mail.ru.}
%\end{center}

% Оформление списка литературы
\smallskip \centerline {\bf Литература} \nopagebreak

1. {\it Шамолин М.В.} Многообразие случаев интегрируемости в динамике
маломерного и многомерного твёрдого тела в неконсервативном поле сил
/ М.~В.~Шамолин // Итоги науки и техники. Сер.~``Современная
математика и её приложения. Тематические обзоры''. --- Т.~125. ---
М.:~ВИНИТИ, 2013. --- С.~5--254.

2. {\it Шамолин М.В.} Новые случаи интегрируемых систем с диссипацией на
касательном расслоении трёхмерного многообразия / М.~В.~Шамолин //
Доклады РАН. --- 2017. --- Т.~477. --- \No~2. --- С.~168--172.

3. {\it Шамолин М.В.} Новые случаи интегрируемых систем с диссипацией на
касательном расслоении четырёхмерного многообразия / М.~В.~Шамолин
// Доклады РАН. --- 2018. --- Т.~479. --- \No~3. --- С.~270--276.

%1. {\it Крейн С.Г.} Линейные дифференциальные уравнения в банаховом пространстве. М.: Наука, 1987. 408 с.

%2. {\it Львовский С. М.} Набор и верстка в системе LATEX. М.: МЦНМО, 2003. — 448 с.
