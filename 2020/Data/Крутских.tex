\vzmstitle{ОБ ОДНОМ СЕМЕЙСТВЕ ОДНОРОДНЫХ ГИПЕРПОВЕРХНОСТЕЙ 4-МЕРНОГО КОМПЛЕКСНОГО ПРОСТРАНСТВА}
\vzmsauthor{Крутских}{В.\,В.}
\vzmsauthor{Лобода}{А.\,В.}
\vzmsinfo{Воронеж; {\it lobvgasu@yandex.ru}}
\vzmscaption

  Согласно [1], любая голоморфно однородная (5-мерная) вещественная гиперповерхность пространства $ \mathbf{C}^3 $,
ассоциированная с 5-мерной нильпотентной алгеброй Ли, либо вырождена по Леви (во всех своих точках), либо
(локально) голоморфно эквивалентна одной из невырожденных квадрик
$$
    Im\, z_3 = |z_1|^2 \pm |z_2|^2.
$$

Ниже показано, что в 4-мерном комплексном пространстве аналогичное утверждение для
вещественных 7-мерных орбит нильпотентных 7-мерных алгебр Ли НЕ верно.

  Для доказательства
рассматривается семейство нильпотентных 7-мерных алгебр Ли, обозначенное в работе [2] через 1357-M и зависящее
 от одного вещественного параметра. Коммутационные соотношения в этом семействе имеют
в некотором базисе вид ($ \lambda \in \mathbf{R}\setminus \{0\} $)
$$
  [e_1, e_2] = e_3, \ [e_1, e_3] = e_5, \ [e_1, e_4] = e_6, \ [e_1, e_5] = e_7, \
\eqno (1)
$$$$
  [e_2, e_4] = e_5, \ [e_2, e_6] = \lambda e_7, \ [e_3, e_4] = ( 1 - \lambda) e_7.
$$

   Авторами построена следующая реализация алгебр этого семейства в виде алгебр Ли голоморфных векторных полей в
 пространстве $ \mathbf{C}^4 $:
$$
   e_1 = i \frac{\partial}{\partial z_1} - z_1 \frac{\partial}{\partial z_2} - z_2 \frac{\partial}{\partial z_3}
        - z_3 \frac{\partial}{\partial z_4}, \
    e_2 = \frac{\partial}{\partial z_1}, \
    e_3 = \frac{\partial}{\partial z_2}, \
$$$$
    e_4 = -i (1 + \lambda) \frac{\partial}{\partial z_2} + z_1 \frac{\partial}{\partial z_3} +
           ( 1 - \lambda) z_2 \frac{\partial}{\partial z_4}, \
\eqno (2)
$$$$
    e_5 =  \frac{\partial}{\partial z_3}, \
    e_6 = -i \lambda \frac{\partial}{\partial z_3} + \lambda z_1 \frac{\partial}{\partial z_4},\
    e_7 =  \frac{\partial}{\partial z_4}.
$$

   При $ \lambda = -1 $ все орбиты такой алгебры в пространстве $ \mathbf{C}^4 $ аффинно
эквивалентны вырожденной по Леви гиперповерхности
$
    y_1 = y_2^2.
$


\textbf{Теорема~1.} {\it
При $ \lambda \ne -1 $ орбитами алгебр из семейства (2) являются (с точностью до аффинных преобразований пространства
$ \mathbf{C}^4 $) алгебраические трубчатые поверхности
$$
    y_4 = y_1 y_3 + A y_2^2 + B y_1^2 y_2 + C y_1^4, \ \mbox{где}
\eqno (3)
$$$$
A =  \frac{1-\lambda}{2(1 + \lambda)}, \
B =  \frac{1}{1 + \lambda}
C =  \frac{1}{4(1 + \lambda)}.
\eqno (4)
$$
}

  При
$ \lambda \ne 1 $ согласованным растяжением переменных уравнение (3)-(4) можно привести к виду
$$
    y_4 = y_1 y_3 + y_2^2 + y_1^2 y_2 + A y_1^4, \quad A = \frac{1-\lambda}8.
\eqno (5)
$$

 Квадратичная форма
$
   y_1 y_3 + y_2^2
$
из правой части этого уравнения превращается в комплексных координатах в знаконеопределенную невырожденную форму Леви
$$
   H(z_1, z_2, z_3) = z_1 \bar z_3 + z_3 \bar z_1 +  |z_2|^2.
$$

Следовательно, при $ \lambda \ne \pm 1 $ все орбиты алгебры (2) являются невырожденными. При этом, согласно [3],
поверхность с уравнением (5) голоморфно эквивалентна соответствующей квадрике
$$
    y_4 = z_1 \bar z_3 + z_3 \bar z_1 +  |z_2|^2
$$
только при $ A = 1/12 $, т.е. при $ \lambda = 1/3 $.





   Тем самым, семейство поверхностей (3) иллюстрирует отличие ситуации в $ \mathbf{C}^4 $ от 3-мерного случая.


\textbf{Замечание 1.} При $ \lambda = 1 $ поверхность (3)-(4), т.е.
$$
    y_4 = y_1 (y_3  + \frac12 y_1 y_2 + \frac18 y_1^3).
\eqno (6)
$$
является вырожденной по Леви.

\textbf{Замечание 2.} Любая поверхность из семейства (5) допускает согласованное растяжение переменных, сохраняющее как поверхность, так и начало координат в $ \mathbf{C}^4 $, через которое она проходит. Это означает, что алгебра с базисом (2) является подалгеброй 8-мерной алгебры, дополнительным базисным полем которой является
$$
    e_8 =
    z_1 \frac{\partial}{\partial z_1}+ 2 z_2 \frac{\partial}{\partial z_2} +
    3 z_3 \frac{\partial}{\partial z_3}+ 4 z_4 \frac{\partial}{\partial z_4}.
$$

\textbf{Замечание 3.} Рассмотренное семейство 1357-M является одним из девяти однопараметрических семейств 7-мер\-ных нильпотентных алгебр Ли.
С использованием техники работы [4] авторами показано, что два других семейства из этих девяти (а именно, 12457-N и 123457-I) могут иметь лишь вырожденные по Леви орбиты в пространстве $ \mathbf{C}^4 $.


% Оформление списка литературы
\smallskip \centerline {\bf Литература} \nopagebreak

1. {\it Акопян Р.С., Лобода А.В.} О голоморфных реализациях нильпотентных алгебр Ли. Функц. анализ и его прил., 53:2 (2019),  59–63.

2. {\it Ming-Peng Gong.} Classification of Nilpotent Lie Algebras of Dimension 7. Waterloo, Canada, 1998.
www.semanticscholar.
\newline
org/paper/f72dbfc64f72f7b3d9a740c77181ae2186d58e22

3. {\it Исаев А.В., Мищенко М. А.} Классификация сферических трубчатых гиперповерхностей, имеющих в сигнатуре формы Леви один минус.
Изв. АН СССР. Сер. матем., 52:6 (1988),  1123–1153.

4. {\it Beloshapka V. K., Kossovskiy I. G.} Homogeneous hyper\-surfaces in $ \mathbf{C}^3 $, associated
with a model CR-cubic, J. Geom. Anal., 20:3 (2010), 538–564.
