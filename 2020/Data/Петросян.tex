\begin{center}
    {\bf О РАЗРЕШИМОСТИ ПЕРИОДИЧЕСКОЙ ЗАДАЧИ ДЛЯ ПОЛУЛИНЕЙНОГО ДИФФЕРЕНЦИАЛЬНОГО ВКЛЮЧЕНИЯ ДРОБНОГО ПОРЯДКА С ОТКЛОНЯЮЩИМСЯ АРГУМЕНТОМ \footnote{Исследование выполнено при финансовой поддержке РФФИ в рамках научного проекта № 19-31-60011.}}\\

    {\it Г.Г. Петросян}

    (Воронеж, ВГУИТ, ВГПУ; {\it garikpetrosyan@yandex.ru})

\end{center}

\addcontentsline{toc}{section}{Петросян Г.Г.}

В работе для полулинейного функционально-диф\-фе\-рен\-ци\-ального включения в сепарабельном банаховом пространстве $E$ следующего вида
$$
^CD^{q}x(t)\in Ax(t)+ F(t,x_{t}),\ t\in[0,T], \eqno(1)
$$
мы исследуем задачу существования интегральных решений удовлетворяющих периодическому краевому условию
$$
x_{0}=x_{T}. \eqno(2)
$$
Символом $^CD^{q}$ обозначается дробная производная Капуто порядка $q \in (0,1),$ $x_{t}$ - предыстория функции $x(t)$ до момента времени $t\in [0,T],$ то есть $x_{t}(s)=x(t+s), s\in [-h,0], 0<h<T.$ $A$ - линейный замкнутый оператор в $E$ удовлетворяющий условию:

$(A)$ $A:D(A) \subseteq E\rightarrow E$ порождает ограниченную $C_{0}$--полугруппу $\left\{U(t)\right\}_{t\geq 0}$ линейных операторов в $E$.

Мы полагаем, что многозначное нелинейное отображение $ F:[0,T]\times C([-h,0];E)\to Kv(E),$ где $Kv(E)$ - совокупность всех непустых выпуклых компактных подмножеств $E,$ подчиняется следующим условиям:

$(F1)$ для каждого $\xi \in C([-h,0];E)$ мультифункция $F\left(\cdot ,\xi \right): \left[0,T\right]\rightarrow Kv\left(E\right) $ допускает измеримое сечение;

$(F2)$ для п.в. $t\in[0,T]$ мультиоператор $F(t,\cdot):E\rightarrow Kv\left( E \right) $ полунепрерывен сверху;

$(F3)$ существует функция $\alpha\in L^\infty_+ ([0,T])$ такая, что
$$
\left\|F(t, x_{t})\right\|_E\leq\alpha(t)(1+\left\|x_{t}\right\|_{C([-h,0];E)})\,\, \mbox{для п.в.} \,\, t\in[0,T];
$$

$(F4)$ существует функция $\mu \in L^{\infty}([0,T])$ такая, что для каждого ограниченого множества $\Delta\subset C([-h,0];E):$
$$ \chi(F(t,\Delta)) \leq \mu(t)\varphi(\Delta),$$
для п.в. $ t \in [0,T],$ где $\varphi(\Delta)=\sup_{s\in [-h,0]}\chi(\Delta(s)),$ $\chi$ мера некомпактности Хаусдорфа в $E,$ $\Delta(s)=\left\{y(s): y\in \Delta\right\}.$

\textbf{Теорема~1.} {\it При выполнении условий $(A),\ (F1) - (F4),$ и дополнительного условия}

$((A1))$ {\it полугруппа $U$ экспоненциально убывающая, то есть
$$\left\|U(t)\right\|\leq e^{-\eta t}, \quad t \geq 0,$$
для некоторого числа $\eta > 0$.}

{\it Если $\frac{k}{\eta} < 1,$ где $k=\max\left\{\|\alpha\|_\infty, \left\|\mu\right\|_{\infty}\right\},$ то задача (1)-(2) имеет решения.}


% Оформление списка литературы
\smallskip \centerline {\bf Литература} \nopagebreak

1. {\it Борисович Ю.Г.} Введение в теорию многозначных ото\-бражений и дифференциальных включений / Ю.Г. Борисович, Б.Д. Гельман, А.Д. Мышкис, В.В. Обуховский. - М.: Книжный дом <<Либроком>>, 2011. - 224 С.

\selectlanguage{english}

2. {\it Kamenskii M.} Condensing Multivalued Maps and Semilinear Differential Inclusions in Banach Spaces / M. Kamenskii, V. Obukhovskii, P. Zecca. --- Berlin--New-York: de Gruyter Series in Nonlinear Analysis and Applications, Walter de Gruyter, 2001. --- 231 P.

3. {\it Kamenskii M.}  Boundary value problems for semilinear differential inclusions of fractional order in a Banach space / M.  Kamenskii, V. Obukhoskii, G. Petrosyan, J.-C. Yao //  Applicable Analysis. - 2017. - Vol. 96. №4.- P. 571-591.

4. {\it Kamenskii M.}  On approximate solutions for a class of semilinear fractional-order differential equations in Banach spa\-ces / M.I. Kamenskii, V.V. Obukhoskii, G.G. Petrosyan, J.C. Yao // Fixed Point Theory and Applications. - 2017. - Vol. 28. №4. -  P. 1-28.

5. {\it Kamenskii M.}  Existence and Approximation of Solutions to Nonlocal Boundary Value Problems for  Fractional Differential Inclusions / M.I. Kamenskii, V.V. Obukhoskii, G.G. Petrosyan, J.C. Yao // Fixed Point Theory and Applications. - 2019. - Vol. 30. №2.

6. {\it Kamenskii M.I.} The Semidiscretization method for differential inclusions of fractional order / M.I. Kamenskii, V.V. Obukhoskii, G.G. Petrosyan // \selectlanguage{russian} Вестник Тамбовского университета. Серия: Естественные и технические науки. - 2018. - Т. 23. №122. -  С. 125-130.

7. {\it Афанасова М.С.} О краевой задаче для функционально"=дифференциального включения дробного порядка с общим начальным условием  в банаховом пространстве / \linebreak М.С.~Афанасова, Г.Г. Петросян // Известия вузов. Математика. – 2019 .- №9. – С. 3-15.

8. {\it Петросян Г.Г.} О формальном представлении решений дифференциальных уравнений дробного порядка / Г.Г. Петросян // Вестник Тамбовского университета. Серия: Естественные и технические науки. -  2018. - Т. 23. №123. - С. 524-530.

9. {\it Петросян Г.Г.} Об одной задаче управляемости для дифференциального включения с дробной производной Капуто  / Г.Г. Петросян, О.Ю. Королева // Вестник Тамбовского университета. Серия: Естественные и технические науки. - 2018. - Т. 23. №124. - С. 679-684.











