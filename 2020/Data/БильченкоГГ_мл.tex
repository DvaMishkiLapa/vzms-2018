\vzmstitle{О ВЛИЯНИИ ЛИНЕЙНО УБЫВАЮЩЕГО ВДУВА И ЛИНЕЙНОГО ТЕМПЕРАТУРНОГО ФАКТОРА НА ОБЛАСТЬ ЗНАЧЕНИЙ ФУНКЦИОНАЛОВ ГИПЕРЗВУКОВОЙ АЭРОДИНАМИКИ}
\vzmsauthor{Бильченко (мл.)}{Г.\,Г.}
\vzmsauthor{Бильченко}{Н.\,Г.}
\vzmsinfo{Казань; {\it ggbil2@gmail.com}; {\it bilchnat@gmail.com}}
\vzmscaption

%% ===============================
%%  Докладчик:
%%
%%  Бильченко
%%  Григорий  (мл.)
%%  Григорьевич
%%
%%  "О  влиянии  линейно  убывающего  вдува
%%  и  линейного  температурного  фактора
%%  на  область  значений  функционалов
%%  гиперзвуковой  аэродинамики"
%%
%%  к.ф.-м.н.
%%
%%  Казанский национальный исследовательский технический университет им. А.Н.Туполева-КАИ
%%
%%  Секция (направление) / Section:  Нелинейный анализ и математическое моделирование
%% ===============================
%%  Содокладчик:
%%
%%  Бильченко
%%  Наталья
%%  Григорьевна
%%
%%  к.ф.-м.н.
%%
%%  Казанский национальный исследовательский технический университет им. А.Н.Туполева-КАИ
%% ===============================



%% ==============================
%% ==============================
    В  данной  работе,
сохраняющей  все  обозначения
и  сокращения  работы
%%%
[1]
%%%
и
продолжающей  исследование  свойств
полученной
с  помощью  метода
обобщённых  интегральных  соотношений
А.~А.~Дородницына
%%%
математической  модели
ЛПС
электропроводящего  газа
на  проницаемых  цилиндрических
и  сферических  поверхностях
ГЛА,
%%%
рассматривается  влияние
следующего  сочетания
управляющих  воздействий:
\textbf{линейно  убывающего}
вдува,
\textbf{линейного}
температурного  фактора
и
\textbf{постоянного}
магнитного  поля
на
интегральные
характеристики
тепломассообмена  и  трения
и
суммарную  мощность
системы,
обеспечивающей  вдув.
%%%% ====================================



%% ==============================
%% ==============================
\textbf{1.  Вычислительные  эксперименты.}
%% ==============================
Результаты
вычислительных  экспериментов,
выполненных
в  условиях
%%% ====
(5){\textbf{--}}(10)
%%% ====
[1]
%%% ====
по  схеме
%%% ====
[7{\textbf{--}}11]
%%% ====
из
%%% ====
[1]
%%% ====
для  управлений
%%% ====
(19),  (20)
%%% ====
[1]
%%% ====
при
$m_{0\,},
m_{1} \,{\in}\,  M^{d}_{25\,}$,
%%% ====
$\tau_{0\,},
\tau_{1} \,{\in}\,  T^{d}_{15\,}$,
%%% ====
$s\equiv  0$,
%%% ====
для  случаев
%%% ====
[2{\textbf{--}}4]
%% =============================== \\
\begingroup\belowdisplayskip=\belowdisplayshortskip
\[
(m^{\prime}\,{=}\,{}-0{,}25;
\tau^{\prime}\,{=}\,{}+ 0{,}15),
\quad
(m^{\prime}\,{=}\,{}-0{,}25;
\tau^{\prime}\,{=}\,{}- 0{,}15),
\]
\endgroup
%% =============================== *
\[
(m^{\prime}\,{=}\,{}-0{,}25;
\tau^{\prime}\,{=}\,{}+ 0{,}45),
\quad
(m^{\prime}\,{=}\,{}-0{,}25;
\tau^{\prime}\,{=}\,{}- 0{,}45),
\]
%% =============================== *
\[
(m^{\prime}\,{=}\,{}-0{,}50;
\tau^{\prime}\,{=}\,{}+ 0{,}15),
\quad
(m^{\prime}\,{=}\,{}-0{,}50;
\tau^{\prime}\,{=}\,{}- 0{,}15)
\hphantom{,}
\]
%% =============================== //
для  воздуха  в  атмосфере  Земли
при
%% =============================== \
$H = 10$~%
$[\text{км}]$,
%% =============================== *
$M_{\infty} =  10$,
%% =============================== *
$R  = 0{,}1$~%
$[\text{м}]$,
%% =============================== /
представлены
(символ  ``$\diamond$'')
на
рис.~1{\textbf{--}}6.
%%%



%% ==============================
%% ==============================
\textbf{2.  Замечания.}
%% ==============================
%% ==============================
1)  Графики
параметров
%%%% ================ \\
$\theta_{0}\left(x\right),
\ldots,
\omega_{1}\left(x\right)%
$\,
%%%% ================ //
математической  модели  ЛПС
представлены  в
%%% ====
[2,
 3],
%%% ====
а
локальных  зависимостей
$q(x)$
и
$f(x)$~%
{\textbf{--}}
в
%%% ====
[2,
 4].
%%% ====
%%%%%%%%%%%%%%%%%%%%%%%%%
%%% Page  1 / Page  2 %%%
%%%%%%%%%%%%%%%%%%%%%%%%%



%% =============================== \\
%\begin{figure}[h!]
\begin{center}
\includegraphics[width=0.95\textwidth]%
{%
Bilchenko_GGjr_NG_2_fig_1_n25p15.eps%
}\\
\emph{Рис.~1.}~%
{Случай
$m^{\prime}\,{=}\,{}-0{,}25$,
$\tau^{\prime}\,{=}\,{}+0{,}15$,
$s\equiv  0$%
}%
\end{center}
%\end{figure}
%% =============================== **



%% =============================== **
%\begin{figure}[h!]
\begin{center}
\includegraphics[width=0.95\textwidth]%
{%
Bilchenko_GGjr_NG_2_fig_2_n25n15.eps%
}\\
\emph{Рис.~2.}~%
{Случай
$m^{\prime}\,{=}\,{}-0{,}25$,
$\tau^{\prime}\,{=}\,{}-0{,}15$,
$s\equiv  0$%
}%
\end{center}
%\end{figure}
%% =============================== //



%% =============================== \\
%\begin{figure}[h!]
\begin{center}
\includegraphics[width=0.95\textwidth]%
{%
Bilchenko_GGjr_NG_2_fig_3_n25p45.eps%
}\\
\emph{Рис.~3.}~%
{Случай
$m^{\prime}\,{=}\,{}-0{,}25$,
$\tau^{\prime}\,{=}\,{}+0{,}45$,
$s\equiv  0$%
}%
\end{center}
%\end{figure}
%% =============================== **



%% ======
2)  Результаты
вычислительных  экспериментов,
а  также  результаты
%%% ===
[1,
 5,
 6]
%%% ===
могут  быть  использованы
в  качестве
моделей  ограничений
%%% ===
(33){\textbf{--}}(35)
%%% ===
[7]
%%% ===
и
%%% ===
$(45_{1})${\textbf{--}}$(45_{r})$
%%% ===
[8]
%%% ===
в  задачах
%%%%%%%%%%%%%%%%%%%%%%%%%
%%% Page  2 / Page  3 %%%
%%%%%%%%%%%%%%%%%%%%%%%%%
синтеза  эффективного  управления,
как  на  всём  участке,
так  и  на  его  фрагментах.
%% ======



%% =============================== **
\begin{center}
\includegraphics[width=0.95\textwidth]%
{%
Bilchenko_GGjr_NG_2_fig_4_n25n45.eps%
}\\
\emph{Рис.~4.}~%
{Случай
$m^{\prime}\,{=}\,{}-0{,}25$,
$\tau^{\prime}\,{=}\,{}-0{,}45$,
$s\equiv  0$%
}%
\end{center}
%\end{figure}
%% =============================== //



%% =============================== \\
%\begin{figure}[h!]
\begin{center}
\includegraphics[width=0.95\textwidth]%
{%
Bilchenko_GGjr_NG_2_fig_5_n50p15.eps%
}\\
\emph{Рис.~5.}~%
{Случай
$m^{\prime}\,{=}\,{}-0{,}50$,
$\tau^{\prime}\,{=}\,{}+0{,}15$,
$s\equiv  0$%
}%
\end{center}
%\end{figure}
%% =============================== **



%% =============================== **
%\begin{figure}[h!]
\begin{center}
\includegraphics[width=0.95\textwidth]%
{%
Bilchenko_GGjr_NG_2_fig_6_n50n15.eps%
}\\
\emph{Рис.~6.}~%
{Случай
$m^{\prime}\,{=}\,{}-0{,}50$,
$\tau^{\prime}\,{=}\,{}-0{,}15$,
$s\equiv  0$%
}%
\end{center}
%\end{figure}
%% =============================== //
%%%%%%%%%%%%%%%%%%%%%%%%%
%%% Page  3 / Page  4 %%%
%%%%%%%%%%%%%%%%%%%%%%%%%



3)  Кроме  представленных
(здесь  и  в
[1])
зависимостей
$Q$,  $F$,
исследовано   влияние
предложенных  сочетаний
$m$  и  $\tau$
на
$Q$,  $F$,  $N$
при  различных  постоянных  значениях
$s \in  S^{d}_{25}$
магнитного  поля.



%% ===============================
% Оформление списка литературы
\litlist
%% ===============================



%% ============================== \\
1.~%
\textit%
{Бильченко~Г.~Г.,
 Бильченко~Н.~Г.~}
{%
  {О  влиянии  линейно  возрастающего  вдува
и  линейного  температурного  фактора
на  область  значений  функционалов
гиперзвуковой  аэродинамики}%
%	~/
%	  {Г.\,Г.\,Биль\-чен\-ко,
%	   Н.\,Г.\,Биль\-чен\-ко}%
~/$\!$/
  <<Воронежская  зимняя  математическая  школа
  С.~Г.~Крейна~{\textbf{--}}  2020>>,
  посвящённая  100-летию
  М.~А.~Красносельского:
  Материалы  международной  конференции
  (27{\textbf{--}}30
  января  2020~г.).~{\textbf{---}}
  Воронеж:  ИПЦ  <<Научная  книга>>,
  2020.%~{\textbf{---}}
  %С.~???{\textbf{--}}???.%
  }
%% ============================== //



%% ============================== \\
2.~%
\textit%
{Бильченко~Г.~Г.,
 Бильченко~Н.~Г.~}
{%
  {Анализ  влияния
линейно  убывающего
вдува
и
линейно  возрастающего
температурного  фактора
на  параметры  математической  модели
и
локальные  характеристики
тепломассообмена  и  трения
на  проницаемых  поверхностях  ГЛА}%
%	~/
%	  {Г.\,Г.\,Биль\-чен\-ко,
%	   Н.\,Г.\,Биль\-чен\-ко}%
~/$\!$/
  Вестник  Воронеж.  гос.  ун-та.
  Сер.  Системный  анализ
  и  информационные  технологии.~{\textbf{---}}
  2019.~{\textbf{---}}
  \No~4.~{\textbf{---}}
  С.~13{\textbf{--}}20.%
  }
%% ============================== //



%% ============================== \\
3.~%
\textit%
{Бильченко~Г.~Г.,
 Бильченко~Н.~Г.~}
{%
  {О  влиянии
линейно  убывающего
вдува  и
линейно  убывающего
температурного  фактора
на  локальные  характеристики
тепломассообмена  и 	трения
на  проницаемых  поверхностях  ГЛА}%
%	~/
%	  {Г.\,Г.\,Биль\-чен\-ко,
%	   Н.\,Г.\,Биль\-чен\-ко}%
~/$\!$/
  <<Некоторые  актуальные  проблемы
  современной  математики
  и  математического  образования.
    Герценовские  чтения{\textbf{--}}2019>>:
  Материалы  научной  конференции,
  8{\textbf{--}}12
  апреля  2019~г.~{\textbf{---}}
  СПб.:  Изд.  \mbox{РГПУ}  им.  А.~И.~Герцена,
  2019.~{\textbf{---}}
  С.~33{\textbf{--}}38.%
  }
%% ============================== //



%% ============================== \\
4.~%
\textit%
{Бильченко~Г.~Г.,
 Бильченко~Н.~Г.~}
{%
  {О  влиянии
линейно  убывающего
вдува  и
линейно  убывающего
температурного  фактора
на  параметры  математической  модели
на  проницаемых  поверхностях  ГЛА}%
%	~/
%	  {Г.\,Г.\,Биль\-чен\-ко,
%	   Н.\,Г.\,Биль\-чен\-ко}%
~/$\!$/
  Современные  методы  прикладной  математики,
  теории  управления  и  компьютерных  технологий:
  Сборник  трудов
  XII  Международной  научной
%%%%%%%%%%%%%%%%%%%%%%%%%
%%% Page  4 / Page  5 %%%
%%%%%%%%%%%%%%%%%%%%%%%%%
  конференции
  <<\mbox{ПМТУКТ}{\textbf{--}}2019>>,
  Воронеж,
  25{\textbf{--}}28
  сентября  2019~г.%
~{\textbf{---}}
  Воронеж:  ВГУИТ,
  2019.~{\textbf{---}}
  С.~87{\textbf{--}}90.%
  }
%% ============================== //



%% ============================== \\
5.~%
\textit%
{Бильченко~Г.~Г.,
 Бильченко~Н.~Г.~}
{%
  {О  влиянии  линейного  вдува
и  постоянного  температурного  фактора
на  интегральные  характеристики  тепломассообмена
и  трения
на  проницаемых  поверхностях
ГЛА}%
%	~/
%	  {Г.\,Г.\,Биль\-чен\-ко,
%	   Н.\,Г.\,Биль\-чен\-ко}%
~/$\!$/
  <<Актуальные  проблемы
   прикладной  математики,  информатики
   и  механики>>:
  Сборник  трудов  Международной
  научно"=технической  конференции,
  Воронеж,
  11{\textbf{--}}13
  ноября  2019~г.~{\textbf{---}}
  Воронеж:
  Изд-во  <<Научно"=исследовательские  публикации>>,
  2019.%~{\textbf{---}}
  %С.~???{\textbf{--}}???.%
  }
%% ============================== //



%% ============================== \\
6.~%
\textit%
{Бильченко~Г.~Г.,
 Бильченко~Н.~Г.~}
{%
  {О  влиянии  линейного  температурного  фактора
и  постоянного  вдува
на  интегральные  характеристики  тепломассообмена
и  трения
на  проницаемых  поверхностях
ГЛА}%
%	~/
%	  {Г.\,Г.\,Биль\-чен\-ко,
%	   Н.\,Г.\,Биль\-чен\-ко}%
~/$\!$/
  <<Актуальные  проблемы
   прикладной  математики,  информатики
   и  механики>>:
  Сборник  трудов  Международной
  научно"=технической  конференции,
  Воронеж,
  11{\textbf{--}}13
  ноября  2019~г.~{\textbf{---}}
  Воронеж:
  Изд-во  <<Научно"=исследовательские  публикации>>,
  2019.%~{\textbf{---}}
  %С.~???{\textbf{--}}???.%
  }
%% ============================== //



%% ============================== \\
7.~%
\textit%
{Бильченко~Г.~Г.,
 Бильченко~Н.~Г.~}
{%
  {Обратные  задачи  тепломассообмена
   на  проницаемых  поверхностях
   гиперзвуковых  летательных  аппаратов.
  IV.  Классификация  задач
   на  всём  участке  управления}%
%	~/
%	  {Г.\,Г.\,Бильченко,
%	   Н.\,Г.\,Бильченко}%
~/$\!$/
  Вестник  Воронеж.  гос.  ун-та.
  Сер.  Системный  анализ
  и  информационные  технологии.~{\textbf{---}}
  2018.~{\textbf{---}}
  \No~3.~{\textbf{---}}
  С.~5{\textbf{--}}12.%
  }
%% ============================== //



%% ============================== \\
8.~%
\textit%
{Бильченко~Г.~Г.,
 Бильченко~Н.~Г.~}
{%
  {Обратные  задачи  тепломассообмена
   на  проницаемых  поверхностях
   гиперзвуковых  летательных  аппаратов.
   V.  Смешанные  задачи
   на  фрагментах  участка  управления}%
%	~/
%	  {Г.\,Г.\,Биль\-чен\-ко,
%	   Н.\,Г.\,Биль\-чен\-ко}%
~/$\!$/
  Вестник  Воронеж.  гос.  ун-та.
  Сер.  Системный  анализ
  и  информационные  технологии.~{\textbf{---}}
  2018.~{\textbf{---}}
  \No~3.~{\textbf{---}}
  С.~13{\textbf{--}}22.%
  }
%% ============================== //
%%%%%%%%%%%%%%%%%%%%%%%%%
%%% Page  5 / ...     %%%
%%%%%%%%%%%%%%%%%%%%%%%%%



%%%%%%%%%%%%%%%%%%%%%%%%%
