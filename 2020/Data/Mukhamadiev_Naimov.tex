

\begin{center}
    {\bf О КОЭРЦИТИВНОЙ ОЦЕНКЕ ОДНОГО НЕЛИНЕЙНОГО
    ДИФФЕРЕНЦИАЛЬНОГО ОПЕРАТОРА}

    {\it Э. Мухамадиев, А. Н. Наимов}

    (Вологда; {\it emuhamadiev@rambler.ru, nan67@rambler.ru})
\end{center}

\addcontentsline{toc}{section}{Мухамадиев Э., Наимов А. Н.}

Рассмотрим дифференциальный оператор
$$
L_{\lambda}x(t)\equiv x'(t)-A(t,\lambda)|x(t)|^{m-1}x(t)
$$
в пространстве $C^1\left(0, \omega; R^n \right)$, где $\omega>0$,
$n, m>1$, $A(t,\lambda)$ - квадратная матрица-функция, непрерывная
по совокупности переменных $(t,\lambda)\in [0, \omega]\times [0,
1]$ и $\omega$-периодическая по $t$. Исследуем вопрос о
коэрцитивной оценке  вида
$$
||L_{\lambda}x||+|x(0)-x(\omega)|^m \geq \sigma ||x||^m \eqno{(1)}
$$
при $||x||\geq M$, где положительные числа $M$ и $\sigma $ не
зависят от  $x(t)$ и $\lambda$.

Имеет место следующая теорема.

\textbf{Теорема~1.} {\it Если в любой точке $(t,\lambda)\in [0,
\omega]\times [0, 1]$ матрица $A(t,\lambda)$ не имеет чисто мнимых
собственных значений, то верна оценка (1).}

Используя оценку (1) можно исследовать разрешимость периодической
задачи
$$
L_{\lambda}x(t)=f(t,x(t)), \quad 0<t<\omega, \quad x(0)=x(\omega).
\eqno{(2)}
$$
Здесь $f: [0, \omega]\times R^n \mapsto R^n$ - непрерывное
отображение, $\omega$-периодическое по $t$ и удовлетворяющее
условию
$$
\max_{0\leq t\leq\omega}|f(t,y)||y|^{-m} \rightarrow 0 \quad
\mbox{ при } \quad |y|\rightarrow\infty. \eqno{(3)}
$$

Из теоремы 1 вытекает, что для решений периодической задачи (2)
имеет место априорная оценка
$$
||x||<M_1,
$$
где $M_1$ не зависит от $x$ и $\lambda$. Следовательно, определено
вращение $\gamma(\Phi_{\lambda},S_r)$ вполне непрерывного
векторного поля
$$
\Phi_{\lambda}x\equiv
x(t)-x(\omega)-\int_0^{t}(L_{\lambda}x(s)-f(s,x(s)))ds
$$
на сферах  $S_r=\{x: ||x||=r\}$ радиуса $r\geq M_1$ (см., напр.,
[1]), при этом $\gamma(\Phi_{\lambda},S_r)$ не зависит от
$\lambda$ и $r$. К тому же,
$\gamma(\Phi_{\lambda},S_r)=\gamma(\Psi_0,S_r)$, где
$$
\Psi_0x\equiv x(t)-x(\omega)-\int_0^{t}L_{0}x(s)ds.
$$
Векторное поле $\Psi_0$ нечётно, поэтому $\gamma(\Psi_0,S_r)\neq
0$ [1]. Отсюда, применяя принцип ненулевого вращения получаем
следующую теорему.


\textbf{Теорема~2.} {\it Пусть выполнено условие теоремы 1. Тогда
задача (2) разрешима при любых $\lambda\in [0, 1]$ и $f$,
удовлетворяющем условию (3).}



\smallskip \centerline {\bf Литература} \nopagebreak

1. {\it Красносельский М. А., Забрейко П. П.} Геометрические
методы нелинейного анализа. М.: Наука. 1975. 512~с.



