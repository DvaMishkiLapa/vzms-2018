\begin{center}
    {\bf  КУЛЬТУРНО-ИСТОРИЧЕСКИЙ ДИСКУРС КАК МЕХАНИЗМ ТРАНСЛЯЦИИ ЦЕННОСТНОГО СОДЕРЖАНИЯ МАТЕМАТИЧЕСКИХ ПОНЯТИЙ}

    {\it Л.В. Жук}

    (Елец, {\it krasnikovalarisa@yandex.ru})
\end{center}

\addcontentsline{toc}{section}{Жук Л.В.}



Актуализируется проблема реализации психодидактического подхода к организации процесса обучения математике в вузе. В качестве эффективного средства достижения личностно-ориентированных целей математического образования рассматривается культурно-исторический дискурс. Представлены примеры интеграции конкретно-историчес-\\кого содержания в практику обучения бакалавров.




{\bf Ключевые слова:} психодидактическая парадигма образования, культурно-исторический дискурс, диалектика \\математики, коммуника\-ция, .


Отличительной особенностью математики как учебной дисциплины является высокая степень абстрактности объектов изучения. Наиболее ярко указанная специфика проявляется при обучении математике в вузе: освоение будущими бакалаврами понятий высшей алгебры, математического анализа, геометрии предполагает овладение теоретическими способами мышления через диалектическое восхождение от абстрактного к конкретному. При этом зачастую математическая теория представляется в виде списка определений, теорем и выводимых из них утверждений, подкрепляется набором упражнений и задач, ориентированных преимущественно на формально-логические действия с математическими объектами в стандартных ситуациях. Описанный формально-дедуктивный подход, традиционный для выс-\\шей школы, препятствует психическому развитию личности обучающихся в отношении таких качеств, как поисковая активность, креативность, достижению ими уровня понимания.

Одним из содержательных резервов реализации лично\-стно-ориентированных целей математического образования бакалавров является культурно-исторический дискурс. Развивающий потенциал культурно-исторического дискурса заключается в формировании у обучающихся интереса к учению, представления об исторически взаимообусловленном единстве математики с другими составляющими духовной культуры, воспитании эстетического восприятия математики.

Культурно-исторический дискурс в математическом образовании предполагает использование математических \\суждений в адекватной целям математического образования интерпретации. Грамотное построение дискурса в процессе коммуникации отражает философское осмысление ос\-ваиваемых понятий, способов деятельности и культурных норм. Приведём некоторые примеры введения культурно-исторического дискурса в процесс обучения математике в вузе.

При изучении темы «Аффинная система координат на плоскости» мы формируем у будущих бакалавров представление о том, что идея координат возникла ещё у древних греков (Архимед, Аполлоний Пергский), однако её развитию помешал невысокий уровень древнегреческой алгебры и слабый интерес к кривым, отличным от прямой и окружности. Позднее, в 14 веке, Николай Орезмский использовал координатное изображение для функции, зависящей от времени, он называл координаты по аналогии с географическими - долготой и широтой. К этому времени развитое понятие о координатах уже существовало в астрономии и географии. Около 1637 года Ферма распространяет мемуар «Введение в изучение плоских и телесных мест», где выписывает и обсуждает уравнения различных кривых 2-го порядка в прямоугольных координатах, наглядно показывая, насколько новый подход проще и плодотворней чисто геометрического. Однако гораздо большее влияние имела «Геометрия» Декарта, вышедшая в том же 1637 году, которая независимо и гораздо более полно развивала те же идеи. Декарт включил в геометрию более широкий класс кривых, в том числе «механические» (трансцендентные, вроде спирали), построил их уравнения и провёл классификацию.

При изучении линий второго порядка студенты с интересом узнают, что эллипс, парабола и гипербола изначально были получены сечением прямого кругового конуса плоскостями, не проходящими через его вершину. Открывателем конических сечений считается Менехм (4 в. до н. э.), ученик Платона. Он использовал параболу и равнобочную гиперболу для решения задачи об удвоении куба. В свою очередь, Аполлоний Пергский (ок. 260-170 гг. до н.э.) в знаменитом трактате «Конические сечения», варьируя угол наклона секущей плоскости, получил все конические сечения, ему мы обязаны и современными их названиями.

Согласно типологии, предложенной Лео Роджерсом, \\можно выделить пять подходов к организации культурно-исторического дискурса в процессе обучения математике:

1) эмпирический подход - хронологическая реконструкция прошлого как исторического прогресса математических идей;

2) социально-культурный подход - рассмотрение истории математики в контексте общественного (экономического, политического, культурного, технологического) развития;

3) личностно-ориентированный подход - анализ индивидуальных творческих процессов, логики математических открытий, психологии изобретений;

4) концептуальный подход - реконструкция математических теорий прошлого в соответствии с современным состоянием математики;

5) герменевтический подход - вовлечение в реконструкцию математического знания уточнённых исторических текстов.

При выборе преподавателем любого из перечисленных подходов культурно-исторический контекст должен быть \\не\-посредственно привязан к математическому образовате\-льному контексту, то есть к конкретному математическому содержанию. В связи с этим актуален вопрос о разработке учебных пособий по различным разделам высшей математики, интегрирующих содержание культурно-исторического дискурса. Среди отечественных разработок, реализующих практику математического культурно-исторического дис-\\курса, следует отметить элективный курс «Введение в алгебру и анализ: культурно-исторический дискурс», предложенный А.Н. Земляковым [2]. В данном курсе ознакомление старшеклассников с основными линиями развития математического знания - числами, уравнениями, функциями, множествами – осуществляется в конкретно-историческом контексте: учащиеся знакомятся с историей развития алгебры и предысторией математического анализа, получают представления о взаимосвязях математики с другими науками и практикой.

Поводя итог, отметим, что интеграция культурно-исто\-ри\-ческого дискурса в процесс математического образования бакалавров способствует демонстрации целесообразности построения и развития математических теорий, даёт возможность раскрыть диалектику математики, показать её как развивающуюся систему взаимосвязанных теорий и различных подходов к решению конкретных задач в их зарождении, представить математику как органичную и неотъемлемую часть системы научного познания мира, раскрыть роль личности математиков и математических сообществ в становлении человеческой цивилизации, сформировать личностное, психологическое отношение к предмету математики.



% Оформление списка литературы
\smallskip \centerline {\bf Литература} \nopagebreak


%\bibitem{Keld}
1. {\it Арутюнова Н.Д.} Дискурс. Речь // Лингвистический энциклопедический словарь. – М.: Научное издательство \\«Большая Российская энциклопедия», 2002. – С. 136–137.

%\bibitem{Kipr}
2. {\it Земляков А.Н.} Введение в алгебру и анализ: культур\-но-исторический дискурс. Элективный курс: Учебное пособие/ Земляков А.Н. – М.: Бином. Лаборатория знаний, 2007. – 320с.

%\bibitem{L1}
3. {\it Земляков А.Н.} Психодидактические аспекты углублённого изучения математики в старших классах общеобразовательной средней школы // Учебно-методическая газета «Математика». «Первое сентября». –№6. –2005. – С. 17-21.
