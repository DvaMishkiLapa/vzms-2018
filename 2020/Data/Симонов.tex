\vzmstitle[\footnote{Работа выполнена при финансовой поддержке РФФИ (проект \No\ 18-01-00332 A).}]{ОБ УСТОЙЧИВОСТИ СИСТЕМЫ ДВУХ ЛИНЕЙНЫХ ГИБРИДНЫХ ФУНКЦИОНАЛЬНО-ДИФФЕРЕНЦИАЛЬНЫХ СИСТЕМ С ПОСЛЕДЕЙСТВИЕМ (ЛГФДСП)}
\vzmsauthor{Симонов}{П.\,М.}
\vzmsinfo{Пермь; {\it simpm@mail.ru}}
\vzmscaption

Постановка задачи: одно уравнение "--- линейное разностное, определённое в дискретном множестве точек (линейное разностное уравнение с последействием (ЛРУП)), а другое "--- линейное функционально"=дифференциальное уравнение с последействием (ЛФДУП) на полуоси.

Исследование продолжает работы [1--7].

Обозначим через $y=\{ y(-1),y(0),y(1),...,y(N),\ldots \}$
бесконечную матрицу со столбцами $y(-1),y(0),y(1),...,y(N),\ldots $ размерами $n$, а через $g=\{ g(0),g(1),...,g(N),\ldots \} $ бесконечную матрицу со столбцами $g(0),g(1),...,g(N),\ldots $ размерами $n.$

Каждой бесконечной матрице $y=\{ y(-1),y(0),y(1),...$, $y(N),\ldots \} $
можно сопоставить вектор"=функцию
$y(t)=y(-1)$ $\chi _{[-1,0)} (t)+y(0)\chi _{[0,1)} (t)+y(1)\chi _{[1,2)} (t)+...+y(N)\chi _{[N,N+1)} (t)+\ldots $

Аналогично, каждой бесконечной матрице $g=\{ g(0),g(1)$, $...,g(N),\ldots \} $ можно сопоставить вектор"=функцию
$$g(t)=g(0)\chi _{[0,1)} (t)+g(1)\chi _{[1,2)} (t)+...+g(N)\chi _{[N,N+1)} (t)+\ldots $$

Символом $y(t)=y[t]$ обозначим вектор"=функцию $y(t)=y([t])$, $t\in [-1,\infty ).$ Символом $g[t]$ обозначим вектор"=функцию $g(t)=g([t])$, $t\in [0,\infty )$.

Множество таких вектор"=функций $y[\cdot ]$ обозначим символом $\ell _{0} $. Множество таких вектор"=функций $g[\cdot ]$ обозначим символом $\ell $. Обозначим $(\Delta y)(t)=y(t)-y(t-1)=y[t]-y[t-1]$ при $t\ge 1$, $(\Delta y)(t)=y(t)=y[t]=y(0)$ при $t\in [0,1)$

Запишем ЛГФДСП в виде
$$
{\mathcal L}_{11} x+{\mathcal L}_{12} y=\dot{x}-F_{11} x-F_{12} y=f,
$$
$$
{\mathcal L}_{21} x+{\mathcal L}_{22} y=\Delta y-F_{21} x-F_{22} y=g.\eqno(1)
$$


Здесь и ниже $R^n$ "--- пространство векторов $\alpha =\{ \alpha ^{1} ,...,\alpha ^{n} \} $ с действительными компонентами и с нормой $\|\alpha \|_{R^n}$. Пусть пространство $L$ локально суммируемых $f:[0,\infty )\to R^n$ с полунормами $\|f\|_{L[0,T]} = {\int _{0}^{T}}\|f(t)\|_{R^n}\, dt$ для всех $T>0$. Пространство $D$ локально абсолютно непрерывных функций $x:[0,\infty )\to R^n $ с полунормами $\|x\|_{D[0,T]} =\|\dot{x}\|_{L[0,T]} +\|x(0)\|_{R^n} $ для всех $T>0$. $L_{\infty}$ "--- банахово пространство (классов эквивалентности) измеримых и ограниченных в существенном функций $z:[0,\infty)\to R^n$ с нормой $\|z\|_{L_{\infty}} = {\rm vrai}\sup\limits_{t\geq 0} \|z(t)\|_{R^n}.$

Пусть пространство $\ell $ вектор"=функций $$g(t)=g(0)\chi _{[0,1)} (t)+g(1)\chi _{[1,2)} (t)+...+g(N)\chi _{[N,N+1)} (t)+\ldots $$
 с полунормами $\|g\|_{\ell _{T} } =\sum\limits_{i=0}^{T}\|g_{i} \|_{R^n} $ для всех $T\geq 0$.

Пространство $\ell _{0} $ вектор"=функций $y(t)=y(-1)\chi _{[-1,0)} (t)+y(0)\chi _{[0,1)} (t)+y(1)\chi _{[1,2)} (t)+$ $...+y(N)\chi _{[N,N+1)} (t)+\ldots $ с полунормами $\|y\|_{\ell _{0T} } =\sum\limits_{i=-1}^{T}\|y_{i} \|_{R^n} $ для всех $T\ge -1$.

Операторы ${\mathcal L}_{11} ,F_{11} :D\to L$, ${\mathcal L}_{12} ,F_{12} :\ell _{0} \to L$, ${\mathcal L}_{21} ,F_{21} :D\to \ell $, $
{\mathcal L}_{22} ,F_{22} :\ell _{0} \to \ell $ предполагаются линейными непрерывными и вольтерровыми.

Пусть модельное уравнений ${\mathcal L}_{11} x=z$ и банахово пространство $B$ с элементами из пространства $L$ ($B\subset L$ и это вложение непрерывно) выбраны так, что решения этого решения этого уравнения обладают интересующими нас асимптотическими свойствами. Пространство $D({\mathcal L}_{11} ,B)$, порождаемое модельным уравнением, будет состоять из решений вида
$$
x(t)=\left({\mathcal C}_{11} z\right)(t)+({\mathcal X}_{11} \alpha )(t)= \int\limits_0^t C_{11}(t,s)z(s)\,ds + X_{11}(t)\alpha
$$
$
(\alpha \in R^n, \quad z\in B).
$

Норму в пространстве $D({\mathcal L}_{11} ,B)$ можно ввести равенством
$\|x\|_{D({\mathcal L}_{11} ,B)} = \|{\mathcal L}_{11} x\|_{B} +\|x(0)\|_{R^n}.$

Предположим, что оператор ${\mathcal C}_{11} $ непрерывно действует из пространства $B$ в пространство $B$, и оператор ${\mathcal X}_{11} $ действует из пространства $R^n$ в пространство $B$. Это условие эквивалентно тому, что пространство $D({\mathcal L}_{11} ,B)$ линейно изоморфно пространству С.Л.Соболева $W_{B}^{(1)} [0,\infty )$ с обычной нормой
$\|x\|_{W_{B}^{(1)} [0,\infty )}$ $=\|\dot{x}\|_{B} +\|x\|_{B}.$

Дальше будем это пространство обозначать как $W_{B} $. При этом, $W_{B} \subset D$, и это вложение непрерывно.

Уравнение ${\mathcal L}_{11} x=z$ с оператором ${\mathcal L}_{11} :W_{B} \to B$ \, $W_{B} $-ус\-т\-о\-й\-чи\-во тогда и только тогда, если оно сильно $B$-ус\-т\-о\-й\-чи\-во. Уравнение ${\mathcal L}_{11} x=z$ сильно $B$-ус\-т\-о\-й\-чи\-во, если для любого $z\in B$ каждое решение $x$ этого уравнения обладает свойством: $x\in B$ и $\dot{x}\in B$.

\smallskip \centerline {\bf 1.Сведение к ЛФДУП} \nopagebreak

Предположим, что общее решение уравнения ${\mathcal L}_{22} y=g$ для $g\in \ell $ принадлежит пространству $\ell _{0} $ и представляется формулой Коши:
$y[t]=Y_{22} [t]y(-1)+\sum\limits_{s=0}^{t}
 C_{22} [t,s]\, g[s].$

Поставим задачу: пусть $g\in M\subset \ell $, где $M$ "--- банахово пространство, и тогда будет $y\in M_{0} \subset \ell _{0} $, где $M_{0} $ "--- банахово пространство, причём $M_{0} $ изоморфно $M$.

Обозначим $({\mathcal C}_{22} g)[t]=\sum\limits_{s=0}^{t}C_{22} [t,s]\,g[s]$, $({\mathcal Y}_{22}$ $y(-1))[t]$ $=$ $Y_{22}[t]$$y(-1)$.

Тогда каждое решение $y$ второго уравнение в (1) имеет вид:
$y=-{\mathcal C}_{22} {\mathcal L}_{21} x+Y_{22} y(-1)+{\mathcal C}_{22} g.$

Подставим в первое уравнения в (1):
$${\mathcal L}_{11} x+{\mathcal L}_{12} y= {\mathcal L}_{11} x- {\mathcal L}_{12} {\mathcal C}_{22} {\mathcal L}_{21} x+{\mathcal L}_{12} Y_{22} y(-1)+ {\mathcal L}_{12} {\mathcal C}_{22} g=f,
$$
$$ {\mathcal L}_{11} x- {\mathcal L}_{12} {\mathcal C}_{22} {\mathcal L}_{21} x=f_{1} =f- {\mathcal L}_{12} Y_{22} y(-1)-{\mathcal L}_{ 12} {\mathcal C}_{22} g.
$$

Введём обозначение ${\mathcal L}_1= {\mathcal L}_{11} - {\mathcal L}_{12} {\mathcal C}_{22} {\mathcal L}_{21}$, тогда первое уравнения в (1) примет вид ${\mathcal L}_1x=f_{1}$.

Предположим, что вольтерров оператор ${\mathcal L}_1:W_{B}^{0} \to B$ вольтеррово обратим, где $W_{B}^{0} =\{ x\in W_{B} :x(0)=0\} $, то есть, когда задача для уравнения ${\mathcal L}_1x=f_{1} $ обладает свойством: при любом $f_{1} \in B$ её решения $x\in W_{B} $. Таким образом, мы решили задачу, когда для уравнения (1) при любом $\{ f,g\} \in B\times M$ её решения $\{ x,y\} \in W_{B} \times M_{0} .$



\smallskip \centerline {\bf 2.Сведение к ЛРУП} \nopagebreak


Для уравнения (1) будем пользоваться такими обозначениями, которые приняты в пункте 1.

Предположим, что общее решение уравнения ${\mathcal L}_{11} x=f$ для $f\in B$ ($B$ непрерывно вложено в $L$) принадлежит пространству $W_{B}$ и представляется формулой Коши $x =$ $X_{11}x(0)$ $+$ ${\mathcal C}_{11}f.$

Из первого уравнения в (1) найдём $x:$
$x=-{\mathcal C}_{11} {\mathcal L}_{12} y+X_{11} x(0)+{\mathcal C}_{11} f.$

Подставим во второе уравнения в (1):
$${\mathcal L}_{21} x+{\mathcal L}_{22} y =
-{\mathcal L}_{21} {\mathcal C}_{11} {\mathcal L}_{12} y+{\mathcal L}_{21} X_{11} x(0)+{\mathcal L}_{21} {\mathcal C}_{11} f+{\mathcal L}_{22} y=g,
$$
$$
- {\mathcal L}_{21} {\mathcal C}_{11} {\mathcal L}_{12} y+{\mathcal L}_{22} y=g_{1} =g- {\mathcal L}_{21} X_{11} x(0)-{\mathcal L}_{21} {\mathcal C}_{11} f.
$$
Введём обозначение ${\mathcal L}_2= {\mathcal L}_{22} -{\mathcal L}_{21} {\mathcal C}_{11} {\mathcal L}_{12} ,$ тогда второе уравнения в (1) примем вид ${\mathcal L}_2y=g_{1} .$

Предположим, что вольтерров оператор ${\mathcal L}_2:M_{0} \to M$ вольтеррово обратим, то есть, когда задача для уравнения ${\mathcal L}_2 y=g_{1} $ при любом $g_{1} \in M$ её решения $y\in M_{0} .$ Таким образом, мы решили задачу, когда для уравнения (1) при любом $\{ f,g\} \in B\times M $ её решения $\{ x,y\} \in W_{B}\times M_{0} .$


\smallskip \centerline{\bf 3.Достаточное условие устойчивости} \nopagebreak

Рассмотрим {\bf пример}. Пусть линейные оператора определены равенствами:

${\mathcal L}_{11}\{x_1,x_2\}_1 = \dot{x}_1 +
a_{11}x_{1\tau_{11}} + a_{12}x_{2\tau_{12}},$

${\mathcal L}_{12}\{y_1,y_2\}_1 = b_{11}y_{1\delta_{11}} + b_{12}y_{2\delta_{12}}$,


${\mathcal L}_{11}\{x_1,x_2\}_2 = \dot{x}_2 +
a_{21}x_{1\tau_{21}} + a_{22}x_{2\tau_{22}},$

${\mathcal L}_{12}\{y_1,y_2\}_2 = b_{21}y_{1\delta_{21}} + b_{22}y_{2\delta_{22}}$,


${\mathcal L}_{21}\{x_1,x_2\}_1 = c_{11}x_{1\rho_{11}} +
c_{12}x_{2\rho_{12}},$

${\mathcal L}_{22}\{y_1,y_2\}_1 = x_1 - d_{11}x_{1\theta_{11}} - d_{12}x_{2\theta_{12}}$,

${\mathcal L}_{21}\{x_1,x_2\}_2 = c_{21}x_{1\rho_{21}} +
c_{22}x_{2\rho_{22}},$

${\mathcal L}_{22}\{y_1,y_2\}_2 = x_2 - d_{21}x_{1\theta_{21}} - d_{22}x_{2\theta_{22}}$,

\noindent где $x_{1\tau_{11}}(t)=x_1(t-\tau_{11})$, если $t \geq 0$, $x_{1\tau_{11}}(t)=0$, если $t < 0$. Аналогичные определения верны
для остальных суперпозиций.

Обозначим:
$$
\ell_{\infty 0} =\{y\in \ell_{0}: \; \|y\|_{\ell_{\infty 0}} = \mathop{\sup}\limits_{k=-1,0,1,\cdots} |y(k)| <+\infty \},
$$
$$
\ell_{\infty} =\{g\in \ell: \; \|g\|_{\ell_{\infty}} = \mathop{\sup}\limits_{k=0,1,\cdots} |g(k)| <+\infty \}.
$$

Будем изучать вопрос, когда для нашего уравнения при любом $\{ f,g\} \in L_{\infty}\times \ell_{\infty}$ его решения $\{ x,y\} \in W_{B}\times \ell_{0\infty}$. Для этого надо найти вольтерровую обратимость оператора ${\mathcal L}_1= {\mathcal L}_{11} - {\mathcal L}_{12} {\mathcal C}_{22} {\mathcal L}_{21}: W_{L_{\infty}}^{0} \to L_{\infty}$. Или, для этого надо найти вольтерровую обратимость оператора ${\mathcal L}_2= {\mathcal L}_{22} - {\mathcal L}_{21} {\mathcal C}_{11} {\mathcal L}_{12}: \ell_{0\infty}^0 \to \ell_{\infty}$.

Для этого оценить норму $\|{\mathcal C}_{11}{\mathcal L}_{12} {\mathcal C}_{22} {\mathcal L}_{21}\|_{W_{L_{\infty}}\to W_{L_{\infty}}}$ или норму $\|{\mathcal C}_{22}{\mathcal L}_{21} {\mathcal C}_{11} {\mathcal L}_{12}\|_{{\ell_{0\infty}}\to {\ell_{0\infty}}}$.

Оценим нормы: $\|{\mathcal L}_{21}\|_{W_{L_{\infty}}\to \ell_{\infty}}$, $\|{\mathcal C}_{22}\|_{\ell_{\infty}\to \ell_{0\infty}}$, $\|{\mathcal L}_{12}\|_{\ell_{0\infty}\to L_{\infty}}$, $\|{\mathcal C}_{11}\|_{L_{\infty}\to W_{L_{\infty}}}$.

Справедливы оценки норм:
$$
\|{\mathcal L}_{21}\|_{W_{L_{\infty}}\to \ell_{\infty}} \leq
\left|\begin{array}{cc}
|c_{11}| & |c_{12}| \\
|c_{21}| & |c_{22}| \\
\end{array}\right|_{R^n},
$$
$$
\|{\mathcal C}_{22}\|_{\ell_{\infty}\to \ell_{0\infty}} \leq
\left|\begin{array}{cc}
\frac{1}{1-d_{11}} & 0 \\
0 & \frac{1}{1-d_{11}} \\
\end{array}\right|_{R^n},
$$
$$
\|{\mathcal L}_{12}\|_{\ell_{0\infty}\to L_{\infty}} \leq
\left|\begin{array}{cc}
|b_{11}| & |b_{12}| \\
|b_{21}| & |b_{22}| \\
\end{array}\right|_{R^n},
$$
$$
\|{\mathcal C}_{11}\|_{L_{\infty}\to W_{L_{\infty}}} \leq
\left|\begin{array}{cc}
\sigma_{11} & \sigma_{12} \\
\sigma_{21} & \sigma_{22} \\
\end{array}\right|_{R^n}.$$

Здесь, для простоты взяли $d_{12}=0$ и $d_{21}=0$, $|d_{11}| < 1$ и $|d_{22}| < 1$.
Обозначили $\sigma_{ij}= \sup\limits_{t \geq 0}\int\limits_{0}^{t}|C_{11ij}(t,s)|\,ds < \infty, $ $i,j=1,2$.

Для нашего уравнения при любом $\{ f,g\} \in L_{\infty}\times \ell_{\infty}$ его решения $\{ x,y\} \in W_{B}\times \ell_{0\infty}$, если выполнено неравенство
$$
\left|\begin{array}{cc}
\sigma_{11} & \sigma_{12} \\
\sigma_{21} & \sigma_{22} \\
\end{array}\right|_{R^n} \times
\left|\begin{array}{cc}
|b_{11}| & |b_{12}| \\
|b_{21}| & |b_{22}| \\
\end{array}\right|_{R^n}\times
$$
$$
\left|\begin{array}{cc}
\frac{1}{1-d_{11}} & 0 \\
0 & \frac{1}{1-d_{11}} \\
\end{array}\right|_{R^n}\times
\left|\begin{array}{cc}
|c_{11}| & |c_{12}| \\
|c_{21}| & |c_{22}| \\
\end{array}\right|_{R^n} < 1.
$$


\smallskip \centerline {\bf Литература} \nopagebreak

1.
{\it Ларионов А.С., Симонов П.М.} Устойчивость гибридных функционально"=дифференциальных систем с последействием (ГФДСП)~//~Вестник РАЕН. Темат. номер ``Дифференциальные уравнения''. 2013. Т.~13, \No~4. С.~34--37.

2.
{\it Симонов П.М.}	Устойчивость линейных гибридных фу\-н\-к\-ци\-о\-на\-ль\-но-ди\-ф\-фе\-рен\-ци\-аль\-ных систем с последействием (ЛГФДСП) // Вестник Тамбовского ун-та. Серия: Естест. и техн. науки. 2013. Т.~18, вып.~5. С.~2670--2672.


3.
{\it Ларионов А.С., Симонов П.М.} Устойчивость гибридных функционально"=дифференциальных систем с последействием (ГФДСП). II.~//~Вестник РАЕН. Темат. номер ``Дифференциальные уравнения''. 2014. Т.~14, \No~5. С.~38--45.

4.
{\it Ларионов А.С., Симонов П.М.} Устойчивость линейных гибридных функционально"=дифференциальных систем с последействием~//~Динамика систем и процессы управления: Труды международной конференции, посв. 90-летию со дня рождения акад. Н.Н. Красовского. Екатеринбург, Россия, 15--20 сентября 2014 г. Изд-во УМЦ УПИ, 2015. С.~243--250.

5.
{\it Симонов П.М.} К вопросу об устойчивости линейных гибридных функционально"=дифференциальных систем с последействием (ЛГФДСП)~// Вестник Тамбовского университета. Серия: Естест. и техн. науки. 2015. Т. 20, вып.~5. С.~1428--1436.

6.
{\it Симонов П.М.} Об устойчивости линейных гибридных функционально"=дифференциальных систем // Известия Института математики и информатики Удмуртского государственного университета. Ижевск: Изд-во УдГУ, 2015. Вып. 2 (46). С. 184--192.


7.
{\it Андрианов Д.Л., Арбузов В.О., Ивлиев С.В., Максимов В.П., Симонов П.М.} Динамические модели экономики: теория, приложения, программная реализация~//~Вестник Пермского университета. Серия: ``Экономика'' = Perm University Herald. Economy. 2015. \No~4 (27). С.~8--32.
