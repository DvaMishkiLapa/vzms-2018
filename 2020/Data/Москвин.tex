\vzmstitle[\footnote{Исследование выполнено в рамках Программы Президента Российской Федерации для государственной поддержки ведущих научных школ РФ (грант НШ-2554.2020.1).}]{АЛГОРИТМИЧЕСКАЯ РЕАЛИЗАЦИЯ БИФРУКАЦИЙ СЛОЕНИЯ ЛИУВИЛЛЯ В ИНТЕГРИРУЕМЫХ БИЛЛИАРДАХ В НЕВЫПУКЛЫХ ОБЛАСТЯХ}
\vzmsauthor{Москвин}{В.\,А.}
\vzmsinfo{Москва; {\it aoshi.k68@gmail.com}}
\vzmscaption
Математический биллиард "--- динамическая система, опи\-сы\-вающая движение без трения материальной точки внутри области с абсолютно упругим отражением от границы (угол падения равен углу отражения). В книге С.Л. Табач\-никова [1] дан обзор актуальных исследований биллиардов. Тополо\-гия совместных поверхностей уровня интег\-ралов опи\-сывается с помощью теории А.Т. Фоменко, которая в случае полных потоков изло\-жена в книге Болсинова--Фоменко [2]. В докладе будет пред\-став\-лено ис\-следование топологии фазового много\-образия плос\-ких бил\-лиардов, потоки в которых не являются полными вследствие наличия не\-выпук\-лых углов на границе области. Как было показано В. Драгович и М. Раднович [4] почти для всех значений интеграла в таких биллиардах совместная поверхность уровня интегралов будет гомеоморфна сфере с ручками и проколами. В докладе будет пред\-ставлено топ\-олог\-ичес\-кое описание трёхмерных ок\-рест\-ностей дву\-мерных ком\-плек\-сов, являющихся прообразами критичес\-ких значения интеграла $\Lambda$.


Мы будем понимать под биллиардной областью $\Omega$ однос\-вязную часть плоскости, ограниченную дугами софокусных квадрик из семейства:
$$(b-\lambda)x^2+(a-\lambda)y^2=(a-\lambda)(b-\lambda),~\lambda \leq a. $$
Биллиард $\Omega$ не должен содержать фокусов. Также любой сегмент фокальной прямой, содержащийся в биллиарде $\Omega$, либо лежит между фокусами, либо лежит вне фокусов. Такие биллиарды будем называть однородными.

Разрежем биллиард $\Omega$ следующим образом: если биллиард $\Omega$ не содержит сегментов фокальной прямой между фокусами, то проведём все эллипсы $\lambda_1, \ldots, \lambda_n$ на которых лежат вершины углов в $3\pi/2$, а если биллиард $\Omega$ не содержит сегментов фокальной прямой вне фокусов, то проведём все гиперболы $\lambda_1, \ldots, \lambda_n$ на которых лежат вершины не выпуклых углов. В результате биллиард $\Omega$ разобьётся на биллиарды $\Sigma_1, \ldots, \Sigma_N$ без не выпуклых углов на границе области. Значения допол\-нитель\-ного интеграла $\Lambda = \lambda_i$ и $\Lambda = b$ будут особыми [4].

Определим 3-атомы для таких биллиардов. В этой работе они будут рассматриваться как CW-комплексы.

\textbf{Определение 1.} { \it Трёхмерным атомом (3-атомом) назо\-вём трёхмерную окрестность $U \subset Q^3$ двумерного слоя $G$, зада\-ваемую неравенством $c - \epsilon \leq \Lambda \leq c + \epsilon$ для достаточно малого $\epsilon$, расслоённую на двумерные поверхности уровни функции $\Lambda$ и рассматриваемую с точностью до послойной экви\-вален\-тности. ($c = b$ или $c = \lambda_i$ для некоторого $i$)}

Рассмотрим биллиард $\Omega$ и сегмент квадрики $\lambda_i$. Обоз\-начим через $\nu$ число компонент связности внутри гиперболы (вне эллипса) $\lambda_i$, а через $\xi$ "--- число компонент связности вне гиперболы (внутри эллипса) с параметром $\lambda_i$.

\textbf{Теорема 1.}
	{\it Рассмотрим однородный биллиард $\Omega$ с выб\-ран\-ным на нем разбиением $\Sigma_1, \ldots, \Sigma_N$. Рассмотрим окрест\-ность значения дополнительного интеграла $\lambda_i - \epsilon \leq \Lambda \leq \lambda_i + \epsilon$ и соответствующий 3-атом $U_i$. Тогда:

	1. Комплекс $U_i \cong \tilde{U_i} \cup (G_g \times I)$, где $G_g$ "--- поверхность рода $g$ с приклеенными к ней $\nu$ цилиндрами $S^1 \times I$ и $g = const$ при изменении интеграла $\Lambda$. Комплекс $\tilde U_i \in Q^3$ проектируется при естественной проекции $\pi$ в окрестность квадрики $\lambda_i$ в биллиарде $\Omega$;

	2. Комплекс $\tilde{U_i} \setminus T_{\lambda_i} |_{\Lambda < \lambda_i} \cong (C_1 \cup \ldots \cup C_{2\nu+2\xi}) \times I$, где $T_{\lambda_i} $ "--- двумерный комплекс, построенный алгорит\-мически;

	3. Комплекс $\tilde{U_i} |_{\Lambda \geq \lambda_i} \cong (C_1 \cup \ldots \cup C_{2\nu}) \times I$;

	4. Трёхмерный комплекс $\tilde{U_i} \setminus T_{\lambda_i} |_{\Lambda < \lambda_i}$ прик\-леи\-вается к двумерному комплексу $T_{\lambda_i}$ послойно. Данная склей\-ка описы\-вается алгоритмом.
}

Теорема 2 описывает строение окрестности $U$ особого слоя $\Lambda = b$.

\textbf{Теорема 2.} {\it
	Рассмотрим однородный биллиард $\Omega$ с выб\-ран\-ным на нем разбиением $\Sigma_1, \ldots, \Sigma_N$. Рассмотрим окрест\-ность значений дополнительного интеграла $b - \epsilon \leq \Lambda \leq b + \epsilon$ и соответствующий 3-атом $U$. Тогда:

1. Комплекс $U \setminus (T_{\lambda_1}\cup\ldots\cup T_{\lambda_n}) \cong V_{\Sigma_1} \times I \cup \ldots \cup V_{\Sigma_N} \times I$, где объединение несвязно. Здесь $T_{\lambda_i}$ "--- двумерные комплексы, построенные алгорит\-мически, а $V_{\Sigma_j}$ "--- 2-атом, соот\-вет\-ствующий 3-атому выпуклого бил\-лиарда $\Sigma_j$, см. [3];

2. Двумерные комплексы $(T_{\lambda_1} \cup \ldots \cup T_{\lambda_n})$ приклеиваются к трёхмерному комплексу $U \setminus (T_{\lambda_1} \cup \ldots \cup T_{\lambda_n})$ послойно. Данная склейка описывается алгоритмом.}







% Оформление списка литературы
\litlist
1. {\it Табачников С.\,Л.}
Геометрия и бильярды.М.;Ижевск:НИЦ Регулярная и хаотическая динамика, 2011.

2. {\it Болсинов А.В., Фоменко А.Т.}
Интегрируемые гамиль\-тоновы системы. Геометрия, топология, классификация. Т. 1. Ижевск: НИЦ Регулярная и хаотическая динамика, 1999.

3. {\it Фокичева В. В.}
Топологическая классификация бильярдов в локально плоских областях, ограниченных дугами софокусных квадрик//
матем. сб. 2015, 206, 10. 127--176.

4. {\it Dragovic V., Radnovic M.}
Bifurcations of Liouville tori in elliptical billiards//
Regullar Chaotic Dyn. РАН. 2009. 14. 479--494.
