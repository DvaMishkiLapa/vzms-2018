\begin{center}
    {\bf ЛИУВИЛЛЕВА КЛАССИФИКАЦИЯ ИНТЕГРИРУЕМОГО ГЕОДЕЗИЧЕСКОГО ПОТОКА В ПОТЕНЦИАЛЬНОМ ПОЛЕ НА ПРОЕКТИВНОЙ ПЛОСКОСТИ}

    {\it Е.И. Антонов}

    (Москва; {\it antonov.zhenya@hotmail.com})
\end{center}

\addcontentsline{toc}{section}{Антонов Е.И.}




Полученные результаты основываются на теории топологической классификации интегрируемых гамильтоновых систем, созданной А.\,Т.~Фоменко и его школой (см. $\left[1\right]$). Подробнее о лиувиллевой классификации (т.е. о вычислении инвариантов Фоменко-Цишанга) для геодезических потоков см.$\left[1\right]$.



Рассмотрим риманово многообразие вращения $M=S^2$ с естественными координатами $(r;\varphi), r \in (0;L), \varphi \in \mathbb{R}/2\pi\mathbb{Z}$, в которых метрика вращения записывается в виде
\[\mathrm{d}s^2=\mathrm{d}r^2+f^2(r)\mathrm{d}\varphi^2.\] В окрестности полюсов введём локальные координаты
\[x=f(r)\cos{\varphi},\qquad y=f(r)\sin{\varphi}.\] При этом метрика в полюсах запишется в виде $\mathrm{d}s^2=\mathrm{d}x^2+\mathrm{d}y^2$.


Рассмотрим интегрируемую гамильтонову систему на кокасательном расслоении $T^*S^2$ с гамильтонианом
\\
$ H = \begin{cases} \displaystyle \frac{p^2_{r}}{2}+\frac{p^2_{\varphi}}{2f^2(r)}+V(r), \qquad & \text{ вне полюсов,} \\ \displaystyle \frac{p^2_{x}+p^2_{y}}{2}+V(0), \qquad & \text{ в полюсах} \end{cases} $
\\
и первым интегралом
\\
 $K= \begin{cases} p_{\varphi}, \qquad &\text{ вне полюсов,} \\ xp_{y}-yp_{x}, \qquad & \text{ в окрестности полюсов.} \end{cases} $



Представим проективную плоскость $\mathbb{RP}^2$ как фактор сферы $S^2$ по инволюции $\eta$, которая в координатах $r,\varphi$ задаётся формулой: \[ \eta (\varphi, r) = (\varphi + \pi, L-r).\] Полюса при инволюции переходят друг в друга: $\eta (S)=N,\eta(N)=S$. Имеем: \begin{gather*}
 T^{*}\mathbb{R}\mathrm{P}^2=T^{*}S^2/\eta ^{*},\\
 \eta^{*}(p_{r},p_{\varphi},r,\varphi)=(-p_{r},p_{\varphi},L-r,\varphi+\pi).
 \end{gather*} Далее мы считаем, что $f(r)=f(L-r)$ и $V(r) = V(L-r)$, чтобы $f$ и $V$ задавали функции на $\mathbb{RP}^2$.


\textbf{Теорема~1} \par
{Рассмотрим систему (то есть геодезический поток с линейным интегралом и с инвариантным при вращениях потенциалом) на многообразии вращения $M\approx \mathbb{R}\mathrm{P}^2$, заданную парой функций $(V(r);f(r))$.
Пусть $Q^3$ - связная компонента неособой изоэнергетической поверхности $Q^3_{h}$. Пусть $W -W$ -- молекула системы на $Q^3$.
\begin{enumerate}

\item Метки на нецентральных рёбрах молекулы следующие:
\begin{enumerate}
  \item на рёбрах между седловыми атомами метки: $r=\infty, \varepsilon=+1$;
 \item
  На рёбрах между седловыми атомами и атомами типа $A$ метка $r=0$, за исключением тех случаев, когда атом $A$ отвечает неподвижной точке инволюции. В этом случае метка $r=\frac{1}{2}$. Метка $\varepsilon$ в обоих случаях равна $+1$.
\end{enumerate}
\item Метки на центральных рёбрах молекулы следующие:
\begin{enumerate}
  \item
  на ребре, соединяющем седловые атомы, метки следующие: $r=\infty, \varepsilon=-1$;
  \item
  Если молекула $W$ --- $W$ имеет тип $A$ --- $A$, то метка $r$ определяется следующим образом:
  \begin{itemize}
  \item
  если $Q^{3}_{S^2}\approx \mathbb{R}\mathrm{P}^3$ то:
\\
  $r=\frac{1}{4}, \varepsilon=+1$;

\item
если $Q^{3}_{S^2}\approx S^1\times S^2$ то возможны два случая:
  \begin{itemize}
  \item  $r=\infty, \varepsilon=+1$;
  \item $r=\infty, \varepsilon=+1$;
    \end{itemize}
\item
если $Q^{3}_{S^2}\approx S^3$ то:
\par
 $r=0, \varepsilon=+1$;
  \end{itemize}
  \end{enumerate}
\item Если молекула $W$ --- $W$ отлична от $A$ --- $A$, то она содержит единственную семью, получаемую отбрасыванием всех атомов $A$. Тогда значение метки $n$ определяется топологическим типом $Q^{3}_{S^2}$ и сечением атомов, соединённых центральным ребром, то есть:
 \begin{itemize}
  \item Если $Q^{3}_{S^2}\approx \mathbb{R}\mathrm{P}^3$, то метка $n$ равна 0.
  \item Если  $Q^{3}_{S^2}\approx S^1\times S^2$, то метка $n$ равна либо 0, либо -1.
  \item Если  $Q^{3}_{S^2}\approx S^3$, то метка $n$ равна 1.
  \end{itemize}
  \end{enumerate}



\smallskip \centerline {\bf Литература} \nopagebreak

1. {\it А.\,В.~Болсинов, А.\,Т.~Фоменко}, ``Интегрируемые гамильтоновы системы. Геометрия, топология, классификация'', Т. 1, 2, Изд. дом “Удмуртский университет”, Ижевск, 1999, 444 с., 447 с.

2. {\itЕ.\,О.~Кантонистова,} ``Топологическая классификация интегрируемых гамильтоновых систем на поверхностях вращения в потенциальном поле'', \textit{Матем. сб.}, \textbf{207}:3 (2016), 47--92;

3. {\it D.\,S.~Timonina}, ``Topological classification of integrable geodesic flows in a potential field on the torus of revolution'',
Lobachevskii J. Math., \textbf{38}:6 (2017), 1108--1120
