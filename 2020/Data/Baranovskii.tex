
\begin{center}
    {\bf НЕЛИНЕЙНАЯ МОДЕЛЬ СДВИГОВОГО ТЕЧЕНИЯ НЕРАВНОМЕРНО НАГРЕТОЙ ВЯЗКОЙ ЖИДКОСТИ}

    {\it Е.С. Барановский, А.А. Домнич, М.А. Артемов}

    (Воронеж, ВГУ; {\it esbaranovskii@gmail.com})
\end{center}

\addcontentsline{toc}{section}{Барановский Е.С., Домнич А.А., Артемов М.А.}

В настоящей работе изучается математическая модель, описывающая установившееся сдвиговое течение несжимаемой вязкой жидкости между плоскостями ${z=-h}$ и~${z=h}$. Предполагается, что течение обусловлено действием постоянного перепада давления ${\partial p}/{\partial x}=-\xi$ и~осуществляется в~условиях пристенного скольжения типа Навье при наличии теплового потока~$\omega$:
$$
\left\{
\begin{array}{l}
\displaystyle
-\frac{d}{dz}\big\{\mu[\theta(z)]u^\prime(z)\big\}=\xi,\;\; z\in [-h,h],
\bigskip
\\
\displaystyle
-\frac{d}{dz}\big\{k[\theta(z)]\theta^\prime(z)\big\}=\omega(z),\;\; z\in [-h,h],
\bigskip
\\
\mu[\theta(h)]u^\prime(h)=-\chi[\theta(h)]u(h),
\bigskip
\\
\mu[\theta(-h)]u^\prime(-h)=\chi[\theta(-h)]u(-h),
\bigskip
\\
k[\theta(h)]\theta^\prime(h)=-\beta \theta(h),
\bigskip
\\
k[\theta(-h)]\theta^\prime(-h)=\beta \theta(-h),
\end{array}
\right.
\eqno({\bf A})
$$
где $u$~--- скорость течения жидкости вдоль оси~$x$; $\theta$~--- температура; $\mu[\theta]$, $k[\theta]$, $\chi[\theta]$~--- коэффициенты вязкости,  теплопроводности и
проскальзывания соответственно; $\beta$~--- постоянный коэффициент теплообмена на стенках канала.

Неизвестными в системе ({\bf A}) являются  скорость~$u$ и температрура~$\theta$, а все остальные функции и величины считаются заданными.
Предположим, что:
\begin{itemize}
\item[({\bf C1})] функция ${\omega\colon[-h,h]\to{\bf R}}$ является чётной и принадлежит пространству Лебега $L^2[-h,h]$;
\item[({\bf C2})] функции ${\chi\colon{\bf R}\to{\bf R}}$, ${\mu\colon{\bf R}\to{\bf R}}$, ${k\colon{\bf R}\to{\bf R}}$ непрерывны;
\item[({\bf C3})] существует константа~$k_0$ такая, что $0<k_0\leq k(s)$
для любого~${s\in{\bf R}}$;
\item[({\bf C4})] для любого ${r>0}$ существуют константы~$\chi_r$ и~$\mu_r$ такие, что ${0<\chi_r\leq\chi(s)}$ и~${0<\mu_r\leq\mu(s)}$ для ${\forall s\in[-r,r]}$;
\item[({\bf C5})] выполнено неравенство $\beta>0$.
\end{itemize}

Введём следующие обозначения:
$$
\overline{v}_h\stackrel{{\rm def}}{=}(2h)^{-1}\int_{-h}^hv(z)\,dz,
$$
$$
H^1_{ even}[-h,h]\stackrel{{\rm def}}{=}\{v\in H^1[-h,h]:\;\, v(-z)=v(z)\; \forall z\in [-h,h]\},
$$
где  $H^1[-h,h]$ --- пространство Соболева.

\medskip
{\bf  Определение.} {\it  Слабым решением} задачи~({\bf A}) назовём пару функций ${(u, \theta)\in H^1_{even}[-h,h]\times H^1_{even}[-h,h]}$ такую, что
$$
\int_{-h}^h\mu[\theta(z)]u^\prime(z) \psi^\prime(z)\,dz
+2\chi[\theta(h)]u(h)\psi(h)
=\xi\int_{-h}^h \psi(z)\,dz,
$$
$$
\int_{-h}^hk[\theta(z)]\theta^\prime(z) \psi^\prime(z)\,dz
+2\beta\theta(h)\psi(h)
=\int_{-h}^h \omega(z)\psi(z)\,dz
$$
для любой пробной функции ${\psi\in H^1_{even}[-h,h]}$.
\medskip

Сформулируем теперь основные результаты работы.

\medskip
\textbf{Теорема.}
{\it
Пусть выполнены условия {\rm({\bf С1})}--{\rm({\bf С5})}. Тогда
\begin{itemize}
\item[{\rm (i)}] задача~{\rm({\bf A})} имеет по крайней мере одно слабое решение;
\item[{\rm (ii)}] если ${(u,\theta)}$ --- слабое решение задачи~{\rm({\bf A})}, то на стенках канала ${z=\pm h}$ выполнены соотношения:
$$
u(h)=u(-h)=\frac{\xi h}{\chi[\overline{\omega}_h h\beta^{-1}]},\quad \theta(h)=\theta(-h)=\overline{\omega}_h h\beta^{-1};
$$
\item[{\rm (iii)}]
если ${(u,\theta)}$ --- слабое решение задачи~{\rm({\bf A})}, то  выполнены следующие энергетические равенства:
$$
\int_{-h}^h\mu[\theta(z)]|u^\prime(z)|^2\,dz
+2\chi[\theta(h)]u^2(h)
=2h\xi\overline{u}_h,
$$
$$
\int_{-h}^hk[\theta(z)]|\theta^\prime(z)|^2\,dz
+2\beta\theta^2(h)
=\int_{-h}^h\omega(z) \theta(z)\,dz;
$$
\item[{\rm (vi)}] если $(u,\theta)$ и $(v,\theta)$ --- слабые решения задачи~{\rm({\bf A})}, то ${u=v}$;
\item[{\rm (v)}] если~$k$ удовлетворяет условию Липшица
с~константой~$M$, где ${M<{\min\{k_0^2,\beta^2\}}/\big({\|\omega\|_{L^1[-h,h]}\max\{1,4h\}}\big)}$,
то задача~{\rm({\bf A})} имеет единственное слабое решение.
\end{itemize}
}

{\bf Замечание.} В линейной постановке задача о протекании неравномерно нагретой вязкой жидкости сквозь ограниченный сосуд c двумя плоскими отверстиями рассмотрена в [1]. В статье [2] изучается обратная задача для эволюционной линейной модели одно\-направленного термогравитационного движения вязкой жидкости в плоском канале. В работе~[3] предложена модель оптимального граничного управления неизотермическим течением жидкости через заданную локально-липшицеву область~${\Omega\subset{\bf R}^d}$, ${d=2,3}$.

\smallskip \centerline {\bf Литература} \nopagebreak

1. {\it Крейн С.Г., Чан Тху Xа.} Задача протекания неравномерно нагретой вязкой жидкости // Журнал вычислительной математики и математической физики. 1989. Т.~29. \No~8. С.~1153--1158.

2. {\it Черемных Е.Н.} Априорные оценки решения задачи об однонаправленном термогравитационном движении вязкой жидкости в плоском канале // Математические заметки. 2018. Т.~103. \No~1. C.~147--157.

3. {\it Baranovskii E.S., Domnich A.A., Artemov M.A.} Optimal boundary control of non-isothermal viscous fluid flow // Fluids. 2019. V.~4. \No~3. Article ID 133.

