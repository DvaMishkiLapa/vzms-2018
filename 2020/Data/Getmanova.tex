\vzmstitle{ON PERIODIC SOLUTIONS OF RANDOM FUNCTIONAL DIFFERENTIAL INCLUSIONS}
\vzmsauthor{Getmanova}{E.\,N.}
\vzmsinfo{Воронеж; {\it ekaterina$\_$getmanova@bk.ru}}
\vzmscaption

\selectlanguage{english}

Let us mention that the method of guiding functions was developed by Krasnoselskii and Perov (see, e.g., [14]) for the investigation of periodic oscillations in dynamical systems go\-ver\-ned by differential equations. The notion of guiding function was generalized in several directions (see, e.g., [1, 2, 4-13]).

Based on the approach given in [1] we present the notion of nonsmooth random generalized integral guiding functions and use those to prove some existence results of random solutions to periodic problem of random functional differential inclusion.

In what follows we will use some known notions and notati\-ons from the theory of multimaps (see, e.g., [2, 4]).

Let $(X,d_X)$ and $(Y,d_Y)$ be metric spaces. By the symbols $P(Y),$ $C(Y)$ and $K(Y)$ we denote the collections of all nonemp\-ty, closed and, respectively, compact subsets of the space $Y.$ If $Y$ is a normed space, $Kv(Y)$ denote the collections of all nonempty convex compact subsets of $Y.$

\textbf{Definition~1.}
A multimap $F:X \to P(Y)$ is called {\it upper semicontinuous (u.s.c.)} at the point $x\in X$ if for each open set $V \subset Y$ such that $F(x) \subset V$ there exists $\delta >0$ such that $d_X(x,x^\prime)<\delta$ implies $F(x^\prime)\subset V.$ A multimap $F:X \to P(Y)$ is called u.s.c. if it is u.s.c. at each point $x\in X.$

Let $I$ be a closed subset of $\mathbb{R}$ with the Lebesgue measure.

\textbf{Definition~2.}
A multifunction $F:I \to K(Y)$ is called {\it measurable} if, for each open subset $W \subset Y,$ its pre-image \linebreak
$F^{-1}(W)=\{t\in I:F(t)\subset W\}$ is the measurable subset of$I.$

Let $(\Omega,\Sigma,\mu)$ be a complete probability space and $I=[0,T]$.

\textbf{Definition~3.} (see [1]). Multimap $\mathcal{F}\colon\Omega\times X\to C(Y)$ is called a {\it random multioperator} if it is product-measurable, i.e. measurable w.r.t. $\Sigma\otimes\mathbb{B}(X)$, where $\Sigma\otimes\mathbb{B}(X)$ is the smallest
$\sigma$-algebra on $\Omega\times X$ which contains all the sets $A\times B$, where $A\in\Sigma$ and $B\in\mathbb{B}(X)$ and
$\mathbb{B}(X)$ denotes the Borel $\sigma$-algebra on $X$. If, moreover, $\mathcal{F}(\omega,\cdot)\colon X\to C(Y)$ is u.s.c. for all
$\omega\in\Omega$, then $\mathcal{F}$ is called a {\it random $u$-multioperator}.

For $\tau>0$ we denote by the symbol $\mathcal{C}$ the space $C([-\tau, 0]; \mathbb{R}^n)$ of continuous functions $x:[-\tau,0] \to\mathbb{R}^n$ with norma $\|x\| = \sup_{t\in[-\tau,0]} \|x(t)\|.$
For $x(\cdot) \in C([-\tau, T]; \mathbb{R}^n)$, the symbol $x_t \in\mathcal{C}$ denotes the function defined as $x_t(\theta ) = x(t + \theta),$ $\theta\in[-\tau,0].$

We consider the periodic problem for a random functional differential inclusion of the following form:
$$
x'(\omega,t)\in \mathcal{F} \bigr (\omega, t, x_t \bigr), \eqno{(1)}
$$
$$
x(\omega, 0) = x(\omega,T), \eqno{(2)}
$$

\noindent where the multimap $\mathcal{F}\colon\Omega \times \mathbb{R} \times \mathcal{C} \multimap \mathbb{R}^{n}$ satisfies conditions:

$(\mathcal{F}_t)$ multifunction $\mathcal{F}$ is a $T$-periodic in the second argument;

$(\mathcal{F}1)$ $\mathcal{F}\colon \Omega \times I \times \mathcal{C} \to Kv(\mathbb{R}^n)$ is a random $u$-multioperator;

$(\mathcal{F}2)$ there exists a function $c:\Omega\times I \to \mathbb{R}$ such that
$(i)$ for each $\omega\in\Omega$ a function $c(\cdot,t)$ is measurable,
$(ii)$ a function $c(\omega,\cdot)$ is locally integrable,
and we have for each $\omega\in\Omega$\linebreak
$
\|\mathcal{F}(\omega,t,\phi)\|:=\sup\{|z|\colon z\in \mathcal{F}(\omega,t,\phi)\} \leq c(\omega,t) (1+|\phi|).
$

\noindent
From $(\mathcal{F}1)$-$(\mathcal{F}2)$ it follows that the superposition multioperator
$\mathcal{P}_\mathcal{F}\colon\Omega\times C(I,\mathbb{R}^{n})\to P(L^{2}(I,\mathbb{R}^{n})),$
	$\mathcal{P}_\mathcal{F}(\omega, x) = \{f\in L^{2}(I,\mathbb{R}^{n})\colon $ $f(s)\in \mathcal{F}(\omega, s, x(s)),\mbox{for a.e.} \, s\in I\}$
is well defined.

By a \emph{random solution} of problem (1), (2) we mean a function $\xi \colon \Omega \times I \to \mathbb{R}^{n}$ such that

(i) the map $\omega \in \Omega \to \xi(\omega, \cdot) \in C([-\tau,T]; \mathbb{R}^n)$ is measurable;

(ii) for each $\omega \in \Omega$ absolutely continuous function $\xi(\omega,\cdot) \in C([-\tau,T]; \mathbb{R}^n)$ satisfies (1), (2) for a.e. $t\in [-\tau,T].$

Let us recall some notions of nonsmooth analysis (see [3]).

Let $V$ be a locally Lipschitz function on the space $\mathbb{R}^{n}.$ For $x_{0} \in \mathbb{R}^{n}$ and $\nu \in \mathbb{R}^{n}$ {\it the Clarke generalized derivative} $V^{0}(x_{0} ;\nu )$ at $x_{0} $ along the direction $\nu $ is given by the formula
$
V^{0}(x_{0} ;\nu ) = \mathop{\overline {\lim}}\limits_{{x \to x_{0},}{t \to 0+}}\frac{{V(x + t\nu ) - V(x)}}{{t}},
$
where $x \in \mathbb{R}^{n}.$ Then {\it the Clarke genera\-li\-zed gradient} $\partial V(x)$ of the function $V$ at the point $x_{0}$ is defined as
$
\partial V(x_{0} ) = \left\{ {x \in \mathbb{R}^{n}: \langle {x,\nu } \rangle \le V^{0}(x_{0} ;\nu )\,\mbox{for all}\,\nu \in \mathbb{R}^{n}} \right\}.
$
Recall that a locally Lipschitz function $V:\mathbb{R}^n\to \mathbb{R}$ is called {\it regular} if for each $x\in \mathbb{R}^{n}$ and $\nu \in \mathbb{R}^{n}$ there exists the derivative along the direction $V'(x,\nu)$ and it coincides with $V^0(x,\nu).$

\textbf{Definition~4.}
A map $V\colon\Omega\times\mathbb{R}^{n}\to\mathbb{R}$ is called a {\it random nonsmooth potential} if the following two conditions are satisfied:
$(i)$ $V(\cdot,x)\colon\Omega\to\mathbb{R}$ is measurable for every $x\in\mathbb{R}^{n}$; \linebreak
$(ii)$ $V(\omega,\cdot)\colon\mathbb{R}^{n}\to\mathbb{R}$ is a locally Lipschitz for every $\omega\in\Omega$.

\textbf{Definition~5.}
            A random nonsmooth potential $V\colon\Omega\times\mathbb{R}^{n}\to\mathbb{R}$ is said to be a {\it random nonsmooth generalized integral guiding function} for inclusion (1) if the following conditions hold:
            $(i)$ there exists $R_0>0$ such that
$
	0\notin\partial V(\omega,x)
$
for all $(\omega,z)\in\Omega\times\mathbb{R}^{n}\colon |z|\geq R_0;$
						$(ii)$ the function $V(\omega,\cdot)$ is regular for every $\omega\in\Omega$;
$(iii)$ there exists $N>0$ such that for all $\omega\in\Omega$ from $x\in C(I,\mathbb{R}^{n})$ with
						${\|x\|}_{2}\geq N$, it follows that
            $
                \int_{0}^{T} \bigl<\upsilon(t),f(t)\bigr> dt \geq 0
            $
            for all $\upsilon\in\mathcal{P}_{\partial V}(\omega,x)$ and for all
						$f\in \mathcal{P}_{F}(\omega,x)$.

\textbf{Theorem~1.}
                Let conditions $(\mathcal{F}_t), (\mathcal{F}1), (\mathcal{F}2)$ hold. If there exists a regular random nonsmooth generalized integral guiding function for inclusion
                (1) such that $\lim_{\|x\|\to +\infty} V(\omega,x) = +\infty$, then problem (1), (2) has a random solution.

% Оформление списка литературы
\smallskip \centerline {\bf Bibliography} \nopagebreak

1. {\it Andres J., G\'orniewicz L.} Random topological degree and random differential inclusions. Topol. Meth. Nonl. Anal. 40 (2012), 337--358.

2. {\it Borisovich Yu.G., Gel'man B.D., Myshkis A.D., Obukho\-vskii V.V.} Introduction to the Theory of Multivalued Maps and Dif\-fe\-ren\-tial Inclusions - 2nd ed. Moscow: Librokom, 2011.

3. {\it Clarke F.H.} Optimization and Nonsmooth Analysis - 2nd ed. Classics in Applied Mathematics, 5. Society for Industrial and Applied Mathematics (SIAM). Philadelphia: PA, 1990.

4. {\it G\'orniewicz L.} Topological Fixed Point Theory of Multi\-valued Mappings - 2nd ed. Berlin: Springer, 2006.

5. {\it Kornev S.V.} On the method of multivalent guiding functi\-ons to the periodic problem of differential inclusions. Autom. Re\-mote Control. 64 (2003), 409--419.

6. {\it Kornev S.V., Obukhovskii V.V.} On nonsmooth multiva\-lent guiding functions. Differ. Equ. 39 (2003), 1578--1584.

7. {\it Kornev S.V., Obukhovskii V.V.} On some developments of the method of integral guiding functions. Funct. Differ. Equ. 12 (2005), 303--310.

8. {\it Kornev S.V., Obukhovskii V.V.} Non-smooth guiding po\-tentials in problems on forced oscillations. Autom. Remote Con\-trol. 68 (2007), 1--8.

9. {\it Kornev S.V., Obukhovskii V.V.} On localization of the guiding function method in the periodic problem of differential inclusions. Russian Mathematics (Iz. VUZ). 5 (2009), 23--32.

10. {\it Kornev S.V.} Nonsmooth integral directing functions in the problems of forced oscillations. Autom. Remote Control. 76 (2015), 1541--1550.

11. {\it Kornev S.V.} Multivalent guiding function in a problem on existence of periodic solutions of some classes of differential inclusions. Russian Mathematics (Iz. VUZ). 11 (2016), 14--26.

12. {\it Kornev S.V., Obukhovskii V.V., Zecca P.} On the method of generalized integral guiding functions in the periodic problem of functional differential inclusions. Differ. Uravn. 52 (2016), no. 10, 1335–-1344.

13. {\it Kornev S.V., Obukhovskii V.V., Zecca P.} Guiding func\-ti\-ons and periodic solutions for inclusions with causal multiope\-rators. Appl. Anal. 96 (2017), issue 3, 418--428.

14. {\it Krasnosel'skii M.A.} The Operator of Translation Along the Trajectories of Differential Equations. Translations of Ma\-the\-ma\-tical Monographs - Vol. 19. Providence, R.I.: American Ma\-the\-matical Society, 1968.
