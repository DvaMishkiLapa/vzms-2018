\vzmstitle[\footnote{Исследование выполнено за счёт гранта Российского научного фонда (проект №~18-11-00199).}]{ВЗАИМОСВЯЗЬ МЕЖДУ ОДНОМЕРНЫМИ~КОНСТАНТАМИ НИКОЛЬСКОГО--БЕРНШТЕЙНА ДЛЯ~ПОЛИНОМОВ И ФУНКЦИЙ}
\vzmsauthor{Горбачев}{Д.\,В.}
\vzmsauthor{Мартьянов}{И.\,А.}
\vzmsinfo{Тула; {\it dvgmail@mail.ru}; {\it martyanow.ivan@yandex.ru}}
\vzmscaption

Пусть $0<p\le \infty$, $\mathcal{C}(n;p;r)=\sup
\frac{\|T^{(r)}\|_{L^{\infty}[0,2\pi)}}{\|T\|_{L^{p}[0,2\pi)}}$ и
$\mathcal{L}(p;r)=\sup
\frac{\|F^{(r)}\|_{L^{\infty}(\mathbb{R})}}{\|F\|_{L^{p}(\mathbb{R})}}$~---
точные константы Никольского--Бернштейна для $r$-х производных
тригонометрических полиномов степени $n$ и целых функций экспоненциального типа
$1$ соответственно. Недавно Е.~Левин и Д.~Любинский (2015) доказали, что для
констант Никольского
\[
\mathcal{C}(n;p;0)=n^{\frac{1}{p}}\mathcal{L}(p;0)(1+o(1)),\quad n\to \infty.
\]
М.~Ганзбург и С.~Тихонов (2017) обобщили этот результат на случай констант
Никольского--Бернштейна:
\[
\mathcal{C}(n;p;r)=n^{r+\frac{1}{p}}\mathcal{L}(p;r)(1+o(1)),\quad n\to \infty.
\]
Также они показали существование в этой задаче экстремальных полинома
$\tilde{T}_{n,r}$ и функции $\tilde{F}_{r}$ соответственно. Ранее мы дали более
точные границы в результате типа Левина--Любинского, доказав, что для всех $p$
и $n$
\[
n^{\frac{1}{p}}\mathcal{L}(p;0)\le \mathcal{C}(n;p;0)\le (n+\lceil
\tfrac{1}{p}\rceil)^{\frac{1}{p}}\mathcal{L}(p;0).
\]

Мы устанавливаем близкие факты для случая констант Никольского--Бернштейна, из
которых также вытекает асимптотическое равенство Ганзбурга--Тихонова.
Результаты формулируется в терминах экстремальных функций $\tilde{T}_{n,r}$,
$\tilde{F}_{r}$ (норма которых равна единице) и коэффициентов Тейлора $A_{s,i}$
ядра типа Джексона--Фейера $(\frac{\sin \pi x}{\pi x})^{2s}$. В частности,
имеем

\textbf{Теорема.} {\it Пусть $p\in (0,\infty]$, $r,n\in \mathbb{Z}_{+}$, $s\in
\mathbb{N}$.

\textup{(i)} Если $2s\ge r+1$, $n\ge s-1$, то
\[
\mathcal{C}(n;p;r)\ge (n-s+1)^{r+\frac{1}{p}}\biggl(\mathcal{L}(p;r)+
\sum_{i=1}^{\lfloor
\frac{r}{2}\rfloor}\frac{(-1)^{i}\binom{r}{2i}A_{s,i}\tilde{F}_{r}^{(r-2i)}(0)}{n^{2i}}\biggr).
\]

\textup{(ii)} Если $n\ge 1$, $ps\ge 1$, то
\[
\mathcal{C}(n;p;r)+\sum_{i=1}^{\lfloor \frac{r}{2}\rfloor}(-1)^{i}\binom{r}{2i}A_{s,i}
\tilde{T}_{n,r}^{(r-2i)}(0)\le n^{r}(n+s)^{\frac{1}{p}}\mathcal{L}(p;r),
\]
где
\[
|\tilde{T}_{n,r}^{(r-2i)}(0)|\le n^{r-2i}(n+\lceil
\tfrac{1}{p}\rceil)^{\frac{1}{p}}\mathcal{L}(p;0).
\]
}