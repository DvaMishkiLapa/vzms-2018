\vzmstitle{НЕПОДВИЖНЫЕ ТОЧКИ И СОВПАДЕНИЯ В УПОРЯДОЧЕННЫХ МНОЖЕСТВАХ И МЕТРИЧЕСКИЕ СЛЕДСТВИЯ}
\vzmsauthor{Фоменко}{Т.\,Н.}
\vzmsinfo{Москва, ВГУ; {\it tn-fomenko@yandex.ru}}
\vzmscaption

Доклад основан на некоторых результатах работы [2]. Рассматривается проблема существования неподвижных точек отображений упорядоченного  множества.

Яхимский (J.Jachymski) в 1997 доказал следующее обобщение известной теоремы Цермело о неподвижой точке.

\textbf{Теорема~1 (Яхимский, см. [1]).}  {\it Пусть  $(X,\preceq)$  "--- частично упорядоченное множество,  $f: X\to X$ "--- регрессивное отображение, то есть $f(x)\preceq x, \forall x\in X$. Пусть для каждой цепи в $X$  задана некоторая её нижняя граница. Тогда отображение $f$ имеет неподвижную точку.}

В [2] получен следующий результат, обобщающий эту теорему Яхимского.

\textbf{Теорема~2 [2].} {\it Пусть $(X,\preceq)$ "--- упорядоченное множество, $f: X\to X$ "--- отображение, и для некоторой точки $x_{0}\in X$ существует хотя бы одна цепь $C\subseteq F(T_{X}(x_{0}))$, где $T_{X}(x_{0}):=\{x\in X : x\preceq_{d\varphi}x_{0}\}$, удовлетворяющая условиям: (1) $ f(x)\preceq x,\forall x\in C;$ (2) если $u,v\in C, u\prec v$, то $u\preceq f(v)$. Пусть, кроме того, для каждой цепи $C$, удовлетворяющей условиям (1)-(2), существует такая нижняя граница $w$, что $f(w)\preceq f(x), x\in C$, и $f(w)\preceq w$. Причём, если $f(w)\prec w$, то $f(f(w))\preceq f(w)$. Тогда множество $Fix(f)$ неподвижных точек отображения $f$ непусто.}

Отметим, что в условиях Теоремы 2 можно утверждать также наличие минимального элемента в множестве $Fix(f)$.


Пусть $(X,d)$ "--- метрическое пространство, и заданы функционал $\varphi: X\to R$ и отображение $f: X\to X$. Обозначим $T_{X}^{d\varphi}(x_{0}):=\{y\in X | y\preceq_{d \varphi}x\}$, где $y\preceq_{d \varphi}x$ означает, что $d(y,x)\le \varphi(x)-\varphi(y)$. Этот способ упорядочения метрического пространства предложен в работе Брондстеда [3].

Следующая теорема является метрическим аналогом Теоремы 2.

\textbf{Теорема~3 [2].} {\it  Пусть $(X,d)$ "--- метрическое пространство, задан  функционал $\varphi :X\to R$ и  отображение $f: X\to X$. Пусть для некоторой точки $x_{0}\in X$ в упорядоченном множестве $(X, \preceq_{d \varphi})$ существует хотя бы одна цепь $C\subseteq f(T_{X}^{d \varphi}(x_{0}))$, удовлетворяющая условиям: (1) $f(x)\preceq_{d \varphi} x, \forall x\in C;$ (2) если $u,v\in C,$ и $u\prec_{d \varphi} v$, то $u\preceq_{d \varphi} f(v)$. Пусть, кроме того, для каждой цепи $C$, удовлетворяющей условиям (1)-(2), существует такая нижняя граница $w$, что $f(w)\preceq_{d \varphi} f(x), x\in C$, $f(w)\preceq_{d \varphi} w$. Причём, если $f(w)\prec_{d \varphi} w$, то $f(f(w))\preceq_{d \varphi} f(w)$. Тогда множество $Fix(f)$ неподвижных точек отображения $f$ непусто.}

В [2] показано, что  из Теоремы 3 следует известная теорема о неподвижной точке Каристи (Caristi).

Отметим, что в Теореме 3 можно утверждать даже больше, а именно, что в множестве $Fix(f)$ имеется минимальная точка (относительно порядка $\preceq_{d\varphi}$), то есть такая точка $a\in Fix(f)$, что для любой точки $b\in Fix(f)$ выполнено условие
$$
\left [\begin{array}{rl}d(a,b)>|\varphi(a)-\varphi(b)|;&
\\d(a,b)\le\varphi(b)-\varphi(a).\\
\end{array}
\right.
$$

Кроме того, доказана более общая теорема о существовании неподвижных точек многозначного отображения упорядоченного множества в себя, получено следствие из неё для  метрических пространств (с помощью упорядочения метрического пространства методом Брондстеда). Это следствие представляет обобщение теоремы 3 (а значит, и теоремы Каристи) на случай многозначного отображения метрического пространства в себя.

Следует отметить, что теорема Каристи не вытекает из метрических аналогов предыдущих совместных результатов автора и Д.А.Подоприхина, поскольку в ней отсутствует требование изотонности отображения $f$ относительно порядка Брондстеда, определяемого заданным функционалом $\varphi$ и метрикой $d$.

Используя упорядочение Брондстеда, исследуются также проблемы существования совпадений многозначных отображений метрических пространств.

% Оформление списка литературы
\smallskip \centerline {\bf Литература} \nopagebreak

1. {\it Kirk W.A. and  Sims B. (eds.)} Handbook of metric fixed point theory. {\it Springer Science \& Business Media},  2001.

2. {\it Фоменко Т.Н.}  Неподвижные точки и совпадения семейств отображений упорядоченных множеств и некоторые метрические следствия. {\it Известия РАН. Серия математическая}, 2019, {\bf 83}, № 1, 168-191.

3. {\it  Br\o ndsted A.} On a lemma of Bishop and Phelps. {\it Pacific J. Math.}, 1974, {\bf 55}, 335--341.
