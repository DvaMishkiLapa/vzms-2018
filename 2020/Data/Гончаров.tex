\vzmstitle{О НЕКОТОРЫХ ЭКСТРЕМАЛЬНЫХ ЗАДАЧАХ ДЛЯ СОБСТВЕННЫХ ЗНАЧЕНИЙ}
\vzmsauthor{Гончаров}{В.\,Ю.}
\vzmsinfo{Московский авиационный институт (национальный исследовательский университет); {\it fulu.happy@gmail.com}}
\vzmsauthor{Муравей}{Л.\,А.}
\vzmsinfo{РГУ имени А.Н. Косыгина; {\it l\_muravey@mail.ru}}
\vzmscaption



\par
Экстремальные задачи, в которых функционал цели
\linebreak
представляет
собой собственное число
некоторой краевой задачи для обыкновенного дифференциального уравнения,
встречаются в различных приложениях.
%
%
%
В работе рассматриваются две такие задачи, относящиеся к классу задач оптимального проектирования
элементов механических конструкций, связанных с потерей устойчивости.
%
%
%
%
%
\par
Первая задача состоит в определении оптимального распределения толщины обшивки тонкого прямого
крыла,
для которого критическая скорость дивергенции
(превышение которой приводит к разрушению крыла)
максимальна, а масса имеет заданное значение.
%
%
%
Краевая задача на собственные значения имеет вид
\[
\begin{gathered}
(a^3 u \theta')'(x) + \lambda \, (a^2 \theta)(x) = 0,
\quad
x \in \Omega \triangleq (0, 1),
\\
\theta(0) = 0,
\quad
(a^3 u \theta')(0) = 0.
\end{gathered}
\tag{$\mathcal{E}_1$}
\]
Здесь
$u$~--- фиксированное распределение толщины обшивки крыла,
выступающее в качестве управляющей переменной;
$a$~--- длина хорды профиля крыла.
%
%
%
Наименьшее собственное значение $\lambda_1[u]$ краевой задачи
($\mathcal{E}_1$)
соответствует критической
скорости дивергенции.
%
%
%
Собственная функция $\theta$,
соответствующая первому собственному
значению краевой задачи
($\mathcal{E}_1$),
представляет собой распределение угла закручивания крыла по
размаху при скорости дивергенции.
%
%
%
Будем предполагать,
что
$a \in C^{0, 1}(\bar{\Omega})$, $a(x) \geq a_0 > 0$ для всех $x \in \Omega$.
%
%
%
Пусть
$0 < \alpha < \beta < +\infty$.
%
%
%
%
%
\par
Масса обшивки крыла определяется формулой
\[
M[u] = \int_0^1 a(x) u(x) \, dx.
\]
Введём множество
\[
\mathcal{U}(m)
=
\{
u \in L_\infty(\Omega) :
\;
\alpha \leq u(x) \leq \beta,
\;
M[u] = m
\}
\]
допустимых распределений толщины обшивки крыла,
предполагая, что параметры $\alpha$, $\beta$ и $m$
выбраны так, что множество $\mathcal{U}(m)$ является непустым.
%
%
%
Математическая формулировка рассматриваемой задачи имеет следующий вид:
\[
\text{найти\;\;} \hat{u} \in \mathcal{U}(m) : \;
\lambda_1[\hat{u}] = \sup_{u \in \mathcal{U}(m)} \lambda_1[u].
\tag{$\mathcal{P}_1$}
\]
%
%
%
\par
Поскольку
допустимые управления принадлежат
классу существенно ограниченных измеримых функций,
оптимальное решение может не представлять практического интереса,
например, в том случае,
когда получаемая форма крыла будет иметь ступенчатые переходы.
%
%
%
В связи с этим возникает необходимость исследования вопроса регулярности оптимального решения,
т.\,е. определения класса функций,
обладающих некоторой степенью гладкости (возможно, обобщённой),
к которому указанное оптимальное решение относится.
%
%
%
\par
Одно из первых исследований,
посвящённых теме задач оптимального проектирования крыльев летательных аппаратов,
было проведено в работе Макинтоша и Истепа [1].
%
%
%
В указанной работе была поставлена близкая к ($\mathcal{P}_1$) задача,
состоящая в минимизации массы обшивки крыла при заданном значении скорости дивергенции, и получено выражение для оптимального решения в случае линейного распределения хорды.
%
%
%
Их подход был обобщён Н.\,В.~Баничуком в работе [2] на случай переменных параметров крыла и
более общего типа краевых условий.
%
%
%
Примечательно,
что в [1] и [2] на распределение толщины обшивки крыла не накладываются никакие ограничения и, как следствие, полученные там оптимальные решения обращаются в нуль в точке, соответствующей свободному концу крыла.
%
%
%
Поэтому представляет интерес рассмотрение случая
наличия положительной нижней грани для значений толщины обшивки крыла,
который учитывается в приведённой нами постановке.
%
%
%
Задача минимизации массы обшивки прямоугольного крыла,
отвечающая случаю $a(x) \equiv 1$,
при заданном значении скорости дивергенции была
рассмотрена Ж.-Л.~Арманом и В.\,Дж.~Витте в [3], где рассмотрен как случай без ограничений, так
и случай общей положительной нижней грани для значений толщины, а также дан вывод аналитических решений.
%
%
%
Кроме того, следует отметить работу Ю.\,А.~Арутюнова и А.\,П.~Сейраняна [4],
в которой рассмотрена задача минимизации массы обшивки крыла при дополнительных интегральных ограничениях.
%
%
%
Подробный литературный обзор по задачам оптимального проектирования
летательных аппаратах в связи с явлением дивергенции
приводится в работе [5].
%
%
%
\par
Приведём основные результаты для задачи ($\mathcal{P}_1$).
\par
\textbf{Теорема~1.} {\it Существует единственное решение задачи \emph{($\mathcal{P}_1$)},
причём элемент $\hat{u} \in \mathcal{U}(m)$ является решением тогда и только тогда,
когда
для некоторой не равной тождественно нулю функции
\[
\hat{y} \in W \triangleq \{ y \in H^1(\Omega) : \; y(1) = 0 \}
\]
пара $(\hat{u}, \hat{y})$ является седловой точкой функционала
\[
\Lambda(u, y) = \int_0^1 \frac{y'^2(x)}{a^2(x)} \, dx \Bigg/
\int_0^1 \frac{y^2(x)}{a^3(x) u(x)} \, dx
\]
на множестве
$\mathcal{U}(m) \times \left(W \setminus \{ 0 \}\right)$,
т.\,е.
выполняются следующие неравенства}
\[
\Lambda(u, \hat{y})
\leq
\Lambda(\hat{u}, \hat{y})
\leq
\Lambda(\hat{u}, y),
\;\;
(u, y) \in \mathcal{U}(m) \times \left(W \setminus \{ 0 \}\right).
\tag{SP}
\]
%
%
%
\par
Необходимое и достаточное условие оптимальности (SP) позволяет заменить задачу ($\mathcal{P}_1$)
эквивалентной задачей поиска седловой точки функционала $\Lambda(\cdot, \cdot)$.
%
%
%
Сделанное замечание лежит в основе доказательств следующих утверждений.
%
%
%
\par
\textbf{Теорема~2.} {\it Решение $\hat{u}$ задачи \emph{($\mathcal{P}_1$)}
является непрерывной по Липшицу функцией.
%
%
%
Кроме того,
задача \emph{($\mathcal{P}_1$)} является корректной в смысле Адамара относительно параметра
$m$,
причём непрерывная зависимость от параметра $m$ для оптимального решения имеет место в смысле сходимости по норме пространства $C^{0,\kappa}(\bar{\Omega})$,
где $\kappa \in (0, 1)$.
}
%
%
%
\par
\textbf{Теорема~3.} {\it Пусть $u_0 \in \mathcal{U}(m)$.
%
%
%
Рассмотрим последовательность $\{ u_n \}$, каждый элемент которой определяется как решение
экстремальной задачи
\[
\Lambda(u_{n + 1}, y_n) = \sup_{u \in \mathcal{U}(m)} \Lambda(u, y_n),
\]
где
$y_n$ обозначает собственную функцию, отвечающую собственному значению $\lambda_1[u_n]$.
%
%
%
Тогда если
\[
\frac{1}{u_n}
\overset{*}{\rightharpoonup} \frac{1}{u_*},
\]
то последовательность $\{ u_n \}$ в действительности сходится
к оптимальному решению $\hat{u}$ в $C^{0,\kappa}(\bar{\Omega})$
для любого значения $\kappa \in (0, 1)$.
}
%
%
%
\par
Теорема~3 в сущности даёт метод отыскания оптимального решения,
причём если в результате численного решения последовательность приближений
сходится поточечно,
то она сходится к оптимальному решению в соответствии
с утверждением теоремы~3 в пространствах Гёльдера.
%
%
%
В рамках обсуждения предполагается также
уделить внимание вопросу о сходимости последовательностей приближений к оптимальному решению,
порождаемыми другими итерационными методами.
%
%
%
\par
Доказательства всех приведённых утверждений для задачи ($\mathcal{P}_1$) приводятся в [5].
%
%
%
\par
Вторая часть обсуждения посвящена задаче об определении оптимальной формы жёстко фиксированной неоднородной колонны,
для которой критическая сила,
приводящая к потере устойчивости колонны, максимальна,
а масса колонны имеет заданное значение.
%
%
%
Эта проблема является вариацией известной задачи Лагранжа о наивыгоднейшем очертании колонны.
%
%
%
\par
Краевая задача на собственные значение имеет следующий вид:
\[
\begin{gathered}
(e u^\nu y'')''(x) + \lambda y''(x) = 0,
\quad
x \in \Omega,
\\
y(0) = y'(0) = y(1) = y'(1) = 0.
\end{gathered}
\tag{$\mathcal{E}_2$}
\]
Будем предполагать,
что
$e, \rho \in L_\infty(\Omega)$,
причём для некоторых положительных постоянных $e_0$, $\rho_0$ и п.\,в. $x \in \Omega$
выполняются неравенства
$e(x) \geq e_0$, $\rho(x) \geq \rho_0$.
%
%
%
При указанных предположениях будем рассматривать обобщённую постановку задачи ($\mathcal{E}_2$).
%
%
%
Здесь $e$~--- модуль упругости материала колонны,
а
$\rho$~--- его переменная плотность.
%
%
%
Как и ранее, $u$ выступает управляющей переменной,
но характеризует площадь поперечных сечений колонны.
%
%
%
Параметр $\nu > 0$ задаёт тип сечения.
%
%
%
Первое собственное значение $\lambda_1[u]$ краевой задачи ($\mathcal{E}_2$) соответствует критической силе,
при которой происходит потеря устойчивости колонны.
%
%
%
Масса колонны определяется формулой
\[
M[u] = \int_0^1 \rho(x) u(x) \, dx.
\]
%
%
%
Для заданного значения $m$ массы колонны определим множество
\[
\mathcal{U}
=
\{
u \in L_\infty(\Omega) :
\;
\alpha \leq u(x) \leq \beta,
\;
M[u] = m
\}
\]
допустимых распределений площади поперечных сечений колонны,
которое будем считать непустым.
%
%
%
Рассматриваемая задача формулируется следующим образом:
\[
\text{найти\;\;} \hat{u} \in \mathcal{U} : \;
\lambda_1[\hat{u}] = \sup_{u \in \mathcal{U}} \lambda_1[u].
\tag{$\mathcal{P}_2$}
\]
Можно показать, что функционал $\lambda_1[\cdot]$ в задаче ($\mathcal{P}_2$)
является слабо* полунепрерывным сверху при $\nu \in (0, 1]$ (см. [6]).
%
%
%
Вместе с тем в приложениях параметр $\nu$ принимает только натуральные значения.
%
%
%
Поэтому возникает необходимость в отыскании другого способа,
позволяющего установить разрешимость задачи ($\mathcal{P}_2$) и для остальных значений параметра $\nu$.
%
%
%
%
%
\par
Задача ($\mathcal{P}_2$) для однородной колонны, отвечающей случаю $e(x) = \rho(x) \equiv 1$,
подробно была рассмотрена в работе [7],
в которой
доказательство существования оптимального решения проведено с использованием
соображений симметрии
и
утверждения о существовании положительной собственной функции, соответствующей первому собственному значению
задачи ($\mathcal{E}_2$).
%
%
%
Указанные особенности доказательства объясняются тем,
что в общей ситуации
при осуществлении предельного перехода получившаяся краевая задача
может отвечать старшему собственному значению.
%
%
%
Отмеченная трудность, естественно, возникает и при исследовании задачи
($\mathcal{P}_2$) в приведённой нами постановке.
%
%
%
Заметим, что эта проблема в принципе является общей для ряда задач оптимального проектирования,
постановка которых вовлекает собственные значения в качестве основного функционала.
%
%
%
Кроме того,
представляют интерес способы доказательства разрешимости
экстремальных задач для собственных значений,
не использующие характерных свойств собственных функций,
сохраняющихся при предельных переходах,
поскольку информации о них просто может и не быть.
%
%
%
%
%
\par
Основной результат для задачи ($\mathcal{P}_2$) заключён в
следующем утверждении.

\textbf{Теорема~4.} {\it Задача \emph{($\mathcal{P}_2$)} об определении оптимальной формы неоднородной колонны обладает решением для любого $\nu > 0$.}
%
%
%
\par



% Оформление списка литературы
\smallskip \centerline {\bf Литература} \nopagebreak

1. {\it McIntosh S.C., Eastep F.E.}
Design of minimum-mass structures with specified stiffness properties
// AIAA Journal. 1968. Vol.~6. No.~5. Pp.~962\nobreakdash--964.

2. {\it Баничук Н.В.}
Минимизация веса крыла при ограничении по скорости дивергенции
// Уч. зап. ЦАГИ. 1978. Т.~9. \textnumero~5. С.~97\nobreakdash--103.

3. {\it Armand J.-L., Vitte W.J.}
Foundations of aeroelastic
\linebreak
optimization and some applications to continuous system.
\linebreak
Stanford: Stanford University, 1970.


4. {\it Арутюнов Ю.А., Сейранян А.П.}
Применение принципа максимума к задаче минимизации веса крыла
летательного аппарата
// Уч. зап. ЦАГИ. 1973. Т.~4. \textnumero~1. С.~55\nobreakdash--70.


5. {\it Гончаров В.Ю., Муравей Л.А.}
Минимизация массы тонкого прямого крыла при ограничении по скорости дивергенции
// ЖВМ. 2019. Т.~59. \textnumero~3. С.~465\nobreakdash--480.

6. {\it Goncharov V.Yu.}
Existence criteria in some extremum problems involving eigenvalues of elliptic operators
// J. Sib. Fed. Uni. 2016. Vol.~9. No.~1. Pp.~37\nobreakdash--47.

7. {\it Cox S.J., Overton M.L.}
On the optimal design of columns against buckling
// SIAM J. Math. Anal. 1992. Vol.~23. No.~2. Pp.~287\nobreakdash--325.
