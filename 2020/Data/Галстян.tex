\vzmstitle[\footnote{Исследование выполнено в рамках Программы Президента Российской Федерации для государственной поддержки ведущих научных школ РФ (грант НШ-2554.2020.1).}]{ПРОБЛЕМА ФЕРМА--ШТЕЙНЕРА В ПРОСТРАНСТВЕ КОМПАКТНЫХ ПОДМНОЖЕСТВ $\mathbb{R}^m$ С МЕТРИКОЙ ХАУСДОРФА}
\vzmsauthor{Галстян}{А.\,Х.}
\vzmsinfo{Москва; {\it ares.1995@mail.ru}}
\vzmscaption

Проблема Ферма--Штейнера состоит в поиске всех точек метрического пространства $Y$ таких, что сумма расстояний от каждой из них до точек из некоторого фиксированного конечного подмножества $A$ пространства $Y$ минимальна [1]. Мы изучаем эту проблему в случае, когда $Y$ "--- это пространство компактных подмножеств евклидового пространства $\mathbb{R}^m$, наделённого метрикой Хаусдорфа, а точки из $A$ "--- это конечные попарно непересекающиеся компакты в $\mathbb{R}^m$.

Множество $A = \{A_1, \ldots, A_n\}$ изначально заданных компактных подмножеств называется \emph{границей}, а каждое $A_i$ — граничным компактом. Положим $A_i = \bigcup_{j=1}^{m_i} \{a^i_j\}$. Подмножества, которые реализуют минимум суммы расстояний до граничных компактов, называются \emph{компактами Штейнера}. Мы обозначаем множество всех компактов Штейнера через $\Sigma(A)$. Оно разбивается на попарно непересекающиеся классы $\Sigma_d(A)$, где $d = (d_1, \ldots, d_n)$, а $d_i$ "--- это расстояние Хаусдорфа от компакта $K\in \Sigma_d(A)$ до $A_i$.

Известно [2], что каждый класс $\Sigma_d(A)$ содержит в себе единственный компакт $K_d$, который максимален по включению, а также некоторое количество минимальных по включению компактов. Через $B_r(y) \subset Y$ мы обозначаем замкнутый шар с центром в точке $y$ радиуса $r$. Мы показываем, что для каждых $d, i$ и $j$ справедливо $B_{d_i} (a^i_j) \cap K_d \neq \emptyset$. Если это множество конечно, мы обозначаем его через $\operatorname{HP}(a^i_j)$, иначе полагаем $\operatorname{HP}(a^i_j)=\emptyset$. Также пусть $\operatorname{HP}(A)=\bigcup \operatorname{HP}(a^i_j)$. Все результаты ниже справедливы для любой конечной границы $A = \{A_1,\ldots,A_n\}$.

\textbf{Теорема~1.} {\it Существуют такие $i \in \{1,\ldots,n\}$ и $j \in \{1,\ldots,m_i\}$, что множество $B_{d_i} (a^i_j) \cap K_d$ имеет лишь конечное число точек.}

\textbf{Следствие~2.} {\it Пусть $K$ "--- компакт Штейнера в классе $\Sigma_d(A)$. Тогда $K\cap \partial K_d\neq \emptyset$.}

\textbf{Теорема~3.} {\it Минимальный компакт $K_\lambda \in \Sigma_d(A)$ является конечным множеством, количество его точек не превосходит $\sum_{i=1}^n m_i - n + 1$, а в случае, когда имеется больше одного $i$, для которого $m_i > 1$, оно не превосходит $\sum_{i=1}^n m_i - n$, где $n$ "--- количество граничных компактов. В обоих случаях оценка точна.}

\textbf{Теорема~4.} {\it Компакт $K \subset {\mathbb R^m}$ является минимальным компактом Штейнера в классе $\Sigma_d(A)$ тогда и только тогда, когда одновременно выполняются три следующие условия$:$
	\begin{itemize}
		\item[$(1)$] $K$ "--- конечное подмножество $K_d;$
		\item[$(2)$] $K\cap B_j^i\neq \emptyset$ для любых $i\in \{1,\ldots,n\}$ и $j\in \{1,\ldots,m_i\};$
		\item[$(3)$] Для любого $p \in K$ существуют $i\in \{1,\ldots,n\}$ и $j\in \{1,\ldots,m_i\}$ такие, что $\bigl(K\setminus \{p\}\bigr)\cap B_j^i=\emptyset$.
\end{itemize}}

\textbf{Теорема~5.} {\it Минимальный компакт Штейнера $K_\lambda$ "--- единственный минимальный в $\Sigma_d(A)$ тогда и только тогда, когда для каждой точки $p \in K_\lambda$ существует точка $a^i_j$ такая, что $\operatorname{HP}(a^i_j)=\{p\}$.}

Будем говорить, что на точке $p\in \mathbb R^m$ \emph{реализуется расстояние $d_i$}, если существует $a_j^i\in A_i$ такая, что $|a_j^ip|=d_i$.

\textbf{Теорема~6.} {\it Для каждого минимального компакта $K_\lambda \in \Sigma_d(A)$ и любого номера $i$ существует точка $p\in K_\lambda$ такая, что на ней реализуются по крайней мере два расстояния $d_i$ и $d_k$ $(i\neq k)$.}

Автор выражает благодарность своим научным руководителям, профессору А.~А.~Тужилину и профессору А.~О.~Иванову, за постановку задачи и постоянное внимание к ней в процессе совместной работы.



% Оформление списка литературы
\litlist

\selectlanguage{english}

1. {\it Ivanov A., Tuzhilin A.} Branching Solutions To One-Dimensional Variational Problems. World Scientific, 2001. 364 p.

2. {\it Ivanov A., Tuzhilin A., Tropin A.} Fermat–Steiner problem in the metric space of compact sets endowed with Hausdorff distance. Journal of Geom. 108, 2017. 575--590 pp.
