\vzmstitle{ОБ ОДНОЙ СТОХАСТИЧЕСКОЙ МОДЕДИ ОПТИМАЛЬНЫХ ДИНАМИЧЕСКИХ ИЗМЕРЕНИЙ}
\vzmsauthor{Келлер}{А.\,В.}
\vzmsauthor{Замышляева}{А.\,А.}
\vzmsauthor{Манакова}{Н.\,А.}
\vzmsinfo{Воронеж, Челябинск; {\it alevtinak@inbox.ru}}
\vzmscaption

	Пусть $\Omega \equiv (\Omega ,{\rm A},P)$ -- полное вероятостное пространство,
$\mathbf{R}$ -- множество вещественных чисел, наделённое борелевской
$\sigma $-алгеброй.
Измеримое отображение $\xi :\Omega \to {\mathbf R}$ -- случайная величина. Множество случайных величин с $E\xi =0$ и конечной дисперсией образует гильбертово пространство $\bf{L}_2$ cо скалярным произведением $<\xi _{1} ,\xi _{2}>=E(\xi _{1} \xi _{2}) $. Пусть
$I \subset {\mathbf R}$ -- некоторый интервал. Отображение $\eta :I \times \Omega \to {\mathbf R}$ вида $\eta =\eta (t,\omega )$ -- одномерный стохастический процесс, для каждого фиксированного $t\in I $ значение отображения $\eta =\eta (t,{\kern 1pt} \, \cdot \, )$
 является случайной величиной, т. е. $\eta =\eta (t,{\kern 1pt} \, \cdot \, )\in {\bf L}_{2} $ и для любого фиксированного $\omega \in \Omega $ значение стохастического процесса $\eta =\eta ({\kern 1pt} \, \cdot \, ,\omega )$ называется (выборочной) траекторией.
 Обозначим: $C{\bf L}_{2} $ -- пространство непрерывных случайных процессов, $\mathop{\eta }\limits^{{\rm o}}{}^{(\ell )} $ -- $\ell $-ю производную Нельсона-Гликлиха случайного процесса $\eta $ [1].
 Множество непрерывных стохастических процессов, имеющих непрерывные производные Нельсона-Гликлиха до порядка $k\in {\mathbf N}$ в каждой точке множества $I$, образуют пространство $C^{k} {\bf L}_{2} $.
	Рассмотрим стохастическую систему леонтьевского типа:
				$$
				L\mathop{\xi }\limits^{{\rm o}} =A\xi +B (u+\varphi), \eqno{(1)}
				$$
где $u:I\to{\mathbf R}^{n}$-- вектор"=функция, $\varphi$ - стохастический процесс. Пусть матрица $A$ -- ${\rm (}L,p{\rm )}$-регулярна, $p\in \{ 0\} \cup {\mathbf N}$, и начальные состояния (1) описываются начальным условием Шоултера-Сидорова:
$$
			\left[(\alpha L-A)^{-1} L\right]^{p+1} (\xi (0)-\xi _{0} )=0, \eqno{(2)}
$$
где $\xi _{0} =\sum _{k=0}^{n}\xi _{0,k} e_{k} $, $\xi _{0,k} $ -- попарно независимые гауссовские случайные величины, а $\left\{e_{k} \right\}_{k=1}^{n} $ является ортонормированным базисом в ${\mathbf R}^{n} $.

\textbf{Теорема~1.} {\it
Для любой вектор"=функции $u \in C^{p+1} (I ,{\mathbf R}^{n} )$, начальных значений $\xi _{0}$ и стохастического процесса $\varphi \in C^{p+1} {\bf L}_{2} (I ,{\mathbf R}^{n} )$, независимых для любого $t\in I $, существует единственное решение $\xi $ задачи (1), (2), заданное формулой
 $$
  \xi (t)=\xi_{u}(t)+\xi_{\varphi} (t), \ \xi_u \in C^{1}(I ,{\mathbf R}^{n}),\ \xi_{\varphi} \in C^{1} {\bf L}_{2} (I ,{\mathbf R}^{n} ),
  $$
где
$$
\xi_{u}(t)=\!\int _{0}^{t}U ^{t-s} L_{1}^{-1} Qu (s)ds+
$$
$$
+\sum _{q=0}^{p}\left(M^{-1} \left(I_{n} -Q\right)L\right)^{q} M^{-1} \left(Q-I_{n} \right)u^{(q)} (t)
$$
- это детерминированная часть, а
$$
\xi_{\varphi} (t)=U ^{t} \xi _{0} +\int _{0}^{t}U ^{t-s} L_{1}^{-1} Q\varphi (s)ds+
$$
$$
+\sum _{q=0}^{p}\left(M^{-1} \left(I_{n} -Q\right)L\right)^{q} M^{-1} \left(Q-I_{n} \right) \mathop{\varphi }\limits^{{\rm o}}{}^{(q)} (t)
$$
- это стохастическая часть решения.
	
	Здесь $U ^{t} =\lim \limits_{r \to \infty } \left( \left( L-{\frac{t}{r}} M\right)^{-1} L \right)^{r} $,
	$Q=\lim \limits_{r \to \infty} \left( rL_{r}^{L} (M) \right)^{p} $,
	$L_{r}^{L} (M) \! = \!\! L \! \left(L-{\frac{1}{r}} M \right)^{-1} \!\!$,
	 $I_{n} \!$ -- единичная матрица порядка n.
}

%Математическая модель измерительного устройства (ИУ) в детерминированном случае представима %системой леонтьевского типа:
%$$
%\left\{\begin{array}{c} {L\dot{x}=Ax+Bu ,} \\ {y=Cx,} \end{array}\right. \eqno{(1)}
%$$
%где $L$ и $A$ -- матрицы, характеризующие структуру ИУ, причем в некоторых случаях возможно, что
%$\det L=0$, $x(t)$ и $\dot{x}(t)$ являются вектор-функциями состояния ИУ и скорости изменения состояния, соответственно; $y(t)$ -- вектор-функция наблюдения; $C$ и $D$ - прямоугольные матрицы, характеризующие взаимосвязь между состоянием системы и наблюдением; $u(t)$ - вектор-функция измерений; $B$ - матрица, характеризующая взаимосвязь между состоянием системы и измерением; $\eta(t)$ - вектор-функция возмущения на выходе ИУ.

%Подчеркнем, что все наблюдения и измерения в системе (1) являются моделируемыми или «виртуальными».
%	Начальное условие Шоултера-Сидорова:
%	$$
%\left[\left(\alpha L-A\right)^{-1} L\right]^{p+1} \left( x(0)-x_{0} \right)=0 			              %\eqno{(2)}
%$$
%определяет состояние ИУ.
%	Основной целью теории оптимальных динамических измерений является восстановление динамически искаженного входного сигнала (измерения) $u(t)$  в соответствии с данным наблюдением $y_{0} (t)$. При использовании этого подхода ключевым понятием является оптимальное динамическое измерение $v(t)$, которое строится как минимум функционала:
%$$
%J(v)=\mathop{\min }\limits_{u\in U_{\partial } } J(x(u),u)
%$$		               	
%на множестве допустимых измерений $U_{\partial } $, где пара $(x(u),u)$ удовлетворяет системе (1), а $U_{\partial }$ содержит априорную информацию об измерениях. Функционал $J(x(u),u)$ отражает оценку близости фактического наблюдения $y_{0} (t)$ и виртуального наблюдения $y(t)$, полученного из (1). В настоящее время в рамках теории оптимальных динамических измерений детерминированный случай хорошо изучен [2], построены алгоритмы решения таких задач [3]. Однако детерминированная проблема не учитывает эффекты случайных помех, которые всегда присутствуют в реальных процессах, поэтому было предложено использовать стохастическую модель ИУ [1]:
 %$$\left\{\begin{array}{c} L{\mathop{\xi }\limits^{{\rm o}}=A\xi+B(u+\varphi) ,} \\ {\eta=C\xi+\nu ,} \end{array}\right.$$
%$$
%\left[\left(\alpha L-A\right)^{-1} L\right]^{p+1} \left(x(0)-x_{0} \right)=0.
%$$

%Здесь матрицы $L,A,B,C$ имеют тот же смысл, что и в (1). Случайные процессы $\varphi$ и $\nu$ определяют шумы в цепях и на выходе ИУ, соответственно.

Аналогично детерминированному случаю введём прост\-ранст\-во измерений $U \!=\! \{ u \in L_{2} (I,{\mathbf R}^{n}) : u^{(p+1)} \in L_{2} (I,{\mathbf R}^{n} )\} $, выделим в нем замкнутое выпуклое множество допустимых измерений $U_{\partial}\subset U$, которое содержит априорную информацию об измерениях. Задача восстановления динамически искажённого сигнала сводится к решению задачи оптимального управления
 $$
 J(v)=\mathop{\min }\limits_{u\in U_{\partial } } J(u),
 $$
для (1) с условием (2), где функционал качества
  $$ J(u)=J(\eta (u))=\sum _{k=0}^{1}\int _{0}^{\tau }E\left\| \mathop{\eta }\limits^{{\rm o}}{}^{(k)} (t)-\eta _{0}^{(k)} (t)\right\| ^{2} dt
  $$
отражает близость реального наблюдения $\eta_{0} (t)$ и виртуального наблюдения $\eta (t)$, полученного на основе математической модели измерительного устройства (ИУ) (1), (2).
	 Точка минимума $v(t)$ функционала на множестве $U_{\partial } $, являющаяся решением задачи оптимального управления, называется оптимальным динамическим измерением. На практике существует только косвенная информация о $v(t)$.
	
\textbf{Теорема~2.} {\it
Оптимальное динамическое измерение не зависит от случайных начальных условий, шумов в цепях и на выходе МТ.	
	}
	
Таким образом, доказательство теоремы приводит к возможности применения численных алгоритмов, разработанных для детерминированного случая [2], при решения задачи восстановления измеренного сигнала. Заметим, что модель ИУ (1), (2) при $\det L=0$ представлена в [3].


% Оформление списка литературы
\smallskip \centerline {\bf Литература} \nopagebreak

1. {\it Gliklikh Yu.E., Mashkov E.Yu.} Stochastic Leontieff Type Equations and Mean Derivatives of Stochastic Processes // Bulletin of the South Ural State University. Series: Mathematical Modelling, Programming and Computer Software, 2013, vol.~6, pp.~25--39.

2. {\it Shestakov A.L., Keller A.V., Sviridyuk G.A.} Optimal Measurements XXI IMEKO World Congress <<Measurement in Research and Industry>>, 2015, pp.~2072--2076.

3. {\it Khudyakov Yu.V.} On mathematical modeling of the measu- \- rement transducers
// Journal of Computational and Engineering Mathematics, 2016, vol.~3, no.~3, pp.~68--73.
