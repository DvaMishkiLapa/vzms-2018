\vzmstitle[\footnote{Исследование выполнено в рамках Программы Президента Российской Федерации для государственной поддержки ведущих научных школ РФ (грант НШ-2554.2020.1).}]{ТОПОЛОГИЧЕСКАЯ КЛАССИФИКАЦИЯ ИНТЕГРИРУЕМЫХ ГЕОДЕЗИЧЕСКИХ БИЛЛИАРДОВ НА КВАДРИКАХ В ТРЕХМЕРНОМ ЕВКЛИДОВОМ ПРОСТРАНСТВЕ}
\vzmsauthor{Белозеров}{Г.\,В.}
\vzmsinfo{Москва; {\it gleb0511beloz@yandex.ru}}
\vzmscaption

Теории математического биллиарда, т. е. задаче о движении материальной
точки в плоской области, ограниченной кусочно"=гладкой кривой с абсолютно
упругим отражением на границе, посвящено много работ.

Биллиарды в областях, ограниченных дугами софокусных квадрик, являются интегрируемыми гамильтоновыми системами.
Такие системы с точностью до лиувиллевой эквивалентности начали изучатьcя
в работах В.~ Драговича, M.~ Раднович [1], [2], а также В.\, В.~ Ведюшкиной (Фокичевой) [3], [4].

Данная работа посвящена интегрируемым геодезическим биллиардам на поверхностях положительной и отрицательной гауссовой
кривизны, a именно на невырожденных квадриках $E$, т.е. на эллипсоиде, однополостном и двуполостном гиперболоидах.
Биллиардным столом на такой квадрике $E$ назовём замкнутую область,
ограниченную конечным числом квадрик, софокусных с данной, и имеющую углы излома на границе равные
$\dfrac{\pi}{2}$. Возникает динамическая система:
материальная точка (шар) движется по биллиардному столу
вдоль геодезических с постоянной по модулю скоростью, отражаясь от границы абсолютно упруго. На
множестве биллиардных столов (на фиксированной квадрике) введём отношение эквивалентности.
Назовём две области эквивалентными
в том и только в том случае, когда они получаются друг из друга элементарными преобразованиями.
Автором получена полная классификация биллиардных столов на квадриках. Оказывается, что
на эллипсоиде есть ровно 21 тип неэквивалентных биллиардных столов, на однополостном гиперболоиде -- 21, а на двуполостном -- 13.


Интегрируемость этих биллиардов следует из известной теоремы Якоби -- Шаля.
Далее автор полностью классифицировал все такие геодезические биллиарды с
точностью до лиувиллевой эквивалентности. Как оказалось, на эллипсоиде их ровно 7, на однополостном гиперболоиде тоже 7, а на двуполостном -- 6.

Далее оказалось, что некоторые геодезические биллиарды на квадриках разного типа лиувиллево эквивалентны.
В итоге, на квадриках в $\mathbb{R}^3$ есть ровно 10 лиувиллево неэквивалентных биллиардов.



Как оказалось, существует частичное соответствие между геодезическими биллиардами на эллипсоиде и плоскими биллиардами
внутри эллипса. При этом омбилические точки заменяются фокусами, а сетка эллиптических координат на
эллипсоиде "--- на сетку на плоскости. Для биллиардов на двуполостном гиперболоиде
автором доказаны следующие 2 теоремы. Первая утверждает существование взаимно однозначного соответствия между
биллиардными столами на двуполостном гиперболоиде и биллиардными столами на плоскости, ограниченными софокусными квадриками. При этом
отношение эквивалентности биллиардных столов сохраняется. Вторая теорема утверждает существование
взаимно однозначного соответствия между плоскими биллиардными системами (ограниченными софокусными квадриками) и геодезическими биллиардными
системами на
двуполостном гиперболоиде. Это соответствие сохраняет лиувиллеву эквивалентность.
Интересно, что для однополостного гиперболоида даже частичного соответствия нет.

Также оказалось, что все найденные геодезические биллиарды на квадриках лиувиллево эквивалентны некоторым известным
интегрируемым системам из физики, механики и геометрии.


% Оформление списка литературы
\litlist


1. {\it V. Dragovich, M. Radnovich}, <<Bifurcations of Liouville tori in elliptical billiards>>, Regul.
Chaotic Dyn., 14:4-5 (2009), 479-494.

2. {\it В. Драгович, М. Раднович}, Интегрируемые биллиарды, квадрики и многомерные
поризмы Понселе, НИЦ <<Регулярная и хаотическая динамика>>, М. - Ижевск, 2010,
338 с.

3. {\it В. В. Фокичева}, <<Описание особенностей системы ``биллиард в эллипсе''>>, Вестн.
Моск. ун-та. Сер. 1 Матем. Мех., 2012, No 5, 31 - 34; англ. пер.: V. V. Fokicheva,
<<Description of singularities for system ``billiard in an ellipse''>>, Moscow Univ. Math.
Bull., 67:5-6 (2012), 217-220.

4. {\it В. В. Фокичева}, <<Описание особенностей системы бильярда в областях, ограниченных
софокусными эллипсами и гиперболами>>, Вестн. Моск. ун-та. Сер. 1 Ма-
тем. Мех., 2014, No 4, 18-27; англ. пер.: V. V. Fokicheva, <<Description of singularities
for billiard systems bounded by confocal ellipses or hyperbolas>>, Moscow Univ. Math.
Bull., 69:4 (2014), 148-158.
