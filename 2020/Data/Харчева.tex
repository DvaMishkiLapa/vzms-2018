\vzmstitle[\footnote{Исследование выполнено в рамках Программы Президента Российской Федерации для государственной поддержки ведущих научных школ РФ (грант НШ-2554.2020.1).}]{МОДЕЛИРОВАНИЕ ГРУБЫХ МОЛЕКУЛ, СОСТОЯЩИХ ИЗ АТОМОВ БЕЗ ЗВЁЗДОЧЕК, ИНТЕГРИРУЕМЫМИИ БИЛЛИАРДАМИ}
\vzmsauthor{Харчева}{И.\,С.}
\vzmsinfo{Москва; {\it irina.harcheva1@yandex.ru}}
\vzmscaption


\textbf{Определение 1.} {Рассмотрим некоторую компактную область $ \Omega $ в плоскости с кусочно-гладкой границей и углами излома $ \pi/2 $. Пусть материальная точка движется по прямой с постоянной скоростью внутри этой области $ \Omega $ и отражается о гладкую часть границы $ \partial \Omega $ без потери скорости и естественным образом: угол падения равен углу отражения. В остальных случаях движение этой материальной точки определяется по непрерывности. Тогда \textit{биллиардом} в области $ \Omega $ называется динамическая система, описываемая движением этой материальной точки.}

Эта динамика задаёт гамильтонову систему на кокасательном расслоении к области $\Omega$. У динамической системы биллиарда есть один первый интеграл -- гамильтониан, равный половине квадрата модуля вектора скорости. Значит, биллиард является гамильтоновой динамической системой с двумя степенями свободы. Из теории гамильтоновых систем следует, что для интегрируемости биллиарда необходим ещё один первый интеграл. В общем случае, для произвольной области $ \Omega $ его может не существовать. Но если подобрать ``хорошую'' область $ \Omega $, то можно найти функцию, которая будет первым интегралом.

Важным классом интегрируемых биллиардов является биллиард в области $\Omega$, ограниченной дугами софокусных эллипсов и гипербол. Оказывается, в таком биллиарде вектор скорости материальной точки на протяжении всей траектории будет направлен по касательной к каустике -- фиксированной квадрике, софокусной с семейством. Поэтому у такой системы появляется ещё один интеграл, независимый с предыдущим -- параметр квадрики $ \Lambda $. Это означает, что динамическая система биллиарда в такой области будет интегрируема по Лиувиллю. Её интегрируемость была показана в работе В.\,В.~Козловa, Д.\,В.~Трещёвa [1].

Расширим постановку биллиардной задачи. Пусть дано n областей $\Omega_1, ... ,\Omega_n$ с кусочно-гладкой границей. Пусть граница этих областей содержит одну и ту же кривую $ l $. Припишем к этой дуге перестановку $ \sigma $ из n элементов. Тогда можно определить более сложный биллиард в объединении областей $\cup_{i = 1}^n \Omega_i$ следующим образом: материальная точка отражается обычным образом от границ, отличных от $ l $, и переходит с одной области на другую по перестановке $ \sigma $, достигая кривой $ l $. Заметим, что общих граничных кривых у областей может быть несколько: $ l_1, l_2, ..., l_k $. Ко всем им можно приписать перестановки $ \sigma_1, \sigma_2, ..., \sigma_k $  и рассмотреть биллиард, в котором материальная точка будет переходить с листа на лист на дугах $ l_1, l_2, ..., l_k $ по перестановкам $ \sigma_1, \sigma_2, ..., \sigma_k $ соответственно. Такие биллиарды будем называть \textit{биллиардными книжками}, а области $ \Omega_i $,  из которых состоит биллиардная книжка -- \textit{листами}. В частном случае, когда $ n = 2 $ такие биллиарды называются топологическими. Топологические биллиарды были полностью классифицированы в работе В.В.\,Фокичевой [2]. В этой работе было обнаружено, что многие известные и важные интегрируемые системы с двумя степенями свободы моделируются топологическими  биллиардами с точностью до лиувиллевой эквивалентности. То есть их инварианты Фоменко-Цишанга (см. [3]) совпадают. В связи с этим А.Т.\,Фоменко предложил следующую гипотезу:

\textbf{Гипотеза. (А.Т. Фоменко)}
{\it Биллиардными книжками можно моделировать:
	
	Гипотеза A. все 3-атомы;
	
	Гипотеза B. все грубые молекулы;
	
	Гипотеза C. все меченые молекулы.
}

\textbf{Teорема 1. (Ведюшкина-Харчева)}  {\it Гипотеза Фоменко A верна, а именно, для любого 3-атома (со звёздочками или без) алгоритмически строится биллиардная книжка, такая что в её изоэнергетической поверхности слоение Лиувилля прообраза окрестности особого значения интеграла $ \Lambda $, отвечающего траекториям, направленным к или от одного из фокусов, послойно гомеоморфно данному атому.}

\textbf{Teорема 2. (Ведюшкина-Харчева)}  {\it Любая грубая молекула, состоящая из атомов без звёздочек, моделируется биллиардными книжками. Более точно: по любой грубой молекуле, состоящей из атомов без звёздочек, алгоритмически строится биллиардная книжка с каноническим квадратичным интегралом $ \Lambda $, отвечающим параметру каустики, такая что грубая молекула, соответствующая этой системе изоморфна заданной изначально грубой молекуле.}



%%%%  ОФОРМЛЕНИЕ СПИСКА ЛИТЕРАТУРЫ %%%
\smallskip \centerline{\bf Литература}\nopagebreak

1. {\it  Козлов В. В., Трещёв Д.В.} Генетическое введение в динамику систем с ударами. М.:  Изд-во МГУ, 1991. 408 с.

2. {\it Фокичева В. В.} Топологическая классификация биллиардов в локально-плоских областях, ограниченных дугами софокусных квадрик. Математический сборник. — 2015. — Т. 206, № 10. С. 127–176.

3. {\it Болсинов А.В., Фоменко А.Т.} Интегрируемые гамильтоновы системы. Геометрия, топология, классификация, том 1. Ижевск НИЦ <<Регулярная и хаотическая динамика>>, 1999.

