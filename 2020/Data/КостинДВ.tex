\vzmstitle{О РЕШЕНИИ ЗАДАЧИ КОШИ ДЛЯ ОДНОГО УРАВНЕНИЯ АЭРОДИНАМИКИ}
\vzmsauthor{Костин}{Д.\,В.}
\vzmsauthor{Прицепов}{М.\,Ю.}
\vzmsauthor{Силаева}{М.\,Н.}
\vzmsinfo{Воронеж; {\it dvk605@mail.ru}; {\it marinanebolsina@yandex.ru}}
\vzmscaption
При исследовании процесса обтекания газовой средой крыла самолёта в [1] c.309 приходят к уравнению
смешанного типа (см. [3])
$$\frac{\partial^{2}u(r,\theta)}{\partial\theta^{2}}=r^{2}(r^{2}-1)\frac{\partial^{2}u(r,\theta)}{\partial r^{2}}+r(r^{2}-1)\frac{\partial u(r,\theta)}{\partial r}, \theta>0.\eqno {(1)}$$

При $r>1$ уравнение (1) описывает сверхзвуковой поток обтекаемой среды и является уравнением гиперболического типа.

При $r\in (0,1)$ "--- это уравнение эллиптического типа и оно описывает звуковой поток.

В настоящем сообщении для $r>1$ рассматривается задача Коши нахождения решения уравнения (1), удовлетворяющего условиям
$$ u(r,0)=\varphi (r),\eqno {(2)}$$
где функция $\varphi(r)$ разложима в ряд
$$\varphi(r)=\sum_{n=0}^{\infty}a_{n}T_{n}(\frac{1}{r}), \eqno {(3)}$$
$T_{n}(x)$- ортогональные многочлены Чебышева 1-го рода.

Справедливо следующее

 \textbf{Утверждение } {\it Решение задачи (1)-(2) существует, единственно и представимо в виде
 $$u(r,\theta)=\sum_{n=0}^{\infty}a_{n}T_{n}(\frac{1}{r})\cos n\theta. \eqno {(4)}$$}
\centerline {\bf Литература}

1. Современное состояние аэродинамики больших скоростей. Т.1. Под редакцией Л.Хоуарта.-Москва, 1955.— 491с.

2. {\it Костин В.А.} $C_0$ — операторные ортогональные многочлены Чебышева и их представления / В.А. Костин, М.Н. Небольсина // Записки научных семинаров ПОМИ, Том 376, 2010, С. 64-88.

3. {\it Смирнов М.М.} Уравнения смешанного типа. М.:Наука, 1970 .— 295 с.

4. {\it Суетин П.К.} Классические ортогональные многочлены. М.:Наука, 1979 .— 415 с.
