\begin{center}
    {\bf МАТЕМАТИЧЕСКИЕ АЛГОРИТМЫ НА ПРАКТИЧЕСКИХ ЗАНЯТИЯХ ПЕРВОКУРСНИКОВ ПО ИНФОРМАТИКЕ И ПРОГРАММИРОВАНИЮ}

    {\it О.Ф. Ускова}

    (Воронеж; {\it sunny.uskova@list.ru})
\end{center}

\addcontentsline{toc}{section}{Ускова О.Ф.}

Роль информационных технологий практически во всех сферах профессиональной деятельности способствует повышению значимости
курса <<Информатика и программирование>>\,, изучение которого начинается на первом курсе факультета прикладной математики,
информатики и механики Воронежского государственного университета \linebreak (ПММ ВГУ).

Изучение информатики и программирования первокурсниками факультета ПММ ВГУ начинается со структурного программирования
на языке C++. Для методической поддержки практических и лабораторных занятий на кафедре математического обеспечения
факультета ПММ разработан задачник-практикум [1]. Основная цель этого учебного пособия --- придать курсу программирования
научно-обоснованный базис, сформировать на его основе определённую культуру разработки программ, структурировать
соответствующим образом учебный процесс. Задачник-~практикум состоит из 12 глав, каждая из которых содержит несколько
разделов: контрольные вопросы, задачи с решениями, тренировочные задания, указания к решению, задания для самостоятельного
решения.

Учитывая особенности специальностей, по которым на факультете ПММ обучаются студенты, достаточное количество заданий
задачника-практикума [1] содержат математические алгоритмы.

Приведём несколько примеров таких заданий для первого семестра студентов 1 курса.

1. Вычислите приближенное значение $\int x^2\,dx$ на отрезке $[a, b]$, используя формулу прямоугольников, если
известно, что отрезок $[a, b]$ разбит на $n$ частей.

2. Пусть даны координаты трёх точек на плоскости. Если они могут быть вершинами треугольника, определите его
вид (прямоугольный, тупоугольный, остроугольный). Вычислите длины его высот и напечатайте их в порядке убывания.

3. При некоторых заданных $x$, $N$ и $E$, определяемых вводом, вычислите

а) сумму $N$ слагаемых заданного вида;

б) сумму тех слагаемых, которые по абсолютной величине больше $E$. Вычисление второй суммы выполните для двух значений $E$,
отличающихся на порядок, при этом определите количество слагаемых, включённых в сумму, вычисляемую для каждого значения
$E$.
Сравните результаты со значением функции, для которой данная сумма определяет приближенное значение при \,$x$, лежащем в интервале
\linebreak
$(-R, R)$, вычисленным с помощью встроенных функций
компилятора.
$$
\frac{\sin x}{x}=1-\frac{x^2}{3!}+\frac{x^4}{5!}-\frac{x^6}{7!}+\dots,
$$
если $R=\infty$.

4. Составить программу нахождения корня $x+\sqrt{x}+\sqrt[3]{x}-2.5=0$ методом деления отрезка $[0.4, 1]$ пополам с точностью
$E=10^{-5}$.

5. Для заданных чисел $a$ и $p$ вычислить $x=\sqrt[p]{a}$ по рекуррентному соотношению (формула Ньютона)
$$
x_{n+1}=\frac{1}{p}[(p-1)x_n+a/x_n^{p-1}]; \quad x_0=a.
$$
Сколько итераций надо выполнить, чтобы для заданной погрешности $E$ выполнялось соотношение $|x_{n+1}-x_n|<E$?

Решая подобные задачи, первокурсники должны овладеть навыками проектирования действий, направленных на решение какой-либо
проблемы или на достижение какой-либо цели~--- основными этапами решения задач с помощью ЭВМ.

Заметим в заключение, что в нашей стране в 1951 году в числе первых разработчиков (с С.А. Авраменко и С.А. Богомолец)
прикладной компьютерной программы для решения дифференциальной краевой задачи второго порядка
$$
y''+y=0, \quad y(0)=y(1)=0,
$$
был математик с мировым именем Селим Григорьевич \linebreak Крейн~[2].
В 60--70-е годы прошлого века С.Г. Крейн, заслуженный деятель
науки, лауреат Государственной премии Украины плодотворно работал в Воронежском государственном университете заведующим кафедрой
уравнений в частных производных, воспитал несколько поколений математиков"=профессионалов. С.Г. Крейн был руководителем
моей дипломной работы, одной из первых в ВГУ дипломных работ по программированию.

% Оформление списка литературы
\smallskip \centerline {\bf Литература} \nopagebreak

1. {\it Ускова О.Ф., Каплиева Н.А., Горбенко О.Д.} Начала структурного программирования на языке C++: задачник-практикум.
Воронеж: Издательский дом ВГУ, 2019. - 261 с.

2. {\it Ускова О.Ф., Каплиева Н.А., Горбенко О.Д.} Российской информатике 70 лет. Актуальные проблемы прикладной математики,
информатики и механики. Сборник трудов Международной научной конференции.
Воронеж: Научно-исследовательские публикации, 2018. - 1672 с. С.~1416--1418.
