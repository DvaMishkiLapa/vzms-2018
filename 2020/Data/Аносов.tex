\vzmstitle{ПРЯМОЕ ДОКАЗАТЕЛЬСТВО НЕОБХОДИМЫХ УСЛОВИЙ М.Г. КРЕЙНА ВЕЩЕСТВЕННОСТИ КОРНЕЙ МНОГОЧЛЕНОВ}
\vzmsauthor{Аносов}{В.\,П.}
\vzmsinfo{Новосибирск, Новосибирский государственный педагогический университет; {\it averi@ngs.ru}}
\vzmscaption

В работе [1] Левина А.Ю. рассматривались функции вида
$$
f(z)=\sum_{k=0}^{n} a_{k} z^{k} \quad (n \leqslant \infty)
$$
с положительными коэффициентами, для вещественности корней которых были приведены необходимые условия, которые сформулируем в виде теоремы~1.

\textbf{Теорема~1.} {\it Условия
	$$
	a_{k}^{2} \geqslant a_{k-1}a_{k+1} \quad (k \geqslant 1)
	$$
	являются необходимыми условиями вещественности корней целой функции с положительными коэффициентами, если её порядок меньше единицы.}

Как утверждает Левин~А.~Ю., эти условия были ранее отмечены М.~Г.~Крейном (см. [1, c.~72]). Мы считаем, что можно дать прямое доказательство этого результата для многочленов. А именно, мы докажем следующее утверждение.

\textbf{Теорема~2.} {\it Если многочлен
	$$
	f_{n}(z)=\sum_{k=0}^{n} a_{k} z^{k} \eqno(1)
	$$
	степени $ n \geqslant 2 $ с положительными коэффициентами имеет только вещественные корни, то необходимо, чтобы для коэффициентов многочлена~(1) была справедлива система неравенств
	$$
	\begin{cases}
	a_{n-1}^{2} \geqslant a_{n} a_{n-2}, & \\
	\dotfill & \\
	a_{n-k}^{2} \geqslant a_{n-k+1} a_{n-k-1}, & \\
	\dotfill & \\
	a_{1}^{2} \geqslant a_{2} a_{0} & .\\
	\end{cases} \eqno(2)
	$$
}

%%% \begin{proof}
	Прежде чем начать доказательство, поясним некоторые обозначения. Систему неравенств в теореме~2 мы обозначили через~(2), а через~$( 2_{k} )$ обозначим $k$-ую строку системы (2) сверху. Это пояснение относится и к последующим системам. А теперь приступим к доказательству теоремы~2 и проведём его методом математической индукции. При $ n = 2 $ данное утверждение легко проверяется.

	Пусть утверждение теоремы~2 справедливо для всех таких многочленов степени $ n = m \geqslant 2 $. Докажем его для многочленов степени $ m + 1 $, то есть докажем неравенство~(2) для $ n = m + 1 $ и $ k = 1, 2, \ldots, m $. Для этого воспользуемся для многочлена $ f_{m+1}(z) $ формулами Виета (см. [2, с.~512])

	$$
	\begin{cases}
		z_{1} + \ldots + z_{m+1} = - \frac{a_{m}}{a_{m+1}}, & \\
		z_{1} z_{2} + \ldots + z_{m} \cdot z_{m+1} = \frac{a_{m-1}}{a_{m+1}}, & \\
		\dotfill & \\
		z_{1} \cdot z_{2} \cdot \ldots \cdot z_{m+1} = (-1)^{m+1} \frac{a_{0}}{a_{m+1}} & ,\\
	\end{cases} \eqno(3)
	$$
	где $ z_{1}, \ldots, z_{m+1} $ — вещественные корни многочлена $ f_{m+1}(z) $. Отметим, что все эти корни отрицательные.

	Наряду с многочленом $ f_{m+1}(z) $ рассмотрим многочлен $ m $-ой степени
	$$
	g_{m}(z) = z^{m} + b_{m-1} z^{m-1} + \ldots + b_{0},
	$$
	корнями которого являются числа $ z_{1}, \ldots , z_{m} $, и для коэффициентов которого $ b_{0}, b_{1}, \ldots, b_{m-1}, b_{m} = 1 $ имеют место следующие равенства
	$$
	\begin{cases}
		z_{1} + \ldots + z_{m} = - b_{m-1}, & \\
		z_{1} \cdot z_{2} + \ldots + z_{m-1} \cdot z_{m} = b_{m-2}, & \\
		\dotfill & \\
		z_{1} \cdot z_{2} \cdot z_{k} + \ldots + z_{m-k+1} \cdot \ldots \cdot z_{m} = (-1)^{k} b_{m-k}, & \\
		\dotfill & \\
		z_{1} \cdot z_{2} \cdot \ldots \cdot z_{m} = (-1)^{m}b_{0} & .\\
	\end{cases} \eqno(4)
	$$
	Из~(4), в силу отрицательности корней $ z_{i} $, вытекает, что $ b_{i} > 0 $ для $ i = 0, 1, \ldots, m-1 $, а также, из предложения индукции, следует справедливость неравенств
	$$ b_{i}^{2} \geqslant b_{i-1} b_{i+1}, \quad (i = 1, 2, \ldots, m-1) . \eqno(5) $$
	Из~(4) и~(3) имеем равенства
	$$
	\begin{cases}
		-b_{m-1} + z_{m+1} = - \frac{a_{m}}{a_{m+1}}, & \\
		b_{m-2} + z_{m+1} (-b_{m-1}) = \frac{a_{m-1}}{a_{m+1}}, & \\
		\dotfill & \\
		(-1)^{k} b_{m-k} + z_{m+1} (-1)^{k-1} b_{m-k+1} = (-1)^{k} \frac{a_{m-k+1}}{a_{m+1}}, & \\
		\dotfill & \\
		b_{0} z_{m+1} = (-1)^{m+1} \frac{a_{0}}{a_{m+1}} & .\\
	\end{cases}\eqno(6)
	$$
	Возведя обе части равенства~$ (6_{1}) $ в квадрат, будем иметь равенство
	$$ b_{m-1}^{2} + 2 z_{m+1} (-b_{m-1}) + z_{m+1}^{2} = \frac{a_{m}^{2}}{a_{m+1}^{2}} , $$
	из которого, учитывая неравенство~(5) при $ i=m-1 $, получаем
	$$ \frac{a_{m}^{2}}{a_{m+1}^{2}} \geqslant b_{m-2} + z_{m+1} (-b_{m-1}) . $$
	Отсюда, с учётом равенства~$ (6_{2}) $, убеждаемся в справедливости неравенства~$ (2_{1}) $ для $ n=m+1 $ и $ k=1 $.

	Далее докажем справедливость неравенства~$ (2_{k}) $ при $ n=m+1 $ и $ 1 < k \leqslant \frac{m+1}{2} $. Для этого возведём сначала обе части равенства~$ (6_{k}) $ в квадрат. В результате получим
	$$ \frac{a_{m-k+1}^{2}}{a_{m+1}^{2}} = b_{m-k}^{2} - 2 b_{m-k} \cdot b_{m-k+1} \cdot z_{m+1} + z_{m+1}^{2} b_{m-k+1}^{2} . \eqno(7) $$
	Затем перемножив, соответственно, левые и правые части равенств~$ (6_{k-1}) $, $ (6_{k+1}) $, будем иметь
	$$
	\frac{a_{m-k} \cdot a_{m-k+2}}{a_{m+1}^{2}} = b_{m-k+1} \cdot b_{m-k-1} - b_{m-k-1} \cdot b_{m-k+2} z_{m+1} - \nonumber $$
	$$ - b_{m-k} \cdot b_{m-k+1} \cdot z_{m+1} + z_{m+1}^{2} \cdot b_{m-k} \cdot b_{m-k+2} . \eqno(8)$$
		Из~(5) при $ i=m-k+1 $ и $ i=m-k $ имеем, что
	$$ b_{m-k+1}^{2} \geqslant b_{m-k} \cdot b_{m-k+2}, b_{m-k}^{2} \geqslant b_{m-k-1} \cdot b_{m-k+1} , \eqno(9) $$
	из которых следует неравенство
	$$ b_{m-k} \cdot b_{m-k+1} \geqslant b_{m-k-1} \cdot b_{m-k+2} . \eqno(10) $$
	Из (9) – (10) вытекает, что правая часть~(8) меньше правой части равенства~(7). Отсюда следует справедливость неравенства~$ (2_{k}) $ при $ n=m+1 $ и $ k \leqslant \frac{m+1}{2} $. Остальные неравенства~(2), то есть неравенства~$ (2) $ при $ n=m+1 $ и $ \frac{m+1}{2} < k \leqslant m $ вытекают из доказанных выше, если учесть что в том случае, когда многочлен $ f_{m+1}(z) $ имеет отличные от нуля вещественные корни, то и многочлен $ v_{m+1}(z) = a_{0} z^{m+1} + \ldots + a_{m+1} $ также имеет вещественные корни. Здесь следует отметить, что утверждение, сформулированное в последнем предложении, следует из равенства
	$$ f_{m+1}(z) = z^{m+1} v_{m+1}(\frac{1}{z}) . $$

	Итак, мы доказали, что если утверждение теоремы~2 справедливо для всех $ n $ от $ 2 $ до $ m $, то оно справедливо и для $ n=m+1 $. Тогда, на основании метода математической индукции, заключаем, что теорема~2 справедлива для любого $ n \geqslant 2 $.
	% \begin{flushright}
		Теорема~2 доказана.
	% \end{flushright}
%%% \end{proof}

\textbf{Замечание~1.} {\it Константу 1, присутствующую в неравенствах $ a_{n-k}^{2} \geqslant 1 \cdot a_{n-k-1} \cdot a_{n-k+1} $ теоремы~2, невозможно увеличить. Сказанное подтверждается следующей аргументацией. Рассмотрим функцию
	$$ f_{n}(z) = (z+1)^{n}. $$
	Коэффициенты $ a_{k} $ этого многочлена равны
	$$ a_{k}=( _{n}^{k}) , \quad k = 0, 1, \ldots, n , $$
	Но для этих коэффициентов не справедливо неравенство
	$$ a_{k}^{2} \geqslant q a_{k-1} a_{k+1} $$
	при некотором $ q > 1 $, достаточно больших $ n \geqslant 2 $ и $ k = 1, 2, \ldots, n-1 $.
}

\textbf{Замечание~2.} {\it Доказательство теоремы~2 можно свести к теореме~53 из работы [3, c.~69], но доказательство которой практически отсутствует. }

Итак, мы привели одно из возможных доказательств \\ необходимых условий М.Г. Крейна вещественности корней многочленов, которое является простым и экономным.

%%% Не смог добиться корректного переноса, поэтому свтавил разрыв строки

% Оформление списка литературы
\smallskip \centerline {\bf Литература} \nopagebreak

1. {\it Левин~А.~Ю.} Элементарный признак вещественности корней целой функции с положительными коэфициентами. Воронеж: Проблемы мат. анализа сложных систем, 1968, №2. С 72–77.

2. {\it Куликов~Л.~Я.} Алгебра и теория чисел. М.: Высшая школа, 1979, — 559 с.

3. {\it Харди~Г., Литтльвуд~Д., Полиа~Г.} Неравенства. М.: Изд-во иностр. лит., 1948.
