\begin{center}
    {\bf О ЛИНЕАРИЗАТОРЕ ГИПЕРБОЛИЧЕСКИХ ПУЧКОВ}

    {\it Л.И. Сухочева}

    (Воронеж; {\it l.suhocheva@yandex.ru})
\end{center}

\addcontentsline{toc}{section}{Сухочева Л.И.}

Объектом нашего рассмотрения являются квадратичные операторные пучки:
$$L = \lambda^2I + \lambda B + C.\eqno{(1)}$$

Один из подходов изучения спектральных свойств операторных пучков состоит в том, чтобы поставить пучку $L$ в соответствие оператор $X$, такой что существует взаимнооднозначное соответствие между спектром пучка $L$ и спектром оператора $X$ и по жордановым цепочкам оператора $X$ можно построить жордановы цепочки пучка $L$ и наоборот. В этом случае говорят, что оператор $X$ является линеаризатором пучка $L$.

М.Г. Крейн и Г. Лангер были одними из первых, кто предложил изучать спектральные свойства квадратичных операторных пучков вида (1), где $B$ - ограниченный самосопряжённый оператор, $C$ - положительный компактный оператор в гильбертовом пространстве $H$, используя ассоциированный оператор
$X = (\begin{smallmatrix}
  0& C^{1/2}\\
  - C^{1/2}& - B
\end{smallmatrix}).$

Пусть пучок (1) является гиперболическим, т.е.
$$(Bf,f)^2 - 4(If,f)(Cf,f) > 0$$
(для всех $f\neq 0$). Тогда существует $\alpha > 0$, такое что оператор $X + \alpha I$ является равномерно положительным в пространстве Крейна $\tilde{H} = H_{+}\oplus H_{-}$,\,\,
$H_{+} = (\begin{smallmatrix}
  x\\
  0
\end{smallmatrix})$, \,\,$H_{-} = (\begin{smallmatrix}
  0\\
  y
\end{smallmatrix}),$ \,\,$[. , .] = (J., .)$,\,\,$J = (\begin{smallmatrix}
  I& 0\\
  0& - I
\end{smallmatrix}).$ Обозначим $X + I = A$.

\textbf{Теорема.} \textit{Пусть $A$ - произвольный равномерно положительный оператор в пространстве Крейна $\tilde{H} = H_{+}\oplus H_{-}$, тогда существует $\alpha > 0$, такое что оператор $A - \alpha I$ будет подобен линеаризатору некоторого гиперболического пучка тогда и только тогда, когда  $\dim H_{+} = \dim H_{-}$. В этом случае  $\alpha > \max\{\lambda | \lambda\in\sigma(A)\}$.
}
