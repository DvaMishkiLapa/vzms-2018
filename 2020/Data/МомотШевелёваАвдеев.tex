\vzmstitle[]{
	Рейтинг орфографической грамотности математических образовательных ресурсов в сети <<Интернет>>
    (декабрь-январь 2019/20)
}
\vzmsauthor{Момот}{Е.\,А.}
\vzmsinfo{Воронеж, ВГУ; {\it y-kate@ukr.net }}
\vzmsauthor{Шевелева}{К.\,В.}
\vzmsinfo{Воронеж, ВГУ; {\it ksyusha.shevelyova@yandex.ru }}
\vzmsauthor{Авдеев}{Н.\,Н.}
\vzmsinfo{Воронеж, ВГУ; {\it nickkolok@mail.ru} }
\small УДК 372.851+811.161.1'32
\vzmscaption


В настоящее время Интернет играет большую роль в разнообразных сферах общественной жизни. Но особенную
обеспокоенность научного сообщества вызывает грамотность интернет-ресурсов, так как в данном промежутке
времени именно интернет оказывает значительное влияние на образование и самообразование современных школьников и
студентов~[1,2].
В 2017 году впервые было проведено междисциплинарное исследование на стыке информатики, математики, педагогики и
лингвистики, посвященное изучению орфографической грамотности интернет-ресурсов, посвящённых математике.
В продолжение исследований [3,4] был проведён третий ежегодный мониторинг.

Ошибки в user-generated content (UGC), как правило, не исправляются;
с авторскими монотекстами (АМТ) ситуация иная:
разосланные авторами письма возымели эффект
(см., напр., \linebreak matematikalegko.ru в табл.),
поэтому при построении рейтинга с 2018 года учитываются только АМТ.
Применимость данной методики построения рейтинга обсуждается в [5].

Список изучаемых сайтов был сформирован на основе из верхних строчек рейтинга LiveInternet в рубрике <<Образование>>
и некоторых других известных сайтов,
за исключением:
сайтов, не относящихся к математике;
UGC (комментариев, форумов, отзывов);
сайтов"=агрегаторов (например, studopedia.org);
а также сайтов, на которых тексты по математике не удалось отделить от текстов по другим,
не смежным дисциплинам.

Результаты приведены в таблице.
$N$ "--- общее количество выявленных ошибок,
$\nu$ "--- количество ошибок на 1000 словоупотреблений,
$P$ "--- количество обработанных словоупотреблений.

В трёхлетней ретроспективе достаточно чётко прослеживается группа сайтов,
объём корпусатекстов на которых достаточно стабилен
(некоторые изменения могут быть вызваны, например,
динамическими блоками <<Рекомендуем также статьи: ...>>,
попадающими в обработку,
или компактными правками).
Эти ресурсы неактивны в производстве контента,
однако сохраняют высокую посещаемость (и, вероятно, рентабельность за счёт рекламы)
благодаря накопленным материалам.

Отдельно прокомментируем ошибки на fipi.ru.
Обе ошибки обнаружены программой в слове <<тренинг>>,
которое на сайте написано как <<треннинг>>.
Это слово входит в состав цитаты или наименования внешнего по отношению к ФИПИ объекта.
Тем не менее, в отличие от цитат наподобие <<и делай с ним что хошь!>>,
авторы сочли такое написание ошибочным.

Заметим, что чувствительность программы к ошибкам неук\-лон\-но увеличивается.
В общем случае это не может оказать решающего влияния на количество найденных ошибок в корпусе~[5];
однако с учётом рассылаемых писем владельцам сайтов работа в этом направлении достаточно важна.
Тем не менее,
итоговое среднее количество ошибок на 1000 СУ не увеличилось по сравнению с прошлым годом, что не может не радовать.

Авторы снова планируют отправить владельцам сайтов письма с подробными отчётами
и продолжать ежегодный мониторинг.

Исходный код программы опубликован на условиях лицензии GNU GPLv3 по адресу:
https://github.com/nickkolok/chas-correct

{\footnotesize
\begin{landscape}
%\scriptsize
%\tiny
%\hspace{-\parskip}
\begin{tabular}{|l|r|l|r|l|r|r|l|r|}
\hline
	\multicolumn{1}{|c|}{Ресурс}
  & \multicolumn{1}{|c|}{$N$, 2017}
  & \multicolumn{1}{|c|}{$\nu$, 2017}%(2017)
  & \multicolumn{1}{|c|}{$N$, 2018}%(2018)
  & \multicolumn{1}{|c|}{$\nu$, 2018}%(2018)
  & \multicolumn{1}{|c|}{$P$, 2018}%(2018)
  & \multicolumn{1}{|c|}{$N$, 2019}%(2019)
  & \multicolumn{1}{|c|}{$\nu$, 2019}%(2019)
  & \multicolumn{1}{|c|}{$P$, 2019}%(2019)
  \\
\hline
algebraclass.ru      &     0  & 0      &   0  & 0      &    54047  &    0  &  0       &    55618  \\
egetrener.ru         &     -  & -      &   0  & 0      &   835394  &    0  &  0       &   835916  \\
cleverstudents.ru    &    15  & 0,0391 &   0  & 0      &   388265  &    2  &  0,0049  &   406487  \\
fipi.ru*             &     -  & -      &   0  & 0      &   311967  &    2  &  0,0061  &   328675  \\
mathprofi.ru         &     -  & -      &   6  & 0,0109 &   549649  &    6  &  0,0101  &   595091  \\
reshuege.ru          &     -  & -      &  17  & 0,0121 &  1407188  &   21  &  0,0128  &  1640583  \\
Википедия**          &     -  & -      & 287  & 0,0255 & 11264755  &    -  &  -       &        -  \\
hijos.ru             &     8  & 0,0311 &   4  & 0,0183 &   219103  &   10  &  0,0276  &   362371  \\
1cov-edu.ru          &     0  & 0      &   0  & 0      &    94353  &    4  &  0,0297  &   134905  \\
webmath.ru           &     -  & -      &  33  & 0,0490 &   673005  &   40  &  0,0359  &  1113337  \\
ru.solverbook.com    &     0  & 0      &  26  & 0,0207 &  1254589  &   28  &  0,0435  &   643346  \\
matematikalegko.ru   &    64  & 0,3708 &  21  & 0,0583 &   360274  &   34  &  0,0506  &   671518  \\
ege-ok.ru            &     -  & -      &  25  & 0,0639 &   391124  &   20  &  0,0570  &   350729  \\
kpolyakov.spb.ru     &     -  & -      &  20  & 0,1119 &   178719  &   12  &  0,0608  &   197398  \\
studizba.com         &     -  & -      &  37  & 0,0578 &   640068  &   60  &  0,0831  &   734375  \\
raum.math.ru         &     -  & -      &   4  & 0,0843 &    47480  &   10  &  0,0833  &   120120  \\
ru.math.wikia.com    &    21  & 0,0961 &  23  & 0,1270 &   181140  &   54  &  0,0845  &   639220  \\
egemaximum.ru        &     -  & -      &  21  & 0,0474 &   442790  &   48  &  0,1069  &   449249  \\
nuru.ru              &     1  & 0,0538 &   2  & 0,1077 &    18575  &    3  &  0,1618  &    18547  \\
alexlarin.net        &     -  & -      &   8  & 0,0861 &    92892  &   17  &  0,1745  &    97452  \\
mathsolution.ru      &     -  & -      & 106  & 0,6548 &   161892  &   29  &  0,1866  &   155397  \\
ru.onlinemschool.com &    21  & 0,0544 &  82  & 0,3674 &   223197  &  142  &  0,3096  &   458674  \\
calcs.su             &     -  & -      &   2  & 0,2125 &     9411  &    6  &  0,5502  &    10905  \\
dxdy.ru (UGC)*       & 18455  & 1,1239 &   -  & -      &        -  &    -  &  -       &        -  \\
\hline
Итого                &     -  & -      & 724  & 0,0995 & 19487910  &  546  &  0,0987  &  9846635  \\
\hline
\end{tabular}
\end{landscape}
}
* --- данные приводятся для справок, в итоговых не учтены
\\
** --- категория <<Математика>> и все вложенные подкатегории
\\

\litlist

1. {\it Каменкова Н. Г.}
 Использование интернет"=технологий при организации изучения курса <<математика и информатика>>
 // Герценовские чтения. Начальное образование. – 2010. – Т. 1. – С. 288-293.

2. {\it Сон Л. П.}
 Интернет-коммуникация и проблема грамотности индивида // Научно-информационный журнал Армия и общество – 2013. – № 4 (36). – С. 87-91.

3. {\it Авдеев Н. Н.}
 Программа анализа грамотности интернет-СМИ // Культура общения и её формирование, межвузовский сборник научных трудов. – 2016. – С. 81-83.

4. {\it Авдеев Н. Н., Шевелева К. В.}
 Анализ орфографической грамотности математических образовательных ресурсов в сети <<Интернет>> // Некоторые вопросы анализа,
 алгебры, геометрии и математического образования – Воронеж: Издательско-полиграфический центр <<Научная книга>>, 2017. – Вып. 7, Часть I – С. 7-8.

5. {\it Авдеев Н. Н., Шевелева К. В.}
 Применимость регулярного выражения как математической модели орфографической ошибки // Сборник Международной конференции <<Актуальные
 проблемы прикладной математики, информатики и механики>> – Воронеж: Издательство <<Научно-исследовательские публикации>> – 2018. – C. 335-340.
