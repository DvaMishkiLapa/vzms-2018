\begin{center}
    {\bf МОДЕЛИРОВАНИЕ СЛОЕНИЙ ИНТЕГРИРУЕМЫХ СИСТЕМ БИЛЛИАРДАМИ НА КЛЕТОЧНЫХ КОМПЛЕКСАХ\footnote{ Работе выполнена при поддержке гранта Президента Российской Федерации для государственной поддержки молодых российских ученых – кандидатов наук.}}\\

    {\it В.В. Ведюшкина}

    (Москва; {\it arinir@yandex.ru})
\end{center}

\addcontentsline{toc}{section}{Ведюшкина В.В.}
В работе [1] А.Т.\,Фоменко выдвинул фундаментальную гипотезу о моделировании (реализации) биллиардами интегрируемых систем с двумя степенями свободы.
В настоящей работе мы анализируем раздел $C$ этой гипотезы.

\textbf{Гипотеза C (реализация меченых молекул).} Широкий класс инвариантов Фоменко--Цишанга (т.е. меченых молекул [2], задающих с точностью до лиувиллевой эквивалентности множество всех интегрируемых систем), которые моделируются интегрируемыми биллиардами. Тем самым, для многих невырожденных интегрируемых систем их слоения Лиувилля на  инвариантных
трехмерных поверхностях (возможно, все такие слоения) послойно гомеоморфны слоениям интегрируемых биллиардных систем из подходящего класса.

В данный момент эта гипотеза доказана для большого числа интегрируемых гамильтоновых систем, известных в математической физике, механике и геометрии. В частности, для многих классических случаев интегрируемости в динамике твердого тела (например, для многих зон энергии случаев Эйлера, Лагранжа, Ковалевской, Горячева--Чаплыгина, Клебша, Соколова, Стеклова, Ковалевской--Яхьи и др.). Гипотеза $C$ также доказана В.\,В.~Ведюшкиной и А.\,Т.~Фоменко для всех интегрируемых при помощи линейных и квадратичных интегралов геодезических потоков на ориентированных замкнутых двумерных поверхностях, т.е. для потоков на торе и на сфере.

 В качестве естественного класса плоских интегрируемых биллиардов рассматриваются биллиарды, ограниченные дугами софокусных квадрик. Такие биллиарды называются в работах В.В.Ведюшкиной элементарными. Их интегрируемость эквивалентна малой теоремы Понселе: любая траектория такого биллиарда лежит на прямых, касательных к некоторой квадрике --- эллипсу или  гиперболе -- софокусных с квадриками, образующими границу биллиарда. Топологические биллиарды и биллиардные книжки получаются склейками таких биллиардов вдоль сегментов их границ. Если вдоль граничного сегмента склеено больше двух элементарных биллиардов-листов (случай биллиардной книжки), то этому ребру склейки необходимо приписать перестановку, которая определяет порядок перехода биллиардной частицы с одного биллиарда на другой при ударе об это ребро склейки. Очевидно, что биллиардные книжки, также как и элементарные биллиарды, интегрируемы с той же парой интегралов.

 Вопрос о справедливости гипотезы Фоменко $C$ в полном объеме пока неясен. В связи с этим А.Т. Фоменко сформулировал ``локальный'' вариант гипотезы $C$, являющийся ``максимальным упрощением'' общей гипотезы $C$.

\textbf{1.} Пусть $\gamma$ --- произвольное ребро с метками $r, \varepsilon$ некоторой меченой молекулы $W^{*}$. Тогда существует интегрируемый биллиард, реализующий такую комбинацию чисел $r, \varepsilon$ на одном из ребер своей меченой молекулы.

Отметим, что имеются следующие четыре варианта:  метка $r = p \slash q$ конечна, и $\varepsilon = \pm 1$; метка $r = \infty$, и $\varepsilon = \pm 1$.

\textbf{2. } (усиление пункта \textbf{1}) В условиях пункта \textbf{1} существует подходящий биллиард, реализующий произвольную пару меток  $r$ и $\varepsilon$ на ребре между любыми, наперед заданными атомами.

\textbf{3.} Пусть $S$ --- семья с целочисленной  меткой $n$ в некоторой меченой молекуле $W^{*}$ интегрируемой системы. Тогда существует  интегрируемый биллиард, реализующий некоторую семью с точно такой же целочисленной меткой $n$.

\textbf{4. } (усиление пункта \textbf{3}) В условиях пункта \textbf{2} существует подходящий биллиард, реализующий не только данную метку $n$, но и саму семью, т.е. граф с нужными атомами и нужным набором ребер.

\textbf{5. } (реализация меченой окрестности любой семьи) Пусть $S$ --- семья с целочисленной меткой $n$ в некоторой меченой молекуле, причем внешние ребра $\gamma_i$ семьи оснащены произвольными метками $r_i, \varepsilon_i$. Тогда существует подходящий биллиард, реализующий такой меченый подграф в своей меченой молекуле.


Анализу и частичному доказательству    этой гипотезы посвящен доклад.




% Оформление списка литературы
\smallskip \centerline {\bf Литература} \nopagebreak

1. {\it Ведюшкина В.В., Фоменко А.Т.} Бильярды и интегрируемость в геометрии и физике. Новый взгляд и новые возможности. Вестн.Моск.Унив., Серия Матем. и Мех. 2019. \textbf{3}. 15--25.

2. {\it Фоменко А.Т., Цишанг Х.} Топологический инвариант и критерий
эквивалентности интегрируемых гамильтоновых систем с двумя степенями
свободы. Изв. АН СССР.Матем.,
1990.\textbf{54}, \textnumero3.
 546--575.

