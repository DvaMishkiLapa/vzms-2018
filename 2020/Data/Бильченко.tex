
\begin{center}
    {\bf  ОБ  ОДНОМ  СВОЙСТВЕ  КРИВЫХ,
ЗАДАВАЕМЫХ  ОПРЕДЕЛЯЮЩИМИ  УРАВНЕНИЯМИ.
ЧАСТЬ  2.}

    {\it  Г.Г.  Бильченко(ст.)}

    (Казань;  {\it  ggbil40@gmail.com})
\end{center}

\addcontentsline{toc}{section}{Бильченко  Г.Г.(ст.)}
%% ===============================
%%  Бильченко
%%  Григорий  (ст.)
%%  Григорьевич
%%
%%  Об  одном  свойстве  кривых,
%%  задаваемых  определяющими  уравнениями.
%%  Часть  2.
%%
%%  к.т.н.,  доцент
%%
%%  Казанский национальный исследовательский технический университет им. А.Н.Туполева-КАИ
%%
%%  Секция (направление) / Section:  Нелинейный анализ и математическое моделирование
%% ===============================



    Рассматривается  движение
механической  системы,
состоящей  из  носителя  и  груза,
совершающего  заданное  движение
по  отношению  к  носителю
%%% ==== \\
[1{\textbf{--}}3].
%%% ==== //
В  алгоритмах  установления  типа
двусторонних  движений  носителя
по  горизонтальной  плоскости
используются  определяющие  выражения
%%% ==== \\
[4].
%%% ==== //



    В  рассматриваемом  сообщении
устанавливается  важное  свойство
определяющих  уравнений,
записанных  для  определяющих  выражений
%% =============== \\
$I_{2}\left(\varphi; \beta\right)$
%% =============== //
и
%% =============== \\
$I_{3}\left(\varphi; \beta\right)$%
%% =============== //
~{\textbf{--}}
отсутствие  других  точек  пересечения,
кроме  единственной  точки
%% =============== \\
$\displaystyle
M_{0}\left(
  0; \beta_{0}
  \right)$
%% =============== //
%%% ==== \\
[5].
%%% ==== //



    \textbf{Ключевые  слова:}
носитель,
груз,
угол  установки  канала,
определяющие  уравнения,
двусторонние  движения  носителя.




%% ===============================
%% ===============================
%% ===============================
\textbf{1.  Дифференциальные  уравнения
движения  носителя.}\nopagebreak
%% ===============================
%% ===============================


%\indent
    Рассматривается  движение
механической  системы,
состоящей  из  носителя  и  груза
%%% ==== \\
[1{\textbf{--}}4].
%%% ==== //
Носитель,  располагаясь  всё  время
в  горизонтальной  плоскости,
двигается  поступательно
по  прямолинейной  траектории.
%%%
Носитель  имеет  прямолинейный  канал,
по  которому  может  перемещаться  груз.
%%%
Ось  канала  расположена  в  вертикальной  плоскости,
проходящей  через  траекторию  носителя.
%%%%
Пусть  закон  движения  груза  в  канале
задан  в  виде
%% =============== \\
$\ell \cdot \sin (\omega t)$,
%% =============== //
где
%% =============== \\
$\ell=const$,
%% =============== **
$\omega=const$,
%% =============== //
а  силы  сопротивления  среды
движению  носителя
моделируются  силами
типа  кулонова  трения.
%%%
Тогда
дифференциальные  уравнения
движения  носителя
(ДУДН),
согласно
%%% ==== \\
[1,
 2],
%%% ==== //
будут  следующими
%% =============================== \\
\begingroup\belowdisplayskip=\belowdisplayshortskip
\[
\ddot{x}
=\beta \cdot
\left (
    \cos \varphi
    +f\cdot \sin \varphi
  \right )
\cdot \sin \left( \omega t\right)
	-\gamma
\phantom{0}
\quad
\mbox{при}
\quad
\dot{x}>0
\,{;}
\eqno(1)
\]
\endgroup
%% =============================== *
%%%%%%%%%%%%%%%%%%%%%%%%%
%%% Page  1 / Page  2 %%%
%%%%%%%%%%%%%%%%%%%%%%%%%
\[
\ddot{x}=
\beta \cdot
\left (
    \cos \varphi
    -f\cdot \sin \varphi
  \right )
\cdot \sin \left( \omega t\right)
	+\gamma
\phantom{0}
\quad
\mbox{при}
\quad
\dot{x}<0
\,{;}
\eqno(2)
\]
%% =============================== *
\[
\ddot{x}=
0
\phantom{\beta \cdot
\left (
    \cos \varphi
    +f\cdot \sin \varphi
  \right )
\cdot \sin \left( \omega t\right)
	-\gamma}
\quad
\mbox{при}
\quad
\dot{x}=0
\,{,}
\eqno(3)
\]
%% =============================== //
где
$x$~{\textbf{--}}
координата  носителя;
%% ===============
$\varphi$~{\textbf{--}}
угол  установки  канала;
%% =============== \\
$\displaystyle
\beta
\,{=}\,
\ell \,{\cdot}\, \omega^{2}
\,{\cdot}\,
\frac{m}{m+M}$;
%% =============== *
$\gamma\,{=}\,g\,{\cdot}\,f$;
%% =============== //
$M$~{\textbf{--}}  масса  носителя;
$m$~{\textbf{--}}  масса  груза;
$g$~{\textbf{--}}  ускорение  свободного  падения;
$f$~{\textbf{--}}  коэффициент
трения  скольжения
в  движении,
равный
коэффициенту
трения  скольжения
в  покое,
для пары материалов
<<носитель~{\textbf{--}}
подстилающая  горизонтальная  плоскость>>.
%%%
Пусть  угол  установки  канала
%% =============================== \\
\[
0\leq
 \varphi < \frac{\pi}{2}
\,{.}
\eqno(4)
\]
%% =============================== //
Предполагается,
что  выполнено  неравенство
%%% ==== \\
[3]
%%% ==== //
%% =============================== \\
\[
\beta \cdot
\sin \varphi
\leq
g
\,{,}
\eqno(5)
\]
%% =============================== //
которое  гарантирует
безотрывное  от  горизонтальной  плоскости
движение  носителя.
%%%
Если
%% =============== \\
$\beta\geq g$,
%% =============== //
то  из
%%% ==== \\
(5)
%%% ==== //
вытекает  вместо
%%% ==== \\
(4)
%%% ==== //
ограничение  на  угол  установки  канала
%% =============================== \\
\[
0
\leq
\varphi
\leq
\arcsin \left(
            \frac{g}{\beta}
          \right)
{.}
\eqno(6)
\]
%% =============================== //




%% ===============================
%% ===============================
%% ===============================
\textbf{2.  Условия  двусторонних  движений  носителя
из  состояния  покоя.}\nopagebreak
%% ===============================
%% ===============================


%\indent
    Условием  того,
чтобы  ДУДН
%%% ==== \\
(1)
%%% ==== //
имело  место
в  действительной  динамике  носителя
является  неравенство
%%% ============================ \\
\[
\beta \cdot
\left (
    \cos \varphi
    + f \cdot \sin \varphi
  \right )
>	 \gamma
\,{,}
\eqno(7)
\]
%%% ============================ //
а  условием  реализации  ДУДН
%%% ==== \\
(2)
%%% ==== //
в  действительной  динамике
будет  неравенство
%%% ============================ \\
\[
\beta \cdot
\left (
    \cos \varphi
    - f \cdot \sin \varphi
  \right )
>	 \gamma
\,{.}
\eqno(8)
\]
%%% ============================ //
%%%%%%%%%%%%%%%%%%%%%%%%%
%%% Page  2 / Page  3 %%%
%%%%%%%%%%%%%%%%%%%%%%%%%
При  этом,
так  как
$\varphi \geq 0$,
то
$\cos \varphi + f \cdot \sin \varphi>0$
всегда,
%%% ========
а  из  требования
$\cos \varphi - f \cdot \sin \varphi>0$
вытекает  ограничение
на  угол  установки  канала
в  виде
%%% ============================ \\
\[
0\leq
\varphi<
\arctg
      \frac{1}
           {f}
\,{.}
\eqno(9)
\]
%%% ============================ //



    Ранее
%%% ==== \\
[3]
%%% ==== //
было  установлено,
что  если
%%% ============================ \\
\[
\beta >
\dfrac{\gamma}{\sqrt{1+f^2}}
\,{,}
\eqno(10)
\]
%%% ============================ //
а
%%% ============================ \\
$
\varphi
\in
\overrightarrow{\,\Phi\,}
\equiv
\left(
    \overrightarrow{\varphi\,}\!_{1};
    \overrightarrow{\varphi\,}\!_{2}
  \right)
$,
%%% ============================ //
где
%%% ============================ \\
\begingroup\belowdisplayskip=\belowdisplayshortskip
\[
\overrightarrow{\varphi\,}\!_{1}
\left(
    \beta
    \right)
=
2\arctg \frac
            {\beta \cdot f
             - \sqrt{\beta^{2}(1+f^{2})
                     -\gamma^{2}
                     }
             }
            {\beta+\gamma}
\,{,}
\]
\endgroup
%% =============================== *
\[
\overrightarrow{\varphi\,}\!_{2}
\left(
    \beta
    \right)
=
2\arctg \frac
            {\beta \cdot f
             + \sqrt{\beta^{2}(1+f^{2})
                     -\gamma^{2}
                     }
             }
            {\beta+\gamma}
\,{,}
\]
%%% ============================ //
то  носитель  может  двигаться
из  состояния  покоя
в  положительном  направлении  оси
$Ox$,
т.е.,
ДУДН
%%% ==== \\
(1)
%%% ==== //
может  иметь  место
в  действительной  динамике  носителя.



	 Если
%%% ==== \\
(10)
%%% ==== //
выполнено,
а
%%% ============================ \\
$
\varphi
\in
\overleftarrow{\,\Phi\,}
\equiv
\left(
\overleftarrow{\varphi\,}\!_{1};
\overleftarrow{\varphi\,}\!_{2}
\right)
$,
%%% ============================ //
где
%%% ============================ \\
$
\overleftarrow{\varphi\,}\!_{1}
=
{}-\overrightarrow{\varphi\,}\!_{2\,}
$,
%%%% ==== **
$
\overleftarrow{\varphi\,}\!_{2}
=
{}-\overrightarrow{\varphi\,}\!_{1\,}
$,
%%% ============================ //
то  носитель  может  двигаться
из  состояния  покоя
в  отрицательном  направлении  оси
$Ox$,
т.е.,  ДУДН
%%% ==== \\
(2)
%%% ==== //
может  иметь  место.



   Было  показано
%%% ==== \\
[3],
%%% ==== //
что  если
%%% ============================ \\
\[
\beta>\gamma
\,{,}
\eqno(11)
\]
%%% ============================ //
а  угол  установки  канала
%%% ============================ \\
$
\varphi
\in
\lefteqn{\overleftarrow{\phantom{\,\,\Phi\,\,}}}
\overrightarrow{\,\,\Phi\,\,}
\equiv
\left(
\overrightarrow{\varphi\,}\!_{1};
\overleftarrow{\varphi\,}\!_{2}
\right)
$,
%%% ============================ //
то  из  состояния  покоя  носитель
может  двигаться
как  в  положительном,
так  и  в  отрицательном  направлении  оси
$Ox$,
т.е.,
могут  реализовываться  ДУДН
%%% ==== \\
(1)
%%% ==== //
и
%%% ==== \\
(2).
%%% ==== //
%%%%%%%%%%%%%%%%%%%%%%%%%
%%% Page  3 / Page  4 %%%
%%%%%%%%%%%%%%%%%%%%%%%%%



   Тогда,
с  учётом
%%% ==== \\
(6),
%%% ==== //
двусторонние  движения  носителя
из  состояния  покоя
в  рассматриваемом  случае  возможны,
если
%% =============================== \\
\begingroup\belowdisplayskip=\belowdisplayshortskip
\[
\varphi
\;{\in}\:\!
\left[
    0
    ;
    \overleftarrow{\varphi\,}_{\! 2}
    \left(\beta\right)
  \right)
\qquad
\qquad
\quad
\text{при}
\qquad
\beta_{1}<\beta < \beta_{2}
\,{;}
%%
\eqno{(12.1)}
\]
\endgroup
%% =============================== *
\[
\varphi
\;{\in}
\left[
    0
    ;
    \arcsin\left(\frac{g}{\beta}\right)
  \right]
\qquad
\;\;
\text{при}
\qquad
\beta_{2}\leq\beta
\,{,}\;
\hphantom{\leq\beta_{1}}
%%
\eqno{(12.2)}
\]
%% =============================== //
где
$\beta_{1}=\gamma$;
%%% ====
$\beta_{2}=
g\cdot
   \sqrt
     {1+4\cdot  f^{2}
      }
$%
~{\textbf{--}}
корень  уравнения
%% =============================== \\
\[
2\arctg
\frac
{{}-\beta_{2}\cdot f
 + \sqrt
     {\beta_{2}^{2} \left(1+f^{2}\right)
      - \gamma^{2}
      }
 }
{\beta_{2}+\gamma}
=
\arcsin\left(\frac{g}{\beta_{2}}
\right)
{.}
\eqno{(13)}
\]
%% =============================== //




%% ===============================
%% ===============================
%% ===============================
\textbf{3.  Определяющие  уравнения.}\nopagebreak
%% ===============================
%% ===============================



%\indent
    Пусть  угол  установки  канала  таков,
что  выполняются  условия
%%% ==== \\
(12).
%%% ==== //
%%%
При  этом  вводятся
%% =============================== \\
\[
\tau _{+}
\,{=}\,
\frac{1}{\omega }
\arcsin
            \frac{\gamma_{+}}
                 {\beta }
\,{,}
%% =============================== *
\qquad
%% =============================== *
\tau _{-}
\,{=}\,
\frac{1}{\omega }
\arcsin
            \frac{\gamma_{-}}
                 {\beta }
\,{,}
\]
%% =============================== //
где
%% =============================== \\
\[
\gamma_{+}
\,{=}\,
    \frac{\gamma}
         {\cos \varphi
          +f \cdot
          \sin \varphi}
\,{,}
%% =============================== *
\qquad
%% =============================== *
\gamma_{-}
\,{=}\,
    \frac{\gamma}
         {\cos \varphi
          -f \cdot
          \sin \varphi}
\,{.}
\]
%% =============================== //
Затем,  следуя
%%% ==== \\
[4],
%%% ==== //
выделяются  определяющие  выражения
%% =============================== \\
\begingroup\belowdisplayskip=\belowdisplayshortskip
\[
I_{1}
\left(
  \varphi; \beta
  \right)
=
\sqrt{\beta^2-\gamma_{+}^2}
+
\sqrt{\beta^2-\gamma_{-}^2}
-
{}
\]
\endgroup
%% =============================== *
\[
{}
-
\gamma_{+} \cdot
\left [\pi+
\arcsin
\left(
    \frac{\gamma_{-}}{\beta}
  \right)
-
\arcsin
\left(
    \frac{\gamma_{+}}{\beta}
  \right)
\right]
\notag
{;}
\]
%% =============================== ***
\[
I_{2}
\left(
  \varphi; \beta
  \right)
=
\sqrt{\beta^2-\gamma_{+}^2}
+
\sqrt{\beta^2-\gamma_{-}^2}
-{}
\]
%% =============================== *
\[
{}-
\gamma_{-} \cdot
\left [\pi+
\arcsin
\left(
    \frac{\gamma_{+}}{\beta}
  \right)
-
\arcsin
\left(
    \frac{\gamma_{-}}{\beta}
  \right)
\right]
{;}
\]
%% =============================== ***
\[
I_{3}
\left(
  \varphi; \beta
  \right)
=
\sqrt{\beta^2-\gamma_{+}^2}
-
\frac{\gamma\pi}
     {\cos\varphi}
+{}
\]
%% =============================== *
%%%%%%%%%%%%%%%%%%%%%%%%%
%%% Page  4 / Page  5 %%%
%%%%%%%%%%%%%%%%%%%%%%%%%
\[
{}
+
\beta
\cdot
\cos
\left [
\arcsin
\left(
    \frac{\gamma_{+}}{\beta}
  \right)
+\pi\cdot
f
\cdot\tg
  \varphi
  \right ]
{,}
\]
%% =============================== //
используя  которые,
можно  записать
определяющие  уравнения
%% =============================== \\
\begingroup\belowdisplayskip=\belowdisplayshortskip
\[
I_{1}
\left(
  \varphi; \beta
  \right)
=
0
\,{;}
\eqno(14)
\]
\endgroup
%% =============================== *
\[
I_{2}
\left(
  \varphi; \beta
  \right)
=
0
\,{;}
\eqno(15)
\]
%% =============================== *
\[
I_{3}
\left(
  \varphi; \beta
  \right)
=
0
\,{.}
\eqno(16)
\]
%% =============================== //



    Определяющее\,
выражение\,
$I_{3}
\left(
  \varphi; \beta
  \right)\,$
привлекается\,
для\,
установ\-ле\-ния
типа  движения  носителя,
когда
$I_{1}
\left(
  \varphi; \beta
  \right)>0$
и
$I_{2}
\left(
  \varphi; \beta
  \right)>  0$.
%%%
При  этом
%%%
[4]
%% =============================== \\
\[
\hphantom{{}-}
\int\limits
_{\tau_{+} }
^{\frac{T}{2} +\tau_{-} }
\Bigl [
    \beta \cdot
    \left(
        \cos \varphi
        +f\cdot \sin \varphi
      \right)
    \cdot
    \sin \left( \omega t\right)
       -{\gamma}
  \Bigr ]
\cdot{dt}
>0
\hphantom{\,{.}}
\eqno{(17)}
\]
%% =============================== **
и
%% =============================== **
\[
{}-\int\limits
_{\frac{T}{2} +\tau_{-} }
^{T +\tau_{+}}
\Bigl [
    \beta \cdot
    \left (
        \cos \varphi
        -f\cdot \sin \varphi
      \right )
    \cdot
    \sin \left( \omega t\right)
    +{\gamma}
  \Bigr ]
\cdot{dt}
>  0
\,{.}
\eqno{(18)}
\]
%% =============================== //
Неравенства
(17)  и  (18)
приводятся   к  виду
%% =============================== \\
\[
\beta
\left [
    \cos
    \left(
        \omega \tau_{+}
      \right)
    +
    \cos
    \left(
        \omega \tau_{-}
      \right)
  \right ]
-
\gamma_{+}
\left [
    \pi
    +\omega \tau_{-}
    -\omega \tau_{+}
  \right ]
  >0
\hphantom{\,{.}}
\eqno{(19)}
\]
%% =============================== **
и
%% =============================== **
\[
\beta
\left [
    \cos
    \left(
        \omega \tau_{+}
      \right)
    +
    \cos
    \left(
        \omega \tau_{-}
      \right)
  \right ]
-
\gamma_{-}
\left [
    \pi
    +\omega \tau_{+}
    -\omega \tau_{-}
  \right ]
<  0
\,{.}
\eqno{(20)}
\]
%% =============================== //
Неравенства
(19)  и  (20)
сводятся
к  одному  неравенству
%% =============================== \\
\[
\gamma_{-}
\left [
    \pi
    +\omega \tau_{+}
    -\omega \tau_{-}
  \right ]
>
\gamma_{+}
\left [
    \pi
    +\omega \tau_{-}
    -\omega \tau_{+}
  \right ]
\,{,}
\]
%% =============================== //
которое  принимает  вид
%% =============================== \\
\[
\arcsin
\left(
    \frac{\gamma_{-}}{\beta}
  \right)
-
\arcsin
\left(
    \frac{\gamma_{+}}{\beta}
  \right)
<
\pi
\cdot
f
\cdot
\tg\varphi
\,{,}
\eqno{(21)}
\]
%% =============================== //
%%%%%%%%%%%%%%%%%%%%%%%%%
%%% Page  5 / Page  6 %%%
%%%%%%%%%%%%%%%%%%%%%%%%%
т.е.,
при
%% =============================== \\
$I_{1}
\left(
  \varphi; \beta
  \right)>0$
%% =============================== //
и
%% =============================== \\
$I_{2}
\left(
  \varphi; \beta
  \right)>  0$
%% =============================== //
рассматриваются
только  такие  механические  системы
<<носитель~{\textbf{--}} груз>>,
параметры  которых
удовлетворяют  условию
%%% ==== \\
(21).
%%% ==== //



    Если
$\varphi = 0$,
то
%% =============================== \\
\begingroup\belowdisplayskip=\belowdisplayshortskip
\[
\gamma_{+}
=\gamma_{-}
=\gamma=g\cdot f
\,{,}
\]
\endgroup
%% =============================== *
\[
I_{2}=I_{3}=I=
2\sqrt{\beta^2-\gamma^2}
-\gamma\cdot\pi
\,{,}
\]
%% =============================== //
а
%%% ==== \\
(15)
%%% ==== //
и
%%% ==== \\
(16)
%%% ==== //
приводятся  к  одному  виду
%% =============================== \\
$I=0$.
%% =============================== //
Откуда
%% =============================== \\
\[
\beta_{0}=
\gamma\cdot
\frac{\sqrt{{\pi}^2+4\,}}
     {2}
\,{.}
\eqno{(22)}
\]
%% =============================== //
Т.е.,  на  плоскости
%% =============================== \\
$\left(
  \varphi; \beta
  \right)$
%% =============================== //
кривые,  задаваемые  уравнениями
%%% ==== \\
(15)
%%% ==== //
и
%%% ==== \\
(16),
%%% ==== //
имеют  одну  общую  точку
%% =============================== \\
$M_{0}\left(
  0; \beta_{0}
  \right)$.
%% =============================== //




   Пусть  теперь
$\varphi \neq 0$.
%%%
Покажем,  что  на  плоскости
%% =============================== \\
$\left(
  \varphi; \beta
  \right)$
%% =============================== //
в  области,
где  определённые  условиями
%%% ==== \\
(12)
%%% ==== //
допустимы  двусторонние  движения  носителя
из  состояния  покоя,
не  существует  ни  одной  точки
%% =============================== \\
$M_{\ast}\left(
  \varphi_{\ast}; \beta_{\ast}
  \right)$,
%% =============================== //
в  которой
%% =============================== \\
$I_{2}\left(
  \varphi_{\ast}; \beta_{\ast}
  \right)=0$
%% =============================== //
и
%% =============================== \\
$I_{3}\left(
  \varphi_{\ast}; \beta_{\ast}
  \right)=0$,
%% =============================== //
отличной  от  точки
%% =============================== \\
$M_{0}\left(
  0; \beta_{0}
  \right)$.
%% =============================== //



    Сделаем  предположение,
что  такая  точка  найдётся,
т.е.,  определяющие  уравнения
%%% ==== \\
(15)
%%% ==== //
и
%%% ==== \\
(16)
%%% ==== //
совместны  при  условии
%%% ==== \\
(21).
%%% ==== //



    При  этом
несовпадающие  члены  уравнений
%%% ==== \\
(15)
%%% ==== //
и
%%% ==== \\
(16)
%%% ==== //
должны  быть  равными,
т.е.,
%% =============================== \\
\begingroup\belowdisplayskip=\belowdisplayshortskip
\[
\sqrt{\beta^2-\gamma_{-}^2}
-
\gamma_{-} \cdot
\left [\pi+
\arcsin
\left(
    \frac{\gamma_{+}}{\beta}
  \right)
-
\arcsin
\left(
    \frac{\gamma_{-}}{\beta}
  \right)
\right]
=
{}
\]
\endgroup
%% =============================== *
\[
{}
=
\beta
\cdot
\cos
\left [
\arcsin
\left(
    \frac{\gamma_{+}}{\beta}
  \right)
+\pi\cdot
f
\cdot\tg
  \varphi
  \right ]
-
\frac{\gamma\pi}
     {\cos\varphi}
\,{.}
\eqno(23)
\]
%% =============================== //



    После  замены
%% =============================== \\
$\beta=k\cdot\gamma_{-\,}$,
%% =============================== //
где
$k > 1$,
условие
(21)
запишется  в  виде
%% =============================== \\
\[
\arcsin
\left(
    \frac{1}{k}
  \right)
-
\arcsin
\left(
    \frac{1}{k}
    \cdot
    \frac{1-z}{1+z}
  \right)
<\pi
 \cdot
z
\,{,}
\eqno(24)
\]
%% =============================== //
%%%%%%%%%%%%%%%%%%%%%%%%%
%%% Page  6 / Page  7 %%%
%%%%%%%%%%%%%%%%%%%%%%%%%
а  уравнение
(23)
будет  иметь  вид
%% =============================== \\
\begingroup\belowdisplayskip=\belowdisplayshortskip
\[
k
\cdot
\cos
\left [
\arcsin
\left(
    \frac{1}{k}
    \cdot
    \frac{1-z}{1+z}
  \right)
+\pi\cdot
z
  \right ]
+
{}
\]
\endgroup
%% =============================== *
\[
{}
+
\arcsin
\left(
    \frac{1}{k}
    \cdot
    \frac{1-z}{1+z}
  \right)
+\pi\cdot
z
=
\sqrt{k^2-1}
+\arcsin
\left(
    \frac{1}{k}
  \right)
{,}
\eqno(25)
\]
%% =============================== //
где
$z=f
\cdot\tg
  \varphi$.
%%%
Уравнение
(25)
удовлетворяется  если
%% =============================== \\
\[
\arcsin
\left(
    \frac{1}{k}
    \cdot
    \frac{1-z}{1+z}
  \right)
+\pi
 \cdot
z
=
\arcsin
\left(
    \frac{1}{k}
  \right)
{,}
\]
%% =============================== //
т.е.,
%% =============================== \\
\[
\arcsin
\left(
    \frac{1}{k}
  \right)
-
\arcsin
\left(
    \frac{1}{k}
    \cdot
    \frac{1-z}{1+z}
  \right)
=
\pi
 \cdot
z
\,{.}
\eqno(26)
\]
%% =============================== //
Уравнение
%%% ==== \\
(26)
%%% ==== //
противоречит  условию
%%% ==== \\
(24),
%%% ==== //
что  доказывает
факт  отсутствия
других  точек  пересечения  кривых,
задаваемых  определяющими  уравнениями
%%% ==== \\
(15)
%%% ==== //
и
%%% ==== \\
(16),
%%% ==== //
кроме  единственной  точки
%% =============================== \\
$\displaystyle
M_{0}\left(
  0; \beta_{0}
  \right)$.
%% =============================== //



    \textbf{Заключение.}
Установленные
в  Частях  1  и  2
свойства  кривых,
задаваемых  определяющими  уравнениями
$I_{1}=0$,
$I_{2}=0$
и
$I_{3}=0$,
гарантируют  разбиение  области
плоскости
%% =============================== \\
$\left(\varphi; \beta\right)$,
%% =============================== //
где
определённые  условиями
(12)
допустимы  двусторонние  движения
носителя  из  состояния  покоя,
на  подобласти,
в  каждой  из  которых
будет  своя  комбинация  значений
определяющих  выражений,
что  используется
в  процессе  установления
типа  движений  носителя.




% Оформление списка литературы
\smallskip \centerline {\bf Литература} \nopagebreak



%% ============================== \\
1.~%
\textit%
{Бильченко~Г.~Г.~}
{%
  {Влияние  подвижного  груза
  на  динамику  носителя}%
%~/  {Г.~Г.~Биль\-чен\-ко}%
~/$\!$/
  Тезисы  докладов  международной  конференции
  <<Конструктивный  негладкий  анализ
  и  смежные  вопросы>>,
  посвящённой  памяти  профессора
  В.~Ф.~Демьянова
  (CNSA~{\textbf{--}}  2017,
  г.~Санкт-Петербург,
  22{\textbf{--}}27
  мая  2017~г.).~{\textbf{---}}
  Ч.~I.~{\textbf{---}}
  СПб.:  Изд-во  ВВМ,
  2017.~{\textbf{---}}
  С.~218{\textbf{--}}224.%
  }
%% ============================== //
%%%%%%%%%%%%%%%%%%%%%%%%%
%%% Page  7 / Page  8 %%%
%%%%%%%%%%%%%%%%%%%%%%%%%



%% ============================== \\
2.~%
\textit%
{Бильченко~Г.~Г.~}
{%
  {Влияние  подвижного  груза
  на  движение  носителя}%
%~/  {Г.~Г.~Биль\-чен\-ко}%
~/$\!$/
  Аналитическая  механика,
  устойчивость  и  управление:
  Труды  XI  Международной
  Четаевской  конференции.~{\textbf{--}}
  Т.~1.
  Секция~1.
  Аналитическая  Механика.
  Казань,
  13{\textbf{--}}17
  июня  2017~г.~{\textbf{---}}
  Казань:  Изд-во  КНИТУ-КАИ,
  2017.~{\textbf{---}}
  С.~37{\textbf{--}}44.%
  }
%% ============================== //



%% ============================== \\
3.~%
\textit%
{Бильченко~Г.~Г.~}
{%
  {Движение  носителя  с  подвижным  грузом
  по  горизонтальной  плоскости}%
%~/  {Г.~Г.~Биль\-чен\-ко}%
~/$\!$/
  Сборник  материалов  международной  конференции
  <<XXVIII  Крымская  Осенняя
  Математическая  Школа-симпозиум
  по  спектральным  и  эволюционным  задачам>>
  (КРОМШ~{\textbf{--}} 2017).
  Секции
  1{\textbf{--}}4.~{\textbf{---}}
  Симферополь:  ДИАЙПИ,
  2017.~{\textbf{---}}
  С.~58{\textbf{--}}61.%
  }
%% ============================== //



%% ============================== \\
4.~%
\textit%
{Бильченко~Г.~Г.~}
{%
  {Алгоритмы
  установления  типа
  двусторонних  движений  носителя
  с  подвижным  грузом
  по  горизонтальной  плоскости}%
%~/  {Г.~Г.~Бильченко}%
~/$\!$/
  <<Актуальные  проблемы
  прикладной  математики,  информатики
  и  механики>>:
  Сборник  трудов  Международной
  научно-технической  конференции,
  Воронеж,
  17{\textbf{--}}19
  декабря  2018~г.~{\textbf{---}}
  Воронеж:
  Изд-во
  <<Научно-%
  ис\-сле\-до\-ва\-тель\-ские
  публикации>>,
  2019.~{\textbf{---}}
  С.~605{\textbf{--}}611.%
  }
%% ============================== //



%% ============================== \\
5.~%
\textit%
{Бильченко~Г.~Г.~}
{%
  {Об  одном  свойстве  кривых,
  задаваемых
  определяющими  уравнениями.
  Часть  1.}%
%~/
%  {Г.~Г.~Бильченко}%
~/$\!$/
  <<Актуальные  проблемы
  прикладной  математики,  информатики
  и  механики>>:
  Сборник  трудов  Международной
  научно-тех\-ни\-чес\-кой
  конференции,
  Воронеж,
  11{\textbf{--}}13
  ноября  2019~г.~{\textbf{---}}
  Воронеж:
  Изд-во
  <<Научно-исследовательские
  публикации>>,
  2019.%~{\textbf{---}}
  %С.~???{\textbf{--}}???.%
  }
%% ============================== //


%%%%%%%%%%%%%%%%%%%%%%%%%
%%% Page  8 / Page  - %%%
%%%%%%%%%%%%%%%%%%%%%%%%%
