
\begin{center}
{\bf ИССЛЕДОВАНИЕ ВОЗМУЩЕННОЙ МОДЕЛИ ЛЕОНТЬЕВА МЕЖОТРАСЛЕВОГО БАЛАНСА}

{\it М.А. Кащенко, В.И. Усков}

(Воронеж; {\it vum1@yandex.ru})
\end{center}

\addcontentsline{toc}{section}{Кащенко М.А., Усков В.И.}

Рассматривается задача Коши для уравнения Леонтьева межотраслевого баланса, возмущенного операторной добавкой:
\[A\frac{dx}{dt}=(B+\varepsilon C)x(t,\varepsilon),\eqno{(1)}\]

\[x(0,\varepsilon)=x^0(\varepsilon).\eqno{(2)}\]

Здесь $A=\mathcal{B}$ --- матрица коэффициентов приростной фондоёмкости, $B=I-\mathcal{A}$, где $\mathcal{A}$ --- матрица прямых затрат, $x$ --- вектор-столбец валовой продукции, $t\in[0,T]$, $\varepsilon\in(0,\varepsilon_0$). Функция $x^0(\varepsilon)$ голоморфна в окрестности точки $\varepsilon=0$.

Вектор-функция $A\frac{dx}{dt}$ называется акселератором Харрода [1].

Такие модели с непрерывными функциями времени отражают условия динамического равновесия валового и конечного продукта в экономике страны.

Целью работы является исследование влияния возмущения, вызываемого малым параметром в задаче (1), (2) с конкретными значениями коэффициентов. Отметим, что в случае вырожденного оператора $A$, стоящего  перед производной, влияние может быть значительным. Модель рассматривается на примере трех отраслей народного хозяйства: $A,B,C:{\bf{R}}^3\to{\bf{R}}^3$, $x(t,\varepsilon)\in {\bf{R}}^3$.

Известно, что оператор $A$, задаваемый вырожденной \\ квадратной матрицей, фредгольмов.

Зададим матрицы коэффициентов в уравнении (1):
\[A=\left( \begin{array}{ccc}
0.2 & 0.3 & 0.4 \\
0.4 & 0.6 & 0.8 \\
0.1 & 0.1 & 0.3 \end{array}
\right), \quad B=\left( \begin{array}{ccc}
0.2 & 0.4 & 0.4 \\
0.1 & 0.4 & 0.1 \\
0.3 & 0.1 & 0.3 \end{array}
\right),\]

\[C=\left( \begin{array}{ccc}
0.1 & 0.1 & 0 \\
0.1 & 0 & 0.3 \\
0 & 0 & 0.1 \end{array}
\right).\]

Ядро оператора $A$ одномерно. Имеем:
\[{\rm Ker}\,A=\left\{\left( \begin{array}{c}
x_1 \\
-\frac{2}{5}x_1 \\
-\frac{1}{5}x_1 \end{array}
\right)\right\}, \quad {\rm Coim}\,A=\left\{\left(\begin{array}{c}
0 \\
\frac{2}{5}x_1+x_2 \\
\frac{1}{5}x_1+x_3 \end{array}
\right)\right\},\]
\[{\rm Im}\,A=\left\{ \begin{array}{c}
\left(\begin{array}{c}
y_1 \\
2y_1 \\
y_3 \end{array}
\right) \end{array}
\right\}, \quad {\rm Coker}\,A=\left\{ \begin{array}{c}
\left( \begin{array}{c}
0 \\
y_2-2y_1 \\
0 \end{array}
\right) \end{array}
\right\},\]
\[e=\left( \begin{array}{c}
1 \\
-\frac{2}{5} \\
-\frac{1}{5} \end{array}
\right), \quad \varphi=\left( \begin{array}{c}
0 \\
1 \\
0 \end{array}
\right),\]
\[Q=\left( \begin{array}{ccc}
0 & 0 & 0 \\
-2 & 1 & 0 \\
0 & 0 & 0 \end{array}
\right), \quad A^{-}=\left( \begin{array}{ccc}
0 & 0 & 0 \\
6 & 0 & -8 \\
-2 & 0 & 6 \end{array}
\right).\]

Применим результаты, полученные в работе [2].

\[d_{00}=<QBe,\varphi >=0,\]

\[d_{01}=<QCe,\varphi >=-0.08\neq 0,\]

\[d_{10}=<QBA^-Be,\varphi >=-0.16\neq 0.\]
Следовательно,
\[\frac{d_{10}}{d_{01}}=2>0.\eqno{(3)}\]

Таким образом, операторная пара $(A,B)$ регулярна; длина $B$-жордановой цепочки равна 2.

Получен следующий результат.

\textbf{Теорема.} \textit{В задаче {\rm (1)}, {\rm (2)} имеет место явление погранслоя.}

Условие (3) --- это условие регулярности вырождения.

\textbf{Замечание.} \textit{Если не выполняется условие регулярности вырождения, то это влечет за собой большое расхождение между планируемым объемом производства ($\varepsilon =0$) и полученным на практике.}


% Оформление списка литературы
\smallskip \centerline {\bf Литература} \nopagebreak


1. Динамические многоотраслевые модели [электронный ресурс]. -- Режим доступа: \\ http://vsh1791.ru/sbks/BKS/EMM2/05.pdf (дата обращения: 19.11.2019).

2. {\it Зубова С. П., Усков В. И.}  Асимптотическое решение задачи Коши для уравнения первого порядка с малым параметром в банаховом пространстве. Регулярный случай // Математические заметки. -- 2018. -- Т. 103, вып. 3. -- С. 393-404.

3. {\it Кащенко М. А.} Решение задачи Коши для динамической модели В. Леонтьева с вырожденным коэффициентом фондоемкости // Материалы Всеукраинской научно-практической конференции соискателей высшего образования и молодых ученых 11-12 апреля 2019 года: Современные и исторические проблемы фундаментальные и прикладныематематической подготовки в учреждениях высшего образования. -- Харьков: ХНАДУ, 2019. -- С. 167-170.



