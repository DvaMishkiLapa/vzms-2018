\vzmstitle{ОБ ОПЕРАТОРНЫХ СЕМЕЙСТВАХ НА МНОГООБРАЗИИ С НЕГЛАДКИМ КРАЕМ}
\vzmsauthor{Васильев}{В.\,Б.}
\vzmsinfo{Белгород; {\it vbv57@inbox.ru}}
\vzmscaption

Материал этого доклада основан на двух авторских препринтах [1,2], которые развивают абстрактные операторные идеи И.Б. Симоненко [3] с одной стороны и факторизационные идеи Г.И. Эскина для эллиптических символов псевдодифференциальных операторов с другой стороны [4].

Рассматриваются специальные классы операторов, действующих в функциональных пространствах на многообразиях. Такие операторы называются операторами локального типа и определяются с точностью до компактного оператора [3]. Этот подход можно назвать операторно"=геометрической трактовкой хорошо известного локального принципа. Описываются в абстрактной форме условия фредгольмовости рассматриваемых операторов и показывается, как эти результаты можно применить к исследованию эллиптических псевдодифференциальных операторов на многообразиях с негладкой границей.

Пусть $M$ -- компактное многообразие с краем $\partial M$, и $A(x)$ -- некоторая оператор"=функция, определённая на $M$. Выделим на границе $\partial M$ подмногообразия $M_k, k=0,1,...,m-1,$ как гладкие $k$-мерные многообразия так, что по определению $M_{m-1}\equiv\partial M, M_0$ представляет собой совокупность изолированных точек $\partial M, M_m\equiv M$. Наконец, введём множество классов операторов ${\rm T}_k, k=0,1,...,m,$ так, что для $x\in M_k$ оператор $A(x): H^{(1)}_k\rightarrow H^{(2)}_k$ является линейным ограниченным оператором, а $H^{(j)}_k, k=0,1,...,m, j=1,2,$ -- некоторые банаховы пространства.


Такое подмногообразие $M_k$ мы называем сингулярным $k$-подмногообразием, если для всех $x\in M_k$ имеет место включение $A(x)\in{\rm T}_k$.

\textbf{Теорема 1.} {\it
Если семейство $A(x)$ состоит из локальных фредгольмовых операторов и непрерывно на каждой компоненте $\overline{M_k\setminus\cup_{i=0}^{k-1}M_i}, k=0,1,...,m,$ то оно порождает единственный фредгольмов оператор
$A'$, действующий в пространствах прямых сумм $\sum\limits_{k=0}^m\oplus H^{(k)}_1\rightarrow\sum\limits_{k=0}^m\oplus H^{(k)}_2$.
}


Такой оператор $A'$ мы называем виртуальным оператором, соответствующим семейству $A(x)$. Виртуальный оператор $A'$ называется эллиптическим, если семейство $A(x)$ состоит из фредгольмовых операторов для всех $x\in M$.

Предложенная абстрактная схема применяется для исследования псевдодифференциального оператора $A$ на $m$-мерном компактном многообразии $M$ с краем, содержащим особенности. Такой оператор обычно определяют с помощью символа $A(x,\xi), (x,\xi)\in{\bf R}^{2m}$ (здесь мы используем локальные координаты; обычно символ задаётся на (ко)касательном расслоении). Предполагается, что на границе $\partial M$ имеются гладкие компактные подмногообразия $M_k$ размерности $0\leq k\leq m-1$, которые представляют собой особенности границы. Эти особенности границы описываются специальными локальными представителями оператора $A$ в точке $x_0\in M$ на карте $U\ni x_0$ следующим образом
\[
(A_{x_0}u)(x)=\int\limits_{D_{x_0}}\int\limits_{{\bf R}^m}e^{i\xi\cdot(x-y)}A(\varphi(x_0),\xi)u(y)d\xi dy,~~~x\in D_{x_0},
\]
где $\varphi :U\to D_{x_0}$ -- диффеоморфизм, и каноническая область $D_{x_0}$ имеет разную форму в зависимости от расположения точки $x_0$ на многообразии $M$. Рассматриваются следующие канонические области:  $D_{x_0}$: ${\bf R}^m, {\bf R}^m_+=\{x\in{\bf R}^m: x=(x',x_m), x_m>0\}, W^k={\bf R}^k\times C^{m-k}$, где $C^{m-k}$ -- выпуклый конус в ${\bf R}^{m-k}$. Другими словами, граница $\partial M$ может содержать конические точки и ребра различных размерностей.

Псевдодифференциальный оператор $A$ изучается в пространствах Соболева--Слободецкого $H^s(M)$, и в качестве локального варианта этих пространств выбираются пространства  $H^s(D_{x_0})$.

Можно предложить следующие достаточные условия фредгольмовости.

\textbf{Теорема 2.} {\it
Предположим, что классический эллиптический символ $A(x,\xi), x\in M_k,$ допускает $k$-волновую факторизацию относительно конусов $C^{m-k}$ с индексами $\kappa_k(x), k=0,1,\cdots,m-2$ [5], удовлетворяющими условиям:
$$
|\kappa_k(x)-s|<1/2,~~~\forall x\in M_k,~~~k=0,1,\cdots,m-1.\eqno(1)
$$

Тогда оператор $A: H^s(M)\rightarrow H^{s-\alpha}(M)$ фредгольмов.
}


\textbf{Замечание.} {\it Если эллиптичность нарушается на подмногообразиях $M_k$, рассматриваются модификации оператора $A$ с привлечением граничных или кограничных операторов $[4,5]$. В частности, это происходит, когда нарушается одно из условий $(1)$.
}



% Оформление списка литературы
\smallskip \centerline {\bf Литература} \nopagebreak

1. {\it Vasilyev V.B.} Operator symbols. ArXiv: Math.FA/1901.06630, pp. 1--11.

2. {\it Vasilyev V.B.} Operator symbols. II. ArXiv: Math.FA/1911.08099, pp. 1--13.

3. {\it Симоненко И.Б.} Локальный метод в теории инвариантных относительно сдвига операторов и их огибающих. Ростов"=на-Дону, ЦВВР, 2007. ---120 с.

4. {\it Эскин Г.И.} Краевые задачи для эллиптических псевдодифференциальных уравнений. М.: Наука, 1973. — 232 с.

5. {\it Васильев В.Б.} Мультипликаторы интегралов Фурье, псевдодифференциальные уравнения, волновая факторизация, краевые задачи. М.: КомКнига, 2010.---135 с.
