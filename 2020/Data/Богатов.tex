\vzmstitle[\footnote{Работа выполнена при поддежке РФФИ, проект 20-011-00402.}]{О РАЗВИТИИ ТЕОРИИ ПОЛОЖИТЕЛЬНЫХ ОПЕРАТОРОВ И ВКЛАДЕ М.А. КРАСНОСЕЛЬСКОГО}
\vzmsauthor{Богатов}{Е.\,М.}
\vzmsinfo{Старый Оскол-Губкин; {\it embogatov@inbox.ru}}
\vzmscaption

В начале XX в. в математическом мире обнаружился интерес к матрицам вида
$$
A=\{a_{ij}\geq 0\}. \ \ \ \eqno(1)
$$

С одной стороны, это было связано с изучением малых колебаний упругих континуумов (развитие идей Ш.Ф. Штурма), а сдругой - с появлением марковских цепей и стохастических матриц.
Первые результаты о положительных собственных значениях матриц (1) были получены О. Перроном и Ф.Г. Фробениусом в 1907-1908 гг. [1]\footnote{Ссылка на работу [1] здесь и далее означает, что в ней имеются выходные данные статей цитируемых в тексте авторов.}.

Исследования И. Фредгольма и Д. Гильберта открыли дорогу для обобщения теоремы Перрона-Фробениуса\footnote{О существовании наибольшего по модулю собственного значения матрицы (1).} на интегральные уравнения Фредгольма 2-го рода с положительным ядром. Это обобщение было выполнено учеником Фробениуса Р. Ентчем в 1912г. [1].

Более интересные результаты были получены московским математиком П.С. Урысоном в 1918 г. [1] в контексте исследования положительной разрешимости нелинейного уравнения
$$
y=\mu\int\limits_a^bK(x,s,y(s))ds. \ \ \ \eqno(2)
$$
Урысон доказал существование положительных собственных значений (2), пользуясь методом последовательных приближений и оценил спектральный интервал порождённого задачей (2) оператора.

В 1935 г. П.С. Александров и Х. Хопф продемонстрировали оригинальный подход к доказательству теоремы существования положительных решений матричного уравнения
$$
Ay=\lambda y, \ \ \ \eqno(3)
$$
применив теорему Брауэра о неподвижной точке [1].

Дальнейшее продвижение в исследованиях положительных решений (3), где $A$ "--- положительный оператор произвольной природы, оказалось более естественным проделывать в функционально"=аналитическом контексте. Оно было связано с именем М.Г. Крейна, который ввёл понятие конуса $K$ в банахово пространство $E$ и позитивного функционала в $K$, исследуя проблему моментов\footnote{Здесь видятся минимум три побудительных мотива. Первый~--- дальнейшая <<геометризация>> банаховых пространств в стиле польской математической школы, обобщение теоремы Асколи-Мазура. Второй - возможность развития идей Александрова-Хопфа на основе теоремы Шаудера о неподвижной точке, в которой фигурирует выпуклое подмножество банахова пространства \textit{E}, инвариантное относительно оператора $A$. И третий - изоморфность множества положительных функций из пространства $C[a,b]$ некоторому конусу $K\subset E$.}.

Отметим, что понятие позитивного функционала было также дано в работе Л.В. Канторовича [2] также в рамках исследования проблемы моментов (1937). Более того, конце 1930-х гг. им была создана теория полуупорядоченных пространств, позволяющая, в частности, ответить на вопрос о положительной разрешимости уравнений вида (3) в одной из самых общих постановок [1].

С конца 1930-х гг. М. Крейн вместе со своим учеником М. Рутманом стали активно разрабатывать теорию конусов в банаховых пространствах. При этом б\textit{о}льшая часть результатов была получена ими для линейных операторов (в пространствах с конусом), и уже на их основе стало возможно обобщение теоремы Ентча на нелинейный случай. Итогом 10-летней работы М. Крейна и М. Рутмана стала фундаментальная статья, опубликованная в УМН в 1948 г. [1] и вошедшая в золотой фонд отечественной науки.

Следующий этап в развитии теории положительных операторов ассоциируется с именем М.А. Красносельского и воронежской математической школой [3]. С 1947 по 1952 г., Красносельский работал в НИИ математики АН УССР, посещал семинары М.Г. Крейна по функциональному анализу и Н.Н. Боголюбова по нелинейной механике. Это оказало существенное влияние на его научное мировоззрение. В результате
\begin{itemize}
	\item теория конусов попала в число приоритетных для Красносельского методов \textit{исследования нелинейных интегральных уравнений}\footnote{Основной темой  исследований М.А. Красносельского в то время.};
		\item знание передовых идей и актуальных задач нелинейной механики\footnote{Модельной задачей, рассматриваемой Красносельским, приводящей к уравнению (2), была задача о продольном изгибе шарнирно"=опёртого стержня переменной жёсткости.} дало возможность для проверки работоспособности разрабатываемых конусных методов.
	\end{itemize}
В исследованиях М.А. Красносельского по теории положительных операторов можно выделить два отправных пункта - теоретический (восходящий к логике развития математики), и прикладной (относящийся к задачам нелинейной механики).
	
	Отметим, что Красносельский не ограничился применением теории конусов к доказательству теорем существования решений уравнений вида (3), как это сделали М. Рутман, Э. Роте и Г. Биркгоф [1]. Он стал рассматривать эту теорию, как дополнительную возможность (наряду с вариационными методами, теорией ветвления и теорией вращения векторных полей) для изучения качественных свойств решений уравнения (3) c нелинейным оператором \textit{A}, действующим в банаховом пространстве. Сюда, в том числе, входил поиск ответа на вопросы о структуре и кратности спектра \textit{A}, о структуре множества собственных функций \textit{A}, об условиях сходимости метода последовательных приближений (3) и др.
	
	Деятельность Красносельского предполагала наличие активных участников научного процесса, в первую очередь студентов и аспирантов. Почти сразу после своего приезда в Воронеж (1952 г.) он привлёк к научной работе в указанной области молодых математиков - Л.А. Ладыженского, И.А. Бахтина, В.Я. Стеценко, Ю.В. Покорного и др. Прицип разработки нового научного направления, характерный для Красносельского, был сформулирован в статье Б.Н. Садовского [4, c.147]:
	
	{\textquotedbl}
	...\textit{Стержнем работы являлся всегда семинар с обязательным привлечением сотрудников факультета, аспирантов и студентов. За 3-4 года по данной тематике готовилось и издавалось в журналах 2-3 десятка публикаций, содержащих оригинальные результаты. Одновременно, с самого начала, обычно писался и текст монографии, который в последующем претерпевал ряд изменений}...
	{\textquotedbl}
	
Итогом 10-летней деятельности созданного Красносельским научного коллектива явилась монография [5], переведённая в 1964 г. на английский язык.

Одним из наиболее интересных (с моей точки зрения) результатов, нашедших своё продолжение в дальнейшем в работах отечественных и зарубежных математиков [6, с. 4-5] явилась так называемая <<\textit{конусная теорема Красносельского о неподвижной точке}>> \, [5, гл. 4, §2,4]:

\textit{Пусть вполне непрерывный оператор A сжимает или растягивает конус K. Тогда оператор A имеет в конусе K по крайней мере одну неподвижную точку.}

Специалист по нелинейному функциональному анализу, профессор М.К. Квонг (КНР) указывает на то, что данная теорема может быть интерпретирована за рамками метрического восприятия и тем самым \textit{поставлена в один ряд с теоремой о неподвижной точке Брауэра-Шаудера} [6, с. 2].

Усилиями Красносельского и его учеников теория положительных операторов внесла весомый вклад в развитие нелинейного функционального анализа (1950-1970 гг.). Отметим здесь появление \textit{новых} типов операторов и теорем о неподвижной точке, \textit{новых}	 методов исследования спектральных свойств операторов и\textit{ ранее неизвестных} применений производных операторов (по конусу).

В докладе будут представлены некоторые подробности развития теории конусов в работах воронежской математической школы и анализ дальнейшего развития этой теории в исследованиях отечественных и зарубежных учёных (1960 -- 1980 гг.) с использованием результатов [7].


% Оформление списка литературы
\smallskip \centerline {\bf Литература} \nopagebreak

1. \textit{Богатов Е.М.} Oб истории теории конусов и полуупорядоченных пространств (в контексте развития нелинейного функционального анализа) / В сб. Алгебра, теория чисел и дискретная геометрия: современные проблемы, приложения и проблемы истории. Материалы XVI междунар. конф., посвящ. 80-летию со дня рождения проф. Мишеля Деза (13-18 мая 2019). - Тула, ТГПУ им. Л.Н. Толстого. 2019. с. 322-325.

2. \textit{Канторович Л.В.} К проблеме моментов для конечного интервала // Доклады АН СССР, Т. XIV (1937), вып. 9. С. 531-536.

3. \textit{Садовский Б.Н.} Научная школа М.А. Красносельского в Воронеже / В сб. Материалы к истории математического факультета ВГУ. Воронеж, ВГУ, 1998. с. 34-50.

4. Марк Александрович Красносельский. К 80-летию со дня рождения. Сб. статей. М.: ИППИ РАН, 2000. - 216 с.

5. \textit{Красносельский М. А.} Положительные решения операторных уравнений. М.: ФИЗМАТГИ3, 1962. - 394 с.

6. \textit{Kwong M. K.} The topological nature of Krasnoselskii's cone fixed point theorem // Nonlinear Analysis: Theory, Methods \verb'&' Applications. - 2008. - V. 69. - Iss. 3. - p. 891-897.

7. \textit{Богатов Е.М.} О развитии теории конусов в работах отечественных математиков / В сб. Годичная научная конференция ИИЕТ РАН им. С.И. Вавилова, 2019. Саратов: Амирит, 2019. с. 224-227.
